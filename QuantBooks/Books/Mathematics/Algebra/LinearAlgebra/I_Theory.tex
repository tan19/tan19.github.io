%!TEX root = LinearAlgebra.tex

\part{Theory}
\chapter{Preliminaries}
\deff{
	A {\bf{field}} is a non-empty set $F$ {\em{closed}} under two operations, usually called {\em{addition}} and {\em{multiplication}}\footnote{Subtraction and division are defined implicitly in terms of the inverse operations of addition and multiplication.}, and denoted by $+$ and $\cdot$ respectively, such that the following {\em{nine}} axioms hold
	\begin{enumerate}
		\item[(1-2).] Associativity of addition and multiplication.
		\item[(3-4).] Commutativity of addition and multiplication.
		\item[(5-6).] Existence and uniqueness of additive and multiplicative identity elements.
		\item[(7-8).] Existence and uniqueness of additive inverses and multiplicative inverses.
		\item[(9).] Distributivity of multiplication over addition.
	\end{enumerate}
}
\deff{
	The characteristic of a ring $R$, $char(R)$, is the smallest positive integer $n$ such that
	$$\underbrace{1+\cdots+1}_{n \text{ summands}} = 0$$
}
\thm{
	Any finite ring has nonzero characteristic.
}

\chapter{Vector Spaces}
\section{Vector Space}
\deff{
	A {\bf{vector space}} over a field $\FF$ is a {\em{nonempty}} set $V$ together with the operations of addition $V\times V \to V$ and scalar multiplication $\FF \times V \to V$ satisfying the following {\em{eight}} properties:
	\begin{enumerate}[(-)]
		\item Additive axioms. For every $\u,\v,\w \in V$, we have
		\begin{enumerate}[(1)]
			\item $\u+\v = \v+\u$
			\item $(\u+\v)+\w = \u+(\v+\w)$
			\item ${\bf{0}}+\u = \u+{\bf{0}}=\u$, where ${\bf{0}} \in V$ is unique for all $\u \in V$
			\item $(-\u)+\u = \u+(-\u) = {\bf{0}}$, where $-\u \in V$ is unique for every $\u \in V$
		\end{enumerate}
		\item Multiplicative axioms. For every $\u \in V$ and scalars $a, b \in \FF$, we have
		\begin{enumerate}[(1)]
			\item $1\x = \x$
			\item $(ab)\x = a(b\x)$
		\end{enumerate}
		\item Distributive axioms. For every $\u, \v \in V$ and scalars $a, b \in \FF$, we have
		\begin{enumerate}[(1)]
			\item a(\u+\v) = a\u + a\v
			\item (a+b)\u = a\u + b\u
		\end{enumerate}
	\end{enumerate}
}
\section{Subspaces}
\deff{
	A subspace of $\RR^n$ is any collection $S$ of vectors in $\RR^n$ such that
	\begin{enumerate}[(1)]
		\item The zero vector $\mathbf{0}$ is in $S$.
		\item If $\u$ and $\v$ are in $S$, then $\u+\v$ is in $S$. \footnote{$S$ is closed under addition.}
		\item If $\u$ is in $S$ and $c$ is a scalar, then $c\u$ is in $S$. \footnote{$S$ is closed under scalar multiplication.}
	\end{enumerate}
}

\deff{
	Let $S, T$ be two subspaces of $\RR^n$. We say $S$ is orthogonal to $T$ if {\em{every}} vector in $S$ is orthogonal to {\em{every}} vector in $T$. The subspace $\{\mathbf{0}\}$ is orthogonal to all subspaces. \footnote{A line can be orthogonal to another line, or it can be orthogonal to a plane, but a plane cannot be orthogonal to a plane.}
}

\deff{
	Let $A$ be an $m \times n$ matrix.
	\begin{enumerate}[(1)]
		\item The {\em{row space}} of $A$ is the subspace $row(A)$ of $\RR^n$ spanned by the rows of $A$.
		\item The {\em{column space}} (or {\em{range}}) of $A$ is the subspace $col(A)$ of $\RR^m$ spanned by the columns of $A$.
	\end{enumerate}
}
\subsection{Four Important Subspaces: the row, column, null, and left null space}
\deff{
	Let $A$ be an $m \times n$ matrix. The {\em{null space}} (or {\em{kernel}}) of $A$ is the subspace of $\RR^n$ consisting of solutions of the homogeneous linear system $A\x=\mathbf{0}$. It is denoted by {\em{null($A$)}}.
}
\deff{
	A {\em{basis}} for a subspace $S$ of $\RR^n$ is a set of vectors in $S$ that
	\begin{enumerate}[(1)]
		\item spans $S$ and 
		\item is linearly independent. \footnote{It does not mean that they are orthogonal.}
	\end{enumerate}
}
\deff{
	If $S$ is a subspace of $\RR^n$, then the number of vectors in a basis for $S$ is called the {\em{dimension}} of $S$, denoted {\em{dim $S$}}. \footnote{The zero vector $\mathbf{0}$ is always a subspace of $\RR^n$. Yet any set containing the zero vector is linearly dependent, so $\mathbf{0}$ cannot have a basis. We define {\em{dim $\mathbf{0}$}} to be 0.}
}
\deff{
	The {\em{rank}} of a matrix $A$ is the dimension of its row and column spaces and is denoted by {\em{rank($A$)}}. \footnote{The row and column spaces of a matrix $A$ have the same dimension.}
}
\deff{
	The {\em{nullity}} of a matrix $A$ is the dimension of its null space and is denoted by {\em{nullity($A$)}}.
}
\thm{
	The Rank Theorem. If $A$ is an $m \times n$ matrix, then $$rank(A) + nullity(A) = n$$.
}
\thm{
	If $A$ is invertible, then $A$ is a product of elementary matrices.
}
\thm{
	Let $A$ be an $m \times n$ matrix. Then $rank(A^TA) = rank(A)$.
}
\deff{
	Let $S$ be a subspace of $\RR^n$ and let $B=\{\v_1,\cdots,\v_k\}$ be a basis for $S$. Let $\v$ be a vector in $S$, and write $\v = c_1\v_1 + \cdots + c_k\v_k$. Then $c_1,\cdots,c_k$ are called the coordinates of $\v$ with respect to $B$, and the column vector $$[\v]_B = [c_1,\cdots,c_k]^T$$ is called the coordinate vector of $\v$ with respect to $B$. \footnote{This coordinate vector is unique.}
}
\deff{
	A transformation $T: \RR^n \to \RR^m$ is called a linear transformation if $$T(c_1\v_1 + c_2\v_2) = c_1T(\v_1) + c_2T(\v_2)$$ for all $\v_1, \v_2$ in $\RR^n$ and scalars $c_1, c_2$.
}
\section{Bases and Dimension}
\section{Coordinates}
\section{Linear Forms: One Vector as Argument}
\section{Bilinear and Quadratic Forms: Two Vectors as Argument}
\section{Jordan Canoical Forms}

\chapter{Eigenvalues and Eigenvectors}
\section{Definitions}
\rmk{
	eigenvectors are non-zero.
}

\deff{
	The set of all eigenvectors corresponding to the same eigenvalue, together with the zero vector, is called an {\em{eigenspace}}.
}

\deff{
	The characteristic polynomial of a matrix $\A$ of order $n$ is
	\begin{align}
		|\A - \lambda\I| = \prod_{i=1}^n (\lambda - \lambda_i)
	\end{align}
}

\thm{
	Every square matrix of order $n$ has $n$ eigenvalues, possibly complex and not necessarily all unique.
}

\deff{
	The algebraic multiplicity $\mu_A(\lambda_i)$ of a eigenvalue $\lambda_i$ is the multiplicity as a root of the characteristic polynomial.
}

\deff{
	The eigenspace $E_{\lambda_i}$ associated with $\lambda_i$ is defined as
	\begin{align}
		E_{\lambda_i} = \{\v : (\A-\lambda_i \I)\v = 0 \}
	\end{align}
}

\deff{
	The dimension of the eigenspace $\E_{\lambda_i}$ is referred to as the geometric multiplicity $\gamma_A(\lambda_i)$ of $\lambda_i$.
}

\chapter{Vector Calculus}
\section{Inner Product (Dot Product)}
\deff{$$\u \otimes \v = \u \v^T$$}
\rmk{The inner product is the trace of the outer product.}

\section{Outer Product}
\section{Cross Product}
\deff{$$\a \times \b = \|\a\| \|\b\| sin(\theta) \n$$ It is also called the vector product.}

\section{Scalar Triple Product}
\section{Vector Triple Product}
\section{Line, Surface, and Volume Integrals}
\section{Integration of Vectors and Matrices}


\chapter{Matrix Calculus}
\section{Matrix Determinant}
\section{Kronecker Product and Vec}
\section{Hadamard Product and Diag}
\section{Matrix Exponential}


%!TEX root = ../LinearAlgebra.tex

\chapter{Vector and Matrix Derivatives}
Suppose $\Y_{m \times n}$ and $\X_{p \times q}$ are both matrices (scalars, vectors are of course special cases). The derivative of $\Y$ with respect to $\X$ involves $mnpq$ partial derivatives, $\left[\frac{\partial Y_{ij}}{\partial X_{kl}}\right]$, for $i = 1, \cdots, m; j = 1, \cdots, n; k = 1, \cdots, p; l = 1, \cdots, q$. This immediately poses a question: What is a convenient (or logic) way of arraying these partial derivatives - as a row vector, as a column vector, or as a matrix (which is a natural choice), and if the latter of what shape/order?

Two competing notational conventions can be distinguished by whether the index of the derivative (matrix) is majored by the numerator or the denominator.
\begin{enumerate}
	\item Numerator layout, i.e. according to $\Y$ and $\X^T$. This is sometimes known as the Jacobian layout.
	\item Denominator layout, i.e. according to $\Y^T$ and $\X$. This is sometimes known as the gradient layout. It is named so because the gradient under this layout is a usual column vector.	
\end{enumerate}
The transpose of one layout is the same as the other. We use the {\bf{numerator-layout}} notation throughout the paper.

{
\renewcommand{\arraystretch}{2}
\begin{center}
    \begin{tabular}{|c|c|c|c|}
    \hline
     & Scalar & Vector & Matrix \\ \hline
    Scalar & $\frac{\partial y}{\partial x}$ & $\frac{\partial \y}{\partial x} = [\frac{\partial y_i}{\partial x}]$ & $\frac{d\Y}{\partial x} = [\frac{\partial y_{ij}}{\partial x}]$ \\ \hline
    Vector & $\frac{\partial y}{\partial \x} = [\frac{\partial y}{\partial x_j}]$ & $\frac{\partial \y}{\partial \x} = [\frac{\partial y_i}{\partial x_j}]$ &  \\ \hline
    Matrix & $\frac{\partial y}{d\X} = [\frac{\partial y}{\partial x_{ji}}]$ &  &  \\
    \hline
    \end{tabular}
\end{center}
}
The partials with respect to the numerator are laid out according to the shape $Y$ while the partials with respect to the denominator are laid out according to the transpose of $X$. For example, $\partial y/\partial \x$ is a row vector\footnote{We distinguish $\partial y/\partial \x$ and the gradient $\nabla_\x y$, which is the transpose of the former and hence a column vector.} while $\partial \y/\partial x$ is a column vector.

Note:
\begin{enumerate}
	\item derivative is a row vector; gradient is its transpose.
	\item Hessian is the derivative of gradient.
\end{enumerate}

\section{Differentials}
\ex{
	\begin{align}
		d\A = \A - \A = \bf{0}
	\end{align}
}

\ex{
	\begin{align}
		d(\alpha\X) = \alpha(X + d\X) - \alpha\X = \alpha d\X
	\end{align}
}

\ex{
	\begin{align}
		d(\X + \Y) = [(\X+\Y) + d(\X+\Y)] - (\X + \Y) = d\X + d\Y
	\end{align}
}

\ex{
	\begin{align}
		d(\tr(\X)) = \tr(\X + d\X) - \tr(\X) = \tr(\X + d\X - \X) = \tr(d\X)
	\end{align}
}

\ex{
	\begin{align}
		d(\X\Y) = (\X+d\X)(\Y+d\Y) - \X\Y = [\X\Y + \X d\Y + (d\X)\Y + d\X d\Y] - \X\Y = \X d\Y + (d\X)\Y
	\end{align}
}

\ex{
	\begin{align}
		\bf{0} = d\I &= d(\X\X^{-1}) = (d\X)\X^{-1} + \X d\X^{-1}\\
		d\X^{-1} &= -\X^{-1}(d\X)\X^{-1}
	\end{align}
}

Another proof is: $$\frac{\A^{-1}(x+h) - \A^{-1}(x)}{h} = \frac{\A^{-1}(x+h)[\A(x+h)-\A(x)]\A^{-1}(x)}{h}$$

\noindent
Next, let's prove something not so trivial.
\prop{
	\begin{align}
		d|\X| = |\X|\tr(\X^{-1}d\X)
	\end{align}
}
\prof{
	First, we see that
	\begin{align}
		\tr(\A^T\B) &= \sum_{i=1}^n \left(\sum_{j=1}^n (\A^T)_{ij}\B_{ji}\right) = \sum_{i=1}^n \sum_{j=1}^n \A_{ji}\B_{ji} = \sum_{i=1}^n \sum_{j=1}^n \A_{ij}\B_{ij} = \vec(\A)^T \vec(\B)
	\end{align}
	which can be computed by first multiply $\A$ and $\B$ element-wise, and then sum all the elements in the resulting matrix (known as the {\cem{Frobenius inner product}})\footnote{The trace operator is a scalar function (of a matrix), that essentially turns matrices into vectors and computes a dot product between them.}.

	Next, applying the Laplace's formula
	\begin{align}
		|\X| = \sum_j x_{ij} \cdot adj^T(\X)_{ij}
	\end{align}
	we have,
	\begin{align}
		d(|\X|) &= \sum_i\sum_j \frac{\partial |\X|}{\partial x_{ij}} d x_{ij} \\
		&= \sum_i\sum_j \frac{\partial\{\sum_k x_{ik} \cdot adj^T(\X)_{ik}\}}{\partial x_{ij}} d x_{ij} ~~~~~ \text{(expand by row $i$)}\\
		&= \sum_i\sum_j \left \{ \sum_k \frac{\partial x_{ik}}{\partial x_{ij}} \cdot adj^T(\X)_{ik} + \sum_k x_{ik} \frac{\partial adj^T(\X)_{ik}}{\partial x_{ij}} \right \} d x_{ij} \\
		&= \sum_i\sum_j adj^T(\X)_{ij} d x_{ij}  ~~~~~ \left(\frac{\partial adj^T(\X)_{ik}}{\partial x_{ij}} = 0, \forall k \ne j \right)\\
	\end{align}

	Now, use $\sum_{i=1}^n \sum_{j=1}^n \A_{ij}\B_{ij} = \tr(\A^T\B)$, we have
	\begin{align}
		d(|\X|) = tr(adj(X) d\X)
	\end{align}

	Since $\X$ is invertible, and $adj(\X) = |\X| \X^{-1}$, finally,
	\begin{align}
		d(|\X|) = |\X| tr(\X^{-1} d\X)
	\end{align}
}


\section{Vector-by-vector Derivatives}
The first two important identities are
\begin{align}
	\frac{\partial A\x}{\partial \x} &= A\\
	\frac{\partial \x^TA}{\partial \x} &= A^T	
\end{align}
In the numerator-layout, the major index of the resulting matrix is based on the numerator, so when $A$ is on the left hand side of $\x$, the derivative is the same size as $A$, on the other hand, if $A$ is on the right hand side of $\x$, it needs to be transposed.

\ex{
	Suppose $a=a(\x)$ is a scalar function and $\u = \u(\x)$ a vector function.
	\begin{align}
		\frac{\partial a\u}{\partial \x} = \frac{\partial a\u}{\partial a}\frac{\partial a}{\partial \x} + \frac{\partial a\u}{\partial \u}\frac{\partial \u}{\partial \x} = \u \frac{\partial a}{\partial \x} + a\frac{\partial \u}{\partial \x}
	\end{align}	
	Recall that $\frac{\partial a\u}{\partial a}$ is a row vector, and the chain rule is expanded from right to left, just as the composition of functions.
}


\section{Derivatives of Vectors and Matrices}
\subsection{Derivatives of a Vector or Matrix with Respect to a Scalar}
Let $\A$ be a matrix, as a matrix-valued function
\begin{align}
	\A(x): \RR \rightarrow \RR^{m \times n}
\end{align}

For vector- and matrix-valued functions there is a further manifestation of the linearity of the derivative: Suppose that $f$ is a fixed linear function defined on $\RR^n$ and that $\A$ is a differentiable vector- or matrix-valued function. Then
\begin{align}
	f(\A)' = f(\A')
\end{align}
A useful example is the trace of $\A$, which is the sum of the diagonal elements of $\A$ (differentiable real-valued functions)
\begin{align}
	tr(\A)' = tr (\A')
\end{align}

Another example is the inner product of two vectors, where we have \footnote{Actually, it should work for all dot product (not necessarily the inner product, which is in the context of Euclidean spaces.)}
\begin{align}
	(\a^T\b)' = \a'^T\b + \a^T\b'
\end{align}

\chapter{Vector and Matrix Integrals}


\chapter{Some Intuitive Explanations}
\section{Eigenvalues and Singular Values}
\section{SVD, PCA, and Change of Basis}