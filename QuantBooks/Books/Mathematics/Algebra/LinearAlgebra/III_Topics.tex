\section{Minors and Cofactors}
\section{Definition}
\deff{General definition of a minor.

Let $\A$ be an $m \times n$ matrix and $k$ an integer with $0 < k \le \min{m,n}$. A $k \times k$ minor of $\A$ is the determinant of a $k \times k$ matrix obtained from $\A$ by deleting $m-k$ rows and $n-k$ columns. For such a matrix there are a total of ${m \choose k} \cdot {n \choose k}$ minors of size $k \times k$.

\deff{First minors and cofactors.

If $A$ is a square matrix, then the minor of the entry in the $i$-th row and $j$-th column (also called the $(i,j)$ minor, or a first minor, is the determinant of the submatrix formed by deleting the $i$-th row and $j$-th column. This number is often denoted $M_{ij}$. The $(i,j)$ cofactor is obtained by multiplying the minor by $(-1)^{i+j}$.
}

\ex{
To illustrate these definitions, consider the following 3 by 3 matrix,
\begin{align}
	\begin{bmatrix}
	1 & 4 & 7 \\
	3 & 0 & 5 \\
	-1 & 9 & 11 \\
	\end{bmatrix}
\end{align}

To compute the minor $M_{23}$ and the cofactor $C_{23}$, we find the determinant of the above matrix with row 2 and column 3 removed.

\begin{align*}
	M_{2,3} = \det \begin{bmatrix}
	1 & 4 & \Box \\
	\Box & \Box & \Box \\
	-1 & 9 & \Box \\
	\end{bmatrix}= \det \begin{bmatrix}
	1 & 4 \\
	-1 & 9 \\
	\end{bmatrix} = (9-(-4)) = 13
\end{align*}

So the cofactor of the (2,3) entry is $C_{23} = (-1)^{2+3}(M_{23}) = -13$.
}

\vspace{5mm}
\noindent
An important application of cofactors is the {\bf{Laplace's formula}} for the expansion of determinants.
\begin{align}
	\det(\A) = \sum_{i=1}^n a_{ij}C_{ij} = \sum_{j=1}^n a_{ij}C_{ij}
\end{align}
If $k \ne i$, we see that
\begin{align}
	\sum_{j=1}^n a_{kj}C_{ij} = 0	
\end{align}
Similarly, if $k \ne j$
\begin{align}
	\sum_{i=1}^n a_{ik}C_{ij} = 0 \label{cofactor}
\end{align}
This is essentially the determinant of a matrix with the $k$-th row the same as the $i$-th row, or the $k$-th column the same as the $j$-th column, which is zero.

\section{The Cramer's Rule and the Adjugate Matrix}
\begin{align}
	\begin{matrix}a_{11}x_1+a_{12}x_2+\cdots+a_{1n}x_n&=&b_1\\a_{21}x_1+a_{22}x_2+\cdots+a_{2n}x_n&=&b_2\\\vdots&\vdots&\vdots\\a_{n1}x_1+a_{n2}x_2+\cdots+a_{nn}x_n&=&b_n\end{matrix}
\end{align}
If we multiply the above by the row vector of cofactors of the $1^{st}$ column, $[C_{11},C_{21},\cdots,C_{n1}]$, we obtain
\begin{align}
	[\det(\A),0,\cdots,0]\begin{bmatrix}x_1\\ \vdots \\ x_n\end{bmatrix} = [C_{11},C_{21},\cdots,C_{n1}]\b
\end{align}
The left hand side used Equation \ref{cofactor}. The right hand side is nothing but the determinant of a matrix with the first column replaced by $\b$.

Similarly, we can multiply the linear system by the row vector of cofactors of the $2^{nd}, 3^{rd}, \cdots, n^{th}$, and we obtain
\begin{align}	
	\det(\A)\x &= \begin{bmatrix}C_{11}& \cdots & C_{n1}\\C_{12} & \cdots &C_{n2}\\ \vdots &  & \vdots\\C_{1n} & \cdots & C_{nn}\end{bmatrix}\b
\end{align}
which gives us
\begin{align}
	\det(\A) &= \begin{bmatrix}C_{11}& \cdots & C_{1n}\\C_{21} & \cdots &C_{2n}\\ \vdots &  & \vdots\\C_{n1} & \cdots & C_{nn}\end{bmatrix}^T\A
\end{align}
The matrix on the right
\begin{align}
	\mathrm{adj}(\A) = \C^T = \begin{bmatrix}C_{11}& \cdots & C_{1n}\\C_{21} & \cdots &C_{2n}\\ \vdots &  & \vdots\\C_{n1} & \cdots & C_{nn}\end{bmatrix}^T
\end{align}
is called the adjugate matrix of $\A$, which is the transpose of the cofactor matrix $\C$.
