\chapter{Calculus}
\section{Convergence Tests}
There are five common techniques to test whether or not an infinite series is convergent. But first of all, a necessary condition:
\thm{
	If the limit of the summand is undefined or nonzero, that is, $\lim \limits_{n \to \infty} a_n \ne 0$, then the series $\sum_{n=1}^\infty a_n$ must diverge.
}

\thm{
	{\bf{Comparison Test.}} If $\{a_n\},\{b_n\}>0$, and the limit $\lim \limits_{n \to \infty} \frac{a_n}{b_n}$ exists, is finite and is not zero, then $\sum_{n=1}^\infty a_n$ converges if and only if $\sum_{n=1}^\infty b_n$ converges.
}

\thm{
	{\bf{Integral Test.}} Let $f:[1,\infty) \to \R_+$ be a positive and monotone decreasing function such that $f(n) = a_n$. Then the series $\{a_n\}$ converges if and only if the integral $\int_1^\infty f(x) dx$ converges.
}

\thm{
	{\bf{Ratio Test.}} Suppose there exists $r$ such that $\lim \limits_{n \to \infty} |\frac{a_{n+1}}{a_n}| = r$. If $r < 1$, then the series converges. If $r > 1$, then the series diverges. If $r = 1$, the ratio test is inconclusive, and the series may converge or diverge.
}

\thm{
	{\bf{Root Test.}} Define $r = \limsup \limits_{n \to \infty} \sqrt[n]{|a_n|}$. If $r < 1$, then the series converges. If $r > 1$, then the series diverges. If $r = 1$, the ratio test is inconclusive, and the series may converge or diverge.
}

\thm{
	{\bf{Alternating Series Test.}} If the alternating series $\sum_{n=1}^\infty (-1)^{n-1}b_n, (b_n > 0)$ satisfies
	\begin{enumerate}
		\item $b_{n+1} \le b_n, ~ \mbox{for all $n$}$; and,
		\item $\lim_{n \to \infty} b_n = 0$.
	\end{enumerate}
	Then the series is convergent.
}

\thm{
	A series is said to be absolutely convergent if $\sum_{i=1}^\infty |a_n|$ converges. Every absolutely convergent series is convergent. But not all convergent series are absolutely convergent. A convergent series that is not absolutely convergent is called conditionally convergent.
}

