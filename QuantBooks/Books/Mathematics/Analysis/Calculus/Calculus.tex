\documentclass{book}
\usepackage{amssymb,amsmath,mathtools}


\usepackage{algorithm2e,algorithmic}

\usepackage{mathrsfs}
\usepackage{paralist}

\usepackage{esint} % for \fint

\allowdisplaybreaks

\usepackage{color}		% enable color characters
\usepackage{graphicx} 	% insert image files
\usepackage{enumerate} 	% enumerate items
\usepackage{caption}
\usepackage{subcaption}
\usepackage{multirow,multicol}
\usepackage[makeroom]{cancel}

\usepackage[colorlinks=true,linkcolor=blue,citecolor=blue]{hyperref}

\usepackage{makeidx}

\newcommand{\cem}[1]{\color{magenta}{\em{#1}}} % color emphasize
\newcommand{\dual}[2]{{#1}^{(#2)}} % color emphasize

\newcommand{\tr}{{\mathrm{tr}}}
\renewcommand{\vec}{{\mathrm{vec}}}

\newcommand{\dd}{{\,\text{d}\,}}
%%
%% horizontal and vertical centering in table p mode
%%
\usepackage{array}
\newcolumntype{P}[1]{>{\centering\arraybackslash}p{#1}} % horizontal centering
\newcolumntype{M}[1]{>{\centering\arraybackslash}m{#1}} % vertical centering

%%
%% define bold font for the alphabet
%%
\usepackage{pgffor}
\foreach \letter in {a,...,z}{ % bold font for a..z
\expandafter\xdef\csname \letter \endcsname{\noexpand\ensuremath{\noexpand\mathbf{\letter}}}
}
\foreach \letter in {A,...,Z}{ % bold font for A..Z
\expandafter\xdef\csname \letter \endcsname{\noexpand\ensuremath{\noexpand\mathbf{\letter}}}
}
\foreach \letter in {A,...,Z}{ % `field' font for AA..ZZ
\expandafter\xdef\csname \letter\letter \endcsname{\noexpand\ensuremath{\noexpand\mathcal{\letter}}}
}
\foreach \letter in {A,...,Z}{ % `field' font for AAA..ZZZ
\expandafter\xdef\csname \letter\letter\letter \endcsname{\noexpand\ensuremath{\noexpand\mathbb{\letter}}}
}
\newcommand{\balpha}{{\boldsymbol{\alpha}}}
\newcommand{\bbeta}{{\boldsymbol{\beta}}}
\newcommand{\bgamma}{{\boldsymbol{\gamma}}}
\newcommand{\bkappa}{{\boldsymbol{\kappa}}}
\newcommand{\bmu}{{\boldsymbol{\mu}}}
\newcommand{\btheta}{{\boldsymbol{\theta}}}
\newcommand{\bTheta}{{\boldsymbol{\Theta}}}
\newcommand{\bPi}{{\boldsymbol{\Pi}}}
\newcommand{\bSigma}{{\boldsymbol{\Sigma}}}
\newcommand{\bPhi}{{\boldsymbol{\Phi}}}
\newcommand{\bLambda}{{\boldsymbol{\Lambda}}}
\newcommand{\bdeta}{{\boldsymbol{\eta}}}
\newcommand{\bphi}{{\boldsymbol{\phi}}}



%%
%% add definitions and theorems
%%
\usepackage[thmmarks,amsmath]{ntheorem}
\theorembodyfont{\normalfont}
\newtheorem{deff}{Definition}[section]
\newtheorem{thm}{Theorem}[section]
\newtheorem{prop}{Proposition}[section]
\newtheorem{lem}{Lemma}[section]
\newtheorem{cor}{Corollary}[section]
\newtheorem{rmk}{Remark}[section]
\newtheorem{alg}{Algorithm}[section]
\newtheorem{ex}{Example}[section]
\newtheorem{ques}{Question}[section]
\newtheorem{ans}{Answer}[section]
\newtheorem{prob}{Problem}[section]
\newtheorem{sol}{Solution}[section]
\newtheorem*{prof}{Proof}[section] 

\title{Calculus}
\author{Xi Tan (tan19@purdue.edu)}
\date{\today}

\begin{document}
\maketitle

\tableofcontents
\newpage

\section*{Preface}
This book reviews calculus, advanced calculus, real analysis, and functional Analysis. The main references to be used are \cite{Stewart} for calculus, \cite{Rudin} for advanced calculus, \cite{Royden} for real analysis, and \cite{Kreyszig} for functional analysis. Other useful texts include: \cite{Folland} and \cite{Torchinsky} for real analysis.
\newpage

\part{Calculus}
\chapter{Infinite Sequences and Series}
\section{Convergence Tests}
There are five common techniques to test whether or not an infinite series is convergent. But first of all, a necessary condition:
\thm{
	If the limit of the summand is undefined or nonzero, that is, $\lim \limits_{n \to \infty} a_n \ne 0$, then the series $\sum_{n=1}^\infty a_n$ must diverge.
}

\thm{
	{\bf{Comparison Test.}} If $\{a_n\},\{b_n\}>0$, and the limit $\lim \limits_{n \to \infty} \frac{a_n}{b_n}$ exists, is finite and is not zero, then $\sum_{n=1}^\infty a_n$ converges if and only if $\sum_{n=1}^\infty b_n$ converges.
}

\thm{
	{\bf{Integral Test.}} Let $f:[1,\infty) \to \R_+$ be a positive and monotone decreasing function such that $f(n) = a_n$. Then the series $\{a_n\}$ converges if and only if the integral $\int_1^\infty f(x) dx$ converges.
}

\thm{
	{\bf{Ratio Test.}} Suppose there exists $r$ such that $\lim \limits_{n \to \infty} |\frac{a_{n+1}}{a_n}| = r$. If $r < 1$, then the series converges. If $r > 1$, then the series diverges. If $r = 1$, the ratio test is inconclusive, and the series may converge or diverge.
}

\thm{
	{\bf{Root Test.}} Define $r = \limsup \limits_{n \to \infty} \sqrt[n]{|a_n|}$. If $r < 1$, then the series converges. If $r > 1$, then the series diverges. If $r = 1$, the ratio test is inconclusive, and the series may converge or diverge.
}

\thm{
	{\bf{Alternating Series Test.}} If the alternating series $\sum_{n=1}^\infty (-1)^{n-1}b_n, (b_n > 0)$ satisfies
	\begin{enumerate}
		\item $b_{n+1} \le b_n, ~ \mbox{for all $n$}$; and, 
		\item $\lim_{n \to \infty} b_n = 0$.
	\end{enumerate}
	Then the series is convergent.
}

\thm{
	A series is said to be absolutely convergent if $\sum_{i=1}^\infty |a_n|$ converges. Every absolutely convergent series is convergent. But not all convergent series are absolutely convergent. A convergent series that is not absolutely convergent is called conditionally convergent.
}

\chapter{Vectors and the Geometry of Space}
\section{Lines in $\RRR^n$}
\deff{
	Given a vector $\p \in \RRR^n$ and a nonzero vector $\v \in \RRR^n$, the set of all points $\y \in \RRR^n$ such that
	\begin{align}
		\y = t\v + \p, ~~~ t \in \RRR
	\end{align}
	is called the \emph{line} through $\p$ in the direction of $\v$.
}

\ex{
	The shortest distance from a point $\q \in \RRR^n$ to a line $L$ with equation $\y = t\v + \p$ is
	\begin{align}
		\left\|(\q-\p) - \frac{(\q-\p)^T\v}{\|\v\|^2} \v \right\|
	\end{align}
}

\section{Hyperplanes in $\RRR^n$}
\deff{
	Suppose $\n$ is a normal vector for a hyperplan $H$ through $\p \in \RRR^n$, then the normal equation for $H$ is
	\begin{align}
		\n^T(\y - \p) = 0
	\end{align}
	If $H$ is in $\RRR^3$, we can use cross-product $\times$ to obtain the normal vector given two vectors on the hyperplane.
}

\ex{
	The shortest distance from a point $\q \in \RRR^n$ to a hyperplane $H$ with equation $\n^T(\y - \p) = 0$ is
	\begin{align}
		\left| \frac{\n^T(\q-\p)}{\|\n\|} \right|
	\end{align}
}

A hyperplane is a set satisfies $H=\{x:w^Tx=b\}$. An equivalent form is $w^T(x-\frac{w}{\|w\|^2}b) = 0$, which suggests that the vector $w$ is perpendicular to the hyperplane, called a normal vector.

Particularly, since $\frac{w^Tw}{\|w\|^2}b = b$, we know $x_0=\frac{w}{\|w\|}\frac{b}{\|w\|}$ is on the hyperplane. The $x_0$ is actually the projection of the origin, since $w$ is orthogonal to the hyperplane and it is nothing but a scaled $w$ on the hyperplane. Therefore, the shortest distance (along the direction of $w$) from the origin to the hyperplane is given by $\frac{b}{\|w\|}$ (could be negative, which means $w$ is on the other side of the hyperplane).

In general, if a hyperplane is given by the equation $f(x) = w^Tx-b = 0$, the distance from any arbitrary vector $p$ to the hyperplane $w^Tx=b$ is given by
\begin{align}
	\frac{f(p)}{\|w\|} = \frac{w^Tp-b}{\|w\|}, ~~~ \mbox{if $p$ is on the opposite side of the origin}\\
	-\frac{f(p)}{\|w\|} = -\frac{w^Tp-b}{\|w\|}, ~~~ \mbox{if $p$ is on the same side of the origin}
\end{align}	

Particularly, when $p=0$ (the origin), the above becomes $\frac{b}{\|w\|}$, which agrees with our previous result.

Proof: Let's prove the first case. Suppose there is a vector $x$ on the hyperplane, such that $p-x = d\frac{w}{\|w\|}$. Since $w$ is orthogonal to the hyperplane, the scalar $d$ is the distance we are after. Now, multiply both sides by $w^T$,

$w^Tp-w^Tx = dw^T\frac{w}{\|w\|}$

$w^Tp-(w^Tx-b) = d\frac{w^Tw}{\|w\|}+b$

$w^Tp = d\|w\| + b$

$d = \frac{w^Tp-b}{\|w\|}$

The proof for the other case is similar.

$\blacksquare$

\end{document}
