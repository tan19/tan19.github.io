\section{Basic Topology}

\section{Finite, Countable, and Uncountable Sets}

\subsection{Definitions}
\begin{deff}
	Consider two sets $A$ and $B$, whose elements may be any objects whatsoever, and suppose that {\underline{with each element $x$ of A}} there is associated, in some manner, an element of $B$, which we denote by $f(x)$. Then $f$ is said to be a {\cem{function}} from $A$ to $B$ (or a {\cem{mapping}} of $A$ into $B$). The set $A$ is called the {\cem{domain}} of $f$ (we also say $f$ is defined on $A$), and the elments $f(x)$ are called the {\cem{values}} of $f$. The set of all values of $f$ is called the {\cem{range}} of $f$.
\end{deff}

\begin{deff}\label{def: mapping}
	Let $A$ and $B$ be two sets and let $f$ be a mapping of $A$ into $B$. If $E \subset A$, $f(E)$ is defined to be the set of all elements $f(x)$, for $x \in E$. We call $f(E)$ the {\cem{image}} of $E$ under $f$. In this notation, $f(A)$ is the range of $f$. It is clear that $f(A) \subset B$. If $f(A) = B$, we say that $f$ maps $A$ {\cem{onto}} $B$. (Note that, according to this usage, {\cem{onto}} is more specific thatn {\cem{into}}.)
\end{deff}

\begin{deff}
	If $E \subset B$ ($E$ is not necessarily a subset of $f(A)$), $f^{-1}(E)$ denotes the set of all $x \in A$ such that $f(x) \in E$. We call $f^{-1}(E)$ the {\cem{inverse image}} of $E$ under $f$. If $y \in B$, $f^{-1}(y)$ is the set of all $x \in A$ such that $f(x) = y$. If, for each $y \in B$, $f^{-1}(y)$ consists of at most one element of $A$, then $f$ is said to be a 1-1 ({\cem{one-to-one}}) mapping of $A$ into $B$. This may also be expressed as follows. $f$ is a 1-1 mapping of $A$ into $B$ provided that $f(x_1) \ne f(x_2)$ whenever $x_1 \ne x_2, x_1 \in A, x_2 \in A$.
\end{deff}

\begin{deff}
	If there exists a 1-1 mapping of $A$ {\cem{onto}} $B$, we say that $A$ and $B$ can be put in {\cem{1-1 correspondence}}, or that $A$ and $B$ have the same {\cem{cardinal number}}, or, briefly, that $A$ and $B$ are {\cem{equivalent}}, and we write $A \sim B$. This relation clearly has the following properties:
	\begin{enumerate}[]
		\item It is reflexive: $A \sim A$
		\item It is symmetric: If $A \sim B$, then $B \sim A$
		\item It is transitive: If $A \sim B$ and $B \sim C$, then $A \sim C$
	\end{enumerate}
	Any relation with theses three properties is called an {\cem{equivalence relation}}.
\end{deff}

\begin{deff}
	For any positive integer $n$, let $J_n$ be the set whose elements are the integers $1,2,\cdots,n$; let $J$ be the set consisting of all positive integers. For any set $A$, we say:
	\begin{enumerate}[(a)]
		\item $A$ is {\cem{finite}} if $A \sim J_n$ for some $n$ (the empty set is also considered to be finite).
		\item $A$ is {\cem{infinite}} if $A$ is not finite.
		\item $A$ is {\cem{countable}} if $A \sim J$.
		\item $A$ is {\cem{uncountable}} if $A$ is neither finite nor countable.
		\item $A$ is {\cem{at most countable}} if $A$ is finite or countable.
	\end{enumerate}
	Countable sets are sometimes called {\cem{enumerable}} or {\cem{denumerable}}.
\end{deff}

\begin{deff}
	By a {\cem{sequence}}, we mean a function $f$ defined on the set $J$ of all positive integers. If $f(n) = x_n$, for $n \in J$, it is customary to denote the sequence $f$ by the symbol $\{x_n\}$, or sometimes by $x_1, x_2, x_3, \cdots$. The values of $f$, that is, the elements $x_n$, are called the {\cem{terms}} of the sequence. If $A$ is a set and if $x_n \in A$ for all $n \in J$, then $\{x_n\}$ is said to be a {\cem{sequence in $A$}}, or a {\cem{sequence of elements of $A$}}.
\end{deff}

\begin{deff}
	Let $A$ and $\Omega$ be sets, and suppose that with each element $\alpha$ of $A$ there is associated a subset of $\Omega$ which we denote by $E_\alpha$. The set whose elements are the sets $E_\alpha$ will be denoted by $\{E_\alpha\}$. Instead of speaking of sets of sets, we shall sometimes speak of a collection of sets, or a family of sets. The {\cem{union}} of the sets $E_\alpha$ is defined to be the set $S$ such that $x \in S$ if and only if $x \in E_\alpha$ for at least one $\alpha \in A$. We use the notation $$S = \bigcup\limits_{\alpha \in A} E_\alpha.$$ The {\cem{intersection}} of the sets $E_\alpha$ is defined to be the set $P$ such that $x \in P$ if and only if $x \in E_\alpha$ for every $\alpha \in A$. We use the notation $$P = \bigcap\limits_{\alpha \in A} E_\alpha.$$
\end{deff}

\subsection{Theorems}
\begin{thm}
	$A$ is infinite if and only if $A$ is equivalent to one of its {\underline{proper subsets}}.
\end{thm}

\begin{thm}
	Every infinite subset of a countable set $A$ is countable.
\end{thm}

\begin{thm}
	Let $\{E_n\}, n = 1, 2, 3, \cdots$, be a sequence of countable sets, and put $$S = \bigcup\limits_{n=1}^\infty E_n.$$ Then $S$ is countable.
\end{thm}

\begin{thm}
	Let $A$ be a countable set, and let $B_n$ be the set of all $n$-tuples $(a_1,\cdots,a_n)$, where $a_k \in A (k = 1, \cdots, n)$, and the elements $a_1, \cdots, a_n$ need not be distinct. Then $B_n$ is countable.
\end{thm}

\begin{cor}
	The set of all rational numbers is countable.
\end{cor}

\begin{thm}
	Let $A$ be the set of all sequences whose elements are the digits $0$ and $1$. This set $A$ is uncountable.
\end{thm}

\section{Metric Spaces}
\subsection{Definitions}
\begin{deff}
	A set $X$, whose elements we shall call {\cem{points}}, is said to be a {\cem{metric space}} if with any two points $p$ and $q$ of $X$ there is associated a real number $d(p,q)$, called the {\cem{distance}} from $p$ to $q$, such that
	\begin{enumerate}[(a)]
		\item $d(p,q)>0$ if $p \ne q$; $d(p,q) = 0$;
		\item $d(p,q)=d(q,p)$;
		\item $d(p,q)\le d(p,r) + d(r,q)$, for any $r \in X$.
	\end{enumerate}
	Any function with these three properties is called a {\cem{distance function}}, or a {\cem{metric}}.
\end{deff}

\begin{deff}
	~
	\begin{enumerate}[(a)]
	\item By the {\cem{segment}} $(a,b)$ we mean the set of all real numbers $x$ such that $a<x<b$.
	\item By the {\cem{interval}} $[a,b]$ we mean the set of all real numbers $x$ such that $a \le x \le b$.
	\item Occasionally we shall also encounter ``half-open intervals'' $[a,b)$ and $(a,b]$; the first consist of all $x$ such that $a \le x < b$, the second of all $x$ such that $a < x \le b$.
	\item If $a_i < b_i$ for $i = 1, \cdots, k$, the set of all points $\x = (x_1, \cdots, x_k)$ in $R^k$ whose coordinates satisfy the inequalities $a_i \le x_i \le b_i (1 \le i \le k)$ is called a {\cem{k-cell}}.
	\item If $\x \in R^k$ and $r>0$, the {\cem{open}} (or {\cem{closed}}) {\cem{ball}} $B$ with center at $\x$ and radius $r$ is defined to be the set of all $y \in R^k$ such that $|\y - \x|<r$ (or $|\y - \x|<r$).
	\end{enumerate}
\end{deff}
\begin{deff}
	 We call a set $E \subset R^k$ {\cem{convex}} if $$\lambda \x + (1-\lambda) \y \in E $$ whenever $\x \in E, \y \in E$, and $0<\lambda<1$.
\end{deff}

\begin{deff}
	Let $X$ be a metric space. All points and sets mentioned below are understood to be elements and subsets of $X$.
	\begin{enumerate}[(a)]
		\item A {\cem{neighborhood}} of $p$ is a set $N_r(p)$ consisting of all $q$ such that $d(p,q) < r$, for some $r > 0$. The number $r$ is called the {\cem{radius}} of $N_r(p)$.
		\item A point $p$ is a limit point of the set $E$ if every neighborhood of $p$ contains a point $q \ne p$ such that $q \in E$.
		%\item If $p \in E$ and $p$ is not a limit point of $E$, then $p$ is called an {\cem{isolated point}} \smallmarginpar{An equivalent deff: There exsits a neighborhood of $p$ such that the only element {\underline{in $E$}} it contains is $p$ itself.} of $E$.
		\item $E$ is {\cem{closed }} if every limit point of $E$ is a point of $E$.
		\item A point $p$ is an {\cem{interior}} point of $E$ if there is a neighborhood $N$ of $p$ such that $N \subset E$.
		\item $E$ is {\cem{open}} if every point of $E$ is an interior point of $E$.
		\item The {\cem{complement}} of $E$ (denoted by $E^c$) is the set of all points $p \in X$ such that $p \notin E$.
		\item $E$ is {\cem{perfect}} if $E$ is closed and if every point of $E$ is a limit point of $E$.
		\item $E$ is {\cem{bounded}} if there is a real number $M$ and a point $q \in X$ such that $d(p,q) < M$ for all $p \in E$.
		\item $E$ is {\cem{dense}} in $X$ if every point of $X$ is a limit point of $E$, or a point of $E$ (or both).
	\end{enumerate}
\end{deff}

\begin{deff}
If $X$ is a metric space, if $E \subset X$, and if $E'$ denotes the set of all limit points of $E$ in $X$, then the {\cem{closure}} of $E$ is the set $\bar E = E \cup E'$.
\end{deff}

\subsection{Theorems}
\begin{thm}
	~
	\begin{enumerate}[(a)]
	\item Balls are convex.
	\item K-cells are convex.
	\end{enumerate}
\end{thm}

\begin{thm}
	Every neighborhood is an open set.
\end{thm}

\begin{thm}
	If $p$ is a limit point of a set $E$, then every neighborhood of $p$ contains infinitely many points of $E$.
\end{thm}

\begin{cor}
	A finite point set has no limit points.
\end{cor}

\begin{thm}
	Let \{$E_n$\} be a (finite or infinite) collection of sets $E_n$. Then $$\left(\bigcup _\alpha E_\alpha\right)^c = \bigcap_\alpha \left(E_\alpha^c\right).$$
\end{thm}

\begin{thm}
	A set $F$ is closed if and only if its complement is open.
\end{thm}
\begin{thm}
	~
	\begin{enumerate}[(a)]
	\item For any collection \{$G_n$\} of open sets, $\bigcup_n G_n$ is open.
	\item For any collection \{$F_n$\} of closed sets, $\bigcap_n F_n$ is closed.
	\item For any finite collection $G_1, \cdots, G_n$ of open sets, $\bigcap_{i=1}^n G_i $ is open.
	\item For any finite collection $F_1, \cdots, F_n$ of closed sets, $\bigcup_{i=1}^n F_i $ is closed.
	\end{enumerate}
\end{thm}

\begin{thm}
	If $X$ is a metric space and $E\subset X$, then
	\begin{enumerate}[(a)]
	\item $\bar E$ is closed,
	\item $E=\bar E$ if and only if $E$ is closed,
	\item $\bar E \subset F$ for every closed set $F\subset X$ such that $E \subset F$.
	\end{enumerate}
	By (a) and (c), $\bar E$ is the smallest closed subset of $X$ that contains $E$,
\end{thm}

\begin{thm}
	Let $E$ be a nonempty set of real numbers which is bounded above. Let $y = sup E$. Then $y \in \bar E$. Hence $y \in E$ if $E$ is closed.
\end{thm}

\begin{thm}
	Suppose $Y \subset X$. A subset $E$ of $Y$ is open relative to $Y$ is and ony if $E = Y \cap G$ for some open subset $G$ of $X$.
\end{thm}

\section{Compact Sets}
\subsection{Definitions}
\begin{deff}
	By an {\cem{open cover}} of a set $E$ in a metric space $X$ we mean a collection $\{G_\alpha\}$ of open subsets of $X$ such that $E \subset \bigcup_\alpha G_\alpha$.
\end{deff}

%\begin{deff}
%	A subset $K$ of a metric space $X$ is said to be {\cem{compact}} if %every open cover of $K$ contains a {\cem{finite}} subcover. %\smallmarginpar{It is clear that every finite set is compact.}
%\end{deff}

\subsection{Theorems}
%\begin{thm}
%	Suppose $K \subset Y \subset X$. Then $K$ is compact relative to $X$ if and only if $K$ is compact relative to $Y$. \smallmarginpar{Every metric space $X$ is an open subset of itself, and is a closed subset of itself.}
%\end{thm}

\begin{thm}
	Compact subsets of metric spaces are closed.
\end{thm}

\begin{thm}
	Cloased subsets of compact sets are compact.
\end{thm}

\begin{thm}
	If $F$ is closed and $K$ is compact, the n$F \cap K$ is compact.
\end{thm}

\begin{thm}
	If $\{K_\alpha\}$ is a collection of compact subsets of a metric space $X$ such that the intersection of every finite subcollection of $\{K_\alpha\}$ is nonempty, then $\cap K_\alpha$ is nonempty.
\end{thm}

\begin{thm}
	If $E$ is an infinite subset of a compact set $K$, then $E$ has a limit point in $K$.
\end{thm}

\begin{thm}
	If $\{I_n\}$ is a sequence of intervals in $R^1$, such that $I_n \supset I_{n+1}$ $(n=1,2,3,\cdots)$, then $\cap_{n=1}^\infty I_n$ is not empty.
\end{thm}

\begin{thm}
	Let $k$ be a positive integer. If $\{I_n\}$ is a sequence of $k$-cells such that $I_n \supset I_{n+1}$ $(n=1,2,3,\cdots)$, then $\cap_{n=1}^infty I_n$ is not empty.
\end{thm}

\begin{thm}
	Every $k$-cell is compact.
\end{thm}

\begin{thm}
	If a set $E$ in $R^k$ has one of the following three properties, then it has the other two:
	\begin{enumerate}[(a)]
		\item $E$ is closed and bounded.
		\item $E$ is compact.
		\item Every infinite subset of $E$ has a limit point in $E$.
	\end{enumerate}
\end{thm}

\begin{thm}
	Every bounded infinite subset of $R^k$ has a limit point in $R^k$.
\end{thm}

\section{Perfect Sets}
\subsection{Theorems}
\begin{thm}
	Let $P$ be a nonempty perfect set in $R^k$. Then $P$ is uncountable.
\end{thm}

\begin{cor}
	Every interval $[a,b]$ $(a < b)$ is uncountable. In particular, the set of all real numbers is uncountable.
\end{cor}

\section{Connected Sets}
\subsection{Definitions}
%\begin{deff}
%	Two subsets $A$ and $B$ of a metric space $X$ are said to be {\cem{separated}} if both $A \cap \bar B$ and $\bar A \cap B$ are empty, i.e., if no point of $A$lies in the closure of $B$ and no point of $B$lies in the closure of $A$. A set $E \subset X$ is siad to be {\cem{connected}} if $E$ is {\cem{not}} a union of two nonempty separated sets. \smallmarginpar{Separated sets are of course disjoint, but disjoint sets need not be sparated.}
%\end{deff}

\subsection{Theorems}
\begin{thm}
	A subset $E$ of the ral line $R^1$ is connected if and only if it has the following property: If $x \in E$, $y \in E$, and $x < z < y$, then $z \in E$.
\end{thm} 