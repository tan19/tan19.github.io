\chapter{Advanced Calculus}

\section{$\liminf$ and $\limsup$}
\deff{
	\begin{align}	
		\liminf\limits_n A_n &= \cup_{n=1}^\infty \cap_{i=n}^\infty A_i = \{x ~|~ x \in A_i ~\text{eventually}\}\\
		\limsup\limits_n A_n &= \cap_{n=1}^\infty \cup_{i=n}^\infty A_i = \{x ~|~ x \in A_i ~\text{for infinitely many}~ i\}
	\end{align}
}

The meaning of $\liminf$ can be seen by re-writing the above definition as: $x \in \liminf\limits_n A_n$ if $\exists n \in \NNN$, s.t. $\forall i \ge n$ and $i \in \NNN$, $x \in A_i$. Hence the elements in $\liminf\limits_n A_n$ are in all but (the first) finitely many sets, though the ``first finitely many sets'' may be different for different elements in $\liminf$. $\limsup$ can be best seen by examining its complement, according to the De Morgan's law.

\prop{
	\begin{align}
		(\limsup\limits_n A_n)^c &= \liminf\limits_n A_n^c\\
		\liminf A_k &\subset \limsup A_k\\
		\limsup(A_k \cup B_k) &= \limsup A_k \cup \limsup B_k\\
		\liminf(A_k \cap B_k) &= \liminf A_k \cap \liminf B_k
	\end{align}	
}

\section{Lines in $\RRR^n$}
\deff{
	Given a vector $\p \in \RRR^n$ and a nonzero vector $\v \in \RRR^n$, the set of all points $\y \in \RRR^n$ such that
	\begin{align}
		\y = t\v + \p, ~~~ t \in \RRR
	\end{align}
	is called the \emph{line} through $\p$ in the direction of $\v$.
}

\ex{
	The shortest distance from a point $\q \in \RRR^n$ to a line $L$ with equation $\y = t\v + \p$ is
	\begin{align}
		\left\|(\q-\p) - \frac{(\q-\p)^T\v}{\|\v\|^2} \v \right\|
	\end{align}
}

\section{Hyperplanes in $\RRR^n$}
\deff{
	Suppose $\n$ is a normal vector for a hyperplan $H$ through $\p \in \RRR^n$, then the normal equation for $H$ is
	\begin{align}
		\n^T(\y - \p) = 0
	\end{align}
	If $H$ is in $\RRR^3$, we can use cross-product $\times$ to obtain the normal vector given two vectors on the hyperplane.
}

\ex{
	The shortest distance from a point $\q \in \RRR^n$ to a hyperplane $H$ with equation $\n^T(\y - \p) = 0$ is
	\begin{align}
		\left| \frac{\n^T(\q-\p)}{\|\n\|} \right|
	\end{align}
}

A hyperplane is a set satisfies $H=\{x:w^Tx=b\}$. An equivalent form is $w^T(x-\frac{w}{\|w\|^2}b) = 0$, which suggests that the vector $w$ is perpendicular to the hyperplane, called a normal vector.

Particularly, since $\frac{w^Tw}{\|w\|^2}b = b$, we know $x_0=\frac{w}{\|w\|}\frac{b}{\|w\|}$ is on the hyperplane. The $x_0$ is actually the projection of the origin, since $w$ is orthogonal to the hyperplane and it is nothing but a scaled $w$ on the hyperplane. Therefore, the shortest distance (along the direction of $w$) from the origin to the hyperplane is given by $\frac{b}{\|w\|}$ (could be negative, which means $w$ is on the other side of the hyperplane).

In general, if a hyperplane is given by the equation $f(x) = w^Tx-b = 0$, the distance from any arbitrary vector $p$ to the hyperplane $w^Tx=b$ is given by
\begin{align}
	\frac{f(p)}{\|w\|} = \frac{w^Tp-b}{\|w\|}, ~~~ \mbox{if $p$ is on the opposite side of the origin}\\
	-\frac{f(p)}{\|w\|} = -\frac{w^Tp-b}{\|w\|}, ~~~ \mbox{if $p$ is on the same side of the origin}
\end{align}	

Particularly, when $p=0$ (the origin), the above becomes $\frac{b}{\|w\|}$, which agrees with our previous result.

Proof: Let's prove the first case. Suppose there is a vector $x$ on the hyperplane, such that $p-x = d\frac{w}{\|w\|}$. Since $w$ is orthogonal to the hyperplane, the scalar $d$ is the distance we are after. Now, multiply both sides by $w^T$,

$w^Tp-w^Tx = dw^T\frac{w}{\|w\|}$

$w^Tp-(w^Tx-b) = d\frac{w^Tw}{\|w\|}+b$

$w^Tp = d\|w\| + b$

$d = \frac{w^Tp-b}{\|w\|}$

The proof for the other case is similar.

$\blacksquare$

\chapter{The Real and Complex Number Systems}
\section{Introduction}
\subsection{Definitions}
\begin{deff}
	If $A$ is any set (whose elements may be numbers or any other objects), we write $x \in A$ to indicate that $x$ is a memeber (or an element) of $A$. If $x$ is not a memeber of $A$, we write: $x \notin A$.
\end{deff}

\begin{deff}
	Throughtout Chap. 1, the set of all rational numbers will be denoted by $Q$.
\end{deff}

\begin{deff}\label{def:supinf}
	Suppose $S$ is an ordered set, $E\subset S$, and $E$ is bounded above. Suppose there exists an $\alpha\in S$ with the following properties:
	\begin{enumerate}[(i)]
		\item $\alpha$ is an upper bound of $E$.
		\item If $\gamma <\alpha$ then $\gamma$ is not an upper bound of $E$.
	\end{enumerate}
	Then $\alpha$ is called the {\cem{least upper bound}} of $E$ [that there is at most one such $\alpha$ is clear from (ii)] or the {\cem{supremum}} of $E$, and we write $$\alpha=\sup E.$$ The {\cem{greatest lower bound}}, or {\cem{infimum}}, of a set $E$ which is bounded below is defined in the same manner: The statement $$\alpha = \inf E$$ means that $\alpha$ is a lower bound of $E$ and that no $\beta$ with $\beta>\alpha$ is a lower bound of $E$.
\end{deff}

\begin{deff}\label{def:infinity}
	The extended real number system consists of the real field $R$ and two symbols, $+\infty$ and $-\infty$. We preserve the original order in $R$, and define $$-\infty<x<+\infty$$ for every $x\in R$.
\end{deff}


\section{Basic Topology}

\section{Finite, Countable, and Uncountable Sets}

\subsection{Definitions}
\begin{deff}
	Consider two sets $A$ and $B$, whose elements may be any objects whatsoever, and suppose that {\underline{with each element $x$ of A}} there is associated, in some manner, an element of $B$, which we denote by $f(x)$. Then $f$ is said to be a {\cem{function}} from $A$ to $B$ (or a {\cem{mapping}} of $A$ into $B$). The set $A$ is called the {\cem{domain}} of $f$ (we also say $f$ is defined on $A$), and the elments $f(x)$ are called the {\cem{values}} of $f$. The set of all values of $f$ is called the {\cem{range}} of $f$.
\end{deff}

\begin{deff}\label{def: mapping}
	Let $A$ and $B$ be two sets and let $f$ be a mapping of $A$ into $B$. If $E \subset A$, $f(E)$ is defined to be the set of all elements $f(x)$, for $x \in E$. We call $f(E)$ the {\cem{image}} of $E$ under $f$. In this notation, $f(A)$ is the range of $f$. It is clear that $f(A) \subset B$. If $f(A) = B$, we say that $f$ maps $A$ {\cem{onto}} $B$. (Note that, according to this usage, {\cem{onto}} is more specific thatn {\cem{into}}.)
\end{deff}

\begin{deff}
	If $E \subset B$ ($E$ is not necessarily a subset of $f(A)$), $f^{-1}(E)$ denotes the set of all $x \in A$ such that $f(x) \in E$. We call $f^{-1}(E)$ the {\cem{inverse image}} of $E$ under $f$. If $y \in B$, $f^{-1}(y)$ is the set of all $x \in A$ such that $f(x) = y$. If, for each $y \in B$, $f^{-1}(y)$ consists of at most one element of $A$, then $f$ is said to be a 1-1 ({\cem{one-to-one}}) mapping of $A$ into $B$. This may also be expressed as follows. $f$ is a 1-1 mapping of $A$ into $B$ provided that $f(x_1) \ne f(x_2)$ whenever $x_1 \ne x_2, x_1 \in A, x_2 \in A$.
\end{deff}

\begin{deff}
	If there exists a 1-1 mapping of $A$ {\cem{onto}} $B$, we say that $A$ and $B$ can be put in {\cem{1-1 correspondence}}, or that $A$ and $B$ have the same {\cem{cardinal number}}, or, briefly, that $A$ and $B$ are {\cem{equivalent}}, and we write $A \sim B$. This relation clearly has the following properties:
	\begin{enumerate}[]
		\item It is reflexive: $A \sim A$
		\item It is symmetric: If $A \sim B$, then $B \sim A$
		\item It is transitive: If $A \sim B$ and $B \sim C$, then $A \sim C$
	\end{enumerate}
	Any relation with theses three properties is called an {\cem{equivalence relation}}.
\end{deff}

\begin{deff}
	For any positive integer $n$, let $J_n$ be the set whose elements are the integers $1,2,\cdots,n$; let $J$ be the set consisting of all positive integers. For any set $A$, we say:
	\begin{enumerate}[(a)]
		\item $A$ is {\cem{finite}} if $A \sim J_n$ for some $n$ (the empty set is also considered to be finite).
		\item $A$ is {\cem{infinite}} if $A$ is not finite.
		\item $A$ is {\cem{countable}} if $A \sim J$.
		\item $A$ is {\cem{uncountable}} if $A$ is neither finite nor countable.
		\item $A$ is {\cem{at most countable}} if $A$ is finite or countable.
	\end{enumerate}
	Countable sets are sometimes called {\cem{enumerable}} or {\cem{denumerable}}.
\end{deff}

\begin{deff}
	By a {\cem{sequence}}, we mean a function $f$ defined on the set $J$ of all positive integers. If $f(n) = x_n$, for $n \in J$, it is customary to denote the sequence $f$ by the symbol $\{x_n\}$, or sometimes by $x_1, x_2, x_3, \cdots$. The values of $f$, that is, the elements $x_n$, are called the {\cem{terms}} of the sequence. If $A$ is a set and if $x_n \in A$ for all $n \in J$, then $\{x_n\}$ is said to be a {\cem{sequence in $A$}}, or a {\cem{sequence of elements of $A$}}.
\end{deff}

\begin{deff}
	Let $A$ and $\Omega$ be sets, and suppose that with each element $\alpha$ of $A$ there is associated a subset of $\Omega$ which we denote by $E_\alpha$. The set whose elements are the sets $E_\alpha$ will be denoted by $\{E_\alpha\}$. Instead of speaking of sets of sets, we shall sometimes speak of a collection of sets, or a family of sets. The {\cem{union}} of the sets $E_\alpha$ is defined to be the set $S$ such that $x \in S$ if and only if $x \in E_\alpha$ for at least one $\alpha \in A$. We use the notation $$S = \bigcup\limits_{\alpha \in A} E_\alpha.$$ The {\cem{intersection}} of the sets $E_\alpha$ is defined to be the set $P$ such that $x \in P$ if and only if $x \in E_\alpha$ for every $\alpha \in A$. We use the notation $$P = \bigcap\limits_{\alpha \in A} E_\alpha.$$
\end{deff}

\subsection{Theorems}
\begin{thm}
	$A$ is infinite if and only if $A$ is equivalent to one of its {\underline{proper subsets}}.
\end{thm}

\begin{thm}
	Every infinite subset of a countable set $A$ is countable.
\end{thm}

\begin{thm}
	Let $\{E_n\}, n = 1, 2, 3, \cdots$, be a sequence of countable sets, and put $$S = \bigcup\limits_{n=1}^\infty E_n.$$ Then $S$ is countable.
\end{thm}

\begin{thm}
	Let $A$ be a countable set, and let $B_n$ be the set of all $n$-tuples $(a_1,\cdots,a_n)$, where $a_k \in A (k = 1, \cdots, n)$, and the elements $a_1, \cdots, a_n$ need not be distinct. Then $B_n$ is countable.
\end{thm}

\begin{cor}
	The set of all rational numbers is countable.
\end{cor}

\begin{thm}
	Let $A$ be the set of all sequences whose elements are the digits $0$ and $1$. This set $A$ is uncountable.
\end{thm}

\section{Metric Spaces}
\subsection{Definitions}
\begin{deff}
	A set $X$, whose elements we shall call {\cem{points}}, is said to be a {\cem{metric space}} if with any two points $p$ and $q$ of $X$ there is associated a real number $d(p,q)$, called the {\cem{distance}} from $p$ to $q$, such that
	\begin{enumerate}[(a)]
		\item $d(p,q)>0$ if $p \ne q$; $d(p,q) = 0$;
		\item $d(p,q)=d(q,p)$;
		\item $d(p,q)\le d(p,r) + d(r,q)$, for any $r \in X$.
	\end{enumerate}
	Any function with these three properties is called a {\cem{distance function}}, or a {\cem{metric}}.
\end{deff}

\begin{deff}
	~
	\begin{enumerate}[(a)]
	\item By the {\cem{segment}} $(a,b)$ we mean the set of all real numbers $x$ such that $a<x<b$.
	\item By the {\cem{interval}} $[a,b]$ we mean the set of all real numbers $x$ such that $a \le x \le b$.
	\item Occasionally we shall also encounter ``half-open intervals'' $[a,b)$ and $(a,b]$; the first consist of all $x$ such that $a \le x < b$, the second of all $x$ such that $a < x \le b$.
	\item If $a_i < b_i$ for $i = 1, \cdots, k$, the set of all points $\x = (x_1, \cdots, x_k)$ in $R^k$ whose coordinates satisfy the inequalities $a_i \le x_i \le b_i (1 \le i \le k)$ is called a {\cem{k-cell}}.
	\item If $\x \in R^k$ and $r>0$, the {\cem{open}} (or {\cem{closed}}) {\cem{ball}} $B$ with center at $\x$ and radius $r$ is defined to be the set of all $y \in R^k$ such that $|\y - \x|<r$ (or $|\y - \x|<r$).
	\end{enumerate}
\end{deff}
\begin{deff}
	 We call a set $E \subset R^k$ {\cem{convex}} if $$\lambda \x + (1-\lambda) \y \in E $$ whenever $\x \in E, \y \in E$, and $0<\lambda<1$.
\end{deff}

\begin{deff}
	Let $X$ be a metric space. All points and sets mentioned below are understood to be elements and subsets of $X$.
	\begin{enumerate}[(a)]
		\item A {\cem{neighborhood}} of $p$ is a set $N_r(p)$ consisting of all $q$ such that $d(p,q) < r$, for some $r > 0$. The number $r$ is called the {\cem{radius}} of $N_r(p)$.
		\item A point $p$ is a limit point of the set $E$ if every neighborhood of $p$ contains a point $q \ne p$ such that $q \in E$.
		%\item If $p \in E$ and $p$ is not a limit point of $E$, then $p$ is called an {\cem{isolated point}} \smallmarginpar{An equivalent deff: There exsits a neighborhood of $p$ such that the only element {\underline{in $E$}} it contains is $p$ itself.} of $E$.
		\item $E$ is {\cem{closed }} if every limit point of $E$ is a point of $E$.
		\item A point $p$ is an {\cem{interior}} point of $E$ if there is a neighborhood $N$ of $p$ such that $N \subset E$.
		\item $E$ is {\cem{open}} if every point of $E$ is an interior point of $E$.
		\item The {\cem{complement}} of $E$ (denoted by $E^c$) is the set of all points $p \in X$ such that $p \notin E$.
		\item $E$ is {\cem{perfect}} if $E$ is closed and if every point of $E$ is a limit point of $E$.
		\item $E$ is {\cem{bounded}} if there is a real number $M$ and a point $q \in X$ such that $d(p,q) < M$ for all $p \in E$.
		\item $E$ is {\cem{dense}} in $X$ if every point of $X$ is a limit point of $E$, or a point of $E$ (or both).
	\end{enumerate}
\end{deff}

\begin{deff}
If $X$ is a metric space, if $E \subset X$, and if $E'$ denotes the set of all limit points of $E$ in $X$, then the {\cem{closure}} of $E$ is the set $\bar E = E \cup E'$.
\end{deff}

\subsection{Theorems}
\begin{thm}
	~
	\begin{enumerate}[(a)]
	\item Balls are convex.
	\item K-cells are convex.
	\end{enumerate}
\end{thm}

\begin{thm}
	Every neighborhood is an open set.
\end{thm}

\begin{thm}
	If $p$ is a limit point of a set $E$, then every neighborhood of $p$ contains infinitely many points of $E$.
\end{thm}

\begin{cor}
	A finite point set has no limit points.
\end{cor}

\begin{thm}
	Let \{$E_n$\} be a (finite or infinite) collection of sets $E_n$. Then $$\left(\bigcup _\alpha E_\alpha\right)^c = \bigcap_\alpha \left(E_\alpha^c\right).$$
\end{thm}

\begin{thm}
	A set $F$ is closed if and only if its complement is open.
\end{thm}
\begin{thm}
	~
	\begin{enumerate}[(a)]
	\item For any collection \{$G_n$\} of open sets, $\bigcup_n G_n$ is open.
	\item For any collection \{$F_n$\} of closed sets, $\bigcap_n F_n$ is closed.
	\item For any finite collection $G_1, \cdots, G_n$ of open sets, $\bigcap_{i=1}^n G_i $ is open.
	\item For any finite collection $F_1, \cdots, F_n$ of closed sets, $\bigcup_{i=1}^n F_i $ is closed.
	\end{enumerate}
\end{thm}

\begin{thm}
	If $X$ is a metric space and $E\subset X$, then
	\begin{enumerate}[(a)]
	\item $\bar E$ is closed,
	\item $E=\bar E$ if and only if $E$ is closed,
	\item $\bar E \subset F$ for every closed set $F\subset X$ such that $E \subset F$.
	\end{enumerate}
	By (a) and (c), $\bar E$ is the smallest closed subset of $X$ that contains $E$,
\end{thm}

\begin{thm}
	Let $E$ be a nonempty set of real numbers which is bounded above. Let $y = sup E$. Then $y \in \bar E$. Hence $y \in E$ if $E$ is closed.
\end{thm}

\begin{thm}
	Suppose $Y \subset X$. A subset $E$ of $Y$ is open relative to $Y$ is and ony if $E = Y \cap G$ for some open subset $G$ of $X$.
\end{thm}

\section{Compact Sets}
\subsection{Definitions}
\begin{deff}
	By an {\cem{open cover}} of a set $E$ in a metric space $X$ we mean a collection $\{G_\alpha\}$ of open subsets of $X$ such that $E \subset \bigcup_\alpha G_\alpha$.
\end{deff}

%\begin{deff}
%	A subset $K$ of a metric space $X$ is said to be {\cem{compact}} if %every open cover of $K$ contains a {\cem{finite}} subcover. %\smallmarginpar{It is clear that every finite set is compact.}
%\end{deff}

\subsection{Theorems}
%\begin{thm}
%	Suppose $K \subset Y \subset X$. Then $K$ is compact relative to $X$ if and only if $K$ is compact relative to $Y$. \smallmarginpar{Every metric space $X$ is an open subset of itself, and is a closed subset of itself.}
%\end{thm}

\begin{thm}
	Compact subsets of metric spaces are closed.
\end{thm}

\begin{thm}
	Cloased subsets of compact sets are compact.
\end{thm}

\begin{thm}
	If $F$ is closed and $K$ is compact, the n$F \cap K$ is compact.
\end{thm}

\begin{thm}
	If $\{K_\alpha\}$ is a collection of compact subsets of a metric space $X$ such that the intersection of every finite subcollection of $\{K_\alpha\}$ is nonempty, then $\cap K_\alpha$ is nonempty.
\end{thm}

\begin{thm}
	If $E$ is an infinite subset of a compact set $K$, then $E$ has a limit point in $K$.
\end{thm}

\begin{thm}
	If $\{I_n\}$ is a sequence of intervals in $R^1$, such that $I_n \supset I_{n+1}$ $(n=1,2,3,\cdots)$, then $\cap_{n=1}^\infty I_n$ is not empty.
\end{thm}

\begin{thm}
	Let $k$ be a positive integer. If $\{I_n\}$ is a sequence of $k$-cells such that $I_n \supset I_{n+1}$ $(n=1,2,3,\cdots)$, then $\cap_{n=1}^infty I_n$ is not empty.
\end{thm}

\begin{thm}
	Every $k$-cell is compact.
\end{thm}

\begin{thm}
	If a set $E$ in $R^k$ has one of the following three properties, then it has the other two:
	\begin{enumerate}[(a)]
		\item $E$ is closed and bounded.
		\item $E$ is compact.
		\item Every infinite subset of $E$ has a limit point in $E$.
	\end{enumerate}
\end{thm}

\begin{thm}
	Every bounded infinite subset of $R^k$ has a limit point in $R^k$.
\end{thm}

\section{Perfect Sets}
\subsection{Theorems}
\begin{thm}
	Let $P$ be a nonempty perfect set in $R^k$. Then $P$ is uncountable.
\end{thm}

\begin{cor}
	Every interval $[a,b]$ $(a < b)$ is uncountable. In particular, the set of all real numbers is uncountable.
\end{cor}

\section{Connected Sets}
\subsection{Definitions}
%\begin{deff}
%	Two subsets $A$ and $B$ of a metric space $X$ are said to be {\cem{separated}} if both $A \cap \bar B$ and $\bar A \cap B$ are empty, i.e., if no point of $A$lies in the closure of $B$ and no point of $B$lies in the closure of $A$. A set $E \subset X$ is siad to be {\cem{connected}} if $E$ is {\cem{not}} a union of two nonempty separated sets. \smallmarginpar{Separated sets are of course disjoint, but disjoint sets need not be sparated.}
%\end{deff}

\subsection{Theorems}
\begin{thm}
	A subset $E$ of the ral line $R^1$ is connected if and only if it has the following property: If $x \in E$, $y \in E$, and $x < z < y$, then $z \in E$.
\end{thm} 
\chapter{Numerical Sequences and Series}

\section{Convergent Sequences}
\subsection{Definitions}
\begin{definition}
	A sequence $\{p_n\}$ in a metric space $X$ is said to {\cem{converge}} \smallmarginpar{If $\{p_n\}$ does not converge, it is said to {\cem{diverge}}.} if there is a point $p \in X$ with the following property: For every $\epsilon > 0$ there is an integer $N$ such that $n \ge N$ implies that $d(p_n,p) < \epsilon$. (Here $d$ denotes the distance in X.)	
\end{definition}

\begin{definition}
	The sequence $\{p_n\}$ is said to be {\cem{bounded}} if its range is bounded.
\end{definition}

\subsection{Thorems}
\begin{theorem}\label{theo:convergent seq}
	Let $\{p_n\}$ be a sequence in a metric space $X$.
	\begin{enumerate}[(a)]
		\item $\{p_n\}$ converges to $p \in X$ if and only if every neighborhood of $p$ contains $p_n$ for all but finitely many $n$.
		\item If $p \in X, p' \in X$, and if $\{p_n\}$ converges to $p$ and to $p'$, then $p'=p$.
		\item If $\{p_n\}$ converges, then $\{p_n\}$ is bounded.
		\item If $E \subset X$ and if $p$ is a limit point \smallmarginpar{A point $p$ is a limit point of a set $E$ if and only if there is a sequence $\{p_n\}$ of {\underline{distinct points of $E$}} converging to $p$.} of $E$, then there is a sequence $\{p_n\}$ in $E$ such that $p = \lim_{n \to \infty} p_n$
	\end{enumerate}
\end{theorem}

\begin{theorem}
	Suppose $\{s_n\}, \{t_n\}$ are complex sequences, and $\lim_{n \to \infty} \{s_n\} = s$ and $\lim \{t_n\} = t$. Then,
	\begin{enumerate}[(a)]
		\item $\lim_{n \to \infty} (s_n + t_n) = s + t$;
		\item $\lim_{n \to \infty} (cs_n) = cs, \lim_{n \to \infty} (c+s_n) = c + s$, for all number $c$;		
		\item $\lim s_n t_n = st$;
		\item $\lim \frac{1}{s_n} = \frac{1}{s}$, provided $s_n \ne 0 (n=1,2,3,\cdots)$, and $s \ne 0$.
	\end{enumerate}	
\end{theorem}

\begin{theorem}
	~
	\begin{enumerate}[(a)]
		\item Suppose $\x_n \in R^k (n=1,2,3,\cdots)$ and $$\x_n = (\alpha_{1,n}, \cdots, \alpha_{k,n}.$$ Then $\{\x_n\}$ converges to $\x=(\alpha_1,\cdots,\alpha_k)$ if and only if $$\lim_{n \to \infty} \alpha_{j,n} = \alpha_j.$$
		\item Suppose $\{\x_n\},\{\y_n\}$ are sequences in $R^k$, $\{\beta_n\}$ is a sequence of real numbers, and $\x_n \to \x, \y_n \to \y, \beta_n \to \beta$. Then $$\lim_{n \to \infty} (\x_n+\y_n) = \x + \y, \lim_{n \to \infty} (\x_n \cdot \y_n) = \x \cdot \y, \lim_{n \to \infty} \beta_n \x_n = \beta \x.$$
	\end{enumerate}
\end{theorem}


\section{Subsequences}
\subsection{Definitions}
\begin{definition}\label{def:sublim}
	Given a sequence $\{p_n\}$, consider a sequence $\{n_k\}$ of positive integers, such that $n_1 < n_2 < n_3 < \cdots$. Then the sequence $\{p_{n_i}\}$ is called a {\cem{subsequence}} of $\{p_n\}$. If $\{p_{n_i}\}$ converges, its limit is called a {\cem{subsequential limit}} of $\{p_n\}$.
\end{definition}

\subsection{Theorems}
\begin{theorem}
	$\{p_n\}$ converges to $p$ if and only if every subsequence of $\{p_n\}$ converges to $p$.
\end{theorem}

\begin{theorem}
	~
	\begin{enumerate}[(a)]
		\item If $\{p_n\}$ is a sequence in a compact metric space $X$, then some subsequence of $\{p_n\}$ converges to a point of $X$.
		\item Every bounded sequence in $R^k$ contains a convergent subsequence.
	\end{enumerate}
\end{theorem}

\begin{theorem}
	The subsequential limits of a sequence $\{p_n\}$ in a metric space $X$ form a closed subset of $X$.
\end{theorem}

\section{Cauchy Sequences}
\subsection{Definitions}
\begin{definition}
	A sequence $\{p_n\}$ in a metric space X is said to be a {\cem{Cauchy sequence}} if for every $\epsilon>0$ there is an integer $N$ such that $d(p_n,p_m)<\epsilon$ if $n\geq N$ and $m\geq N$.
\end{definition}

\begin{definition}
	Let $E$ be a nonempty subset of a metric space $X$, and let $S$ be the set of all real numbers of the form $d(p,q)$, with $p\in E$ and $q\in E$. The sup of $S$ is called the {\cem{diameter}} of $E$.\\
	If $\{p_n\}$ is a sequence in $X$ and if $E_N$ consists of the points $p_N, p_{N+1}, p_{N+2}, \cdots$, it is clear from the two preceding definitions that $\{p_n\}$ is a {\cem{Cauchy sequence if and only if}} $$\lim_{N \to \infty} \text{diam }E_N =0.$$
\end{definition}

\begin{definition}
	A metric space in which every Cauchy sequence converges is said to be {\cem{complete}}. 
\end{definition}

\begin{definition}
	A sequence $\{s_n\}$ of real numbers is said to be
	\begin{enumerate}[(a)]
		\item {\cem{monotonically}} increasing if $s_n \leq s_{n+1} (n=1,2,3,\cdots)$;
		\item {\cem{monotonically}} decreasing if $s_n \geq s_{n+1} (n=1,2,3,\cdots)$;
	\end{enumerate}
\end{definition}

\subsection{Theorems}
\begin{theorem}
	~
	\begin{enumerate}[(a)]
		\item If $\bar E$ is the closure of a set $E$ in a metric space $X$, then $$\text{diam } \bar E = \text{diam }E.$$
		\item If $K_n$ is a sequence of compact sets in $X$ such that $K_n \supset K_{n+1} (n=1,2,3,\cdots)$ and if $$\lim_{n \to \infty} \text{diam } K_n =0,$$ then $\bigcap_1^\infty K_n$ consists of exactly one point.
	\end{enumerate}
\end{theorem}

\begin{theorem}
	~
	\begin{enumerate}[(a)]
		\item In any metric space $X$, every convergent sequence is a Cauchy sequence.
		\item If X is a compact metric space and if $\{p_n\}$ is a Cauchy sequence in $X$, then $\{p_n\}$ converges to some point of $X$.
		\item in $R^k$, every Cauchy sequence converges.
	\end{enumerate}
	The fact that a sequence converges in $R^k$ if and only it is a Cauchy sequence is usually called the {\cem{Cauchy criterion}} for convergence.\\
	This theorem says that {\cem{all compact metric spaces and all Euclidean spaces are complete}}. It implies also that {\cem{every closed subset of $E$ of a complete metric space $X$ is complete}}.  
\end{theorem}

\begin{theorem}
	Suppose $\{s_n\}$ is monotonic. Then $\{s_n\}$ converges if and only if it is bounded.
\end{theorem}

\section{Upper and Lower Limits}
\subsection{Definitions}
\begin{definition}
	Let $\{s_n\}$ be a sequence of real numbers with the following property: For every real $M$ there is an interger $N$ such that $n\geq N$ implies $s_n \geq M$. We then write $$s_n\to+\infty.$$ Similarly, if for every real $M$ there is an integer $N$ such that $n\geq N$ implies $s_n\leq M$, we write $$s_n\to-\infty.$$
\end{definition}

\begin{definition}\label{def:liminf}
	Let $\{s_n\}$ be a sequence of real numbers. Let $E$ be the set of numbers $x$ (in the extended real number system) such that $s_{n_k}\to x$ for some subsequence $\{s_{n_k}\}$. This set $E$ contains all subsequential limits as defined in Definition ~\ref{def:sublim}, plus possibly the numbers $+\infty, -\infty$.\\
	We now recall Definition ~\ref{def:supinf} and ~\ref{def:infinity} and put $$s^*=\sup E,$$ $$s_*=\inf E.$$ The numbers $s^*, s_*$ are called the {\cem{upper}} and {\cem{lower limits}} of $\{s_n\}$; we use the notation $$\limsup\limits_{n\rightarrow\infty}s_n=s^*,~~\liminf\limits_{n\rightarrow\infty}s_n=s_*$$
\end{definition}

\subsection{Theorems}
\begin{theorem}
	Let$\{s_n\}$ be a sequence of real numbers. Let $E$ and $s^*$ have the same meaning as in Definition ~\ref{def:liminf}. Then $s^*$ has the following two properties:
	\begin{enumerate}[(a)]
		\item $s^*\in E$
		\item If $x>s^*$, there is an integer $N$ such that $n\geq N$ implies $s_n<x$.
	\end{enumerate}
	Moreover, $s^*$ is the only number with the properties (a) and (b). \\
	Of course, an analogous result is true for $s_*$.
\end{theorem}

\begin{theorem}
	If $s_n\leq t_n$ for $n\geq N$, where N is fixed, then
	$$\liminf\limits_{n\rightarrow\infty}s_n\leq \liminf\limits_{n\rightarrow\infty}t_n,$$
	$$\limsup\limits_{n\rightarrow\infty}s_n\leq \limsup\limits_{n\rightarrow\infty}t_n,$$
\end{theorem}

\section{Some Special Sequences}
\subsection{Theorems}
\begin{theorem}
~
	\begin{enumerate}[(a)]
			\item If $p>0$, then $\lim_{n\to\infty}\frac{1}{n^p}=0$.
			\item If $p>0$, then $\lim_{n\to\infty}\sqrt[n] p=1$.
			\item $\lim_{n\to\infty} \sqrt[n]{n}=1.$
			\item If $p>0$ and $\alpha$ is real, then $\lim_{n\to\infty}\frac{n^\alpha}{(1+p)^n}=0$.
			\item If $|x|<1$, then $\lim_{n\to\infty}x^n=0$.
	\end{enumerate}
\end{theorem}

\section{Series}
\subsection{Definitions}
\begin{definition}
	Given a sequence $\{a_n\}$, we use the notation $$\sum^q_{n=p} a_n ~~(p\leq q)$$ to denote the sum $a_p+a_{p+1}+\cdots+a_q$. With $\{a_n\}$ we associate a sequence $\{s_n\}$, where $$s_n=\sum_{k=1}^n a_k.$$
	For $\{s_n\}$ we also use the symbolic expression $$a_1+a_2+a_3+\cdots$$ or, more concisely, $$\sum_{n=1}^\infty a_n.$$ The above symbol we call an {\cem{infinite series}}, or just a {\cem{series}}. The numbers $\{s_n\}$ are called the {\cem{partial sums}} of the series. If $\{s_n\}$ converges to $s$, we say that the series $converges$, and write $$\sum_{n=1}^\infty a_n =s.$$ The number $s$ is called the sum of the series; but it should be clearly understood that {\cem{s is the limit of a sequence of sums}}, and is not obtained simply by addition.\\
	If $\{s_n\}$ diverges, the series is said to diverge.
\end{definition}

\subsection{Theorems}
\theo{
	$\sum a_n$ converges if and only if for every $\epsilon>0$ there is an integer $N$ such that $$|\sum^m_{k=m}a_k|\leq\epsilon$$ if $m\geq n \geq N$.
}

\theo{
	If $\sum a_n$ converges, then $\lim_{n\to\infty}a_n=0$.
}

\theo{
	A series of nonnegative terms converges if and only if its partial sums form a bounded sequence.
}

\theo{
	~
	\begin{enumerate}[(a)]
			\item If $|a_n|\leq c_n$ for $n\geq N_0$, where $N_0$ is some fixed integer, and if $\sum c_n$ converges, then $\sum a_n$ converges.
			\item If $a_n \geq d_n \geq 0$ for $n\geq N_0$, and if $\sum d_n$ diverges, then $\sum a_n$ diverges.
	\end{enumerate}
}

\section{Series of Nonnegative Terms}
\subsection{Theorems}
\theo{
	If $0\leq x<1$, then $$\sum_{n=0}^\infty x^n = \frac{1}{1-x}.$$
	If $x\geq 1$, the series diverges.
}

\theo{
	Suppose $a_1\geq a_2\geq a_3 \geq \cdots \geq 0$. Then the series $\sum_{n=1}^\infty a_n$ converges if and only if the series
	$$\sum_{k=0}^\infty 2^k a_{2^k}=a_1+2a_2+4a_4+8a_8+\cdots$$ converges.
}
\theo{
	$\sum\frac{1}{n^p}$ converges if $p>1$ and diverges if $p\leq 1$.
}
\theo{
	If $p>1$, $$\sum_{n=2}^\infty \frac{1}{n(\log n)^p}$$ converges; if $p\leq 1$, the series diverges.
}

\section{The Number $e$}
\subsection{Definitions}
\deff{
	$$e=\sum_{n=0}^\infty \frac{1}{n!}$$
}
\subsection{Theorems}
\theo{
	$$\lim_{n\to\infty}\left(1+\frac{1}{n}\right)^n=e.$$
}
\theo{
	$e$ is irrational.
}

\section{The Root and Ratio Tests}
\subsection{Theorems}
\theo{
	(Root Test) Given $\sum a_n$, put $\alpha=\limsup\limits_{n\to\infty} \sqrt[n]{|a_n|}$.\\
	Then
	\begin{enumerate}[(a)]
			\item if $\alpha<1$, $\sum a_n$ converges;
			\item if $\alpha>1$, $\sum a_n$ diverges;
			\item if $\alpha=1$, the test gives no information.
	\end{enumerate}
}
\theo{
	(Ratio Test) The series $\sum a_n$
	\begin{enumerate}[(a)]
			\item converges if $\limsup\limits_{n\to\infty} \left|\frac{a_{n+1}}{a_n}\right|<1$,
			\item diverges if $\left|\frac{a_{n+1}}{a_n}\right|\geq 1$ for all $n\geq n_0$, where $n_0$ is some fixed integer.
	\end{enumerate}
}
\theo{
	For any sequence $\{c_n\}$ of positive numbers,
	$$\liminf\limits_{n\to\infty}\frac{c_{n+1}}{c_n}\leq \liminf\limits_{n\to\infty}\sqrt[n]{c_n},$$
	$$\limsup\limits_{n\to\infty}\sqrt[n]{c_n}\leq \limsup\limits_{n\to\infty}\frac{c_{n+1}}{c_n}.$$
}

\section{Power Series}
\subsection{Definitions}
\deff{
	Given a sequence $\{c_n\}$ of complex numbers, the series $$\sum_{n=0}^\infty c_n z^n$$ is called a {\cem{power series}}. The numbers $\{c_n\}$ are called the {\cem{coefficients}} of the series; $z$ is a complex number.
}
\subsection{Theorems}
\theo{
	Given the power series $\sum c_n z^n$, put
	$$\alpha = \limsup\limits_{n\to\infty} \sqrt[n]{|c_n|}, ~~ R=\frac{1}{\alpha}.$$
	(if $\alpha=0, R=+\infty$; if $\alpha=+\infty, R=0$.) Then $\sum c_n z^n$ converges if $|z|<R$, and diverges if $|z|>R$.
}

\section{Summation by Parts}
\subsection{Theorems}
\theo{
	Given two sequences $\{a_n\},\{b_n\}$, put $$A_n=\sum_{k=0}^n a_k$$ if $n\geq 0$; put $A_{-1}=0$. Then, if $0\leq p\leq q$, we have
	$$\sum_{n=p}^q a_n b_n = \sum_{n=p}^{q-1}A_n(b_n-b_{n+1})+A_q b_q-A_{p-1}b_p.$$
}
\theo{
	Suppose
	\begin{enumerate}[(a)]
			\item the partial sums $A_n$ of $\sum a_n$ form a bounded sequences;
			\item $b_0\geq b_1 \geq b_2 \geq \cdots$;
			\item $\lim_{n\to \infty} b_n = 0.$
	\end{enumerate}
}
\theo{
	Suppose
	\begin{enumerate}[(a)]
			\item $|c_1|\geq|c_2|\geq|c_3|\geq \cdots;$
			\item $c_{2m-1}\geq 0, c_{2m}\leq 0 ~~ (m=1,2,3,\cdots);$
			\item $\lim_{n\to\infty}c_n=0$.
	\end{enumerate}
	Then $\sum c_n$ converges.
}
\theo{
	Suppose the radius of convergence of $\sum c_n z^n$ is 1, and suppose $c_0\geq c_1 \geq c_2 \geq \cdots, \lim_{n\to\infty}c_n=0$. Then $\sum c_n z^n$ converges at every point on the circle $|z|=1$, except possibly at $z=1$.
}
\section{Absolute Convergence}
\subsection{Definitions}
\deff{The series $\sum a_n$ is said to {\cem{converge absolutely}} if the series $\sum|a_n|$ converges.}
\deff{
	If $\sum a_n$ converges, but $\sum|a_n|$ diverges, we say that $\sum a_n$ converges {\cem{nonabsolutely}}.
}
\subsection{Theorems}
\theo{
	If $\sum{a_n}$ converges absolutely, then $\sum a_n$ converges.
}


\section{Addition and Multiplication of Series}
\subsection{Definitions}
\deff{
	Given $\sum a_n$ and $\sum b_n$, we put $$c_n=\sum_{k=0}^n a_k b_{n-k} ~~ (n=0,1,2,\cdots)$$ and call $\sum c_n$ the {\cem{product}} of the two given series.
}
\subsection{Theorems}
\theo{
	If $\sum a_n=A$, and $\sum b_n =B$, then $\sum(a_n+b_n)=A+B$, and $\sum c a_n = cA$, for any fixed $c$.
}
\theo{
	Suppose
	\begin{enumerate}[(a)]
			\item $\sum_{n=0}^\infty a_n$ converges absolutely,
			\item $\sum_{n=0}^\infty a_n=A$,
			\item $\sum_{n=0}^\infty b_n=B$,
			\item $c_n=\sum_{k=0}^n a_k b_{n-k} ~~ (n=0,1,2,\cdots).$
	\end{enumerate}
	Then $$\sum_{n=0}^\infty c_n = AB.$$\\
	That is, the product of two convergent series converges, and to the right value, if at least one of the two series converges absolutely.
}
\theo{
	If the series $\sum a_n, \sum b_n, \sum c_n$ converge to $A,B,C$, and $c_n=a_0 b_n+\cdots+a_nb_0$ then $C=AB$.
}
\section{Rearrangements}
\subsection{Definitions}
\deff{
	Let $\{k_n\}, n=1,2,3,\cdots$, be a sequence in which every positive integer appears once and only once (that is, $\{k_n\}$ is a 1-1 function from $J$ onto $J$, in the notation of Definition ~\ref{def: mapping}). Putting $$a_n^\prime=a_{k_n}~~(n=1,2,3,\cdots),$$ we say that $\sum a_n^\prime$ is a {\cem{rearrangement}} of $\sum a_n$.
}
\subsection{Theorems}
\theo{
	Let $\sum a_n$ be a series of real numbers which converges, but not absolutely. Suppose $$-\infty \leq \alpha\leq\beta\leq\infty.$$ Then there exist a rearrangement $\sum a_m^\prime$ with partial sums $s_n^\prime$ such that
	$$\liminf\limits_{n\to\infty}s_n^\prime = \alpha, ~~ \limsup\limits_{n\to\infty}s_n^\prime=\beta.$$
}
\theo{
	If $\sum a_n$ is a series of complex numbers which converges absolutely, then every rearrangement of $\sum a_n$ converges, and they all converges to the same sum.
}
\chapter{Continuity}

\section{Limits of Functions}
\subsection{Definitions}
\begin{deff}\label{def:function limit}
Let $X$ and $Y$ be metric spaces; suppose $E \subset X$, $f$ maps $E$ into $Y$, and $p$ is a limit point of $E$. \smallmarginpar{The deff does not say anything about $f(p)$.} We write $f(x) \to q$ as $x \to p$, or $$\lim_{x \to p} f(x) = q$$ if there is a point $q \in Y$ with the following property: For every $\epsilon > 0$ there exists a $\delta > 0$ such that $$d_Y(f(x),q) < \epsilon$$ for all points $x \in E$ for which $$0 < d_X(x,p) < \delta.$$
\end{deff}

\subsection{Theorems}
\begin{thm}
	Let $X, Y, E, f, \mbox{ and } p$ be as in Definition ~\ref{def:function limit}. Then $$\lim_{x \to p} f(x) = q$$ if and only if $$\lim_{n \to \infty} f(p_n) = q$$ for every sequence $\{p_n\}$ in $E$ such that $$p_n \ne p, ~\lim_{n \to \infty}p_n = p.$$
\end{thm}

\begin{cor}
	If $f$ has a limit at $p$, this limit is unique.
\end{cor}

\begin{thm}
	Suppose $E \subset X$, a metric space, $p$ is a limit point of $E, f \mbox{ and } g$ are complex functions on $E$, and $$\lim_{x \to p}f(x) = A, ~ \lim_{x \to p} g(x) = B.$$ Then
	\begin{enumerate}[(a)]
		\item $\lim_{x \to p} (f+g)(x) = A + B$;
		\item $\lim_{x \to p} (fg)(x) = AB$;
		\item $\lim_{x \to p} (\frac{f}{g})(x) = \frac{A}{B}, ~ if ~ B \ne 0$.
	\end{enumerate}
\end{thm}

\section{Continuous Functions}
\subsection{Definitions}
\begin{deff}\label{def:function continuous}
	Suppose $X$ and $Y$ are metric spaces, $E \subset X$, $p \in E$, and $f$ maps $E$ into $Y$. Then $f$ is said to be {\cem{continuous at p}}  if for every $\epsilon > 0$ there exists a $\delta >0 $ such that $$\delta_Y(f(x),f(p)) < \epsilon$$ for all points $x \in E$ for which $d_X(x,p) < \delta.$
\end{deff}

\subsection{Theorems}
\begin{thm}
	In the situation given in Definition~\ref{def:function continuous}, assume also that {\underline{$p$ is a limit point}} \smallmarginpar{If $p$ is an {\underline{isolated point}} of $E$, then every function $f$ which has $E$ as its domain of defintion is continuous at $p$.} of $E$. Then $f$ is continuous at $p$ if and only if $\lim_{x \to p} f(x) = f(p).$
\end{thm}

\begin{thm}
	Suppose $X, Y, Z$ are metric spaces, $E \subset X$, $f$ maps $E$ into $Y$, $g$ maps the range of $f$, $f(E)$, into $Z$, and $h$ is the mapping of $E$ into $Z$ defined by $$h(x) = g(f(x)) ~ ~ (x \in E).$$ If $f$ is continuous at a point $p \in E$ and if $g$ is continuous at the point $f(p)$, then $h$ is continuous at $p$.
\end{thm}

\thm{
	A mapping $f$ of a metric space $X$ into a metric space $Y$ is onctinuous on $X$ if and only if $f^{-1}(V)$ is open in $X$ for every open set $V$ in $Y$.
}

\cor{
	A mapping $f$ of a metric space $X$ into a metric space $Y$ is continuous if and only if $f^{-1}(C)$ is closed in $X$ for every closed set $C$ in $Y$.
}

\thm{
	Let $f$ and $g$ be complex {\underline{continuous}} functions on a metric space $X$. Then $f+g, fg, \mbox{ and } f/g$ are {\underline{continuous}} on $X$.
}

\thm{
	~
	\begin{enumerate}[(a)]
		\item Let $f_1,\cdots,f_k$ be real functions on a metric space $X$, and let $\f$ be the mapping of $X$ into $R^k$ defined by $$\f(x) = (f_1(x),\cdots,f_k(x)) ~ ~ (x \in X);$$ then $\f$ is continuous if and only if each of the functions $f_1,\cdots,f_k$ is continuous.
		\item if $\f$ and $\g$ are continuous mappings of $X$ into $R^k$, then $\f+\g$ and $\f \cdot \g$ are continuous on $X$.
	\end{enumerate}
}

\section{Continuity and Compactness}

\subsection{Definitions}
\deff{
	A mapping $\f$ of a set $E$ into $R^k$ is said to be {\cem{bounded}} if there is a real number $M$ such that $|\f(x)| \le M$ for all $x \in E$.
}

\begin{deff}
	Let $f$ be a mapping of a metric space $X$ into a metric space $Y$. We say that $f$ is {\cem{uniformly continuous}} on $X$ if for every $\epsilon > 0$ there exists $\delta > 0$ such that $$d_Y(f(p),f(q)) < \epsilon$$ for all $p$ and $q$ in $X$ for which $d_X(p,q) < \delta$.
\end{deff}

\subsection{Theorems}
\thm{
	Suppose $f$ is a continuous mapping of a compact metric space $X$ into a metric space $Y$. Then $f(X)$ is compact.
}

\thm{
	If $\f$ is a continuous mapping of a compact metric space $X$ into $R^k$, then $\f(X)$ is closed and bounded. Thus, $\f$ is bounded.
}

\thm{
	Suppose $f$ is a continuous real function on a compact metric space $X$, and $$M = \sup_{p \in X} f(p), ~ ~ m = \inf_{p \in X}f(p).$$ Then there exist points $p, q \in X$ such that $f(p) = M$ and $f(q) = m$. \smallmarginpar{This is to say, $f$ attains its maximum (at $p$) and its minimum (at $q$).}
}

\thm{
	Suppose $f$ is a continuous 1-1 mapping of a compact metric space $X$ {\underline{onto}} a metric space $Y$. Then the inverse mapping $f^{-1}$ defined on $Y$ by $$f^{-1}(f(x)) = x ~ ~ (x \in X)$$ is a continuous mapping of $Y$ {\underline{onto}} $X$.
}

\thm{
	Let $f$ be a continuous mapping of a compact metric space $X$ into a metric space $Y$. Then $f$ is uniformly continuous on $X$.
}

\thm{
	Let $E$ be a noncompact set in $R^1$. Then
	\begin{enumerate}[(a)]
		\item there exists a continuous function on $E$ which is not bounded;
		\item there exists a continuous and bounded function on $E$ which has no maximum. If, in addition, $E$ is bounded, then 
		\item there exists a continuous function on $E$ which is not uniformly continuous.
	\end{enumerate}
}

\section{Continuity and Connectedness}
\subsection{Theorems}
\thm{
	If $f$ is a continuous mapping of a metric space $X$ into a metric space $Y$, and if $E$ is a connected subset of $X$, then $f(E)$ is connected.
}

\thm{
	Let $f$ be a continuous real function on the interval $[a,b]$. If $f(a) < f(b)$ and if $c$ is a number such that $f(a) < c < f(b)$, then there exists a point $x \in (a,b)$ such that $f(x) = c$.
}

\section{Discontinuities}
\subsection{Definitions}
\deff{
	Let $f$ be defined on $(a,b)$. Consider any point $x$ such that $a \le x < b$. We write $$f(x+) = q$$ if $f(t_n) \to q$ as $n \to \infty$, for all sequences $\{t_n\}$ in $(x,b)$ such that $t_n \to x$. To obtain the deff of $f(x-)$, for $a < x \le b$, we restrict ourselves to sequences $\{t_n\}$ in $(a,x)$. It is clear that any point $x$ of $(a,b)$, $\lim_{t \to x} f(t)$ exists if and only if $$f(x+) = f(x-) = \lim_{t \to x} f(t).$$ 
}

\deff{
	Let $f$ be defined on $(a,b)$. If $f$ is discontinuous at a point $x$, and if $f(x+)$ and $f(x-)$ exist, then $f$ is said to have a discontinuity of the {\cem{first kind}}, or a {\cem{simple discontinuity}} at $x$. \smallmarginpar{There are two ways in which a function can have a simple discontinuity: either $f(x+) \ne f(x-)$, in which case the value $f(x)$ is immaterial, or $f(x+) = f(x-) \ne f(x)$.} Otherwise the discontinuity is said to be of the {\cem{second kind}}.
}

\section{Monotonic Functions}
\subsection{Definitions}
\deff{
	Let $f$ be real on $(a,b)$. Then $f$ is said to be {\cem{monotonically increasing}} on $(a,b)$ if $a < x < y < b$ implies $f(x) \le f(y)$. If the last inequality is reversed, we obtain the deff of a {\cem{monotonically decreasing}} function. The class of monotonic functions consists of both the increasing and the deceasing functions.
}

\subsection{Theorems}
\thm{
	Let $f$ be monotonically increasing on $(a,b)$. Then $f(x+)$ and $f(x-)$ exist at every point of $x$ of $(a,b)$. More precisely, $$\sup_{a<t<x}f(t) = f(x-) \le f(x) \le f(x+) = \inf_{x<t<b}f(t).$$ Furthermore, if $a<x<y<b$, then $$f(x+) \le f(y-).$$ Analogous results evidently hod for monotonically decreasing functions.
}

\cor{
	Monotonic functions have no discontinuities of the second kind. \smallmarginpar{Compare with Corollary~\ref{cor: derivative functions do not have discontinuities of the first kind.}}
}\label{cor: monotone functions do not have discontinuities of the second kind.}

\thm{
	Let $f$ be monotonic on $(a,b)$. Then the set of points of $(a,b)$ at which $f$ is discontinuous is at most countable.
}

\section{Infinite Limits and Limits at Infinity}
\subsection{Definitions}
\deff{
	For any real $c$, the set of real numbers $x$ such that $x > c$ is called a neighborhood of $+\infty$ and is written $(c,+\infty)$. Similarly, the set $(-\infty,c)$ is a neighborhood of $-\infty$.
}

\deff{
	Let $f$ be a real function defined on $E \subset R$. We say that $$f(t) \to A \mbox{ as } t \to x,$$ where $A$ and $x$ are in the extended real number system, if for every neighborhood $U$ of $A$ there is a neighborhood $V$ of $x$ such that $V \cap E$ is not empty, and such that $f(t) \in U$ for all $t \in V \cap E, t \ne x$.
}

\subsection{Theorems}
\thm{
	Let $f$ and $g$ be defined on $E \subset R$. Suppose $$f(t) \to A, ~ ~ g(t) \to B \mbox{ as } t \to x.$$ Then
	\begin{enumerate}[(a)]
		\item $f(t) \to A' \mbox{ implies } A' = A.$
		\item $(f+g)(t) \to A + B,$
		\item $(fg)(t) \to AB,$
		\item $(f/g)(t) \to A/B,$
	\end{enumerate}
	provided the right member of (b), (c), and (d) are defined.
}
\section{Differentiation}

\section{The Derivative of a Real Function}
\subsection{Definitions}
\begin{deff}
	Let $f$ be defined (and real-valued) on $[a,b]$. For any $x \in [a,b]$ form the quotient $$\phi(t) = \frac{f(t)-f(x)}{t-x} (a<t<b, t \ne x),$$ and define $$f'(x) = \lim_{t \to x} \phi(t),$$ provided this limit exists in accordance with Definition~\ref{def:function limit}. We thus associate with the function $f$ a function $f'$ whose domain is the set of points $x$ at which the limit exists; $f'$ is called the {\cem{derivative}} of $f$. If $f'$ is defined at a point $x$, we say that $f$ is {\cem{differentiable}} at $x$. If $f'$ is defined at every point of a set $E \subset [a,b]$, we say that $f$ is differentiable on $E$.
\end{deff}

\subsection{Theorems}
%\begin{thm}
%	Let $f$ be defined on $[a,b]$. If $f$ is differentiable at a point $x \in [a,b]$, then $f$ is continuous at $x$. \smallmarginpar{Prove by using the fact that limit of a product is the product of limits.}
%\end{thm}

\thm{
	Suppose $f$ and $g$ are defined on $[a,b]$ and are differentiable at a point $x \in [a,b]$. Then $f+g, fg, \mbox{ and } f/g$ are differentiable at $x$, and
	\begin{enumerate}[(a)]
		\item $(f+g)'(x) = f'(x) + g'(x)$;
		\item $(fg)'(x) = f'(x)g(x) + f(x)g'(x);$
		\item $(\frac{f}{g})'(x) = \frac{g(x)f'(x)-g'(x)f(x)}{g^2(x)}$
	\end{enumerate}
	In (c), we assume of course that $g(x) \ne 0$.
}

\thm{
	Suppose $f$ is continuous on $[a,b]$, $f'(x)$ exists at some point $x \in [a,b]$, $g$ is defined on an interval $I$ which contains the range of $f$, and $g$ is differentiable at the point $f(x)$. If $$h(t) = g(f(t)) ~ ~ (a \le t \le b),$$ then $h$ is differentiable at $x$, and $$h'(x) = g'(f(x))f'(x).$$
}

\section{Mean Value Theorems}
\subsection{Definitions}
\begin{deff}
	Let $f$ be a real function defined on a metric space $X$. We say that $f$ has a {\cem{local maximum}} at a point $p \in X$ if there exists $\delta > 0$ such that $f(q) \le f(p)$ for all $q \in X$ with $d(p,q) < \delta.$
\end{deff}

\subsection{Theorems}
%\begin{thm}
%	Let $f$ be defined on $[a,b]$; if $f$ has a local maximum at a point $x \in (a,b)$, and if $f'(x)$ exists, then $f'(x) = 0.$ \smallmarginpar{Prove by showing the left-hand and right-hand derivatives}
%\end{thm}

\thm{
	If $f$ and $g$ are continuous real functions on $[a,b]$ which are differentiable in $(a,b)$, then there is a point $x \in (a,b)$ at which $$[f(b)-f(a)]g'(x) = [g(b)-g(a)]f'(x).$$ Note that differentiability is not required at the endpoints.
}

\thm{
	If $f$ is a real continuous function on $[a,b]$ which is differentiable in $(a,b)$, then there is a point $x \in (a,b)$ at which $$f(b)-f(a) = (b-a)f'(x).$$
}

\thm{
	Suppose $f$ is differentiable in $(a,b)$.
	\begin{enumerate}[(a)]
		\item If $f'(x) \ge 0$ for all $x \in (a,b)$, then $f$ is monotonically increasing.
		\item If $f'(x) = 0$ for all $x \in (a,b)$, then $f$ is constant.
		\item If $f'(x) \le 0$ for all $x \in (a,b)$, then $f$ is monotonically decreasing.
	\end{enumerate}
}

\section{The Continuity of Derivatives}
\subsection{Theorems}
\thm{
	Suppose $f$ is a real differentiable function on $[a,b]$ and suppose $f'(a) < \lambda < f'(b)$. Then there is a point $x \in (a,b)$ such that $f'(x) = \lambda$.
}

%\cor{
%	If $f$ is differentiable on $[a,b]$, then $f'$ cannot have any simple discontinuities on $[a,b]$. \smallmarginpar{Compare with Corollary~\ref{cor: monotone functions do not have discontinuities of the second kind.}}
%}\label{cor: derivative functions do not have discontinuities of the first kind.}

\section{L'Hospital's Rule}
\subsection{Theorems}
\thm{
	Suppose $f$ and $g$ are real and differentiable in $(a,b)$, and $g'(x) \ne 0$ for all $x \in (a,b)$, where $-\infty \le a < b \le +\infty$. Suppose $$\frac{f'(x)}{g‘(x)} \to A ~ as ~ x \to a.$$ If $$f(x) \to 0 ~ and ~ g(x) \to 0 ~ as ~ x \to a,$$ or if $$g(x) \to +\infty ~ as ~ x \to a,$$ then $$\frac{f(x)}{g(x)} \to A ~ as ~ x \to a.$$
}

\section{Derivatives of Higher Order}
\subsection{Definitions}
\deff{
	If $f$ has a derivative $f'$ on an interval, and if $f'$ is itself differentiable, we denote the derivative of $f'$ b $f''$ and call $f''$ the second derivative of $f$. Continuing in this manner , we obtain functions $$f,f',f'',f^{(3)},\cdots,f^{(n)},$$ each of which is the derivative of the preceding one. $f^{(n)}$ is called the $n$th derivative, or the derivative of order $n$, of $f$.
}

\section{Taylor's Theorem}
\subsection{Theorems}
\thm{
	Suppose $f$ is a real function on $[a,b]$, $n$ is a positive integer, $f^{(n-1)}$ is continuous on $[a,b]$, $f^{(n)}(t)$ exists for every $t \in (a,b)$. Let $\alpha, \beta$ be distincet points of $[a,b]$, and define $$P(t) = \sum_{k=0}^{n-1} \frac{f^{(k)}(\alpha)}{k!}(t-\alpha)^k.$$ Then there exists a point $x$ between $\alpha$ and $\beta$ such that $$f(\beta) = P(\beta) + \frac{f^{(n)}(x)}{n!} (\beta-\alpha)^n.$$
}

\section{Differentiation of Vector-valued Functions}
\subsection{Theorems}
\thm{
	Suppose $\f$ is a continuous mapping of $[a,b]$ into $R^k$ and $\f$ is differentiable in $(a,b)$. Then there exists $x \in (a,b)$ such that $$|\f(b)-\f(a)| \le (b-a)|\f'(x)|.$$
}



\chapter{The Riemann-Stieltjes Integral}
\section{Definition and Existence of the Integral}
\subsection{Definitions}
\deff{
	We say that the partition $P*$ is a {\cem{refinement}} of $P$ if $P* \supset P$ (that is, if every point of $P$ is a point of $P*$). Given two partitions, $P_1$ and $P_2$, we say that $P*$ is their {\cem{common refinement}} if $P* = P_1 \cup P_2$.
}
\subsection{Theorems}
\theo{
	If $P^*$ is a refinement of $P$, then $$L(P,f,\alpha) \le L(P^*,f,\alpha)$$ and $$U(P^*,f,\alpha) \le U(P,f,\alpha).$$
}

\theo{
	$\underline{\int_{a}^{b}} f d\alpha \le \overline{\int_{a}^{b}} f d\alpha$
}

\theo{
	$f \in \R(\alpha)$ on $[a,b]$ if and only if for every $\epsilon > 0$ there exists a partition $P$ such that $$U(P,f,\alpha) - L(P,f,\alpha) < \epsilon.$$
}\label{theo: integrable}

\theo{
	~
	\begin{enumerate}[(a)]
		\item If Theorem~\ref{theo: integrable} holds for some $P$ and some $\epsilon$, then Theorem~\ref{theo: integrable} holds (with the same $\epsilon$) for every refinement of $P$.
		\item If Theorem~\ref{theo: integrable} holds for $P = \{x_0,\cdots,x_n\}$ and if $s_i,t_i$ are arbitrary points in $[x_{i-1},x_i]$, then $$\sum_{i=1}^n |f(s_i)-f(t_i)| \Delta\alpha_i < \epsilon.$$
		\item If $f \in \R(\alpha)$ and the hypotheses of (b) hold, then $$|\sum_{i=1}^n |f(s_i)-f(t_i)| \Delta\alpha_i - \int_a^b f d\alpha| < \epsilon.$$
	\end{enumerate}
}

\theo{
	If $f$ is continuous on $[a,b]$ then $f \in \R(\alpha)$ on $[a,b]$.
}

\theo{
	If $f$ is monotonic on $[a,b]$, and if $\alpha$ is continuous on $[a,b]$, then $f \in \R(\alpha)$. (We still assume, of course, that $\alpha$ is monotonic.)
}

\theo{
	Suppose $f$ is bonded on $[a,b]$, $f$ has only finitely many points of discontinuity on $[a,b]$, and $\alpha$ is continuous at every point at which $f$ is discontinuous. Then $f \in \R(\alpha)$.
}

\theo{
	Suppose $f \in \R(\alpha)$ on $[a,b]$, $m \le f \le M$, $\phi$ is continuous on $[m,M]$, and $h(x) = \phi(f(x))$ on $[a,b]$. Then $h \in \R(\alpha)$ on $[a,b]$.
}

\section{Properties of the Integral}
\subsection{Definitions}
\deff{
	The {\cem{unit step function}} $I$ is defined by $$I(x) = \begin{cases}0 ~ ~ (x \le 0)\\1 ~ ~ (x > 0)\end{cases}$$
}

\subsection{Theorems}
\theo{
	If $f \in \R(\alpha)$ and $g \in \R(\alpha)$ on $[a,b]$, then
	\begin{enumerate}[(a)]
		\item $fg \in \R(\alpha)$;
		\item $|f| \in \R(\alpha)$ and $\left|\int_a^b f d\alpha\right| \le \int_a^b |f| d\alpha$.
	\end{enumerate}
}

\theo{
	If $a < s < b$, $f$ is bounded on $[a,b]$, $f$ is continuous at $s$, and $\alpha(x) = I(x-s)$, then $$\int_a^b f d\alpha = f(s).$$
}

\theo{
	Suppose $c_n \ge 0$ for $1,2,3,\cdots$, $\sum c_n$ converges, $\{s_n\}$ is a sequence of distinct points in $(a,b)$, and $$\alpha(x) = \sum_{n=1}^\infty c_n I(x-s_n).$$ Let $f$ be continuous on $[a,b]$. Then $$\int_a^b f d\alpha = \sum_{n=1}^\infty c_n f(s_n).$$
}

\theo{
	Assume $\alpha$ increases monotonically and $\alpha' \in \R$ on $[a,b]$. Let $f$ be a bounded real function on $[a,b]$. Then $f \in \R(\alpha)$ if and only if $f\alpha' \in \R$. In that case, $$\int_a^b f d\alpha = \int_a^b f(x) \alpha'(x) dx.$$
}

\theo{
	Suppose $\phi$ is a strictly increasing continuous function that maps an interval $[A,B]$ onto $[a,b]$. Suppose $\alpha$ is monotonnically increasing on $[a,b]$ and $f \in \R(\alpha)$ on $[a,b]$. Define $\beta$ and $g$ on $[A,B]$ by $$\beta(y) = \alpha(\phi(y)), ~ g(y) = f(\phi(y)).$$ Then $g \in \R(\beta)$ and $$\int_A^B g d\beta = \int_a^b f d\alpha.$$
}

\section{Integration and Differentiation}
\subsection{Theorems}
\theo{
	Let $f \in \R$ on $[a,b]$. For $a \le x \le b$, put $$F(x) = \int_a^x f(t) dt.$$ Then $F$ is continuous on $[a,b]$; furthermore, if $f$ is continuous at a point $x_0$ of $[a,b]$, then $F$ is differentiable at $x_0$, and $$F'(x_0) = f(x_0).$$
}
\theo{
	If $f \in \R$ on $[a,b]$ and if there is a differentiable function $F$ on $[a,b]$ such that $F'=f$, then $$\int_a^b f(x) dx = F(b) - F(a).$$
}
\theo{
	Suppose $F$ and $G$ are differentiable functions on $[a,b]$, $F' = f \in \R$, and $G' = g \in \R$. Then $$\int_a^b F(x)g(x) dx = F(b)G(b) - F(a)G(a) - \int_a^b f(x) G(x) dx.$$
}


\chapter{Sequences and Series of Functions}
\section{Discussion of Main Problem}
\subsection{Definitions}
\deff{
	Suppose $\{f_n\}$, $n=1,2,3,\cdots$, is a sequence of functions defined on a set $E$, and suppose that the sequence of numbers $\{f_n(x)$ converges for every $x \in E$. We can then define a function $f$ by $$f(x) = \lim_{n \to \infty} f_n(x) ~ ~ (x \in E).$$ Under these circumstances we say that $\{f_n\}$ converges on $E$ and that $f$ is the {\cem{limit}}, or the {\cem{limit function}}, of $\{f_n\}$. Sometimes we shall use a more descriptive terminology and shall say that ``$\{f_n\}$ converges to $f$ {\cem{pointwise}} on $E$'' if the above holds. Similarly, if $\sum f_n(x)$ converges for every $x \in E$, and if we define $$f(x) = \sum_{n=1}^\infty f_n(x) ~ ~ (x \in E),$$ the function $f$ is called the {\cem{sum}} of the series $\sum f_n$.
}

\section{Uniform Convergence}
\subsection{Definitions}
\begin{definition}
	We say that a sequence of functions $\{f_n\}, n = 1, 2, 3, \cdots$, converges {\cem{uniformly}} on $E$ to a function $f$ if for every $\epsilon > 0$ there is an integer $N$ such that $n \ge N$ implies $$|f_n(x)-f(x)| \le \epsilon$$ {\underline{for all $x \in E$}}.
\end{definition}

\subsection{Theorems}
\begin{theorem}
	Suppose $K$ is compact, and
	\begin{enumerate}[(a)]
		\item $\{f_n\}$ is a sequence of continuous functions on $K$,
		\item $\{f_n\}$ converges pointwise to a continuous function $f$ on $K$,
		\item $f_n(x) \ge f_{n+1}(x)$ for all $x \in K, n = 1, 2, 3, \cdots$.		
	\end{enumerate}
	Then $f_n \to f$ uniformly on $K$.
\end{theorem}

\theo{
	Supose $$\lim_{n \to \infty} f_n(x) = f(x) ~ ~ (x \in E).$$ Put $$M_n = \sup_{x \in E} |f_n(x) - f(x)|.$$ Then $f_n \to f$ uniformly on $E$ if and only if $M_n \to 0$ as $n \to \infty$.
}

\theo{
	Suppose $\{f_n\}$ is a sequence of functions defined on $E$, and suppose $$|f_n(x)| \le M_n ~~ (x \in E, n=1,2,3,\cdots).$$ Then $\sum f_n$ converges uniformly on $E$ if $\sum M_n$ converges.
}

\section{Uniform Convergence and Continuity}
\subsection{Definitions}
\deff{
	If $X$ is a metric space, $\C(X)$ will denote the set of all complex-valued, continuous, bounded functions with domain $X$. We associate with each $f \in \C(X)$ its supreme norm $$\|f\| = \sup_{x \in X} |f(x)|.$$ We also define the distance between $f \in \C(X)$ and $g \in \C(X)$ to be $\|f-g\|$.
}

\subsection{Theorems}
\theo{
	Suppose $f_n \to f$ uniformly on a set $E$ in a metric space. Let $x$ be a limit point of $E$, and suppose that $$\lim_{t \in x} f_n(t) = A_n ~ ~ (n=1,2,3,\cdots).$$ Then $\{A_n\}$ converges, and $$\lim_{t \in x} f(t) = \lim_{n \to \infty} A_n.$$ In other words, the conclusion is that $$\lim_{t \to x} \lim_{n \to \infty} f_n(t) = \lim_{n \to \infty} \lim_{t \to x} f_n(t).$$
}
\theo{
	If $\{f_n\}$ is a sequence of continuous functions on $E$, and if $f_n \to f$ uniformly on $E$, then $f$ is continuous on $E$.
}
\theo{
	Suppose $K$ is compact, and
	\begin{enumerate}[(a)]
		\item $\{f_n\}$ is a sequence of continuous functions on $K$,
		\item $\{f_n\}$ converges pointwise to a continuous function $f$ on $K$,
		\item $f_n(x) \ge f_{n+1}(x)$ for all $x \in K, n=1,2,3,\cdots$
	\end{enumerate}
	Then $f_n \to f$ uniformly on $K$.
}
\theo{
	The above metric makes $\C(X)$ into a complete metric space.
}

\section{Uniform Convergence and Integration}
\subsection{Theorems}
\theo{
	Let $\alpha$ be monotonically increasing on $[a,b]$. Suppose $f_n \in \R(\alpha)$ on $[a,b]$, for $n=1,2,3,\cdots$, and suppose $f_n \to f$ uniformly on $[a,b]$. Then $f \in \R(\alpha)$ on $[a,b]$, and $$\int_a^b f d\alpha = \lim_{n \to \infty} \int_a^b f_n d\alpha.$$ (The existence of the limit is part of the conclusion.)
}
\coro{
	If $f_n \in \R(\alpha)$ on $[a,b]$ and if $$f(x) = \sum_{n=1}^\infty f_n(x) ~ ~ (a \le x \le b),$$ the series converging uniformly on $[a,b]$, then $$\int_a^b f d\alpha = \sum_{n=1}^\infty \int_a^b f_n d\alpha.$$ In other words, the series may be integrated term by term.
}

\section{Uniform Convergence and Differentiation}
\subsection{Theorems}
\theo{
	Suppose $\{f_n\}$ is a sequence of functions, differentiable on $[a,b]$ and such that $\{f_n(x_0)\}$ converges for some point $x_0$ on $[a,b]$. If $\{f_n'\}$ converges uniformly on $[a,b]$, then $\{f_n\}$ converges uniformly on $[a,b]$, to a function $f$, and $$f'(x) = \lim_{n \to \infty} f'_n(x) ~ ~ (a \le x \le b).$$
}
\theo{
	There exists a real continuous function on the real line which is nowhere differentiable.
}

\section{Equicontinuous Families of Functions}
\subsection{Definitions}
\deff{
	Let $\{f_n\}$ be a sequence of functions defined on a set $E$. We say that $\{f_n\}$ is {\cem{pointwise bounded}} on $E$ if the sequence $\{f_n(x)\}$ is bounded for every $x \in E$, that is, if there exists a finite-valued function $\phi$ defined on $E$ such that $$|f_n(x)| < \phi(x) ~ ~ (x \in E, n=1,2,3,\cdots).$$ We say that $\{f_n\}$ is {\cem{uniformly bounded}} on $E$ if there exists a number $M$ such that $$|f_n(x)| < M ~ ~ (x \in E, n=1,2,3,\cdots).$$
}
\deff{
	A family $\F$ of complex functions $f$ defined on a set $E$ in a metric space $X$ is said to be {\cem{equicontinuous}} on $E$ if for every $\epsilon > 0$ there exists a $\delta > 0$ such that $$|f(x)-f(y)| < \epsilon$$ whenever $d(x,y) < \delta, x \in E, y \in E, ~ and ~ f \in \F$.
 }
 \subsection{Theorems}
 \theo{
 	If $\{f_n\}$ is a pointwise bounded sequence of complex functions on a countable set $E$, then $\{f_n\}$ has a subsequence $\{f_{n_k}\}$ such that $\{f_{n_k}\}$ converges for every $x \ in E$.
 }
 \theo{
 	If $K$ is a compact metric space, if $f_n \in \C(K)$ for $n=1,2,3,\cdots$, and if $\{f_n\}$ converges uniformly on $K$, then $\{f_n\}$ is equicontinuous on $K$.
 }
 \theo{
 	If $K$ is compact, if $f_n \in \C(K)$ for $n=1,2,3,\cdots$, and if $\{f_n\}$ is pointwise bounded and equicontinuous on $K$, then
 	\begin{enumerate}[(a)]
 		\item $\{f_n\}$ is uniformly bounded on $K$,
 		\item $\{f_n\}$ contains a uniformly convergent subsequence.
 	\end{enumerate}
 }

 \section{The Stone-Weierstrass Theorem}
 \subsection{Theorems}
 \theo{
 	If $f$ is a continuous complex function on $[a,b]$, there exists a sequence of polynomials $P_n$ such that $$\lim_{n \to \infty} P_n(x) = f(x)$$ uniformly on $[a,b]$. If $f$ is real, the $P_n$ may be taken real.
 }
 \coro{
 	For every interval $[-a,a]$ there is a sequence of real polynomials $P_n$ such that $P_n(0)=0$ and such that $$\lim_{n \to \infty} P_n(x) = |x|$$ uniformly on $[-a,a]$.
 }







\section{Some Special Functions}


\chapter{Functions of Several Variables}
\section{The Contraction Principle}
\subsection{Definitions}
\deff{
	Let $X$ be a metric space, with metric $d$. If $\phi$ maps $X$ into $X$ and if there is a number $c < 1$ such that $$d(\phi(x),\phi(y)) \le c ~ d(x,y)$$ for all $x, y \in X$, then $\phi$ is said to be a {\underline{{\cem{contraction}} of $X$ into $X$}}. \smallmarginpar{If $f$ is a contraction mapping then it is also a continuous mapping. The reverse is not true.}
}

\subsection{Theorems}
\theo{
	If $X$ is a complete metric space, and if $\phi$ is a contraction of $X$ into $X$, then there exists one and only one $x \in X$ such that $\phi(x) = x$.
}
\section{Exercises}
\subsection{Concept Questions}
\prob{
	A sequence $\{a_n\}$ converges if and only if it is bounded.
\\
	- FALSE. $\{sin(n)\}$ is bounded but not convergent. However, if a sequence converges, then it is bounded. See Theorem~\ref{theo:convergent seq}
}

