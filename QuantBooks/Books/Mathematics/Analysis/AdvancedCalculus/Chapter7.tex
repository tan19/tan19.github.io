\chapter{Sequences and Series of Functions}
\section{Discussion of Main Problem}
\subsection{Definitions}
\deff{
	Suppose $\{f_n\}$, $n=1,2,3,\cdots$, is a sequence of functions defined on a set $E$, and suppose that the sequence of numbers $\{f_n(x)$ converges for every $x \in E$. We can then define a function $f$ by $$f(x) = \lim_{n \to \infty} f_n(x) ~ ~ (x \in E).$$ Under these circumstances we say that $\{f_n\}$ converges on $E$ and that $f$ is the {\cem{limit}}, or the {\cem{limit function}}, of $\{f_n\}$. Sometimes we shall use a more descriptive terminology and shall say that ``$\{f_n\}$ converges to $f$ {\cem{pointwise}} on $E$'' if the above holds. Similarly, if $\sum f_n(x)$ converges for every $x \in E$, and if we define $$f(x) = \sum_{n=1}^\infty f_n(x) ~ ~ (x \in E),$$ the function $f$ is called the {\cem{sum}} of the series $\sum f_n$.
}

\section{Uniform Convergence}
\subsection{Definitions}
\begin{deff}
	We say that a sequence of functions $\{f_n\}, n = 1, 2, 3, \cdots$, converges {\cem{uniformly}} on $E$ to a function $f$ if for every $\epsilon > 0$ there is an integer $N$ such that $n \ge N$ implies $$|f_n(x)-f(x)| \le \epsilon$$ {\underline{for all $x \in E$}}.
\end{deff}

\subsection{Theorems}
\begin{thm}
	Suppose $K$ is compact, and
	\begin{enumerate}[(a)]
		\item $\{f_n\}$ is a sequence of continuous functions on $K$,
		\item $\{f_n\}$ converges pointwise to a continuous function $f$ on $K$,
		\item $f_n(x) \ge f_{n+1}(x)$ for all $x \in K, n = 1, 2, 3, \cdots$.		
	\end{enumerate}
	Then $f_n \to f$ uniformly on $K$.
\end{thm}

\thm{
	Supose $$\lim_{n \to \infty} f_n(x) = f(x) ~ ~ (x \in E).$$ Put $$M_n = \sup_{x \in E} |f_n(x) - f(x)|.$$ Then $f_n \to f$ uniformly on $E$ if and only if $M_n \to 0$ as $n \to \infty$.
}

\thm{
	Suppose $\{f_n\}$ is a sequence of functions defined on $E$, and suppose $$|f_n(x)| \le M_n ~~ (x \in E, n=1,2,3,\cdots).$$ Then $\sum f_n$ converges uniformly on $E$ if $\sum M_n$ converges.
}

\section{Uniform Convergence and Continuity}
\subsection{Definitions}
\deff{
	If $X$ is a metric space, $\C(X)$ will denote the set of all complex-valued, continuous, bounded functions with domain $X$. We associate with each $f \in \C(X)$ its supreme norm $$\|f\| = \sup_{x \in X} |f(x)|.$$ We also define the distance between $f \in \C(X)$ and $g \in \C(X)$ to be $\|f-g\|$.
}

\subsection{Theorems}
\thm{
	Suppose $f_n \to f$ uniformly on a set $E$ in a metric space. Let $x$ be a limit point of $E$, and suppose that $$\lim_{t \in x} f_n(t) = A_n ~ ~ (n=1,2,3,\cdots).$$ Then $\{A_n\}$ converges, and $$\lim_{t \in x} f(t) = \lim_{n \to \infty} A_n.$$ In other words, the conclusion is that $$\lim_{t \to x} \lim_{n \to \infty} f_n(t) = \lim_{n \to \infty} \lim_{t \to x} f_n(t).$$
}
\thm{
	If $\{f_n\}$ is a sequence of continuous functions on $E$, and if $f_n \to f$ uniformly on $E$, then $f$ is continuous on $E$.
}
\thm{
	Suppose $K$ is compact, and
	\begin{enumerate}[(a)]
		\item $\{f_n\}$ is a sequence of continuous functions on $K$,
		\item $\{f_n\}$ converges pointwise to a continuous function $f$ on $K$,
		\item $f_n(x) \ge f_{n+1}(x)$ for all $x \in K, n=1,2,3,\cdots$
	\end{enumerate}
	Then $f_n \to f$ uniformly on $K$.
}
\thm{
	The above metric makes $\C(X)$ into a complete metric space.
}

\section{Uniform Convergence and Integration}
\subsection{Theorems}
\thm{
	Let $\alpha$ be monotonically increasing on $[a,b]$. Suppose $f_n \in \R(\alpha)$ on $[a,b]$, for $n=1,2,3,\cdots$, and suppose $f_n \to f$ uniformly on $[a,b]$. Then $f \in \R(\alpha)$ on $[a,b]$, and $$\int_a^b f d\alpha = \lim_{n \to \infty} \int_a^b f_n d\alpha.$$ (The existence of the limit is part of the conclusion.)
}
\cor{
	If $f_n \in \R(\alpha)$ on $[a,b]$ and if $$f(x) = \sum_{n=1}^\infty f_n(x) ~ ~ (a \le x \le b),$$ the series converging uniformly on $[a,b]$, then $$\int_a^b f d\alpha = \sum_{n=1}^\infty \int_a^b f_n d\alpha.$$ In other words, the series may be integrated term by term.
}

\section{Uniform Convergence and Differentiation}
\subsection{Theorems}
\thm{
	Suppose $\{f_n\}$ is a sequence of functions, differentiable on $[a,b]$ and such that $\{f_n(x_0)\}$ converges for some point $x_0$ on $[a,b]$. If $\{f_n'\}$ converges uniformly on $[a,b]$, then $\{f_n\}$ converges uniformly on $[a,b]$, to a function $f$, and $$f'(x) = \lim_{n \to \infty} f'_n(x) ~ ~ (a \le x \le b).$$
}
\thm{
	There exists a real continuous function on the real line which is nowhere differentiable.
}

\section{Equicontinuous Families of Functions}
\subsection{Definitions}
\deff{
	Let $\{f_n\}$ be a sequence of functions defined on a set $E$. We say that $\{f_n\}$ is {\cem{pointwise bounded}} on $E$ if the sequence $\{f_n(x)\}$ is bounded for every $x \in E$, that is, if there exists a finite-valued function $\phi$ defined on $E$ such that $$|f_n(x)| < \phi(x) ~ ~ (x \in E, n=1,2,3,\cdots).$$ We say that $\{f_n\}$ is {\cem{uniformly bounded}} on $E$ if there exists a number $M$ such that $$|f_n(x)| < M ~ ~ (x \in E, n=1,2,3,\cdots).$$
}
\deff{
	A family $\F$ of complex functions $f$ defined on a set $E$ in a metric space $X$ is said to be {\cem{equicontinuous}} on $E$ if for every $\epsilon > 0$ there exists a $\delta > 0$ such that $$|f(x)-f(y)| < \epsilon$$ whenever $d(x,y) < \delta, x \in E, y \in E, ~ and ~ f \in \F$.
 }
 \subsection{Theorems}
 \thm{
 	If $\{f_n\}$ is a pointwise bounded sequence of complex functions on a countable set $E$, then $\{f_n\}$ has a subsequence $\{f_{n_k}\}$ such that $\{f_{n_k}\}$ converges for every $x \ in E$.
 }
 \thm{
 	If $K$ is a compact metric space, if $f_n \in \C(K)$ for $n=1,2,3,\cdots$, and if $\{f_n\}$ converges uniformly on $K$, then $\{f_n\}$ is equicontinuous on $K$.
 }
 \thm{
 	If $K$ is compact, if $f_n \in \C(K)$ for $n=1,2,3,\cdots$, and if $\{f_n\}$ is pointwise bounded and equicontinuous on $K$, then
 	\begin{enumerate}[(a)]
 		\item $\{f_n\}$ is uniformly bounded on $K$,
 		\item $\{f_n\}$ contains a uniformly convergent subsequence.
 	\end{enumerate}
 }

 \section{The Stone-Weierstrass Theorem}
 \subsection{Theorems}
 \thm{
 	If $f$ is a continuous complex function on $[a,b]$, there exists a sequence of polynomials $P_n$ such that $$\lim_{n \to \infty} P_n(x) = f(x)$$ uniformly on $[a,b]$. If $f$ is real, the $P_n$ may be taken real.
 }
 \cor{
 	For every interval $[-a,a]$ there is a sequence of real polynomials $P_n$ such that $P_n(0)=0$ and such that $$\lim_{n \to \infty} P_n(x) = |x|$$ uniformly on $[-a,a]$.
 }






