\section{Differentiation}

\section{The Derivative of a Real Function}
\subsection{Definitions}
\begin{deff}
	Let $f$ be defined (and real-valued) on $[a,b]$. For any $x \in [a,b]$ form the quotient $$\phi(t) = \frac{f(t)-f(x)}{t-x} (a<t<b, t \ne x),$$ and define $$f'(x) = \lim_{t \to x} \phi(t),$$ provided this limit exists in accordance with Definition~\ref{def:function limit}. We thus associate with the function $f$ a function $f'$ whose domain is the set of points $x$ at which the limit exists; $f'$ is called the {\cem{derivative}} of $f$. If $f'$ is defined at a point $x$, we say that $f$ is {\cem{differentiable}} at $x$. If $f'$ is defined at every point of a set $E \subset [a,b]$, we say that $f$ is differentiable on $E$.
\end{deff}

\subsection{Theorems}
%\begin{thm}
%	Let $f$ be defined on $[a,b]$. If $f$ is differentiable at a point $x \in [a,b]$, then $f$ is continuous at $x$. \smallmarginpar{Prove by using the fact that limit of a product is the product of limits.}
%\end{thm}

\thm{
	Suppose $f$ and $g$ are defined on $[a,b]$ and are differentiable at a point $x \in [a,b]$. Then $f+g, fg, \mbox{ and } f/g$ are differentiable at $x$, and
	\begin{enumerate}[(a)]
		\item $(f+g)'(x) = f'(x) + g'(x)$;
		\item $(fg)'(x) = f'(x)g(x) + f(x)g'(x);$
		\item $(\frac{f}{g})'(x) = \frac{g(x)f'(x)-g'(x)f(x)}{g^2(x)}$
	\end{enumerate}
	In (c), we assume of course that $g(x) \ne 0$.
}

\thm{
	Suppose $f$ is continuous on $[a,b]$, $f'(x)$ exists at some point $x \in [a,b]$, $g$ is defined on an interval $I$ which contains the range of $f$, and $g$ is differentiable at the point $f(x)$. If $$h(t) = g(f(t)) ~ ~ (a \le t \le b),$$ then $h$ is differentiable at $x$, and $$h'(x) = g'(f(x))f'(x).$$
}

\section{Mean Value Theorems}
\subsection{Definitions}
\begin{deff}
	Let $f$ be a real function defined on a metric space $X$. We say that $f$ has a {\cem{local maximum}} at a point $p \in X$ if there exists $\delta > 0$ such that $f(q) \le f(p)$ for all $q \in X$ with $d(p,q) < \delta.$
\end{deff}

\subsection{Theorems}
%\begin{thm}
%	Let $f$ be defined on $[a,b]$; if $f$ has a local maximum at a point $x \in (a,b)$, and if $f'(x)$ exists, then $f'(x) = 0.$ \smallmarginpar{Prove by showing the left-hand and right-hand derivatives}
%\end{thm}

\thm{
	If $f$ and $g$ are continuous real functions on $[a,b]$ which are differentiable in $(a,b)$, then there is a point $x \in (a,b)$ at which $$[f(b)-f(a)]g'(x) = [g(b)-g(a)]f'(x).$$ Note that differentiability is not required at the endpoints.
}

\thm{
	If $f$ is a real continuous function on $[a,b]$ which is differentiable in $(a,b)$, then there is a point $x \in (a,b)$ at which $$f(b)-f(a) = (b-a)f'(x).$$
}

\thm{
	Suppose $f$ is differentiable in $(a,b)$.
	\begin{enumerate}[(a)]
		\item If $f'(x) \ge 0$ for all $x \in (a,b)$, then $f$ is monotonically increasing.
		\item If $f'(x) = 0$ for all $x \in (a,b)$, then $f$ is constant.
		\item If $f'(x) \le 0$ for all $x \in (a,b)$, then $f$ is monotonically decreasing.
	\end{enumerate}
}

\section{The Continuity of Derivatives}
\subsection{Theorems}
\thm{
	Suppose $f$ is a real differentiable function on $[a,b]$ and suppose $f'(a) < \lambda < f'(b)$. Then there is a point $x \in (a,b)$ such that $f'(x) = \lambda$.
}

%\cor{
%	If $f$ is differentiable on $[a,b]$, then $f'$ cannot have any simple discontinuities on $[a,b]$. \smallmarginpar{Compare with Corollary~\ref{cor: monotone functions do not have discontinuities of the second kind.}}
%}\label{cor: derivative functions do not have discontinuities of the first kind.}

\section{L'Hospital's Rule}
\subsection{Theorems}
\thm{
	Suppose $f$ and $g$ are real and differentiable in $(a,b)$, and $g'(x) \ne 0$ for all $x \in (a,b)$, where $-\infty \le a < b \le +\infty$. Suppose $$\frac{f'(x)}{g‘(x)} \to A ~ as ~ x \to a.$$ If $$f(x) \to 0 ~ and ~ g(x) \to 0 ~ as ~ x \to a,$$ or if $$g(x) \to +\infty ~ as ~ x \to a,$$ then $$\frac{f(x)}{g(x)} \to A ~ as ~ x \to a.$$
}

\section{Derivatives of Higher Order}
\subsection{Definitions}
\deff{
	If $f$ has a derivative $f'$ on an interval, and if $f'$ is itself differentiable, we denote the derivative of $f'$ b $f''$ and call $f''$ the second derivative of $f$. Continuing in this manner , we obtain functions $$f,f',f'',f^{(3)},\cdots,f^{(n)},$$ each of which is the derivative of the preceding one. $f^{(n)}$ is called the $n$th derivative, or the derivative of order $n$, of $f$.
}

\section{Taylor's Theorem}
\subsection{Theorems}
\thm{
	Suppose $f$ is a real function on $[a,b]$, $n$ is a positive integer, $f^{(n-1)}$ is continuous on $[a,b]$, $f^{(n)}(t)$ exists for every $t \in (a,b)$. Let $\alpha, \beta$ be distincet points of $[a,b]$, and define $$P(t) = \sum_{k=0}^{n-1} \frac{f^{(k)}(\alpha)}{k!}(t-\alpha)^k.$$ Then there exists a point $x$ between $\alpha$ and $\beta$ such that $$f(\beta) = P(\beta) + \frac{f^{(n)}(x)}{n!} (\beta-\alpha)^n.$$
}

\section{Differentiation of Vector-valued Functions}
\subsection{Theorems}
\thm{
	Suppose $\f$ is a continuous mapping of $[a,b]$ into $R^k$ and $\f$ is differentiable in $(a,b)$. Then there exists $x \in (a,b)$ such that $$|\f(b)-\f(a)| \le (b-a)|\f'(x)|.$$
}


