\section{Continuity}

\section{Limits of Functions}
\subsection{Definitions}
%\begin{deff}\label{def:function limit}
%Let $X$ and $Y$ be metric spaces; suppose $E \subset X$, $f$ maps $E$ into $Y$, and $p$ is a limit point of $E$. \smallmarginpar{The deff does not say anything about $f(p)$.} We write $f(x) \to q$ as $x \to p$, or $$\lim_{x \to p} f(x) = q$$ if there is a point $q \in Y$ with the following property: For every $\epsilon > 0$ there exists a $\delta > 0$ such that $$d_Y(f(x),q) < \epsilon$$ for all points $x \in E$ for which $$0 < d_X(x,p) < \delta.$$
%\end{deff}

\subsection{Theorems}
\begin{thm}
	Let $X, Y, E, f, \mbox{ and } p$ be as in Definition ~\ref{def:function limit}. Then $$\lim_{x \to p} f(x) = q$$ if and only if $$\lim_{n \to \infty} f(p_n) = q$$ for every sequence $\{p_n\}$ in $E$ such that $$p_n \ne p, ~\lim_{n \to \infty}p_n = p.$$
\end{thm}

\begin{cor}
	If $f$ has a limit at $p$, this limit is unique.
\end{cor}

\begin{thm}
	Suppose $E \subset X$, a metric space, $p$ is a limit point of $E, f \mbox{ and } g$ are complex functions on $E$, and $$\lim_{x \to p}f(x) = A, ~ \lim_{x \to p} g(x) = B.$$ Then
	\begin{enumerate}[(a)]
		\item $\lim_{x \to p} (f+g)(x) = A + B$;
		\item $\lim_{x \to p} (fg)(x) = AB$;
		\item $\lim_{x \to p} (\frac{f}{g})(x) = \frac{A}{B}, ~ if ~ B \ne 0$.
	\end{enumerate}
\end{thm}

\section{Continuous Functions}
\subsection{Definitions}
\begin{deff}\label{def:function continuous}
	Suppose $X$ and $Y$ are metric spaces, $E \subset X$, $p \in E$, and $f$ maps $E$ into $Y$. Then $f$ is said to be {\cem{continuous at p}}  if for every $\epsilon > 0$ there exists a $\delta >0 $ such that $$\delta_Y(f(x),f(p)) < \epsilon$$ for all points $x \in E$ for which $d_X(x,p) < \delta.$
\end{deff}

\subsection{Theorems}
%\begin{thm}
%	In the situation given in Definition~\ref{def:function continuous}, assume also that {\underline{$p$ is a limit point}} \smallmarginpar{If $p$ is an {\underline{isolated point}} of $E$, then every function $f$ which has $E$ as its domain of defintion is continuous at $p$.} of $E$. Then $f$ is continuous at $p$ if and only if $\lim_{x \to p} f(x) = f(p).$
%\end{thm}

\begin{thm}
	Suppose $X, Y, Z$ are metric spaces, $E \subset X$, $f$ maps $E$ into $Y$, $g$ maps the range of $f$, $f(E)$, into $Z$, and $h$ is the mapping of $E$ into $Z$ defined by $$h(x) = g(f(x)) ~ ~ (x \in E).$$ If $f$ is continuous at a point $p \in E$ and if $g$ is continuous at the point $f(p)$, then $h$ is continuous at $p$.
\end{thm}

\thm{
	A mapping $f$ of a metric space $X$ into a metric space $Y$ is onctinuous on $X$ if and only if $f^{-1}(V)$ is open in $X$ for every open set $V$ in $Y$.
}

\cor{
	A mapping $f$ of a metric space $X$ into a metric space $Y$ is continuous if and only if $f^{-1}(C)$ is closed in $X$ for every closed set $C$ in $Y$.
}

\thm{
	Let $f$ and $g$ be complex {\underline{continuous}} functions on a metric space $X$. Then $f+g, fg, \mbox{ and } f/g$ are {\underline{continuous}} on $X$.
}

\thm{
	~
	\begin{enumerate}[(a)]
		\item Let $f_1,\cdots,f_k$ be real functions on a metric space $X$, and let $\f$ be the mapping of $X$ into $R^k$ defined by $$\f(x) = (f_1(x),\cdots,f_k(x)) ~ ~ (x \in X);$$ then $\f$ is continuous if and only if each of the functions $f_1,\cdots,f_k$ is continuous.
		\item if $\f$ and $\g$ are continuous mappings of $X$ into $R^k$, then $\f+\g$ and $\f \cdot \g$ are continuous on $X$.
	\end{enumerate}
}

\section{Continuity and Compactness}

\subsection{Definitions}
\deff{
	A mapping $\f$ of a set $E$ into $R^k$ is said to be {\cem{bounded}} if there is a real number $M$ such that $|\f(x)| \le M$ for all $x \in E$.
}

\begin{deff}
	Let $f$ be a mapping of a metric space $X$ into a metric space $Y$. We say that $f$ is {\cem{uniformly continuous}} on $X$ if for every $\epsilon > 0$ there exists $\delta > 0$ such that $$d_Y(f(p),f(q)) < \epsilon$$ for all $p$ and $q$ in $X$ for which $d_X(p,q) < \delta$.
\end{deff}

\subsection{Theorems}
\thm{
	Suppose $f$ is a continuous mapping of a compact metric space $X$ into a metric space $Y$. Then $f(X)$ is compact.
}

\thm{
	If $\f$ is a continuous mapping of a compact metric space $X$ into $R^k$, then $\f(X)$ is closed and bounded. Thus, $\f$ is bounded.
}

%\thm{
%	Suppose $f$ is a continuous real function on a compact metric space $X$, and $$M = \sup_{p \in X} f(p), ~ ~ m = \inf_{p \in X}f(p).$$ Then there exist points $p, q \in X$ such that $f(p) = M$ and $f(q) = m$. \smallmarginpar{This is to say, $f$ attains its maximum (at $p$) and its minimum (at $q$).}
%}

\thm{
	Suppose $f$ is a continuous 1-1 mapping of a compact metric space $X$ {\underline{onto}} a metric space $Y$. Then the inverse mapping $f^{-1}$ defined on $Y$ by $$f^{-1}(f(x)) = x ~ ~ (x \in X)$$ is a continuous mapping of $Y$ {\underline{onto}} $X$.
}

\thm{
	Let $f$ be a continuous mapping of a compact metric space $X$ into a metric space $Y$. Then $f$ is uniformly continuous on $X$.
}

\thm{
	Let $E$ be a noncompact set in $R^1$. Then
	\begin{enumerate}[(a)]
		\item there exists a continuous function on $E$ which is not bounded;
		\item there exists a continuous and bounded function on $E$ which has no maximum. If, in addition, $E$ is bounded, then
		\item there exists a continuous function on $E$ which is not uniformly continuous.
	\end{enumerate}
}

\section{Continuity and Connectedness}
\subsection{Theorems}
\thm{
	If $f$ is a continuous mapping of a metric space $X$ into a metric space $Y$, and if $E$ is a connected subset of $X$, then $f(E)$ is connected.
}

\thm{
	Let $f$ be a continuous real function on the interval $[a,b]$. If $f(a) < f(b)$ and if $c$ is a number such that $f(a) < c < f(b)$, then there exists a point $x \in (a,b)$ such that $f(x) = c$.
}

\section{Discontinuities}
\subsection{Definitions}
\deff{
	Let $f$ be defined on $(a,b)$. Consider any point $x$ such that $a \le x < b$. We write $$f(x+) = q$$ if $f(t_n) \to q$ as $n \to \infty$, for all sequences $\{t_n\}$ in $(x,b)$ such that $t_n \to x$. To obtain the deff of $f(x-)$, for $a < x \le b$, we restrict ourselves to sequences $\{t_n\}$ in $(a,x)$. It is clear that any point $x$ of $(a,b)$, $\lim_{t \to x} f(t)$ exists if and only if $$f(x+) = f(x-) = \lim_{t \to x} f(t).$$
}

%\deff{
%	Let $f$ be defined on $(a,b)$. If $f$ is discontinuous at a point $x$, and if $f(x+)$ and $f(x-)$ exist, then $f$ is said to have a discontinuity of the {\cem{first kind}}, or a {\cem{simple discontinuity}} at $x$. \smallmarginpar{There are two ways in which a function can have a simple discontinuity: either $f(x+) \ne f(x-)$, in which case the value $f(x)$ is immaterial, or $f(x+) = f(x-) \ne f(x)$.} Otherwise the discontinuity is said to be of the {\cem{second kind}}.
%}

\section{Monotonic Functions}
\subsection{Definitions}
\deff{
	Let $f$ be real on $(a,b)$. Then $f$ is said to be {\cem{monotonically increasing}} on $(a,b)$ if $a < x < y < b$ implies $f(x) \le f(y)$. If the last inequality is reversed, we obtain the deff of a {\cem{monotonically decreasing}} function. The class of monotonic functions consists of both the increasing and the deceasing functions.
}

\subsection{Theorems}
\thm{
	Let $f$ be monotonically increasing on $(a,b)$. Then $f(x+)$ and $f(x-)$ exist at every point of $x$ of $(a,b)$. More precisely, $$\sup_{a<t<x}f(t) = f(x-) \le f(x) \le f(x+) = \inf_{x<t<b}f(t).$$ Furthermore, if $a<x<y<b$, then $$f(x+) \le f(y-).$$ Analogous results evidently hod for monotonically decreasing functions.
}

%\cor{
%	Monotonic functions have no discontinuities of the second kind. \smallmarginpar{Compare with Corollary~\ref{cor: derivative functions do not have discontinuities of the first kind.}}
%}\label{cor: monotone functions do not have discontinuities of the second kind.}

\thm{
	Let $f$ be monotonic on $(a,b)$. Then the set of points of $(a,b)$ at which $f$ is discontinuous is at most countable.
}

\section{Infinite Limits and Limits at Infinity}
\subsection{Definitions}
\deff{
	For any real $c$, the set of real numbers $x$ such that $x > c$ is called a neighborhood of $+\infty$ and is written $(c,+\infty)$. Similarly, the set $(-\infty,c)$ is a neighborhood of $-\infty$.
}

\deff{
	Let $f$ be a real function defined on $E \subset R$. We say that $$f(t) \to A \mbox{ as } t \to x,$$ where $A$ and $x$ are in the extended real number system, if for every neighborhood $U$ of $A$ there is a neighborhood $V$ of $x$ such that $V \cap E$ is not empty, and such that $f(t) \in U$ for all $t \in V \cap E, t \ne x$.
}

\subsection{Theorems}
\thm{
	Let $f$ and $g$ be defined on $E \subset R$. Suppose $$f(t) \to A, ~ ~ g(t) \to B \mbox{ as } t \to x.$$ Then
	\begin{enumerate}[(a)]
		\item $f(t) \to A' \mbox{ implies } A' = A.$
		\item $(f+g)(t) \to A + B,$
		\item $(fg)(t) \to AB,$
		\item $(f/g)(t) \to A/B,$
	\end{enumerate}
	provided the right member of (b), (c), and (d) are defined.
} 