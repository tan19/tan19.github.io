\chapter{Functions of Several Variables}
\section{The Contraction Principle}
\subsection{Definitions}
\deff{
	Let $X$ be a metric space, with metric $d$. If $\phi$ maps $X$ into $X$ and if there is a number $c < 1$ such that $$d(\phi(x),\phi(y)) \le c ~ d(x,y)$$ for all $x, y \in X$, then $\phi$ is said to be a {\underline{{\cem{contraction}} of $X$ into $X$}}. \smallmarginpar{If $f$ is a contraction mapping then it is also a continuous mapping. The reverse is not true.}
}

\subsection{Theorems}
\thm{
	If $X$ is a complete metric space, and if $\phi$ is a contraction of $X$ into $X$, then there exists one and only one $x \in X$ such that $\phi(x) = x$.
}