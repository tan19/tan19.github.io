\section{The Real and Complex Number Systems}
\subsection{Definitions}
\begin{deff}
	If $A$ is any set (whose elements may be numbers or any other objects), we write $x \in A$ to indicate that $x$ is a memeber (or an element) of $A$. If $x$ is not a memeber of $A$, we write: $x \notin A$.
\end{deff}

\begin{deff}
	Throughtout Chap. 1, the set of all rational numbers will be denoted by $Q$.
\end{deff}

\begin{deff}\label{def:supinf}
	Suppose $S$ is an ordered set, $E\subset S$, and $E$ is bounded above. Suppose there exists an $\alpha\in S$ with the following properties:
	\begin{enumerate}[(i)]
		\item $\alpha$ is an upper bound of $E$.
		\item If $\gamma <\alpha$ then $\gamma$ is not an upper bound of $E$.
	\end{enumerate}
	Then $\alpha$ is called the {\cem{least upper bound}} of $E$ [that there is at most one such $\alpha$ is clear from (ii)] or the {\cem{supremum}} of $E$, and we write $$\alpha=\sup E.$$ The {\cem{greatest lower bound}}, or {\cem{infimum}}, of a set $E$ which is bounded below is defined in the same manner: The statement $$\alpha = \inf E$$ means that $\alpha$ is a lower bound of $E$ and that no $\beta$ with $\beta>\alpha$ is a lower bound of $E$.
\end{deff}

\begin{deff}\label{def:infinity}
	The extended real number system consists of the real field $R$ and two symbols, $+\infty$ and $-\infty$. We preserve the original order in $R$, and define $$-\infty<x<+\infty$$ for every $x\in R$.
\end{deff}

