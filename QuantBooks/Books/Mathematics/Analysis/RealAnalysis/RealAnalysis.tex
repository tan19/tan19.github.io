\documentclass{article}
\usepackage{amssymb,amsmath,mathtools}


\usepackage{algorithm2e,algorithmic}

\usepackage{mathrsfs}
\usepackage{paralist}

\usepackage{esint} % for \fint

\allowdisplaybreaks

\usepackage{color}		% enable color characters
\usepackage{graphicx} 	% insert image files
\usepackage{enumerate} 	% enumerate items
\usepackage{caption}
\usepackage{subcaption}
\usepackage{multirow,multicol}
\usepackage[makeroom]{cancel}

\usepackage[colorlinks=true,linkcolor=blue,citecolor=blue]{hyperref}

\usepackage{makeidx}

\newcommand{\cem}[1]{\color{magenta}{\em{#1}}} % color emphasize
\newcommand{\dual}[2]{{#1}^{(#2)}} % color emphasize

\newcommand{\tr}{{\mathrm{tr}}}
\renewcommand{\vec}{{\mathrm{vec}}}

\newcommand{\dd}{{\,\mathrm{d}\,}}
%%
%% horizontal and vertical centering in table p mode
%%
\usepackage{array}
\newcolumntype{P}[1]{>{\centering\arraybackslash}p{#1}} % horizontal centering
\newcolumntype{M}[1]{>{\centering\arraybackslash}m{#1}} % vertical centering

%%
%% define bold font for the alphabet
%%
\usepackage{pgffor}
\foreach \letter in {a,...,z}{ % bold font for a..z
\expandafter\xdef\csname \letter \endcsname{\noexpand\ensuremath{\noexpand\mathbf{\letter}}}
}
\foreach \letter in {A,...,Z}{ % bold font for A..Z
\expandafter\xdef\csname \letter \endcsname{\noexpand\ensuremath{\noexpand\mathbf{\letter}}}
}
\foreach \letter in {A,...,Z}{ % `field' font for AA..ZZ
\expandafter\xdef\csname \letter\letter \endcsname{\noexpand\ensuremath{\noexpand\mathcal{\letter}}}
}
\foreach \letter in {A,...,Z}{ % `field' font for AAA..ZZZ
\expandafter\xdef\csname \letter\letter\letter \endcsname{\noexpand\ensuremath{\noexpand\mathbb{\letter}}}
}
\newcommand{\balpha}{{\boldsymbol{\alpha}}}
\newcommand{\bbeta}{{\boldsymbol{\beta}}}
\newcommand{\bgamma}{{\boldsymbol{\gamma}}}
\newcommand{\bkappa}{{\boldsymbol{\kappa}}}
\newcommand{\bmu}{{\boldsymbol{\mu}}}
\newcommand{\btheta}{{\boldsymbol{\theta}}}
\newcommand{\bTheta}{{\boldsymbol{\Theta}}}
\newcommand{\bPi}{{\boldsymbol{\Pi}}}
\newcommand{\bSigma}{{\boldsymbol{\Sigma}}}
\newcommand{\bPhi}{{\boldsymbol{\Phi}}}
\newcommand{\bLambda}{{\boldsymbol{\Lambda}}}
\newcommand{\bdeta}{{\boldsymbol{\eta}}}
\newcommand{\bphi}{{\boldsymbol{\phi}}}



%%
%% add definitions and theorems
%%
\usepackage[thmmarks,amsmath]{ntheorem}
\theorembodyfont{\normalfont}
\newtheorem{deff}{Definition}[section]
\newtheorem{thm}{Theorem}[section]
\newtheorem{prop}{Proposition}[section]
\newtheorem{lem}{Lemma}[section]
\newtheorem{cor}{Corollary}[section]
\newtheorem{rmk}{Remark}[section]
\newtheorem{alg}{Algorithm}[section]
\newtheorem{ex}{Example}[section]
\newtheorem{ques}{Question}[section]
\newtheorem{ans}{Answer}[section]
\newtheorem{prob}{Problem}[section]
\newtheorem{sol}{Solution}[section]
\newtheorem*{prof}{Proof}[section] 

\title{Analysis}
\author{Xi Tan (tan19@purdue.edu)}
\date{\today}

\begin{document}
\maketitle

\tableofcontents
\newpage

\section*{Preface}
This book reviews calculus, advanced calculus, real analysis, and functional Analysis. The main references to be used are \cite{Stewart} for calculus, \cite{Rudin} for advanced calculus, \cite{Royden} for real analysis, and \cite{Kreyszig} for functional analysis. Other useful texts include: \cite{Folland} and \cite{Torchinsky} for real analysis.
\newpage

\section{Introduction}
The core material of real analysis is that of Lebesgue integral, which extends the application of Riemann integral to a larger family of functions. The prerequisite of Lebesgue integral is measure theory. We begin from important concepts of sets, point topology, and the real number system, then continue with measurable functions before discussing Lebesgue integral.

\section{Set Theory}


\section{Point Topology}
\section{Real Number System}
The real number system can be characterized by three axioms: 1) the filed axiom, 2) the order axiom, and 3) the completeness axiom.

Of particular interest is the completeness axiom. Depending on the construction of real numbers, it can take the form of axioms (the completeness axiom), or a theorem from the construction. These include:
\begin{enumerate}
	\item Lease upper bound property
	\item Dedekind completeness
	\item Cauchy completeness
	\item Nested intervals theorem
	\item Monotone convergence theorem
	\item Bolzano-Weierstrass theorem
\end{enumerate}

\section{Measure Theory}

\section{Measurable Sets and Measurable Functions}

\section{Lebesgue Integration}

\begin{thebibliography}{100} % 100 is a random guess of the total number of %references  
\bibitem{Stewart} James Stewart {\em{Calsulus - Early Transcendentals}}. Cengage Learning, 2012
\bibitem{Rudin} Walter Rudin {\em{Principles of Mathematical Analysis}}. McGraw-Hill Companies, Inc., 1976.
\bibitem{Royden} H. L. Royden {\em{Real Analysis}}. Pearson Eduction, Inc., 1988.
\bibitem{Kreyszig} Erwin Kreyszig {\em{Introductory Functional Analysis with Applications}}. Wiley, 1989.
\bibitem{Folland} Gerald B. Folland {\em{Real Analysis: Modern Techniques and Their Applications}}. Wiley, 1999.
\bibitem{Torchinsky} Alberto Torchinsky {\em{Real Variables}}. Westview Press, 1995.
\end{thebibliography}

\end{document}
