\chapter*{Preface}
This book project, which consists of four subjects: Finance, Mathematics, Statistics, and Computer Science, is tailored specifically to prepare someone for a quant career. It originated from my general belief of the hierarchy of solving a problem --- problems are solved at strategic, tactical, and operational levels.

{\em{Microeconomics}} and {\em{Macroeconomics}} explain the driving forces of capital markets, from a legislator's perspective. {\em{Accounting}} and {\em{Corporate Finance}} take a closer and necessary look at these forces, from a different angle. {\em{Stochastic Calculus}} and {\em{Asset Pricing}} provide with a set of tools and ideas that enables us to {\bf{strategically}} model one of the central problems in Quantitative Finance.

Generally speaking, there are two paths to solve a quantitative finance problem at the {\bf{tactical}} level: the mathematical way and the statistical way. There are only two pieces of math we need to know: {\em{Analysis}}, in particular measure-theoretical probability and differential equations; and {\em{Linear Algebra}}, with functional analysis in mind. Statistics, on the other hand, should start with {\em{Statistical Experiment Design}}, from which we learn how to collect data for statistical models. Next, the study of {\em{Random Variables}} and {\em{Stochastic Processes}} introduce the building blocks of the statistical ``pillbox'', with {\em{Mathematical Statistics}} the ``scaffold''. Once the ``pillbox'' is ready, we are equipped to tackle our problems using {\em{Machine Learning}}, which is essentially a collection of statistical models and optimization algorithms.

{\em{Computer Architecture}} and {\em{Operating System}} are respectively about the ``hardware'' and ``software'' of a single computer. The interaction of multiple computers is understood in {\em{Computer Network}}. Once we are comfortable with these concepts, we will be able to use {\em{Data Structure and Algorithms}} to solve problems at the {\bf{operational}} level, and use {\em{C++}} and/or {\em{Java}} to implement our ideas.

I am aware that it can take a while, and even multiple advanced degrees, to finish this curriculum, but let's remember the motto from the Leipzig Gewandhaus Orchestra: ``{\bf{\em{Res severa est verum gaudium}}}''.

\vspace{8mm}
\hfill {\em{Xi Tan}}

\hfill {\em{West Lafayette, IN}}

\hfill {\em{October, 2013}}