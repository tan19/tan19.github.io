\documentclass{book}
\usepackage{amssymb,amsmath,mathtools}


\usepackage{algorithm2e,algorithmic}

\usepackage{mathrsfs}
\usepackage{paralist}

\usepackage{esint} % for \fint

\allowdisplaybreaks

\usepackage{color}		% enable color characters
\usepackage{graphicx} 	% insert image files
\usepackage{enumerate} 	% enumerate items
\usepackage{caption}
\usepackage{subcaption}
\usepackage{multirow,multicol}
\usepackage[makeroom]{cancel}

\usepackage[colorlinks=true,linkcolor=blue,citecolor=blue]{hyperref}

\usepackage{makeidx}

\newcommand{\cem}[1]{\color{magenta}{\em{#1}}} % color emphasize
\newcommand{\dual}[2]{{#1}^{(#2)}} % color emphasize

\newcommand{\tr}{{\mathrm{tr}}}
\renewcommand{\vec}{{\mathrm{vec}}}

\newcommand{\dd}{{\,\mathrm{d}\,}}
%%
%% horizontal and vertical centering in table p mode
%%
\usepackage{array}
\newcolumntype{P}[1]{>{\centering\arraybackslash}p{#1}} % horizontal centering
\newcolumntype{M}[1]{>{\centering\arraybackslash}m{#1}} % vertical centering

%%
%% define bold font for the alphabet
%%
\usepackage{pgffor}
\foreach \letter in {a,...,z}{ % bold font for a..z
\expandafter\xdef\csname \letter \endcsname{\noexpand\ensuremath{\noexpand\mathbf{\letter}}}
}
\foreach \letter in {A,...,Z}{ % bold font for A..Z
\expandafter\xdef\csname \letter \endcsname{\noexpand\ensuremath{\noexpand\mathbf{\letter}}}
}
\foreach \letter in {A,...,Z}{ % `field' font for AA..ZZ
\expandafter\xdef\csname \letter\letter \endcsname{\noexpand\ensuremath{\noexpand\mathcal{\letter}}}
}
\foreach \letter in {A,...,Z}{ % `field' font for AAA..ZZZ
\expandafter\xdef\csname \letter\letter\letter \endcsname{\noexpand\ensuremath{\noexpand\mathbb{\letter}}}
}
\newcommand{\balpha}{{\boldsymbol{\alpha}}}
\newcommand{\bbeta}{{\boldsymbol{\beta}}}
\newcommand{\bgamma}{{\boldsymbol{\gamma}}}
\newcommand{\bkappa}{{\boldsymbol{\kappa}}}
\newcommand{\bmu}{{\boldsymbol{\mu}}}
\newcommand{\btheta}{{\boldsymbol{\theta}}}
\newcommand{\bTheta}{{\boldsymbol{\Theta}}}
\newcommand{\bPi}{{\boldsymbol{\Pi}}}
\newcommand{\bSigma}{{\boldsymbol{\Sigma}}}
\newcommand{\bPhi}{{\boldsymbol{\Phi}}}
\newcommand{\bLambda}{{\boldsymbol{\Lambda}}}
\newcommand{\bdeta}{{\boldsymbol{\eta}}}
\newcommand{\bphi}{{\boldsymbol{\phi}}}



%%
%% add definitions and theorems
%%
\usepackage[thmmarks,amsmath]{ntheorem}
\theorembodyfont{\normalfont}
\newtheorem{deff}{Definition}[section]
\newtheorem{thm}{Theorem}[section]
\newtheorem{prop}{Proposition}[section]
\newtheorem{lem}{Lemma}[section]
\newtheorem{cor}{Corollary}[section]
\newtheorem{rmk}{Remark}[section]
\newtheorem{alg}{Algorithm}[section]
\newtheorem{ex}{Example}[section]
\newtheorem{ques}{Question}[section]
\newtheorem{ans}{Answer}[section]
\newtheorem{prob}{Problem}[section]
\newtheorem{sol}{Solution}[section]
\newtheorem*{prof}{Proof}[section] 

\title{Quantitative Finance}
\author{Xi Tan (xtan3.1415926@gmail.com)}

\makeindex

\begin{document}

\maketitle
\tableofcontents
\chapter*{Preface}
This book project, which consists of four subjects: Finance, Mathematics, Statistics, and Computer Science, is tailored specifically to prepare someone for a quant career. It originated from my general belief of the hierarchy of solving a problem --- problems are solved at strategic, tactical, and operational levels.

{\em{Microeconomics}} and {\em{Macroeconomics}} explain the driving forces of capital markets, from a legislator's perspective. {\em{Accounting}} and {\em{Corporate Finance}} take a closer and necessary look at these forces, from a different angle. {\em{Stochastic Calculus}} and {\em{Asset Pricing}} provide with a set of tools and ideas that enables us to {\bf{strategically}} model one of the central problems in Quantitative Finance.

Generally speaking, there are two paths to solve a quantitative finance problem at the {\bf{tactical}} level: the mathematical way and the statistical way. There are only two pieces of math we need to know: {\em{Analysis}}, in particular measure-theoretical probability and differential equations; and {\em{Linear Algebra}}, with functional analysis in mind. Statistics, on the other hand, should start with {\em{Statistical Experiment Design}}, from which we learn how to collect data for statistical models. Next, the study of {\em{Random Variables}} and {\em{Stochastic Processes}} introduce the building blocks of the statistical ``pillbox'', with {\em{Mathematical Statistics}} the ``scaffold''. Once the ``pillbox'' is ready, we are equipped to tackle our problems using {\em{Machine Learning}}, which is essentially a collection of statistical models and optimization algorithms.

{\em{Computer Architecture}} and {\em{Operating System}} are respectively about the ``hardware'' and ``software'' of a single computer. The interaction of multiple computers is understood in {\em{Computer Network}}. Once we are comfortable with these concepts, we will be able to use {\em{Data Structure and Algorithms}} to solve problems at the {\bf{operational}} level, and use {\em{C++}} and/or {\em{Java}} to implement our ideas.

I am aware that it can take a while, and even multiple advanced degrees, to finish this curriculum, but let's remember the motto from the Leipzig Gewandhaus Orchestra: ``{\bf{\em{Res severa est verum gaudium}}}''.

\vspace{8mm}
\hfill {\em{Xi Tan}}

\hfill {\em{West Lafayette, IN}}

\hfill {\em{October, 2013}}

\addcontentsline{toc}{chapter}{Preface}

\part{Background}
\chapter{Economical Finance}
\section{Capital Market Overview}
\section{Trading and Exchanges}
\section{Macroeconomic Environment}
\section{Classic Finance Theory}
\subsection{Time Value of Money}
\subsection{Capital Asset Pricing Model (CAPM)}


\chapter{Mathematical Finance}
\section{Probability Theory}
\subsection{Probability Space}
\subsection{Information and $\sigma$-Algebra}
\subsection{Conditional Expectation}
\subsection{Martingales}
\subsection{Change of Measure and Girsanov's Theorem}

\section{Brownian Motion and Stochastic Calculus}
\subsection{Stochastic Processes}
\subsection{Brownian Motion}
\subsection{Geometric Brownian Motion}
\subsection{It\^o Integral and It\^o Process}
\subsection{Function of It\^o Processes and It\^o's Lemma}

\section{Stochastic Differential Equations}
\subsection{The Feynman--Kac Formula}
\subsection{Kolmogorov Forward and Backward Equations}
\subsection{Explicitly Solvable Stochastic Differential Equations}
\subsection{Backward Induction (BI)}
\subsection{Autonomous System}
\subsection{Milstein Method}
\subsection{Euler--Maruyama Method}
\subsection{Runge--Kutta Method}

\section{Probability Distribution}
\subsection{Poisson Distribution}
\subsection{Gaussian Distribution}
\subsection{Lognormal Distribution}
\subsection{$\chi^2$ Distribution}


\chapter{Statistical Finance}
\section{Regression}
\section{Classification}
\section{Machine Learning}
\section{Monte Carlo Simulation Methods}
\section{Moment Matching Methods}
\section{Copula Methods}

\chapter{Computational Finance}
\section{Linear Algebra}
\section{Numerical Analysis}
\section{Interpolation Methods}
\section{Root-finding Methods}
\subsection{The Bisection Method}
\subsection{The Newton--Raphson Method}
\subsection{The Secant Method}
\section{Optimization Methods}
\subsection{Linear Optimization}
\subsection{Non-linear Optimization}
\section{Software Engineering}


\part{Pricing, Investment, and Risk Management}
\chapter{Financial Modeling}
\section{Overview of (Arbitrage) Pricing Strategies}
\section{The Black--Scholes Model and Its Variants}
\subsection{The Black--Scholes Model}
\subsection{The Back--Scholes Model with Constant Dividend Yield (BS-D)}
\subsection{The Black's Model}
\section{Forward Models}
\subsection{Funding Spread Models}
\subsection{Dividend Models}
\section{Short Rate Models}
\subsection{Overview}
\subsection{Ornstein--Uhlenbeck Process}
\subsection{Square-root Process}
\section{Volatility Models}
\subsection{Implied Volatility Surface}
\subsection{Local Volatility Model}
\begin{align}
	C(K, T; S_0) &= 
\end{align}
Differentiating this twice with respect to $K$ to obtain
\begin{align}
	\frac{\partial C^2}{\partial K^2} &= \frac{1}{\partial K^2} \left[\int_K^\infty \dd S_T \phi(S_T, T; S_0) S_T - K \int_K^\infty \dd S_T \phi(S_T, T; S_0) \right]\\
	&= \frac{1}{\partial K} \left[K \phi(K, T; S_0) K - \int_K^\infty \dd S_T \phi(S_T, T; S_0) - K K \phi(K, T; S_0) \right]\\
	&= - \frac{1}{\partial K} \left[ \int_K^\infty \dd S_T \phi(S_T, T; S_0)\right]\\
	&= - K \phi(K, T; S_0) \\
\end{align}


\subsection{Stochastic Volatility Models}
\section{Jump Models}



\chapter{Investment Management}
\section{Asset Classes}
\subsection{Fixed Income}
\subsection{Equities}
\subsection{Foreign Exchange}
\subsection{Credit Derivatives}
\subsection{Interest Rate Derivatives}

\chapter{Risk Management}
\section{Risk Management Overview}
\subsection{Short-term Risk Management: Politics, Macroeconomics, Fundamental Analysis}
\subsection{Mid-term Risk Management: Technical Analysis, Sensitivity Analysis}
\subsection{Short-term Risk Management}

\section{Option Greeks}
\subsection{Delta}
\index{Delta}
\subsection{Gamma}
\index{Gamma}
\subsection{Vega}
\index{Vega}
\subsection{Theta}
\index{Theta}
\subsection{Hedging}

\section{Correlation and Skew}
\subsection{Implied Volatility Surface}
\subsection{Correlation}
\subsection{Skew}

\section{Term Structure, Duration, and Convexity}
\subsection{Term Structure of Interest Rate}
\subsection{Duration}
\subsection{Convexity}

\part{Miscellaneous}
\chapter{Study Notes}




\begin{thebibliography}{100} % 100 is a random guess of the total number of %references
\bibitem{Stewart} James Stewart {\em{Calsulus - Early Transcendentals}}. Cengage Learning, 2012
\bibitem{Rudin} Walter Rudin {\em{Principles of Mathematical Analysis}}. McGraw-Hill Companies, Inc., 1976.
\bibitem{Royden} H. L. Royden {\em{Real Analysis}}. Pearson Eduction, Inc., 1988.
\bibitem{Kreyszig} Erwin Kreyszig {\em{Introductory Functional Analysis with Applications}}. Wiley, 1989.
\bibitem{Folland} Gerald B. Folland {\em{Real Analysis: Modern Techniques and Their Applications}}. Wiley, 1999.
\bibitem{Torchinsky} Alberto Torchinsky {\em{Real Variables}}. Westview Press, 1995.

\bibitem{Wasserman} {\url{http://normaldeviate.wordpress.com/2012/11/17/what-is-bayesianfrequentist-inference/}}
\bibitem{Hochster} {\url{http://www.quora.com/What-is-the-difference-between-Bayesian-and-frequentist-statisticians}}
\bibitem{bayesian-inference advantage} {\url{http://www.bayesian-inference.com/advantagesbayesian}}
\bibitem{bayesian-inference likelihood} {\url{http://www.bayesian-inference.com/likelihood#likelihoodprinciple}}
\bibitem{Rossi} Rossi P, Allenby G, McCulloch R. {\emph{Bayesian Statistics and Marketing (pp. 4)}}. John Wiley \& Sons, 2005.
\bibitem{Efron 1978} Efron, Bradley. {\emph{Controversies in the Foundations of Statistics}}. The American Mathematical Monthly, Vol. 85, No. 4 (Apr., 1978), pp. 231-246.
\bibitem{Efron 2013} Efron, Bradley. {\emph{A 250-year Argument: Belief, Behavior, and the Bootstrap}}. Bull. Amer. Math. Soc. 50 (2013), 129-146.
\bibitem{quora CI} {\url{http://www.quora.com/Statistics-academic-discipline/What-is-a-confidence-interval-in-laymans-terms}}
\bibitem{quora diff} {\url{http://www.quora.com/What-is-the-difference-between-Bayesian-and-frequentist-statisticians}}
\bibitem{wiki} {\url{http://en.wikipedia.org/wiki/Confidence_interval#Meaning_and_interpretation}}

\bibitem{Minka} Thomas P. Minka. {\em{Old and New Matrix Algebra Useful for Statistics}}. December 28, 2000.
\bibitem{Wikepedia} \url{http://en.wikipedia.org/wiki/Matrix_calculus}. Accessed on \today
\bibitem{Searle} S. R. Searle and H. V. Henderson. {\em{A Primer on Differential Calculus for Vectors and Matrices}}. BU-1047-MB, 1993.
\bibitem{Nydick} Steven W. Nydick. {\em{A Different(ial) Way Matrix Derivatives Again}}. May 17, 2012.
\bibitem{Nydick} Steven W. Nydick. {\em{With(out) A Trace Matrix Derivatives the Easy Way}}. May 16, 2012.
\bibitem{Roweis} Sam Roweis. {\em{Matrix Identities}}. June 1999.
\bibitem{Tao} Terry Tao. {\em{Matrix identities as derivatives of determinant identities}}. January 13, 2013
\end{thebibliography}

\printindex

\end{document}
