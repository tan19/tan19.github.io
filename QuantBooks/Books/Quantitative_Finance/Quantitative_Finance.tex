\chapter{Economical Finance}
\section{Capital Market Overview}
\section{Trading and Exchanges}
\section{Macroeconomic Environment}
\section{Classic Finance Theory}
\subsection{Time Value of Money}
\subsection{Capital Asset Pricing Model (CAPM)}


\chapter{Mathematical Finance}
\section{Probability Theory}
\subsection{Probability Space}
\subsection{Information and $\sigma$-Algebra}
\subsection{Conditional Expectation}
\subsection{Martingales}
\subsection{Change of Measure and Girsanov's Theorem}

\section{Brownian Motion and Stochastic Calculus}
\subsection{Stochastic Processes}
\subsection{Brownian Motion}
\subsection{Geometric Brownian Motion}
\subsection{It\^o Integral and It\^o Process}
\subsection{Function of It\^o Processes and It\^o's Lemma}

\section{Stochastic Differential Equations}
\subsection{The Feynman--Kac Formula}
\subsection{Kolmogorov Forward and Backward Equations}
Start with the SDE defined by
\begin{align}
  \dd X_t & = \mu(X_t)\dd t + \sigma(X_t) \dd W_t.
\end{align}
The transition density $\rho(x,t|y,s)$ is defined by
\begin{align}
  \int_A \rho(x,t|y,s) \dd x & = \PPP[X_{t+s} \in A | X_s = y] = \PPP[X_t \in A | X_0 = y].
\end{align}
The density $\rho(x, t | y, s)$ is time-invariance since $\mu(X_t)$ and $\sigma(X_t)$ are assumed to be time invariance, and consequently, that $X_t$ is assumed to be stationary.

Consider a differentiable function $V(X_t, t) = V(x, t)$ with $V(X_t, t) = 0$ for $t \notin (0,T)$. Then by It\^o's lemma
\begin{align}
  \dd V & = \left[\frac{\partial V}{\partial t} + \mu\frac{\partial V}{\partial x} + \frac{1}{2}\sigma^2 \frac{\partial^2 V}{\partial x^2}\right]\dd t + \left[\sigma \frac{\partial V}{\partial x}\right]\dd W_t
\end{align}
so that
\begin{align}
  V(X_T, T) - V(X_0, 0) & = \int_0^T \left[\frac{\partial V}{\partial t} + \mu\frac{\partial V}{\partial x} + \frac{1}{2}\sigma^2 \frac{\partial^2 V}{\partial x^2}\right]\dd t + \int_0^T \left[\sigma \frac{\partial V}{\partial x}\right]\dd W_t
\end{align}
where $\mu = \mu(X_t)$ and $\sigma = \sigma(X_t)$ for notational convenience. Take the conditional expectation of both sides of the above equation given $X_0$
\begin{align}
    \EEE[V(X_T, T) - V(X_0, 0)] & = \EEE\left[\int_0^T \left[\frac{\partial V}{\partial t} + \mu\frac{\partial V}{\partial x} + \frac{1}{2}\sigma^2 \frac{\partial^2 V}{\partial x^2}\right]\dd t\right]\nonumber \\
    &= \int_\RRR \left\{\int_0^T \left[\frac{\partial V}{\partial t} + \mu\frac{\partial V}{\partial x} + \frac{1}{2}\sigma^2 \frac{\partial^2 V}{\partial x^2}\right]\dd t\right\}\rho(x,t|y,s) \dd x
\end{align}
We write this equation as the sum of three integrals:
\begin{align}
    I_1 &\equiv \int_\RRR \int_0^T \rho \frac{\partial V}{\partial t} \dd t \dd x \\
    I_2 &\equiv \int_\RRR \int_0^T \rho \mu \frac{\partial V}{\partial x} \dd t \dd x \\
    I_3 &\equiv \frac{1}{2}\int_\RRR \int_0^T \rho \sigma^2 \frac{\partial^2 V}{\partial x^2} \dd t \dd x
\end{align}
The objective of the derivation is to apply integration by parts to get rid of the derivatives of $V$.

\subsubsection{Evaluation of the Integrals}
The trick is that $I_1$ is evaluated using integration by parts on $t$, while $I_2$ and $I_3$ are each evaluated using integration by parts on $x$.

\paragraph{Evaluation of $I_1$}
\begin{align}
  I_1 & = \int_\RRR \left[\int_0^T \rho \frac{\partial V}{\partial t} \dd t \right] \dd x = \int_\RRR \left[\rho V |_0^T -\int_0^T \frac{\partial \rho}{\partial t} V \dd t\right]\dd x  = -\int_\RRR \int_0^T \frac{\partial \rho}{\partial t} V \dd t \dd x
\end{align}
where we have used the fact that at boundaries $0$ and $T$, $V = 0$.

\paragraph{Evaluation of $I_2$}
\begin{align}
  I_2 & = \int_0^T \left[\int_\RRR \rho \mu \frac{\partial V}{\partial x} \dd x \right] \dd t = \int_0^T\left[\rho\mu V|_\RRR - \int_\RRR V \frac{\partial(\rho\mu)}{\partial x} \dd x\right] \dd t = -\int_\RRR \int_0^T \frac{\partial (\rho \mu)}{\partial x} V \dd t \dd x
\end{align}
where again we have used the fact that $\rho(X_t = \pm \infty, t | X_0, 0) = 0$.

\paragraph{Evaluation of $I_3$}
\begin{align}
  I_3 & = \frac{1}{2} \int_0^T \left[\int_\RRR \rho\sigma^2 \frac{\partial^2 V}{\partial x^2} \dd x\right] \dd t = \frac{1}{2}\int_0^T \left[\rho\sigma^2\frac{\partial V}{\partial x}\bigg\vert_\RRR - \int_\RRR \frac{\partial V}{\partial x} \frac{\partial (\rho \sigma^2)}{\partial x} \dd x\right] \dd t \nonumber\\
  &= -\frac{1}{2}\int_0^T \left[\int_\RRR \frac{\partial V}{\partial x}\frac{\partial(\rho\sigma^2)}{\partial x} \dd x\right] \dd t = -\frac{1}{2}\int_0^T \left[\frac{\partial(\rho \sigma^2)}{\partial x}V \bigg\vert_\RRR - \int_\RRR \frac{\partial^2 (\rho\sigma^2)}{\partial x^2} V \dd x\right] \dd t \nonumber\\
  &= \frac{1}{2}\int_\RRR\int_0^T \frac{\partial^2(\rho\sigma^2)}{\partial x^2}V \dd t \dd x
\end{align}

Since $\EEE[V(X_T,T) - V(X_0,0)] \equiv 0$, we have
\begin{align}
   \int_\RRR \int_0^T \frac{\partial \rho}{\partial t} V \dd t \dd x + \int_\RRR \int_0^T \frac{\partial (\rho \mu)}{\partial x} V \dd t \dd x = \frac{1}{2}\int_\RRR\int_0^T \frac{\partial^2(\rho\sigma^2)}{\partial x^2}V \dd t \dd x
\end{align}
and
\begin{align}
  \frac{\partial \rho}{\partial t} = - \frac{\partial(\rho \mu)}{\partial x} + \frac{1}{2} \frac{\partial^2(\rho \sigma^2)}{\partial x^2}
\end{align}
Suppose the dynamics of $X_t$ is instead specified by a geometric Brownian motion,
\begin{align}
  \dd X_t & \mu(X_t, t)X_t \dd t + \sigma(X_t, t)X_t \dd W_t
\end{align}
Then
\begin{align}\label{Fokker--Planck Forward Equation}
  \frac{\partial \rho}{\partial t} = - \frac{\partial(\rho \mu x)}{\partial x} + \frac{1}{2} \frac{\partial^2(\rho \sigma^2 x^2)}{\partial x^2}
\end{align}
This is the Fokker--Planck forward equation\index{Fokker--Planck forward equation} \cite{Rouah, Harrison05}.



\subsection{Explicitly Solvable Stochastic Differential Equations}
\subsection{Backward Induction (BI)}
\subsection{Autonomous System}
\subsection{Milstein Method}
\subsection{Euler--Maruyama Method}
\subsection{Runge--Kutta Method}

\section{Probability Distribution}
\subsection{Poisson Distribution}
\subsection{Gaussian Distribution}
\subsection{Lognormal Distribution}
\subsection{$\chi^2$ Distribution}


\chapter{Statistical Finance}
\section{Regression}
\section{Classification}
\section{Machine Learning}
\section{Monte Carlo Simulation Methods}
\section{Moment Matching Methods}
\section{Copula Methods}

\chapter{Computational Finance}
\section{Linear Algebra}
\section{Numerical Analysis}
\section{Interpolation Methods}
\section{Root-finding Methods}
\subsection{The Bisection Method}
\subsection{The Newton--Raphson Method}
\subsection{The Secant Method}
\section{Optimization Methods}
\subsection{Linear Optimization}
\subsection{Non-linear Optimization}
\section{Software Engineering}

\chapter{Financial Modeling}
\section{Overview of (Arbitrage) Pricing Strategies}
\section{The Black--Scholes Model and Its Variants}
\subsection{The Black--Scholes Model}
\subsection{The Back--Scholes Model with Constant Dividend Yield (BS-D)}
\subsection{The Black's Model}
\section{Forward Models}
\subsection{Funding Spread Models}
\subsection{Dividend Models}
\section{Short Rate Models}
\subsection{Overview}
\subsection{Ornstein--Uhlenbeck Process}
\subsection{Square-root Process}
\section{Volatility Models}
\subsection{Implied Volatility Surface}
\subsection{Local Volatility Model}
\begin{align}
	C(K, T; S_0) &= \int_K^\infty \phi(S_T, T; S_0) [S_T - K] \dd S_T
\end{align}
Differentiating this twice with respect to $K$ to obtain
\begin{align}
	\frac{\partial C^2}{\partial K^2} &= \frac{1}{\partial K^2} \left[\int_K^\infty \phi(S_T, T; S_0) S_T \dd S_T  - K \int_K^\infty \phi(S_T, T; S_0) \dd S_T \right] \nonumber \\
	&= \frac{1}{\partial K} \left[- K \phi(K, T; S_0) - \int_K^\infty \phi(S_T, T; S_0) \dd S_T + K \phi(K, T; S_0) \right] \nonumber \\
	&= - \frac{1}{\partial K} \left[ \int_K^\infty \phi(S_T, T; S_0) \dd S_T \right] \nonumber \\
    &= \phi(K, T; S_0)
\end{align}
and differentiating this with respect to $T$ we obtain
\begin{align}
  \frac{\partial C}{\partial T} & = \int_{K}^{\infty} \left[\frac{\partial}{\partial T} \phi(S_T, T;S_0) \right] (S_T - K) \dd S_T \nonumber \\
\end{align}


\subsection{Stochastic Volatility Models}
\section{Jump Models}



\chapter{Investment Management}
\section{Asset Classes}
\subsection{Fixed Income}
\subsection{Equities}
\subsection{Foreign Exchange}
\subsection{Credit Derivatives}
\subsection{Interest Rate Derivatives}

\chapter{Risk Management}
\section{Risk Management Overview}
\subsection{Short-term Risk Management: Politics, Macroeconomics, Fundamental Analysis}
\subsection{Mid-term Risk Management: Technical Analysis, Sensitivity Analysis}
\subsection{Short-term Risk Management}

\section{Option Greeks}
\subsection{Delta}
\index{Delta}
\subsection{Gamma}
\index{Gamma}
\subsection{Vega}
\index{Vega}
\subsection{Theta}
\index{Theta}
\subsection{Hedging}

\section{Correlation and Skew}
\subsection{Implied Volatility Surface}
\subsection{Correlation}
\subsection{Skew}

\section{Term Structure, Duration, and Convexity}
\subsection{Term Structure of Interest Rate}
\subsection{Duration}
\subsection{Convexity}