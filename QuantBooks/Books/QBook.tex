\documentclass{book}
\usepackage{amssymb,amsmath,mathtools}


\usepackage{algorithm2e,algorithmic}

\usepackage{mathrsfs}
\usepackage{paralist}

\usepackage{esint} % for \fint

\allowdisplaybreaks

\usepackage{color}		% enable color characters
\usepackage{graphicx} 	% insert image files
\usepackage{enumerate} 	% enumerate items
\usepackage{caption}
\usepackage{subcaption}
\usepackage{multirow,multicol}
\usepackage[makeroom]{cancel}

\usepackage[colorlinks=true,linkcolor=blue,citecolor=blue]{hyperref}

\usepackage{makeidx}

\newcommand{\cem}[1]{\color{magenta}{\em{#1}}} % color emphasize
\newcommand{\dual}[2]{{#1}^{(#2)}} % color emphasize

\newcommand{\tr}{{\mathrm{tr}}}
\renewcommand{\vec}{{\mathrm{vec}}}

\newcommand{\dd}{{\,\text{d}\,}}
%%
%% horizontal and vertical centering in table p mode
%%
\usepackage{array}
\newcolumntype{P}[1]{>{\centering\arraybackslash}p{#1}} % horizontal centering
\newcolumntype{M}[1]{>{\centering\arraybackslash}m{#1}} % vertical centering

%%
%% define bold font for the alphabet
%%
\usepackage{pgffor}
\foreach \letter in {a,...,z}{ % bold font for a..z
\expandafter\xdef\csname \letter \endcsname{\noexpand\ensuremath{\noexpand\mathbf{\letter}}}
}
\foreach \letter in {A,...,Z}{ % bold font for A..Z
\expandafter\xdef\csname \letter \endcsname{\noexpand\ensuremath{\noexpand\mathbf{\letter}}}
}
\foreach \letter in {A,...,Z}{ % `field' font for AA..ZZ
\expandafter\xdef\csname \letter\letter \endcsname{\noexpand\ensuremath{\noexpand\mathcal{\letter}}}
}
\foreach \letter in {A,...,Z}{ % `field' font for AAA..ZZZ
\expandafter\xdef\csname \letter\letter\letter \endcsname{\noexpand\ensuremath{\noexpand\mathbb{\letter}}}
}
\newcommand{\balpha}{{\boldsymbol{\alpha}}}
\newcommand{\bbeta}{{\boldsymbol{\beta}}}
\newcommand{\bgamma}{{\boldsymbol{\gamma}}}
\newcommand{\bkappa}{{\boldsymbol{\kappa}}}
\newcommand{\bmu}{{\boldsymbol{\mu}}}
\newcommand{\btheta}{{\boldsymbol{\theta}}}
\newcommand{\bTheta}{{\boldsymbol{\Theta}}}
\newcommand{\bPi}{{\boldsymbol{\Pi}}}
\newcommand{\bSigma}{{\boldsymbol{\Sigma}}}
\newcommand{\bPhi}{{\boldsymbol{\Phi}}}
\newcommand{\bLambda}{{\boldsymbol{\Lambda}}}
\newcommand{\bdeta}{{\boldsymbol{\eta}}}
\newcommand{\bphi}{{\boldsymbol{\phi}}}



%%
%% add definitions and theorems
%%
\usepackage[thmmarks,amsmath]{ntheorem}
\theorembodyfont{\normalfont}
\newtheorem{deff}{Definition}[section]
\newtheorem{thm}{Theorem}[section]
\newtheorem{prop}{Proposition}[section]
\newtheorem{lem}{Lemma}[section]
\newtheorem{cor}{Corollary}[section]
\newtheorem{rmk}{Remark}[section]
\newtheorem{alg}{Algorithm}[section]
\newtheorem{ex}{Example}[section]
\newtheorem{ques}{Question}[section]
\newtheorem{ans}{Answer}[section]
\newtheorem{prob}{Problem}[section]
\newtheorem{sol}{Solution}[section]
\newtheorem*{prof}{Proof}[section] 

\title{Quantitative Finance Handbook}
\author{Xi Tan (xtan3.1415926@gmail.com)}

\makeindex

\begin{document}

\maketitle
\tableofcontents
\chapter*{Preface}
This book project, which consists of four subjects: Finance, Mathematics, Statistics, and Computer Science, is tailored specifically to prepare someone for a quant career. It originated from my general belief of the hierarchy of solving a problem --- problems are solved at strategic, tactical, and operational levels.

{\em{Microeconomics}} and {\em{Macroeconomics}} explain the driving forces of capital markets, from a legislator's perspective. {\em{Accounting}} and {\em{Corporate Finance}} take a closer and necessary look at these forces, from a different angle. {\em{Stochastic Calculus}} and {\em{Asset Pricing}} provide with a set of tools and ideas that enables us to {\bf{strategically}} model one of the central problems in Quantitative Finance.

Generally speaking, there are two paths to solve a quantitative finance problem at the {\bf{tactical}} level: the mathematical way and the statistical way. There are only two pieces of math we need to know: {\em{Analysis}}, in particular measure-theoretical probability and differential equations; and {\em{Linear Algebra}}, with functional analysis in mind. Statistics, on the other hand, should start with {\em{Statistical Experiment Design}}, from which we learn how to collect data for statistical models. Next, the study of {\em{Random Variables}} and {\em{Stochastic Processes}} introduce the building blocks of the statistical ``pillbox'', with {\em{Mathematical Statistics}} the ``scaffold''. Once the ``pillbox'' is ready, we are equipped to tackle our problems using {\em{Machine Learning}}, which is essentially a collection of statistical models and optimization algorithms.

{\em{Computer Architecture}} and {\em{Operating System}} are respectively about the ``hardware'' and ``software'' of a single computer. The interaction of multiple computers is understood in {\em{Computer Network}}. Once we are comfortable with these concepts, we will be able to use {\em{Data Structure and Algorithms}} to solve problems at the {\bf{operational}} level, and use {\em{C++}} and/or {\em{Java}} to implement our ideas.

I am aware that it can take a while, and even multiple advanced degrees, to finish this curriculum, but let's remember the motto from the Leipzig Gewandhaus Orchestra: ``{\bf{\em{Res severa est verum gaudium}}}''.

\vspace{8mm}
\hfill {\em{Xi Tan}}

\hfill {\em{West Lafayette, IN}}

\hfill {\em{October, 2013}}
\addcontentsline{toc}{chapter}{Preface}

\part{Finance and Economics}
\documentclass{book}
\usepackage{amssymb,amsmath,mathtools}


\usepackage{algorithm2e,algorithmic}

\usepackage{mathrsfs}
\usepackage{paralist}

\usepackage{esint} % for \fint

\allowdisplaybreaks

\usepackage{color}		% enable color characters
\usepackage{graphicx} 	% insert image files
\usepackage{enumerate} 	% enumerate items
\usepackage{caption}
\usepackage{subcaption}
\usepackage{multirow,multicol}
\usepackage[makeroom]{cancel}

\usepackage[colorlinks=true,linkcolor=blue,citecolor=blue]{hyperref}

\usepackage{makeidx}

\newcommand{\cem}[1]{\color{magenta}{\em{#1}}} % color emphasize
\newcommand{\dual}[2]{{#1}^{(#2)}} % color emphasize

\newcommand{\tr}{{\mathrm{tr}}}
\renewcommand{\vec}{{\mathrm{vec}}}

\newcommand{\dd}{{\,\text{d}\,}}
%%
%% horizontal and vertical centering in table p mode
%%
\usepackage{array}
\newcolumntype{P}[1]{>{\centering\arraybackslash}p{#1}} % horizontal centering
\newcolumntype{M}[1]{>{\centering\arraybackslash}m{#1}} % vertical centering

%%
%% define bold font for the alphabet
%%
\usepackage{pgffor}
\foreach \letter in {a,...,z}{ % bold font for a..z
\expandafter\xdef\csname \letter \endcsname{\noexpand\ensuremath{\noexpand\mathbf{\letter}}}
}
\foreach \letter in {A,...,Z}{ % bold font for A..Z
\expandafter\xdef\csname \letter \endcsname{\noexpand\ensuremath{\noexpand\mathbf{\letter}}}
}
\foreach \letter in {A,...,Z}{ % `field' font for AA..ZZ
\expandafter\xdef\csname \letter\letter \endcsname{\noexpand\ensuremath{\noexpand\mathcal{\letter}}}
}
\foreach \letter in {A,...,Z}{ % `field' font for AAA..ZZZ
\expandafter\xdef\csname \letter\letter\letter \endcsname{\noexpand\ensuremath{\noexpand\mathbb{\letter}}}
}
\newcommand{\balpha}{{\boldsymbol{\alpha}}}
\newcommand{\bbeta}{{\boldsymbol{\beta}}}
\newcommand{\bgamma}{{\boldsymbol{\gamma}}}
\newcommand{\bkappa}{{\boldsymbol{\kappa}}}
\newcommand{\bmu}{{\boldsymbol{\mu}}}
\newcommand{\btheta}{{\boldsymbol{\theta}}}
\newcommand{\bTheta}{{\boldsymbol{\Theta}}}
\newcommand{\bPi}{{\boldsymbol{\Pi}}}
\newcommand{\bSigma}{{\boldsymbol{\Sigma}}}
\newcommand{\bPhi}{{\boldsymbol{\Phi}}}
\newcommand{\bLambda}{{\boldsymbol{\Lambda}}}
\newcommand{\bdeta}{{\boldsymbol{\eta}}}
\newcommand{\bphi}{{\boldsymbol{\phi}}}



%%
%% add definitions and theorems
%%
\usepackage[thmmarks,amsmath]{ntheorem}
\theorembodyfont{\normalfont}
\newtheorem{deff}{Definition}[section]
\newtheorem{thm}{Theorem}[section]
\newtheorem{prop}{Proposition}[section]
\newtheorem{lem}{Lemma}[section]
\newtheorem{cor}{Corollary}[section]
\newtheorem{rmk}{Remark}[section]
\newtheorem{alg}{Algorithm}[section]
\newtheorem{ex}{Example}[section]
\newtheorem{ques}{Question}[section]
\newtheorem{ans}{Answer}[section]
\newtheorem{prob}{Problem}[section]
\newtheorem{sol}{Solution}[section]
\newtheorem*{prof}{Proof}[section] 

\title{Quantitative Finance}
\author{Xi Tan (xtan3.1415926@gmail.com)}

\makeindex

\begin{document}

\maketitle
\tableofcontents
\chapter*{Preface}
This book project, which consists of four subjects: Finance, Mathematics, Statistics, and Computer Science, is tailored specifically to prepare someone for a quant career. It originated from my general belief of the hierarchy of solving a problem --- problems are solved at strategic, tactical, and operational levels.

{\em{Microeconomics}} and {\em{Macroeconomics}} explain the driving forces of capital markets, from a legislator's perspective. {\em{Accounting}} and {\em{Corporate Finance}} take a closer and necessary look at these forces, from a different angle. {\em{Stochastic Calculus}} and {\em{Asset Pricing}} provide with a set of tools and ideas that enables us to {\bf{strategically}} model one of the central problems in Quantitative Finance.

Generally speaking, there are two paths to solve a quantitative finance problem at the {\bf{tactical}} level: the mathematical way and the statistical way. There are only two pieces of math we need to know: {\em{Analysis}}, in particular measure-theoretical probability and differential equations; and {\em{Linear Algebra}}, with functional analysis in mind. Statistics, on the other hand, should start with {\em{Statistical Experiment Design}}, from which we learn how to collect data for statistical models. Next, the study of {\em{Random Variables}} and {\em{Stochastic Processes}} introduce the building blocks of the statistical ``pillbox'', with {\em{Mathematical Statistics}} the ``scaffold''. Once the ``pillbox'' is ready, we are equipped to tackle our problems using {\em{Machine Learning}}, which is essentially a collection of statistical models and optimization algorithms.

{\em{Computer Architecture}} and {\em{Operating System}} are respectively about the ``hardware'' and ``software'' of a single computer. The interaction of multiple computers is understood in {\em{Computer Network}}. Once we are comfortable with these concepts, we will be able to use {\em{Data Structure and Algorithms}} to solve problems at the {\bf{operational}} level, and use {\em{C++}} and/or {\em{Java}} to implement our ideas.

I am aware that it can take a while, and even multiple advanced degrees, to finish this curriculum, but let's remember the motto from the Leipzig Gewandhaus Orchestra: ``{\bf{\em{Res severa est verum gaudium}}}''.

\vspace{8mm}
\hfill {\em{Xi Tan}}

\hfill {\em{West Lafayette, IN}}

\hfill {\em{October, 2013}}
\addcontentsline{toc}{chapter}{Preface}

\part{Background}
\chapter{Economical Finance}
\section{Capital Market Overview}
\section{Trading and Exchanges}
\section{Macroeconomic Environment}
\section{Classic Finance Theory}
\subsection{Time Value of Money}
\subsection{Capital Asset Pricing Model (CAPM)}


\chapter{Mathematical Finance}
\section{Probability Theory}
\subsection{Probability Space}
\subsection{Information and $\sigma$-Algebra}
\subsection{Conditional Expectation}
\subsection{Martingales}
\subsection{Change of Measure and Girsanov's Theorem}

\section{Brownian Motion and Stochastic Calculus}
\subsection{Stochastic Processes}
\subsection{Brownian Motion}
\subsection{Geometric Brownian Motion}
\subsection{It\^o Integral and It\^o Process}
\subsection{Function of It\^o Processes and It\^o's Lemma}

\section{Stochastic Differential Equations}
\subsection{The Feynman--Kac Formula}
\subsection{Kolmogorov Forward and Backward Equations}
\subsection{Explicitly Solvable Stochastic Differential Equations}
\subsection{Backward Induction (BI)}

\section{Probability Distribution}
\subsection{Poisson Distribution}
\subsection{Gaussian Distribution}
\subsection{Lognormal Distribution}
\subsection{$\chi^2$ Distribution}


\chapter{Statistical Finance}
\section{Regression}
\section{Classification}
\section{Machine Learning}
\section{Monte Carlo Simulation Methods}
\section{Moment Matching Methods}
\section{Copula Methods}

\chapter{Computational Finance}
\section{Linear Algebra}
\section{Numerical Analysis}
\section{Interpolation Methods}
\section{Root-finding Methods}
\subsection{The Bisection Method}
\subsection{The Newton--Raphson Method}
\subsection{The Secant Method}
\section{Optimization Methods}
\subsection{Linear Optimization}
\subsection{Non-linear Optimization}
\section{Software Engineering}


\part{Pricing, Investment, and Risk Management}
\chapter{Financial Modeling}
\section{Overview of (Arbitrage) Pricing Strategies}
\section{The Black--Scholes Model and Its Variants}
\subsection{The Black--Scholes Model}
\subsection{The Back--Scholes Model with Constant Dividend Yield (BS-D)}
\subsection{The Black's Model}
\section{Forward Models}
\subsection{Funding Spread Models}
\subsection{Dividend Models}
\section{Short Rate Models}
\subsection{Overview}
\subsection{Ornstein--Uhlenbeck Process}
\subsection{Square-root Process}
\section{Volatility Models}
\subsection{Implied Volatility Surface}
\subsection{Local Volatility Model}
\subsection{Stochastic Volatility Models}
\section{Jump Models}



\chapter{Investment Management}
\section{Asset Classes}
\subsection{Fixed Income}
\subsection{Equities}
\subsection{Foreign Exchange}
\subsection{Credit Derivatives}
\subsection{Interest Rate Derivatives}

\chapter{Risk Management}
\section{Risk Management Overview}
\subsection{Short-term Risk Management: Politics, Macroeconomics, Fundamental Analysis}
\subsection{Mid-term Risk Management: Technical Analysis, Sensitivity Analysis}
\subsection{Short-term Risk Management}

\section{Option Greeks}
\subsection{Delta}
\index{Delta}
\subsection{Gamma}
\index{Gamma}
\subsection{Vega}
\index{Vega}
\subsection{Theta}
\index{Theta}
\subsection{Hedging}

\section{Correlation and Skew}
\subsection{Implied Volatility Surface}
\subsection{Correlation}
\subsection{Skew}

\section{Term Structure, Duration, and Convexity}
\subsection{Term Structure of Interest Rate}
\subsection{Duration}
\subsection{Convexity}

\part{Miscellaneous}
\chapter{Study Notes}



\begin{thebibliography}{100} % 100 is a random guess of the total number of %references
\bibitem{Stewart} James Stewart {\em{Calsulus - Early Transcendentals}}. Cengage Learning, 2012
\bibitem{Rudin} Walter Rudin {\em{Principles of Mathematical Analysis}}. McGraw-Hill Companies, Inc., 1976.
\bibitem{Royden} H. L. Royden {\em{Real Analysis}}. Pearson Eduction, Inc., 1988.
\bibitem{Kreyszig} Erwin Kreyszig {\em{Introductory Functional Analysis with Applications}}. Wiley, 1989.
\bibitem{Folland} Gerald B. Folland {\em{Real Analysis: Modern Techniques and Their Applications}}. Wiley, 1999.
\bibitem{Torchinsky} Alberto Torchinsky {\em{Real Variables}}. Westview Press, 1995.

\bibitem{Wasserman} {\url{http://normaldeviate.wordpress.com/2012/11/17/what-is-bayesianfrequentist-inference/}}
\bibitem{Hochster} {\url{http://www.quora.com/What-is-the-difference-between-Bayesian-and-frequentist-statisticians}}
\bibitem{bayesian-inference advantage} {\url{http://www.bayesian-inference.com/advantagesbayesian}}
\bibitem{bayesian-inference likelihood} {\url{http://www.bayesian-inference.com/likelihood#likelihoodprinciple}}
\bibitem{Rossi} Rossi P, Allenby G, McCulloch R. {\emph{Bayesian Statistics and Marketing (pp. 4)}}. John Wiley \& Sons, 2005.
\bibitem{Efron 1978} Efron, Bradley. {\emph{Controversies in the Foundations of Statistics}}. The American Mathematical Monthly, Vol. 85, No. 4 (Apr., 1978), pp. 231-246.
\bibitem{Efron 2013} Efron, Bradley. {\emph{A 250-year Argument: Belief, Behavior, and the Bootstrap}}. Bull. Amer. Math. Soc. 50 (2013), 129-146.
\bibitem{quora CI} {\url{http://www.quora.com/Statistics-academic-discipline/What-is-a-confidence-interval-in-laymans-terms}}
\bibitem{quora diff} {\url{http://www.quora.com/What-is-the-difference-between-Bayesian-and-frequentist-statisticians}}
\bibitem{wiki} {\url{http://en.wikipedia.org/wiki/Confidence_interval#Meaning_and_interpretation}}

\bibitem{Minka} Thomas P. Minka. {\em{Old and New Matrix Algebra Useful for Statistics}}. December 28, 2000.
\bibitem{Wikepedia} \url{http://en.wikipedia.org/wiki/Matrix_calculus}. Accessed on \today
\bibitem{Searle} S. R. Searle and H. V. Henderson. {\em{A Primer on Differential Calculus for Vectors and Matrices}}. BU-1047-MB, 1993.
\bibitem{Nydick} Steven W. Nydick. {\em{A Different(ial) Way Matrix Derivatives Again}}. May 17, 2012.
\bibitem{Nydick} Steven W. Nydick. {\em{With(out) A Trace Matrix Derivatives the Easy Way}}. May 16, 2012.
\bibitem{Roweis} Sam Roweis. {\em{Matrix Identities}}. June 1999.
\bibitem{Tao} Terry Tao. {\em{Matrix identities as derivatives of determinant identities}}. January 13, 2013
\end{thebibliography}

\printindex

\end{document}


\part{Mathematics and Physics}
\documentclass{book}
\usepackage{amssymb,amsmath,mathtools}


\usepackage{algorithm2e,algorithmic}

\usepackage{mathrsfs}
\usepackage{paralist}

\usepackage{esint} % for \fint

\allowdisplaybreaks

\usepackage{color}		% enable color characters
\usepackage{graphicx} 	% insert image files
\usepackage{enumerate} 	% enumerate items
\usepackage{caption}
\usepackage{subcaption}
\usepackage{multirow,multicol}
\usepackage[makeroom]{cancel}

\usepackage[colorlinks=true,linkcolor=blue,citecolor=blue]{hyperref}

\usepackage{makeidx}

\newcommand{\cem}[1]{\color{magenta}{\em{#1}}} % color emphasize
\newcommand{\dual}[2]{{#1}^{(#2)}} % color emphasize

\newcommand{\tr}{{\mathrm{tr}}}
\renewcommand{\vec}{{\mathrm{vec}}}

\newcommand{\dd}{{\,\text{d}\,}}
%%
%% horizontal and vertical centering in table p mode
%%
\usepackage{array}
\newcolumntype{P}[1]{>{\centering\arraybackslash}p{#1}} % horizontal centering
\newcolumntype{M}[1]{>{\centering\arraybackslash}m{#1}} % vertical centering

%%
%% define bold font for the alphabet
%%
\usepackage{pgffor}
\foreach \letter in {a,...,z}{ % bold font for a..z
\expandafter\xdef\csname \letter \endcsname{\noexpand\ensuremath{\noexpand\mathbf{\letter}}}
}
\foreach \letter in {A,...,Z}{ % bold font for A..Z
\expandafter\xdef\csname \letter \endcsname{\noexpand\ensuremath{\noexpand\mathbf{\letter}}}
}
\foreach \letter in {A,...,Z}{ % `field' font for AA..ZZ
\expandafter\xdef\csname \letter\letter \endcsname{\noexpand\ensuremath{\noexpand\mathcal{\letter}}}
}
\foreach \letter in {A,...,Z}{ % `field' font for AAA..ZZZ
\expandafter\xdef\csname \letter\letter\letter \endcsname{\noexpand\ensuremath{\noexpand\mathbb{\letter}}}
}
\newcommand{\balpha}{{\boldsymbol{\alpha}}}
\newcommand{\bbeta}{{\boldsymbol{\beta}}}
\newcommand{\bgamma}{{\boldsymbol{\gamma}}}
\newcommand{\bkappa}{{\boldsymbol{\kappa}}}
\newcommand{\bmu}{{\boldsymbol{\mu}}}
\newcommand{\btheta}{{\boldsymbol{\theta}}}
\newcommand{\bTheta}{{\boldsymbol{\Theta}}}
\newcommand{\bPi}{{\boldsymbol{\Pi}}}
\newcommand{\bSigma}{{\boldsymbol{\Sigma}}}
\newcommand{\bPhi}{{\boldsymbol{\Phi}}}
\newcommand{\bLambda}{{\boldsymbol{\Lambda}}}
\newcommand{\bdeta}{{\boldsymbol{\eta}}}
\newcommand{\bphi}{{\boldsymbol{\phi}}}



%%
%% add definitions and theorems
%%
\usepackage[thmmarks,amsmath]{ntheorem}
\theorembodyfont{\normalfont}
\newtheorem{deff}{Definition}[section]
\newtheorem{thm}{Theorem}[section]
\newtheorem{prop}{Proposition}[section]
\newtheorem{lem}{Lemma}[section]
\newtheorem{cor}{Corollary}[section]
\newtheorem{rmk}{Remark}[section]
\newtheorem{alg}{Algorithm}[section]
\newtheorem{ex}{Example}[section]
\newtheorem{ques}{Question}[section]
\newtheorem{ans}{Answer}[section]
\newtheorem{prob}{Problem}[section]
\newtheorem{sol}{Solution}[section]
\newtheorem*{prof}{Proof}[section] 

\title{Calculus}
\author{Xi Tan (tan19@purdue.edu)}
\date{\today}

\begin{document}
\maketitle

\tableofcontents
\newpage

\section*{Preface}
This book reviews calculus, advanced calculus, real analysis, and functional Analysis. The main references to be used are \cite{Stewart} for calculus, \cite{Rudin} for advanced calculus, \cite{Royden} for real analysis, and \cite{Kreyszig} for functional analysis. Other useful texts include: \cite{Folland} and \cite{Torchinsky} for real analysis.
\newpage

\part{Calculus}
\chapter{Infinite Sequences and Series}
\section{Convergence Tests}
There are five common techniques to test whether or not an infinite series is convergent. But first of all, a necessary condition:
\thm{
	If the limit of the summand is undefined or nonzero, that is, $\lim \limits_{n \to \infty} a_n \ne 0$, then the series $\sum_{n=1}^\infty a_n$ must diverge.
}

\thm{
	{\bf{Comparison Test.}} If $\{a_n\},\{b_n\}>0$, and the limit $\lim \limits_{n \to \infty} \frac{a_n}{b_n}$ exists, is finite and is not zero, then $\sum_{n=1}^\infty a_n$ converges if and only if $\sum_{n=1}^\infty b_n$ converges.
}

\thm{
	{\bf{Integral Test.}} Let $f:[1,\infty) \to \R_+$ be a positive and monotone decreasing function such that $f(n) = a_n$. Then the series $\{a_n\}$ converges if and only if the integral $\int_1^\infty f(x) dx$ converges.
}

\thm{
	{\bf{Ratio Test.}} Suppose there exists $r$ such that $\lim \limits_{n \to \infty} |\frac{a_{n+1}}{a_n}| = r$. If $r < 1$, then the series converges. If $r > 1$, then the series diverges. If $r = 1$, the ratio test is inconclusive, and the series may converge or diverge.
}

\thm{
	{\bf{Root Test.}} Define $r = \limsup \limits_{n \to \infty} \sqrt[n]{|a_n|}$. If $r < 1$, then the series converges. If $r > 1$, then the series diverges. If $r = 1$, the ratio test is inconclusive, and the series may converge or diverge.
}

\thm{
	{\bf{Alternating Series Test.}} If the alternating series $\sum_{n=1}^\infty (-1)^{n-1}b_n, (b_n > 0)$ satisfies
	\begin{enumerate}
		\item $b_{n+1} \le b_n, ~ \mbox{for all $n$}$; and, 
		\item $\lim_{n \to \infty} b_n = 0$.
	\end{enumerate}
	Then the series is convergent.
}

\thm{
	A series is said to be absolutely convergent if $\sum_{i=1}^\infty |a_n|$ converges. Every absolutely convergent series is convergent. But not all convergent series are absolutely convergent. A convergent series that is not absolutely convergent is called conditionally convergent.
}

\chapter{Vectors and the Geometry of Space}
\section{Lines in $\RRR^n$}
\deff{
	Given a vector $\p \in \RRR^n$ and a nonzero vector $\v \in \RRR^n$, the set of all points $\y \in \RRR^n$ such that
	\begin{align}
		\y = t\v + \p, ~~~ t \in \RRR
	\end{align}
	is called the \emph{line} through $\p$ in the direction of $\v$.
}

\ex{
	The shortest distance from a point $\q \in \RRR^n$ to a line $L$ with equation $\y = t\v + \p$ is
	\begin{align}
		\left\|(\q-\p) - \frac{(\q-\p)^T\v}{\|\v\|^2} \v \right\|
	\end{align}
}

\section{Hyperplanes in $\RRR^n$}
\deff{
	Suppose $\n$ is a normal vector for a hyperplan $H$ through $\p \in \RRR^n$, then the normal equation for $H$ is
	\begin{align}
		\n^T(\y - \p) = 0
	\end{align}
	If $H$ is in $\RRR^3$, we can use cross-product $\times$ to obtain the normal vector given two vectors on the hyperplane.
}

\ex{
	The shortest distance from a point $\q \in \RRR^n$ to a hyperplane $H$ with equation $\n^T(\y - \p) = 0$ is
	\begin{align}
		\left| \frac{\n^T(\q-\p)}{\|\n\|} \right|
	\end{align}
}

A hyperplane is a set satisfies $H=\{x:w^Tx=b\}$. An equivalent form is $w^T(x-\frac{w}{\|w\|^2}b) = 0$, which suggests that the vector $w$ is perpendicular to the hyperplane, called a normal vector.

Particularly, since $\frac{w^Tw}{\|w\|^2}b = b$, we know $x_0=\frac{w}{\|w\|}\frac{b}{\|w\|}$ is on the hyperplane. The $x_0$ is actually the projection of the origin, since $w$ is orthogonal to the hyperplane and it is nothing but a scaled $w$ on the hyperplane. Therefore, the shortest distance (along the direction of $w$) from the origin to the hyperplane is given by $\frac{b}{\|w\|}$ (could be negative, which means $w$ is on the other side of the hyperplane).

In general, if a hyperplane is given by the equation $f(x) = w^Tx-b = 0$, the distance from any arbitrary vector $p$ to the hyperplane $w^Tx=b$ is given by
\begin{align}
	\frac{f(p)}{\|w\|} = \frac{w^Tp-b}{\|w\|}, ~~~ \mbox{if $p$ is on the opposite side of the origin}\\
	-\frac{f(p)}{\|w\|} = -\frac{w^Tp-b}{\|w\|}, ~~~ \mbox{if $p$ is on the same side of the origin}
\end{align}	

Particularly, when $p=0$ (the origin), the above becomes $\frac{b}{\|w\|}$, which agrees with our previous result.

Proof: Let's prove the first case. Suppose there is a vector $x$ on the hyperplane, such that $p-x = d\frac{w}{\|w\|}$. Since $w$ is orthogonal to the hyperplane, the scalar $d$ is the distance we are after. Now, multiply both sides by $w^T$,

$w^Tp-w^Tx = dw^T\frac{w}{\|w\|}$

$w^Tp-(w^Tx-b) = d\frac{w^Tw}{\|w\|}+b$

$w^Tp = d\|w\| + b$

$d = \frac{w^Tp-b}{\|w\|}$

The proof for the other case is similar.

$\blacksquare$

\end{document}

\chapter{Advanced Calculus}

\section{$\liminf$ and $\limsup$}
\deff{
	\begin{align}	
		\liminf\limits_n A_n &= \cup_{n=1}^\infty \cap_{i=n}^\infty A_i = \{x ~|~ x \in A_i ~\text{eventually}\}\\
		\limsup\limits_n A_n &= \cap_{n=1}^\infty \cup_{i=n}^\infty A_i = \{x ~|~ x \in A_i ~\text{for infinitely many}~ i\}
	\end{align}
}

The meaning of $\liminf$ can be seen by re-writing the above definition as: $x \in \liminf\limits_n A_n$ if $\exists n \in \NNN$, s.t. $\forall i \ge n$ and $i \in \NNN$, $x \in A_i$. Hence the elements in $\liminf\limits_n A_n$ are in all but (the first) finitely many sets, though the ``first finitely many sets'' may be different for different elements in $\liminf$. $\limsup$ can be best seen by examining its complement, according to the De Morgan's law.

\prop{
	\begin{align}
		(\limsup\limits_n A_n)^c &= \liminf\limits_n A_n^c\\
		\liminf A_k &\subset \limsup A_k\\
		\limsup(A_k \cup B_k) &= \limsup A_k \cup \limsup B_k\\
		\liminf(A_k \cap B_k) &= \liminf A_k \cap \liminf B_k
	\end{align}	
}

\section{Lines in $\RRR^n$}
\deff{
	Given a vector $\p \in \RRR^n$ and a nonzero vector $\v \in \RRR^n$, the set of all points $\y \in \RRR^n$ such that
	\begin{align}
		\y = t\v + \p, ~~~ t \in \RRR
	\end{align}
	is called the \emph{line} through $\p$ in the direction of $\v$.
}

\ex{
	The shortest distance from a point $\q \in \RRR^n$ to a line $L$ with equation $\y = t\v + \p$ is
	\begin{align}
		\left\|(\q-\p) - \frac{(\q-\p)^T\v}{\|\v\|^2} \v \right\|
	\end{align}
}

\section{Hyperplanes in $\RRR^n$}
\deff{
	Suppose $\n$ is a normal vector for a hyperplan $H$ through $\p \in \RRR^n$, then the normal equation for $H$ is
	\begin{align}
		\n^T(\y - \p) = 0
	\end{align}
	If $H$ is in $\RRR^3$, we can use cross-product $\times$ to obtain the normal vector given two vectors on the hyperplane.
}

\ex{
	The shortest distance from a point $\q \in \RRR^n$ to a hyperplane $H$ with equation $\n^T(\y - \p) = 0$ is
	\begin{align}
		\left| \frac{\n^T(\q-\p)}{\|\n\|} \right|
	\end{align}
}

A hyperplane is a set satisfies $H=\{x:w^Tx=b\}$. An equivalent form is $w^T(x-\frac{w}{\|w\|^2}b) = 0$, which suggests that the vector $w$ is perpendicular to the hyperplane, called a normal vector.

Particularly, since $\frac{w^Tw}{\|w\|^2}b = b$, we know $x_0=\frac{w}{\|w\|}\frac{b}{\|w\|}$ is on the hyperplane. The $x_0$ is actually the projection of the origin, since $w$ is orthogonal to the hyperplane and it is nothing but a scaled $w$ on the hyperplane. Therefore, the shortest distance (along the direction of $w$) from the origin to the hyperplane is given by $\frac{b}{\|w\|}$ (could be negative, which means $w$ is on the other side of the hyperplane).

In general, if a hyperplane is given by the equation $f(x) = w^Tx-b = 0$, the distance from any arbitrary vector $p$ to the hyperplane $w^Tx=b$ is given by
\begin{align}
	\frac{f(p)}{\|w\|} = \frac{w^Tp-b}{\|w\|}, ~~~ \mbox{if $p$ is on the opposite side of the origin}\\
	-\frac{f(p)}{\|w\|} = -\frac{w^Tp-b}{\|w\|}, ~~~ \mbox{if $p$ is on the same side of the origin}
\end{align}	

Particularly, when $p=0$ (the origin), the above becomes $\frac{b}{\|w\|}$, which agrees with our previous result.

Proof: Let's prove the first case. Suppose there is a vector $x$ on the hyperplane, such that $p-x = d\frac{w}{\|w\|}$. Since $w$ is orthogonal to the hyperplane, the scalar $d$ is the distance we are after. Now, multiply both sides by $w^T$,

$w^Tp-w^Tx = dw^T\frac{w}{\|w\|}$

$w^Tp-(w^Tx-b) = d\frac{w^Tw}{\|w\|}+b$

$w^Tp = d\|w\| + b$

$d = \frac{w^Tp-b}{\|w\|}$

The proof for the other case is similar.

$\blacksquare$

\chapter{The Real and Complex Number Systems}
\section{Introduction}
\subsection{Definitions}
\begin{definition}
	If $A$ is any set (whose elements may be numbers or any other objects), we write $x \in A$ to indicate that $x$ is a memeber (or an element) of $A$. If $x$ is not a memeber of $A$, we write: $x \notin A$.
\end{definition}

\begin{definition}
	Throughtout Chap. 1, the set of all rational numbers will be denoted by $Q$.
\end{definition}

\begin{definition}\label{def:supinf}
	Suppose $S$ is an ordered set, $E\subset S$, and $E$ is bounded above. Suppose there exists an $\alpha\in S$ with the following properties:
	\begin{enumerate}[(i)]
		\item $\alpha$ is an upper bound of $E$.
		\item If $\gamma <\alpha$ then $\gamma$ is not an upper bound of $E$.
	\end{enumerate}
	Then $\alpha$ is called the {\cem{least upper bound}} of $E$ [that there is at most one such $\alpha$ is clear from (ii)] or the {\cem{supremum}} of $E$, and we write $$\alpha=\sup E.$$ The {\cem{greatest lower bound}}, or {\cem{infimum}}, of a set $E$ which is bounded below is defined in the same manner: The statement $$\alpha = \inf E$$ means that $\alpha$ is a lower bound of $E$ and that no $\beta$ with $\beta>\alpha$ is a lower bound of $E$.
\end{definition}

\begin{definition}\label{def:infinity}
	The extended real number system consists of the real field $R$ and two symbols, $+\infty$ and $-\infty$. We preserve the original order in $R$, and define $$-\infty<x<+\infty$$ for every $x\in R$.
\end{definition}


\section{Basic Topology}

\section{Finite, Countable, and Uncountable Sets}

\subsection{Definitions}
\begin{deff}
	Consider two sets $A$ and $B$, whose elements may be any objects whatsoever, and suppose that {\underline{with each element $x$ of A}} there is associated, in some manner, an element of $B$, which we denote by $f(x)$. Then $f$ is said to be a {\cem{function}} from $A$ to $B$ (or a {\cem{mapping}} of $A$ into $B$). The set $A$ is called the {\cem{domain}} of $f$ (we also say $f$ is defined on $A$), and the elments $f(x)$ are called the {\cem{values}} of $f$. The set of all values of $f$ is called the {\cem{range}} of $f$.
\end{deff}

\begin{deff}\label{def: mapping}
	Let $A$ and $B$ be two sets and let $f$ be a mapping of $A$ into $B$. If $E \subset A$, $f(E)$ is defined to be the set of all elements $f(x)$, for $x \in E$. We call $f(E)$ the {\cem{image}} of $E$ under $f$. In this notation, $f(A)$ is the range of $f$. It is clear that $f(A) \subset B$. If $f(A) = B$, we say that $f$ maps $A$ {\cem{onto}} $B$. (Note that, according to this usage, {\cem{onto}} is more specific thatn {\cem{into}}.)
\end{deff}

\begin{deff}
	If $E \subset B$ ($E$ is not necessarily a subset of $f(A)$), $f^{-1}(E)$ denotes the set of all $x \in A$ such that $f(x) \in E$. We call $f^{-1}(E)$ the {\cem{inverse image}} of $E$ under $f$. If $y \in B$, $f^{-1}(y)$ is the set of all $x \in A$ such that $f(x) = y$. If, for each $y \in B$, $f^{-1}(y)$ consists of at most one element of $A$, then $f$ is said to be a 1-1 ({\cem{one-to-one}}) mapping of $A$ into $B$. This may also be expressed as follows. $f$ is a 1-1 mapping of $A$ into $B$ provided that $f(x_1) \ne f(x_2)$ whenever $x_1 \ne x_2, x_1 \in A, x_2 \in A$.
\end{deff}

\begin{deff}
	If there exists a 1-1 mapping of $A$ {\cem{onto}} $B$, we say that $A$ and $B$ can be put in {\cem{1-1 correspondence}}, or that $A$ and $B$ have the same {\cem{cardinal number}}, or, briefly, that $A$ and $B$ are {\cem{equivalent}}, and we write $A \sim B$. This relation clearly has the following properties:
	\begin{enumerate}[]
		\item It is reflexive: $A \sim A$
		\item It is symmetric: If $A \sim B$, then $B \sim A$
		\item It is transitive: If $A \sim B$ and $B \sim C$, then $A \sim C$
	\end{enumerate}
	Any relation with theses three properties is called an {\cem{equivalence relation}}.
\end{deff}

\begin{deff}
	For any positive integer $n$, let $J_n$ be the set whose elements are the integers $1,2,\cdots,n$; let $J$ be the set consisting of all positive integers. For any set $A$, we say:
	\begin{enumerate}[(a)]
		\item $A$ is {\cem{finite}} if $A \sim J_n$ for some $n$ (the empty set is also considered to be finite).
		\item $A$ is {\cem{infinite}} if $A$ is not finite.
		\item $A$ is {\cem{countable}} if $A \sim J$.
		\item $A$ is {\cem{uncountable}} if $A$ is neither finite nor countable.
		\item $A$ is {\cem{at most countable}} if $A$ is finite or countable.
	\end{enumerate}
	Countable sets are sometimes called {\cem{enumerable}} or {\cem{denumerable}}.
\end{deff}

\begin{deff}
	By a {\cem{sequence}}, we mean a function $f$ defined on the set $J$ of all positive integers. If $f(n) = x_n$, for $n \in J$, it is customary to denote the sequence $f$ by the symbol $\{x_n\}$, or sometimes by $x_1, x_2, x_3, \cdots$. The values of $f$, that is, the elements $x_n$, are called the {\cem{terms}} of the sequence. If $A$ is a set and if $x_n \in A$ for all $n \in J$, then $\{x_n\}$ is said to be a {\cem{sequence in $A$}}, or a {\cem{sequence of elements of $A$}}.
\end{deff}

\begin{deff}
	Let $A$ and $\Omega$ be sets, and suppose that with each element $\alpha$ of $A$ there is associated a subset of $\Omega$ which we denote by $E_\alpha$. The set whose elements are the sets $E_\alpha$ will be denoted by $\{E_\alpha\}$. Instead of speaking of sets of sets, we shall sometimes speak of a collection of sets, or a family of sets. The {\cem{union}} of the sets $E_\alpha$ is defined to be the set $S$ such that $x \in S$ if and only if $x \in E_\alpha$ for at least one $\alpha \in A$. We use the notation $$S = \bigcup\limits_{\alpha \in A} E_\alpha.$$ The {\cem{intersection}} of the sets $E_\alpha$ is defined to be the set $P$ such that $x \in P$ if and only if $x \in E_\alpha$ for every $\alpha \in A$. We use the notation $$P = \bigcap\limits_{\alpha \in A} E_\alpha.$$
\end{deff}

\subsection{Theorems}
\begin{thm}
	$A$ is infinite if and only if $A$ is equivalent to one of its {\underline{proper subsets}}.
\end{thm}

\begin{thm}
	Every infinite subset of a countable set $A$ is countable.
\end{thm}

\begin{thm}
	Let $\{E_n\}, n = 1, 2, 3, \cdots$, be a sequence of countable sets, and put $$S = \bigcup\limits_{n=1}^\infty E_n.$$ Then $S$ is countable.
\end{thm}

\begin{thm}
	Let $A$ be a countable set, and let $B_n$ be the set of all $n$-tuples $(a_1,\cdots,a_n)$, where $a_k \in A (k = 1, \cdots, n)$, and the elements $a_1, \cdots, a_n$ need not be distinct. Then $B_n$ is countable.
\end{thm}

\begin{cor}
	The set of all rational numbers is countable.
\end{cor}

\begin{thm}
	Let $A$ be the set of all sequences whose elements are the digits $0$ and $1$. This set $A$ is uncountable.
\end{thm}

\section{Metric Spaces}
\subsection{Definitions}
\begin{deff}
	A set $X$, whose elements we shall call {\cem{points}}, is said to be a {\cem{metric space}} if with any two points $p$ and $q$ of $X$ there is associated a real number $d(p,q)$, called the {\cem{distance}} from $p$ to $q$, such that
	\begin{enumerate}[(a)]
		\item $d(p,q)>0$ if $p \ne q$; $d(p,q) = 0$;
		\item $d(p,q)=d(q,p)$;
		\item $d(p,q)\le d(p,r) + d(r,q)$, for any $r \in X$.
	\end{enumerate}
	Any function with these three properties is called a {\cem{distance function}}, or a {\cem{metric}}.
\end{deff}

\begin{deff}
	~
	\begin{enumerate}[(a)]
	\item By the {\cem{segment}} $(a,b)$ we mean the set of all real numbers $x$ such that $a<x<b$.
	\item By the {\cem{interval}} $[a,b]$ we mean the set of all real numbers $x$ such that $a \le x \le b$.
	\item Occasionally we shall also encounter ``half-open intervals'' $[a,b)$ and $(a,b]$; the first consist of all $x$ such that $a \le x < b$, the second of all $x$ such that $a < x \le b$.
	\item If $a_i < b_i$ for $i = 1, \cdots, k$, the set of all points $\x = (x_1, \cdots, x_k)$ in $R^k$ whose coordinates satisfy the inequalities $a_i \le x_i \le b_i (1 \le i \le k)$ is called a {\cem{k-cell}}.
	\item If $\x \in R^k$ and $r>0$, the {\cem{open}} (or {\cem{closed}}) {\cem{ball}} $B$ with center at $\x$ and radius $r$ is defined to be the set of all $y \in R^k$ such that $|\y - \x|<r$ (or $|\y - \x|<r$).
	\end{enumerate}
\end{deff}
\begin{deff}
	 We call a set $E \subset R^k$ {\cem{convex}} if $$\lambda \x + (1-\lambda) \y \in E $$ whenever $\x \in E, \y \in E$, and $0<\lambda<1$.
\end{deff}

\begin{deff}
	Let $X$ be a metric space. All points and sets mentioned below are understood to be elements and subsets of $X$.
	\begin{enumerate}[(a)]
		\item A {\cem{neighborhood}} of $p$ is a set $N_r(p)$ consisting of all $q$ such that $d(p,q) < r$, for some $r > 0$. The number $r$ is called the {\cem{radius}} of $N_r(p)$.
		\item A point $p$ is a limit point of the set $E$ if every neighborhood of $p$ contains a point $q \ne p$ such that $q \in E$.
		%\item If $p \in E$ and $p$ is not a limit point of $E$, then $p$ is called an {\cem{isolated point}} \smallmarginpar{An equivalent deff: There exsits a neighborhood of $p$ such that the only element {\underline{in $E$}} it contains is $p$ itself.} of $E$.
		\item $E$ is {\cem{closed }} if every limit point of $E$ is a point of $E$.
		\item A point $p$ is an {\cem{interior}} point of $E$ if there is a neighborhood $N$ of $p$ such that $N \subset E$.
		\item $E$ is {\cem{open}} if every point of $E$ is an interior point of $E$.
		\item The {\cem{complement}} of $E$ (denoted by $E^c$) is the set of all points $p \in X$ such that $p \notin E$.
		\item $E$ is {\cem{perfect}} if $E$ is closed and if every point of $E$ is a limit point of $E$.
		\item $E$ is {\cem{bounded}} if there is a real number $M$ and a point $q \in X$ such that $d(p,q) < M$ for all $p \in E$.
		\item $E$ is {\cem{dense}} in $X$ if every point of $X$ is a limit point of $E$, or a point of $E$ (or both).
	\end{enumerate}
\end{deff}

\begin{deff}
If $X$ is a metric space, if $E \subset X$, and if $E'$ denotes the set of all limit points of $E$ in $X$, then the {\cem{closure}} of $E$ is the set $\bar E = E \cup E'$.
\end{deff}

\subsection{Theorems}
\begin{thm}
	~
	\begin{enumerate}[(a)]
	\item Balls are convex.
	\item K-cells are convex.
	\end{enumerate}
\end{thm}

\begin{thm}
	Every neighborhood is an open set.
\end{thm}

\begin{thm}
	If $p$ is a limit point of a set $E$, then every neighborhood of $p$ contains infinitely many points of $E$.
\end{thm}

\begin{cor}
	A finite point set has no limit points.
\end{cor}

\begin{thm}
	Let \{$E_n$\} be a (finite or infinite) collection of sets $E_n$. Then $$\left(\bigcup _\alpha E_\alpha\right)^c = \bigcap_\alpha \left(E_\alpha^c\right).$$
\end{thm}

\begin{thm}
	A set $F$ is closed if and only if its complement is open.
\end{thm}
\begin{thm}
	~
	\begin{enumerate}[(a)]
	\item For any collection \{$G_n$\} of open sets, $\bigcup_n G_n$ is open.
	\item For any collection \{$F_n$\} of closed sets, $\bigcap_n F_n$ is closed.
	\item For any finite collection $G_1, \cdots, G_n$ of open sets, $\bigcap_{i=1}^n G_i $ is open.
	\item For any finite collection $F_1, \cdots, F_n$ of closed sets, $\bigcup_{i=1}^n F_i $ is closed.
	\end{enumerate}
\end{thm}

\begin{thm}
	If $X$ is a metric space and $E\subset X$, then
	\begin{enumerate}[(a)]
	\item $\bar E$ is closed,
	\item $E=\bar E$ if and only if $E$ is closed,
	\item $\bar E \subset F$ for every closed set $F\subset X$ such that $E \subset F$.
	\end{enumerate}
	By (a) and (c), $\bar E$ is the smallest closed subset of $X$ that contains $E$,
\end{thm}

\begin{thm}
	Let $E$ be a nonempty set of real numbers which is bounded above. Let $y = sup E$. Then $y \in \bar E$. Hence $y \in E$ if $E$ is closed.
\end{thm}

\begin{thm}
	Suppose $Y \subset X$. A subset $E$ of $Y$ is open relative to $Y$ is and ony if $E = Y \cap G$ for some open subset $G$ of $X$.
\end{thm}

\section{Compact Sets}
\subsection{Definitions}
\begin{deff}
	By an {\cem{open cover}} of a set $E$ in a metric space $X$ we mean a collection $\{G_\alpha\}$ of open subsets of $X$ such that $E \subset \bigcup_\alpha G_\alpha$.
\end{deff}

%\begin{deff}
%	A subset $K$ of a metric space $X$ is said to be {\cem{compact}} if %every open cover of $K$ contains a {\cem{finite}} subcover. %\smallmarginpar{It is clear that every finite set is compact.}
%\end{deff}

\subsection{Theorems}
%\begin{thm}
%	Suppose $K \subset Y \subset X$. Then $K$ is compact relative to $X$ if and only if $K$ is compact relative to $Y$. \smallmarginpar{Every metric space $X$ is an open subset of itself, and is a closed subset of itself.}
%\end{thm}

\begin{thm}
	Compact subsets of metric spaces are closed.
\end{thm}

\begin{thm}
	Cloased subsets of compact sets are compact.
\end{thm}

\begin{thm}
	If $F$ is closed and $K$ is compact, the n$F \cap K$ is compact.
\end{thm}

\begin{thm}
	If $\{K_\alpha\}$ is a collection of compact subsets of a metric space $X$ such that the intersection of every finite subcollection of $\{K_\alpha\}$ is nonempty, then $\cap K_\alpha$ is nonempty.
\end{thm}

\begin{thm}
	If $E$ is an infinite subset of a compact set $K$, then $E$ has a limit point in $K$.
\end{thm}

\begin{thm}
	If $\{I_n\}$ is a sequence of intervals in $R^1$, such that $I_n \supset I_{n+1}$ $(n=1,2,3,\cdots)$, then $\cap_{n=1}^\infty I_n$ is not empty.
\end{thm}

\begin{thm}
	Let $k$ be a positive integer. If $\{I_n\}$ is a sequence of $k$-cells such that $I_n \supset I_{n+1}$ $(n=1,2,3,\cdots)$, then $\cap_{n=1}^infty I_n$ is not empty.
\end{thm}

\begin{thm}
	Every $k$-cell is compact.
\end{thm}

\begin{thm}
	If a set $E$ in $R^k$ has one of the following three properties, then it has the other two:
	\begin{enumerate}[(a)]
		\item $E$ is closed and bounded.
		\item $E$ is compact.
		\item Every infinite subset of $E$ has a limit point in $E$.
	\end{enumerate}
\end{thm}

\begin{thm}
	Every bounded infinite subset of $R^k$ has a limit point in $R^k$.
\end{thm}

\section{Perfect Sets}
\subsection{Theorems}
\begin{thm}
	Let $P$ be a nonempty perfect set in $R^k$. Then $P$ is uncountable.
\end{thm}

\begin{cor}
	Every interval $[a,b]$ $(a < b)$ is uncountable. In particular, the set of all real numbers is uncountable.
\end{cor}

\section{Connected Sets}
\subsection{Definitions}
%\begin{deff}
%	Two subsets $A$ and $B$ of a metric space $X$ are said to be {\cem{separated}} if both $A \cap \bar B$ and $\bar A \cap B$ are empty, i.e., if no point of $A$lies in the closure of $B$ and no point of $B$lies in the closure of $A$. A set $E \subset X$ is siad to be {\cem{connected}} if $E$ is {\cem{not}} a union of two nonempty separated sets. \smallmarginpar{Separated sets are of course disjoint, but disjoint sets need not be sparated.}
%\end{deff}

\subsection{Theorems}
\begin{thm}
	A subset $E$ of the ral line $R^1$ is connected if and only if it has the following property: If $x \in E$, $y \in E$, and $x < z < y$, then $z \in E$.
\end{thm} 
\section{Numerical Sequences and Series}

\section{Convergent Sequences}
\subsection{Definitions}
%\begin{deff}
%	A sequence $\{p_n\}$ in a metric space $X$ is said to {\cem{converge}} \smallmarginpar{If $\{p_n\}$ does not converge, it is said to {\cem{diverge}}.} if there is a point $p \in X$ with the following property: For every $\epsilon > 0$ there is an integer $N$ such that $n \ge N$ implies that $d(p_n,p) < \epsilon$. (Here $d$ denotes the distance in X.)	
%\end{deff}

\begin{deff}
	The sequence $\{p_n\}$ is said to be {\cem{bounded}} if its range is bounded.
\end{deff}

\subsection{Thorems}
\begin{thm}\label{thm:convergent seq}
	Let $\{p_n\}$ be a sequence in a metric space $X$.
	\begin{enumerate}[(a)]
		\item $\{p_n\}$ converges to $p \in X$ if and only if every neighborhood of $p$ contains $p_n$ for all but finitely many $n$.
		\item If $p \in X, p' \in X$, and if $\{p_n\}$ converges to $p$ and to $p'$, then $p'=p$.
		\item If $\{p_n\}$ converges, then $\{p_n\}$ is bounded.
		\item If $E \subset X$ and if $p$ is a limit point %\smallmarginpar{A point $p$ is a limit point of a set $E$ if and only if there is a sequence $\{p_n\}$ of {\underline{distinct points of $E$}} converging to $p$.} of $E$, then there is a sequence $\{p_n\}$ in $E$ such that $p = \lim_{n \to \infty} p_n$
	\end{enumerate}
\end{thm}

\begin{thm}
	Suppose $\{s_n\}, \{t_n\}$ are complex sequences, and $\lim_{n \to \infty} \{s_n\} = s$ and $\lim \{t_n\} = t$. Then,
	\begin{enumerate}[(a)]
		\item $\lim_{n \to \infty} (s_n + t_n) = s + t$;
		\item $\lim_{n \to \infty} (cs_n) = cs, \lim_{n \to \infty} (c+s_n) = c + s$, for all number $c$;		
		\item $\lim s_n t_n = st$;
		\item $\lim \frac{1}{s_n} = \frac{1}{s}$, provided $s_n \ne 0 (n=1,2,3,\cdots)$, and $s \ne 0$.
	\end{enumerate}	
\end{thm}

\begin{thm}
	~
	\begin{enumerate}[(a)]
		\item Suppose $\x_n \in R^k (n=1,2,3,\cdots)$ and $$\x_n = (\alpha_{1,n}, \cdots, \alpha_{k,n}.$$ Then $\{\x_n\}$ converges to $\x=(\alpha_1,\cdots,\alpha_k)$ if and only if $$\lim_{n \to \infty} \alpha_{j,n} = \alpha_j.$$
		\item Suppose $\{\x_n\},\{\y_n\}$ are sequences in $R^k$, $\{\beta_n\}$ is a sequence of real numbers, and $\x_n \to \x, \y_n \to \y, \beta_n \to \beta$. Then $$\lim_{n \to \infty} (\x_n+\y_n) = \x + \y, \lim_{n \to \infty} (\x_n \cdot \y_n) = \x \cdot \y, \lim_{n \to \infty} \beta_n \x_n = \beta \x.$$
	\end{enumerate}
\end{thm}


\section{Subsequences}
\subsection{Definitions}
\begin{deff}\label{def:sublim}
	Given a sequence $\{p_n\}$, consider a sequence $\{n_k\}$ of positive integers, such that $n_1 < n_2 < n_3 < \cdots$. Then the sequence $\{p_{n_i}\}$ is called a {\cem{subsequence}} of $\{p_n\}$. If $\{p_{n_i}\}$ converges, its limit is called a {\cem{subsequential limit}} of $\{p_n\}$.
\end{deff}

\subsection{Theorems}
\begin{thm}
	$\{p_n\}$ converges to $p$ if and only if every subsequence of $\{p_n\}$ converges to $p$.
\end{thm}

\begin{thm}
	~
	\begin{enumerate}[(a)]
		\item If $\{p_n\}$ is a sequence in a compact metric space $X$, then some subsequence of $\{p_n\}$ converges to a point of $X$.
		\item Every bounded sequence in $R^k$ contains a convergent subsequence.
	\end{enumerate}
\end{thm}

\begin{thm}
	The subsequential limits of a sequence $\{p_n\}$ in a metric space $X$ form a closed subset of $X$.
\end{thm}

\section{Cauchy Sequences}
\subsection{Definitions}
\begin{deff}
	A sequence $\{p_n\}$ in a metric space X is said to be a {\cem{Cauchy sequence}} if for every $\epsilon>0$ there is an integer $N$ such that $d(p_n,p_m)<\epsilon$ if $n\geq N$ and $m\geq N$.
\end{deff}

\begin{deff}
	Let $E$ be a nonempty subset of a metric space $X$, and let $S$ be the set of all real numbers of the form $d(p,q)$, with $p\in E$ and $q\in E$. The sup of $S$ is called the {\cem{diameter}} of $E$.\\
	If $\{p_n\}$ is a sequence in $X$ and if $E_N$ consists of the points $p_N, p_{N+1}, p_{N+2}, \cdots$, it is clear from the two preceding deffs that $\{p_n\}$ is a {\cem{Cauchy sequence if and only if}} $$\lim_{N \to \infty} \text{diam }E_N =0.$$
\end{deff}

\begin{deff}
	A metric space in which every Cauchy sequence converges is said to be {\cem{complete}}.
\end{deff}

\begin{deff}
	A sequence $\{s_n\}$ of real numbers is said to be
	\begin{enumerate}[(a)]
		\item {\cem{monotonically}} increasing if $s_n \leq s_{n+1} (n=1,2,3,\cdots)$;
		\item {\cem{monotonically}} decreasing if $s_n \geq s_{n+1} (n=1,2,3,\cdots)$;
	\end{enumerate}
\end{deff}

\subsection{Theorems}
\begin{thm}
	~
	\begin{enumerate}[(a)]
		\item If $\bar E$ is the closure of a set $E$ in a metric space $X$, then $$\text{diam } \bar E = \text{diam }E.$$
		\item If $K_n$ is a sequence of compact sets in $X$ such that $K_n \supset K_{n+1} (n=1,2,3,\cdots)$ and if $$\lim_{n \to \infty} \text{diam } K_n =0,$$ then $\bigcap_1^\infty K_n$ consists of exactly one point.
	\end{enumerate}
\end{thm}

\begin{thm}
	~
	\begin{enumerate}[(a)]
		\item In any metric space $X$, every convergent sequence is a Cauchy sequence.
		\item If X is a compact metric space and if $\{p_n\}$ is a Cauchy sequence in $X$, then $\{p_n\}$ converges to some point of $X$.
		\item in $R^k$, every Cauchy sequence converges.
	\end{enumerate}
	The fact that a sequence converges in $R^k$ if and only it is a Cauchy sequence is usually called the {\cem{Cauchy criterion}} for convergence.\\
	This thm says that {\cem{all compact metric spaces and all Euclidean spaces are complete}}. It implies also that {\cem{every closed subset of $E$ of a complete metric space $X$ is complete}}.
\end{thm}

\begin{thm}
	Suppose $\{s_n\}$ is monotonic. Then $\{s_n\}$ converges if and only if it is bounded.
\end{thm}

\section{Upper and Lower Limits}
\subsection{Definitions}
\begin{deff}
	Let $\{s_n\}$ be a sequence of real numbers with the following property: For every real $M$ there is an interger $N$ such that $n\geq N$ implies $s_n \geq M$. We then write $$s_n\to+\infty.$$ Similarly, if for every real $M$ there is an integer $N$ such that $n\geq N$ implies $s_n\leq M$, we write $$s_n\to-\infty.$$
\end{deff}

\begin{deff}\label{def:liminf}
	Let $\{s_n\}$ be a sequence of real numbers. Let $E$ be the set of numbers $x$ (in the extended real number system) such that $s_{n_k}\to x$ for some subsequence $\{s_{n_k}\}$. This set $E$ contains all subsequential limits as defined in Definition ~\ref{def:sublim}, plus possibly the numbers $+\infty, -\infty$.\\
	We now recall Definition ~\ref{def:supinf} and ~\ref{def:infinity} and put $$s^*=\sup E,$$ $$s_*=\inf E.$$ The numbers $s^*, s_*$ are called the {\cem{upper}} and {\cem{lower limits}} of $\{s_n\}$; we use the notation $$\limsup\limits_{n\rightarrow\infty}s_n=s^*,~~\liminf\limits_{n\rightarrow\infty}s_n=s_*$$
\end{deff}

\subsection{Theorems}
\begin{thm}
	Let$\{s_n\}$ be a sequence of real numbers. Let $E$ and $s^*$ have the same meaning as in Definition ~\ref{def:liminf}. Then $s^*$ has the following two properties:
	\begin{enumerate}[(a)]
		\item $s^*\in E$
		\item If $x>s^*$, there is an integer $N$ such that $n\geq N$ implies $s_n<x$.
	\end{enumerate}
	Moreover, $s^*$ is the only number with the properties (a) and (b). \\
	Of course, an analogous result is true for $s_*$.
\end{thm}

\begin{thm}
	If $s_n\leq t_n$ for $n\geq N$, where N is fixed, then
	$$\liminf\limits_{n\rightarrow\infty}s_n\leq \liminf\limits_{n\rightarrow\infty}t_n,$$
	$$\limsup\limits_{n\rightarrow\infty}s_n\leq \limsup\limits_{n\rightarrow\infty}t_n,$$
\end{thm}

\section{Some Special Sequences}
\subsection{Theorems}
\begin{thm}
~
	\begin{enumerate}[(a)]
			\item If $p>0$, then $\lim_{n\to\infty}\frac{1}{n^p}=0$.
			\item If $p>0$, then $\lim_{n\to\infty}\sqrt[n] p=1$.
			\item $\lim_{n\to\infty} \sqrt[n]{n}=1.$
			\item If $p>0$ and $\alpha$ is real, then $\lim_{n\to\infty}\frac{n^\alpha}{(1+p)^n}=0$.
			\item If $|x|<1$, then $\lim_{n\to\infty}x^n=0$.
	\end{enumerate}
\end{thm}

\section{Series}
\subsection{Definitions}
\begin{deff}
	Given a sequence $\{a_n\}$, we use the notation $$\sum^q_{n=p} a_n ~~(p\leq q)$$ to denote the sum $a_p+a_{p+1}+\cdots+a_q$. With $\{a_n\}$ we associate a sequence $\{s_n\}$, where $$s_n=\sum_{k=1}^n a_k.$$
	For $\{s_n\}$ we also use the symbolic expression $$a_1+a_2+a_3+\cdots$$ or, more concisely, $$\sum_{n=1}^\infty a_n.$$ The above symbol we call an {\cem{infinite series}}, or just a {\cem{series}}. The numbers $\{s_n\}$ are called the {\cem{partial sums}} of the series. If $\{s_n\}$ converges to $s$, we say that the series $converges$, and write $$\sum_{n=1}^\infty a_n =s.$$ The number $s$ is called the sum of the series; but it should be clearly understood that {\cem{s is the limit of a sequence of sums}}, and is not obtained simply by addition.\\
	If $\{s_n\}$ diverges, the series is said to diverge.
\end{deff}

\subsection{Theorems}
\thm{
	$\sum a_n$ converges if and only if for every $\epsilon>0$ there is an integer $N$ such that $$|\sum^m_{k=m}a_k|\leq\epsilon$$ if $m\geq n \geq N$.
}

\thm{
	If $\sum a_n$ converges, then $\lim_{n\to\infty}a_n=0$.
}

\thm{
	A series of nonnegative terms converges if and only if its partial sums form a bounded sequence.
}

\thm{
	~
	\begin{enumerate}[(a)]
			\item If $|a_n|\leq c_n$ for $n\geq N_0$, where $N_0$ is some fixed integer, and if $\sum c_n$ converges, then $\sum a_n$ converges.
			\item If $a_n \geq d_n \geq 0$ for $n\geq N_0$, and if $\sum d_n$ diverges, then $\sum a_n$ diverges.
	\end{enumerate}
}

\section{Series of Nonnegative Terms}
\subsection{Theorems}
\thm{
	If $0\leq x<1$, then $$\sum_{n=0}^\infty x^n = \frac{1}{1-x}.$$
	If $x\geq 1$, the series diverges.
}

\thm{
	Suppose $a_1\geq a_2\geq a_3 \geq \cdots \geq 0$. Then the series $\sum_{n=1}^\infty a_n$ converges if and only if the series
	$$\sum_{k=0}^\infty 2^k a_{2^k}=a_1+2a_2+4a_4+8a_8+\cdots$$ converges.
}
\thm{
	$\sum\frac{1}{n^p}$ converges if $p>1$ and diverges if $p\leq 1$.
}
\thm{
	If $p>1$, $$\sum_{n=2}^\infty \frac{1}{n(\log n)^p}$$ converges; if $p\leq 1$, the series diverges.
}

\section{The Number $e$}
\subsection{Definitions}
\deff{
	$$e=\sum_{n=0}^\infty \frac{1}{n!}$$
}
\subsection{Theorems}
\thm{
	$$\lim_{n\to\infty}\left(1+\frac{1}{n}\right)^n=e.$$
}
\thm{
	$e$ is irrational.
}

\section{The Root and Ratio Tests}
\subsection{Theorems}
\thm{
	(Root Test) Given $\sum a_n$, put $\alpha=\limsup\limits_{n\to\infty} \sqrt[n]{|a_n|}$.\\
	Then
	\begin{enumerate}[(a)]
			\item if $\alpha<1$, $\sum a_n$ converges;
			\item if $\alpha>1$, $\sum a_n$ diverges;
			\item if $\alpha=1$, the test gives no information.
	\end{enumerate}
}
\thm{
	(Ratio Test) The series $\sum a_n$
	\begin{enumerate}[(a)]
			\item converges if $\limsup\limits_{n\to\infty} \left|\frac{a_{n+1}}{a_n}\right|<1$,
			\item diverges if $\left|\frac{a_{n+1}}{a_n}\right|\geq 1$ for all $n\geq n_0$, where $n_0$ is some fixed integer.
	\end{enumerate}
}
\thm{
	For any sequence $\{c_n\}$ of positive numbers,
	$$\liminf\limits_{n\to\infty}\frac{c_{n+1}}{c_n}\leq \liminf\limits_{n\to\infty}\sqrt[n]{c_n},$$
	$$\limsup\limits_{n\to\infty}\sqrt[n]{c_n}\leq \limsup\limits_{n\to\infty}\frac{c_{n+1}}{c_n}.$$
}

\section{Power Series}
\subsection{Definitions}
\deff{
	Given a sequence $\{c_n\}$ of complex numbers, the series $$\sum_{n=0}^\infty c_n z^n$$ is called a {\cem{power series}}. The numbers $\{c_n\}$ are called the {\cem{coefficients}} of the series; $z$ is a complex number.
}
\subsection{Theorems}
\thm{
	Given the power series $\sum c_n z^n$, put
	$$\alpha = \limsup\limits_{n\to\infty} \sqrt[n]{|c_n|}, ~~ R=\frac{1}{\alpha}.$$
	(if $\alpha=0, R=+\infty$; if $\alpha=+\infty, R=0$.) Then $\sum c_n z^n$ converges if $|z|<R$, and diverges if $|z|>R$.
}

\section{Summation by Parts}
\subsection{Theorems}
\thm{
	Given two sequences $\{a_n\},\{b_n\}$, put $$A_n=\sum_{k=0}^n a_k$$ if $n\geq 0$; put $A_{-1}=0$. Then, if $0\leq p\leq q$, we have
	$$\sum_{n=p}^q a_n b_n = \sum_{n=p}^{q-1}A_n(b_n-b_{n+1})+A_q b_q-A_{p-1}b_p.$$
}
\thm{
	Suppose
	\begin{enumerate}[(a)]
			\item the partial sums $A_n$ of $\sum a_n$ form a bounded sequences;
			\item $b_0\geq b_1 \geq b_2 \geq \cdots$;
			\item $\lim_{n\to \infty} b_n = 0.$
	\end{enumerate}
}
\thm{
	Suppose
	\begin{enumerate}[(a)]
			\item $|c_1|\geq|c_2|\geq|c_3|\geq \cdots;$
			\item $c_{2m-1}\geq 0, c_{2m}\leq 0 ~~ (m=1,2,3,\cdots);$
			\item $\lim_{n\to\infty}c_n=0$.
	\end{enumerate}
	Then $\sum c_n$ converges.
}
\thm{
	Suppose the radius of convergence of $\sum c_n z^n$ is 1, and suppose $c_0\geq c_1 \geq c_2 \geq \cdots, \lim_{n\to\infty}c_n=0$. Then $\sum c_n z^n$ converges at every point on the circle $|z|=1$, except possibly at $z=1$.
}
\section{Absolute Convergence}
\subsection{Definitions}
\deff{The series $\sum a_n$ is said to {\cem{converge absolutely}} if the series $\sum|a_n|$ converges.}
\deff{
	If $\sum a_n$ converges, but $\sum|a_n|$ diverges, we say that $\sum a_n$ converges {\cem{nonabsolutely}}.
}
\subsection{Theorems}
\thm{
	If $\sum{a_n}$ converges absolutely, then $\sum a_n$ converges.
}


\section{Addition and Multiplication of Series}
\subsection{Definitions}
\deff{
	Given $\sum a_n$ and $\sum b_n$, we put $$c_n=\sum_{k=0}^n a_k b_{n-k} ~~ (n=0,1,2,\cdots)$$ and call $\sum c_n$ the {\cem{product}} of the two given series.
}
\subsection{Theorems}
\thm{
	If $\sum a_n=A$, and $\sum b_n =B$, then $\sum(a_n+b_n)=A+B$, and $\sum c a_n = cA$, for any fixed $c$.
}
\thm{
	Suppose
	\begin{enumerate}[(a)]
			\item $\sum_{n=0}^\infty a_n$ converges absolutely,
			\item $\sum_{n=0}^\infty a_n=A$,
			\item $\sum_{n=0}^\infty b_n=B$,
			\item $c_n=\sum_{k=0}^n a_k b_{n-k} ~~ (n=0,1,2,\cdots).$
	\end{enumerate}
	Then $$\sum_{n=0}^\infty c_n = AB.$$\\
	That is, the product of two convergent series converges, and to the right value, if at least one of the two series converges absolutely.
}
\thm{
	If the series $\sum a_n, \sum b_n, \sum c_n$ converge to $A,B,C$, and $c_n=a_0 b_n+\cdots+a_nb_0$ then $C=AB$.
}
\section{Rearrangements}
\subsection{Definitions}
\deff{
	Let $\{k_n\}, n=1,2,3,\cdots$, be a sequence in which every positive integer appears once and only once (that is, $\{k_n\}$ is a 1-1 function from $J$ onto $J$, in the notation of Definition ~\ref{def: mapping}). Putting $$a_n^\prime=a_{k_n}~~(n=1,2,3,\cdots),$$ we say that $\sum a_n^\prime$ is a {\cem{rearrangement}} of $\sum a_n$.
}
\subsection{Theorems}
\thm{
	Let $\sum a_n$ be a series of real numbers which converges, but not absolutely. Suppose $$-\infty \leq \alpha\leq\beta\leq\infty.$$ Then there exist a rearrangement $\sum a_m^\prime$ with partial sums $s_n^\prime$ such that
	$$\liminf\limits_{n\to\infty}s_n^\prime = \alpha, ~~ \limsup\limits_{n\to\infty}s_n^\prime=\beta.$$
}
\thm{
	If $\sum a_n$ is a series of complex numbers which converges absolutely, then every rearrangement of $\sum a_n$ converges, and they all converges to the same sum.
} 
\chapter{Continuity}

\section{Limits of Functions}
\subsection{Definitions}
\begin{deff}\label{def:function limit}
Let $X$ and $Y$ be metric spaces; suppose $E \subset X$, $f$ maps $E$ into $Y$, and $p$ is a limit point of $E$. \smallmarginpar{The deff does not say anything about $f(p)$.} We write $f(x) \to q$ as $x \to p$, or $$\lim_{x \to p} f(x) = q$$ if there is a point $q \in Y$ with the following property: For every $\epsilon > 0$ there exists a $\delta > 0$ such that $$d_Y(f(x),q) < \epsilon$$ for all points $x \in E$ for which $$0 < d_X(x,p) < \delta.$$
\end{deff}

\subsection{Theorems}
\begin{thm}
	Let $X, Y, E, f, \mbox{ and } p$ be as in Definition ~\ref{def:function limit}. Then $$\lim_{x \to p} f(x) = q$$ if and only if $$\lim_{n \to \infty} f(p_n) = q$$ for every sequence $\{p_n\}$ in $E$ such that $$p_n \ne p, ~\lim_{n \to \infty}p_n = p.$$
\end{thm}

\begin{cor}
	If $f$ has a limit at $p$, this limit is unique.
\end{cor}

\begin{thm}
	Suppose $E \subset X$, a metric space, $p$ is a limit point of $E, f \mbox{ and } g$ are complex functions on $E$, and $$\lim_{x \to p}f(x) = A, ~ \lim_{x \to p} g(x) = B.$$ Then
	\begin{enumerate}[(a)]
		\item $\lim_{x \to p} (f+g)(x) = A + B$;
		\item $\lim_{x \to p} (fg)(x) = AB$;
		\item $\lim_{x \to p} (\frac{f}{g})(x) = \frac{A}{B}, ~ if ~ B \ne 0$.
	\end{enumerate}
\end{thm}

\section{Continuous Functions}
\subsection{Definitions}
\begin{deff}\label{def:function continuous}
	Suppose $X$ and $Y$ are metric spaces, $E \subset X$, $p \in E$, and $f$ maps $E$ into $Y$. Then $f$ is said to be {\cem{continuous at p}}  if for every $\epsilon > 0$ there exists a $\delta >0 $ such that $$\delta_Y(f(x),f(p)) < \epsilon$$ for all points $x \in E$ for which $d_X(x,p) < \delta.$
\end{deff}

\subsection{Theorems}
\begin{thm}
	In the situation given in Definition~\ref{def:function continuous}, assume also that {\underline{$p$ is a limit point}} \smallmarginpar{If $p$ is an {\underline{isolated point}} of $E$, then every function $f$ which has $E$ as its domain of defintion is continuous at $p$.} of $E$. Then $f$ is continuous at $p$ if and only if $\lim_{x \to p} f(x) = f(p).$
\end{thm}

\begin{thm}
	Suppose $X, Y, Z$ are metric spaces, $E \subset X$, $f$ maps $E$ into $Y$, $g$ maps the range of $f$, $f(E)$, into $Z$, and $h$ is the mapping of $E$ into $Z$ defined by $$h(x) = g(f(x)) ~ ~ (x \in E).$$ If $f$ is continuous at a point $p \in E$ and if $g$ is continuous at the point $f(p)$, then $h$ is continuous at $p$.
\end{thm}

\thm{
	A mapping $f$ of a metric space $X$ into a metric space $Y$ is onctinuous on $X$ if and only if $f^{-1}(V)$ is open in $X$ for every open set $V$ in $Y$.
}

\cor{
	A mapping $f$ of a metric space $X$ into a metric space $Y$ is continuous if and only if $f^{-1}(C)$ is closed in $X$ for every closed set $C$ in $Y$.
}

\thm{
	Let $f$ and $g$ be complex {\underline{continuous}} functions on a metric space $X$. Then $f+g, fg, \mbox{ and } f/g$ are {\underline{continuous}} on $X$.
}

\thm{
	~
	\begin{enumerate}[(a)]
		\item Let $f_1,\cdots,f_k$ be real functions on a metric space $X$, and let $\f$ be the mapping of $X$ into $R^k$ defined by $$\f(x) = (f_1(x),\cdots,f_k(x)) ~ ~ (x \in X);$$ then $\f$ is continuous if and only if each of the functions $f_1,\cdots,f_k$ is continuous.
		\item if $\f$ and $\g$ are continuous mappings of $X$ into $R^k$, then $\f+\g$ and $\f \cdot \g$ are continuous on $X$.
	\end{enumerate}
}

\section{Continuity and Compactness}

\subsection{Definitions}
\deff{
	A mapping $\f$ of a set $E$ into $R^k$ is said to be {\cem{bounded}} if there is a real number $M$ such that $|\f(x)| \le M$ for all $x \in E$.
}

\begin{deff}
	Let $f$ be a mapping of a metric space $X$ into a metric space $Y$. We say that $f$ is {\cem{uniformly continuous}} on $X$ if for every $\epsilon > 0$ there exists $\delta > 0$ such that $$d_Y(f(p),f(q)) < \epsilon$$ for all $p$ and $q$ in $X$ for which $d_X(p,q) < \delta$.
\end{deff}

\subsection{Theorems}
\thm{
	Suppose $f$ is a continuous mapping of a compact metric space $X$ into a metric space $Y$. Then $f(X)$ is compact.
}

\thm{
	If $\f$ is a continuous mapping of a compact metric space $X$ into $R^k$, then $\f(X)$ is closed and bounded. Thus, $\f$ is bounded.
}

\thm{
	Suppose $f$ is a continuous real function on a compact metric space $X$, and $$M = \sup_{p \in X} f(p), ~ ~ m = \inf_{p \in X}f(p).$$ Then there exist points $p, q \in X$ such that $f(p) = M$ and $f(q) = m$. \smallmarginpar{This is to say, $f$ attains its maximum (at $p$) and its minimum (at $q$).}
}

\thm{
	Suppose $f$ is a continuous 1-1 mapping of a compact metric space $X$ {\underline{onto}} a metric space $Y$. Then the inverse mapping $f^{-1}$ defined on $Y$ by $$f^{-1}(f(x)) = x ~ ~ (x \in X)$$ is a continuous mapping of $Y$ {\underline{onto}} $X$.
}

\thm{
	Let $f$ be a continuous mapping of a compact metric space $X$ into a metric space $Y$. Then $f$ is uniformly continuous on $X$.
}

\thm{
	Let $E$ be a noncompact set in $R^1$. Then
	\begin{enumerate}[(a)]
		\item there exists a continuous function on $E$ which is not bounded;
		\item there exists a continuous and bounded function on $E$ which has no maximum. If, in addition, $E$ is bounded, then 
		\item there exists a continuous function on $E$ which is not uniformly continuous.
	\end{enumerate}
}

\section{Continuity and Connectedness}
\subsection{Theorems}
\thm{
	If $f$ is a continuous mapping of a metric space $X$ into a metric space $Y$, and if $E$ is a connected subset of $X$, then $f(E)$ is connected.
}

\thm{
	Let $f$ be a continuous real function on the interval $[a,b]$. If $f(a) < f(b)$ and if $c$ is a number such that $f(a) < c < f(b)$, then there exists a point $x \in (a,b)$ such that $f(x) = c$.
}

\section{Discontinuities}
\subsection{Definitions}
\deff{
	Let $f$ be defined on $(a,b)$. Consider any point $x$ such that $a \le x < b$. We write $$f(x+) = q$$ if $f(t_n) \to q$ as $n \to \infty$, for all sequences $\{t_n\}$ in $(x,b)$ such that $t_n \to x$. To obtain the deff of $f(x-)$, for $a < x \le b$, we restrict ourselves to sequences $\{t_n\}$ in $(a,x)$. It is clear that any point $x$ of $(a,b)$, $\lim_{t \to x} f(t)$ exists if and only if $$f(x+) = f(x-) = \lim_{t \to x} f(t).$$ 
}

\deff{
	Let $f$ be defined on $(a,b)$. If $f$ is discontinuous at a point $x$, and if $f(x+)$ and $f(x-)$ exist, then $f$ is said to have a discontinuity of the {\cem{first kind}}, or a {\cem{simple discontinuity}} at $x$. \smallmarginpar{There are two ways in which a function can have a simple discontinuity: either $f(x+) \ne f(x-)$, in which case the value $f(x)$ is immaterial, or $f(x+) = f(x-) \ne f(x)$.} Otherwise the discontinuity is said to be of the {\cem{second kind}}.
}

\section{Monotonic Functions}
\subsection{Definitions}
\deff{
	Let $f$ be real on $(a,b)$. Then $f$ is said to be {\cem{monotonically increasing}} on $(a,b)$ if $a < x < y < b$ implies $f(x) \le f(y)$. If the last inequality is reversed, we obtain the deff of a {\cem{monotonically decreasing}} function. The class of monotonic functions consists of both the increasing and the deceasing functions.
}

\subsection{Theorems}
\thm{
	Let $f$ be monotonically increasing on $(a,b)$. Then $f(x+)$ and $f(x-)$ exist at every point of $x$ of $(a,b)$. More precisely, $$\sup_{a<t<x}f(t) = f(x-) \le f(x) \le f(x+) = \inf_{x<t<b}f(t).$$ Furthermore, if $a<x<y<b$, then $$f(x+) \le f(y-).$$ Analogous results evidently hod for monotonically decreasing functions.
}

\cor{
	Monotonic functions have no discontinuities of the second kind. \smallmarginpar{Compare with Corollary~\ref{cor: derivative functions do not have discontinuities of the first kind.}}
}\label{cor: monotone functions do not have discontinuities of the second kind.}

\thm{
	Let $f$ be monotonic on $(a,b)$. Then the set of points of $(a,b)$ at which $f$ is discontinuous is at most countable.
}

\section{Infinite Limits and Limits at Infinity}
\subsection{Definitions}
\deff{
	For any real $c$, the set of real numbers $x$ such that $x > c$ is called a neighborhood of $+\infty$ and is written $(c,+\infty)$. Similarly, the set $(-\infty,c)$ is a neighborhood of $-\infty$.
}

\deff{
	Let $f$ be a real function defined on $E \subset R$. We say that $$f(t) \to A \mbox{ as } t \to x,$$ where $A$ and $x$ are in the extended real number system, if for every neighborhood $U$ of $A$ there is a neighborhood $V$ of $x$ such that $V \cap E$ is not empty, and such that $f(t) \in U$ for all $t \in V \cap E, t \ne x$.
}

\subsection{Theorems}
\thm{
	Let $f$ and $g$ be defined on $E \subset R$. Suppose $$f(t) \to A, ~ ~ g(t) \to B \mbox{ as } t \to x.$$ Then
	\begin{enumerate}[(a)]
		\item $f(t) \to A' \mbox{ implies } A' = A.$
		\item $(f+g)(t) \to A + B,$
		\item $(fg)(t) \to AB,$
		\item $(f/g)(t) \to A/B,$
	\end{enumerate}
	provided the right member of (b), (c), and (d) are defined.
}
\section{Differentiation}

\section{The Derivative of a Real Function}
\subsection{Definitions}
\begin{deff}
	Let $f$ be defined (and real-valued) on $[a,b]$. For any $x \in [a,b]$ form the quotient $$\phi(t) = \frac{f(t)-f(x)}{t-x} (a<t<b, t \ne x),$$ and define $$f'(x) = \lim_{t \to x} \phi(t),$$ provided this limit exists in accordance with Definition~\ref{def:function limit}. We thus associate with the function $f$ a function $f'$ whose domain is the set of points $x$ at which the limit exists; $f'$ is called the {\cem{derivative}} of $f$. If $f'$ is defined at a point $x$, we say that $f$ is {\cem{differentiable}} at $x$. If $f'$ is defined at every point of a set $E \subset [a,b]$, we say that $f$ is differentiable on $E$.
\end{deff}

\subsection{Theorems}
%\begin{thm}
%	Let $f$ be defined on $[a,b]$. If $f$ is differentiable at a point $x \in [a,b]$, then $f$ is continuous at $x$. \smallmarginpar{Prove by using the fact that limit of a product is the product of limits.}
%\end{thm}

\thm{
	Suppose $f$ and $g$ are defined on $[a,b]$ and are differentiable at a point $x \in [a,b]$. Then $f+g, fg, \mbox{ and } f/g$ are differentiable at $x$, and
	\begin{enumerate}[(a)]
		\item $(f+g)'(x) = f'(x) + g'(x)$;
		\item $(fg)'(x) = f'(x)g(x) + f(x)g'(x);$
		\item $(\frac{f}{g})'(x) = \frac{g(x)f'(x)-g'(x)f(x)}{g^2(x)}$
	\end{enumerate}
	In (c), we assume of course that $g(x) \ne 0$.
}

\thm{
	Suppose $f$ is continuous on $[a,b]$, $f'(x)$ exists at some point $x \in [a,b]$, $g$ is defined on an interval $I$ which contains the range of $f$, and $g$ is differentiable at the point $f(x)$. If $$h(t) = g(f(t)) ~ ~ (a \le t \le b),$$ then $h$ is differentiable at $x$, and $$h'(x) = g'(f(x))f'(x).$$
}

\section{Mean Value Theorems}
\subsection{Definitions}
\begin{deff}
	Let $f$ be a real function defined on a metric space $X$. We say that $f$ has a {\cem{local maximum}} at a point $p \in X$ if there exists $\delta > 0$ such that $f(q) \le f(p)$ for all $q \in X$ with $d(p,q) < \delta.$
\end{deff}

\subsection{Theorems}
%\begin{thm}
%	Let $f$ be defined on $[a,b]$; if $f$ has a local maximum at a point $x \in (a,b)$, and if $f'(x)$ exists, then $f'(x) = 0.$ \smallmarginpar{Prove by showing the left-hand and right-hand derivatives}
%\end{thm}

\thm{
	If $f$ and $g$ are continuous real functions on $[a,b]$ which are differentiable in $(a,b)$, then there is a point $x \in (a,b)$ at which $$[f(b)-f(a)]g'(x) = [g(b)-g(a)]f'(x).$$ Note that differentiability is not required at the endpoints.
}

\thm{
	If $f$ is a real continuous function on $[a,b]$ which is differentiable in $(a,b)$, then there is a point $x \in (a,b)$ at which $$f(b)-f(a) = (b-a)f'(x).$$
}

\thm{
	Suppose $f$ is differentiable in $(a,b)$.
	\begin{enumerate}[(a)]
		\item If $f'(x) \ge 0$ for all $x \in (a,b)$, then $f$ is monotonically increasing.
		\item If $f'(x) = 0$ for all $x \in (a,b)$, then $f$ is constant.
		\item If $f'(x) \le 0$ for all $x \in (a,b)$, then $f$ is monotonically decreasing.
	\end{enumerate}
}

\section{The Continuity of Derivatives}
\subsection{Theorems}
\thm{
	Suppose $f$ is a real differentiable function on $[a,b]$ and suppose $f'(a) < \lambda < f'(b)$. Then there is a point $x \in (a,b)$ such that $f'(x) = \lambda$.
}

%\cor{
%	If $f$ is differentiable on $[a,b]$, then $f'$ cannot have any simple discontinuities on $[a,b]$. \smallmarginpar{Compare with Corollary~\ref{cor: monotone functions do not have discontinuities of the second kind.}}
%}\label{cor: derivative functions do not have discontinuities of the first kind.}

\section{L'Hospital's Rule}
\subsection{Theorems}
\thm{
	Suppose $f$ and $g$ are real and differentiable in $(a,b)$, and $g'(x) \ne 0$ for all $x \in (a,b)$, where $-\infty \le a < b \le +\infty$. Suppose $$\frac{f'(x)}{g‘(x)} \to A ~ as ~ x \to a.$$ If $$f(x) \to 0 ~ and ~ g(x) \to 0 ~ as ~ x \to a,$$ or if $$g(x) \to +\infty ~ as ~ x \to a,$$ then $$\frac{f(x)}{g(x)} \to A ~ as ~ x \to a.$$
}

\section{Derivatives of Higher Order}
\subsection{Definitions}
\deff{
	If $f$ has a derivative $f'$ on an interval, and if $f'$ is itself differentiable, we denote the derivative of $f'$ b $f''$ and call $f''$ the second derivative of $f$. Continuing in this manner , we obtain functions $$f,f',f'',f^{(3)},\cdots,f^{(n)},$$ each of which is the derivative of the preceding one. $f^{(n)}$ is called the $n$th derivative, or the derivative of order $n$, of $f$.
}

\section{Taylor's Theorem}
\subsection{Theorems}
\thm{
	Suppose $f$ is a real function on $[a,b]$, $n$ is a positive integer, $f^{(n-1)}$ is continuous on $[a,b]$, $f^{(n)}(t)$ exists for every $t \in (a,b)$. Let $\alpha, \beta$ be distincet points of $[a,b]$, and define $$P(t) = \sum_{k=0}^{n-1} \frac{f^{(k)}(\alpha)}{k!}(t-\alpha)^k.$$ Then there exists a point $x$ between $\alpha$ and $\beta$ such that $$f(\beta) = P(\beta) + \frac{f^{(n)}(x)}{n!} (\beta-\alpha)^n.$$
}

\section{Differentiation of Vector-valued Functions}
\subsection{Theorems}
\thm{
	Suppose $\f$ is a continuous mapping of $[a,b]$ into $R^k$ and $\f$ is differentiable in $(a,b)$. Then there exists $x \in (a,b)$ such that $$|\f(b)-\f(a)| \le (b-a)|\f'(x)|.$$
}



\section{The Riemann-Stieltjes Integral}
\section{Definition and Existence of the Integral}
\subsection{Definitions}
\deff{
	We say that the partition $P*$ is a {\cem{refinement}} of $P$ if $P* \supset P$ (that is, if every point of $P$ is a point of $P*$). Given two partitions, $P_1$ and $P_2$, we say that $P*$ is their {\cem{common refinement}} if $P* = P_1 \cup P_2$.
}
\subsection{Theorems}
\thm{
	If $P^*$ is a refinement of $P$, then $$L(P,f,\alpha) \le L(P^*,f,\alpha)$$ and $$U(P^*,f,\alpha) \le U(P,f,\alpha).$$
}

\thm{
	$\underline{\int_{a}^{b}} f d\alpha \le \overline{\int_{a}^{b}} f d\alpha$
}

\thm{
	$f \in \R(\alpha)$ on $[a,b]$ if and only if for every $\epsilon > 0$ there exists a partition $P$ such that $$U(P,f,\alpha) - L(P,f,\alpha) < \epsilon.$$
}\label{the: integrable}

\thm{
	~
	\begin{enumerate}[(a)]
		\item If Theorem~\ref{the: integrable} holds for some $P$ and some $\epsilon$, then Theorem~\ref{the: integrable} holds (with the same $\epsilon$) for every refinement of $P$.
		\item If Theorem~\ref{the: integrable} holds for $P = \{x_0,\cdots,x_n\}$ and if $s_i,t_i$ are arbitrary points in $[x_{i-1},x_i]$, then $$\sum_{i=1}^n |f(s_i)-f(t_i)| \Delta\alpha_i < \epsilon.$$
		\item If $f \in \R(\alpha)$ and the hypotheses of (b) hold, then $$|\sum_{i=1}^n |f(s_i)-f(t_i)| \Delta\alpha_i - \int_a^b f d\alpha| < \epsilon.$$
	\end{enumerate}
}

\thm{
	If $f$ is continuous on $[a,b]$ then $f \in \R(\alpha)$ on $[a,b]$.
}

\thm{
	If $f$ is monotonic on $[a,b]$, and if $\alpha$ is continuous on $[a,b]$, then $f \in \R(\alpha)$. (We still assume, of course, that $\alpha$ is monotonic.)
}

\thm{
	Suppose $f$ is bonded on $[a,b]$, $f$ has only finitely many points of discontinuity on $[a,b]$, and $\alpha$ is continuous at every point at which $f$ is discontinuous. Then $f \in \R(\alpha)$.
}

\thm{
	Suppose $f \in \R(\alpha)$ on $[a,b]$, $m \le f \le M$, $\phi$ is continuous on $[m,M]$, and $h(x) = \phi(f(x))$ on $[a,b]$. Then $h \in \R(\alpha)$ on $[a,b]$.
}

\section{Properties of the Integral}
\subsection{Definitions}
\deff{
	The {\cem{unit step function}} $I$ is defined by $$I(x) = \begin{cases}0 ~ ~ (x \le 0)\\1 ~ ~ (x > 0)\end{cases}$$
}

\subsection{Theorems}
\thm{
	If $f \in \R(\alpha)$ and $g \in \R(\alpha)$ on $[a,b]$, then
	\begin{enumerate}[(a)]
		\item $fg \in \R(\alpha)$;
		\item $|f| \in \R(\alpha)$ and $\left|\int_a^b f d\alpha\right| \le \int_a^b |f| d\alpha$.
	\end{enumerate}
}

\thm{
	If $a < s < b$, $f$ is bounded on $[a,b]$, $f$ is continuous at $s$, and $\alpha(x) = I(x-s)$, then $$\int_a^b f d\alpha = f(s).$$
}

\thm{
	Suppose $c_n \ge 0$ for $1,2,3,\cdots$, $\sum c_n$ converges, $\{s_n\}$ is a sequence of distinct points in $(a,b)$, and $$\alpha(x) = \sum_{n=1}^\infty c_n I(x-s_n).$$ Let $f$ be continuous on $[a,b]$. Then $$\int_a^b f d\alpha = \sum_{n=1}^\infty c_n f(s_n).$$
}

\thm{
	Assume $\alpha$ increases monotonically and $\alpha' \in \R$ on $[a,b]$. Let $f$ be a bounded real function on $[a,b]$. Then $f \in \R(\alpha)$ if and only if $f\alpha' \in \R$. In that case, $$\int_a^b f d\alpha = \int_a^b f(x) \alpha'(x) dx.$$
}

\thm{
	Suppose $\phi$ is a strictly increasing continuous function that maps an interval $[A,B]$ onto $[a,b]$. Suppose $\alpha$ is monotonnically increasing on $[a,b]$ and $f \in \R(\alpha)$ on $[a,b]$. Define $\beta$ and $g$ on $[A,B]$ by $$\beta(y) = \alpha(\phi(y)), ~ g(y) = f(\phi(y)).$$ Then $g \in \R(\beta)$ and $$\int_A^B g d\beta = \int_a^b f d\alpha.$$
}

\section{Integration and Differentiation}
\subsection{Theorems}
\thm{
	Let $f \in \R$ on $[a,b]$. For $a \le x \le b$, put $$F(x) = \int_a^x f(t) dt.$$ Then $F$ is continuous on $[a,b]$; furthermore, if $f$ is continuous at a point $x_0$ of $[a,b]$, then $F$ is differentiable at $x_0$, and $$F'(x_0) = f(x_0).$$
}
\thm{
	If $f \in \R$ on $[a,b]$ and if there is a differentiable function $F$ on $[a,b]$ such that $F'=f$, then $$\int_a^b f(x) dx = F(b) - F(a).$$
}
\thm{
	Suppose $F$ and $G$ are differentiable functions on $[a,b]$, $F' = f \in \R$, and $G' = g \in \R$. Then $$\int_a^b F(x)g(x) dx = F(b)G(b) - F(a)G(a) - \int_a^b f(x) G(x) dx.$$
}


\chapter{Sequences and Series of Functions}
\section{Discussion of Main Problem}
\subsection{Definitions}
\deff{
	Suppose $\{f_n\}$, $n=1,2,3,\cdots$, is a sequence of functions defined on a set $E$, and suppose that the sequence of numbers $\{f_n(x)$ converges for every $x \in E$. We can then define a function $f$ by $$f(x) = \lim_{n \to \infty} f_n(x) ~ ~ (x \in E).$$ Under these circumstances we say that $\{f_n\}$ converges on $E$ and that $f$ is the {\cem{limit}}, or the {\cem{limit function}}, of $\{f_n\}$. Sometimes we shall use a more descriptive terminology and shall say that ``$\{f_n\}$ converges to $f$ {\cem{pointwise}} on $E$'' if the above holds. Similarly, if $\sum f_n(x)$ converges for every $x \in E$, and if we define $$f(x) = \sum_{n=1}^\infty f_n(x) ~ ~ (x \in E),$$ the function $f$ is called the {\cem{sum}} of the series $\sum f_n$.
}

\section{Uniform Convergence}
\subsection{Definitions}
\begin{deff}
	We say that a sequence of functions $\{f_n\}, n = 1, 2, 3, \cdots$, converges {\cem{uniformly}} on $E$ to a function $f$ if for every $\epsilon > 0$ there is an integer $N$ such that $n \ge N$ implies $$|f_n(x)-f(x)| \le \epsilon$$ {\underline{for all $x \in E$}}.
\end{deff}

\subsection{Theorems}
\begin{thm}
	Suppose $K$ is compact, and
	\begin{enumerate}[(a)]
		\item $\{f_n\}$ is a sequence of continuous functions on $K$,
		\item $\{f_n\}$ converges pointwise to a continuous function $f$ on $K$,
		\item $f_n(x) \ge f_{n+1}(x)$ for all $x \in K, n = 1, 2, 3, \cdots$.		
	\end{enumerate}
	Then $f_n \to f$ uniformly on $K$.
\end{thm}

\thm{
	Supose $$\lim_{n \to \infty} f_n(x) = f(x) ~ ~ (x \in E).$$ Put $$M_n = \sup_{x \in E} |f_n(x) - f(x)|.$$ Then $f_n \to f$ uniformly on $E$ if and only if $M_n \to 0$ as $n \to \infty$.
}

\thm{
	Suppose $\{f_n\}$ is a sequence of functions defined on $E$, and suppose $$|f_n(x)| \le M_n ~~ (x \in E, n=1,2,3,\cdots).$$ Then $\sum f_n$ converges uniformly on $E$ if $\sum M_n$ converges.
}

\section{Uniform Convergence and Continuity}
\subsection{Definitions}
\deff{
	If $X$ is a metric space, $\C(X)$ will denote the set of all complex-valued, continuous, bounded functions with domain $X$. We associate with each $f \in \C(X)$ its supreme norm $$\|f\| = \sup_{x \in X} |f(x)|.$$ We also define the distance between $f \in \C(X)$ and $g \in \C(X)$ to be $\|f-g\|$.
}

\subsection{Theorems}
\thm{
	Suppose $f_n \to f$ uniformly on a set $E$ in a metric space. Let $x$ be a limit point of $E$, and suppose that $$\lim_{t \in x} f_n(t) = A_n ~ ~ (n=1,2,3,\cdots).$$ Then $\{A_n\}$ converges, and $$\lim_{t \in x} f(t) = \lim_{n \to \infty} A_n.$$ In other words, the conclusion is that $$\lim_{t \to x} \lim_{n \to \infty} f_n(t) = \lim_{n \to \infty} \lim_{t \to x} f_n(t).$$
}
\thm{
	If $\{f_n\}$ is a sequence of continuous functions on $E$, and if $f_n \to f$ uniformly on $E$, then $f$ is continuous on $E$.
}
\thm{
	Suppose $K$ is compact, and
	\begin{enumerate}[(a)]
		\item $\{f_n\}$ is a sequence of continuous functions on $K$,
		\item $\{f_n\}$ converges pointwise to a continuous function $f$ on $K$,
		\item $f_n(x) \ge f_{n+1}(x)$ for all $x \in K, n=1,2,3,\cdots$
	\end{enumerate}
	Then $f_n \to f$ uniformly on $K$.
}
\thm{
	The above metric makes $\C(X)$ into a complete metric space.
}

\section{Uniform Convergence and Integration}
\subsection{Theorems}
\thm{
	Let $\alpha$ be monotonically increasing on $[a,b]$. Suppose $f_n \in \R(\alpha)$ on $[a,b]$, for $n=1,2,3,\cdots$, and suppose $f_n \to f$ uniformly on $[a,b]$. Then $f \in \R(\alpha)$ on $[a,b]$, and $$\int_a^b f d\alpha = \lim_{n \to \infty} \int_a^b f_n d\alpha.$$ (The existence of the limit is part of the conclusion.)
}
\cor{
	If $f_n \in \R(\alpha)$ on $[a,b]$ and if $$f(x) = \sum_{n=1}^\infty f_n(x) ~ ~ (a \le x \le b),$$ the series converging uniformly on $[a,b]$, then $$\int_a^b f d\alpha = \sum_{n=1}^\infty \int_a^b f_n d\alpha.$$ In other words, the series may be integrated term by term.
}

\section{Uniform Convergence and Differentiation}
\subsection{Theorems}
\thm{
	Suppose $\{f_n\}$ is a sequence of functions, differentiable on $[a,b]$ and such that $\{f_n(x_0)\}$ converges for some point $x_0$ on $[a,b]$. If $\{f_n'\}$ converges uniformly on $[a,b]$, then $\{f_n\}$ converges uniformly on $[a,b]$, to a function $f$, and $$f'(x) = \lim_{n \to \infty} f'_n(x) ~ ~ (a \le x \le b).$$
}
\thm{
	There exists a real continuous function on the real line which is nowhere differentiable.
}

\section{Equicontinuous Families of Functions}
\subsection{Definitions}
\deff{
	Let $\{f_n\}$ be a sequence of functions defined on a set $E$. We say that $\{f_n\}$ is {\cem{pointwise bounded}} on $E$ if the sequence $\{f_n(x)\}$ is bounded for every $x \in E$, that is, if there exists a finite-valued function $\phi$ defined on $E$ such that $$|f_n(x)| < \phi(x) ~ ~ (x \in E, n=1,2,3,\cdots).$$ We say that $\{f_n\}$ is {\cem{uniformly bounded}} on $E$ if there exists a number $M$ such that $$|f_n(x)| < M ~ ~ (x \in E, n=1,2,3,\cdots).$$
}
\deff{
	A family $\F$ of complex functions $f$ defined on a set $E$ in a metric space $X$ is said to be {\cem{equicontinuous}} on $E$ if for every $\epsilon > 0$ there exists a $\delta > 0$ such that $$|f(x)-f(y)| < \epsilon$$ whenever $d(x,y) < \delta, x \in E, y \in E, ~ and ~ f \in \F$.
 }
 \subsection{Theorems}
 \thm{
 	If $\{f_n\}$ is a pointwise bounded sequence of complex functions on a countable set $E$, then $\{f_n\}$ has a subsequence $\{f_{n_k}\}$ such that $\{f_{n_k}\}$ converges for every $x \ in E$.
 }
 \thm{
 	If $K$ is a compact metric space, if $f_n \in \C(K)$ for $n=1,2,3,\cdots$, and if $\{f_n\}$ converges uniformly on $K$, then $\{f_n\}$ is equicontinuous on $K$.
 }
 \thm{
 	If $K$ is compact, if $f_n \in \C(K)$ for $n=1,2,3,\cdots$, and if $\{f_n\}$ is pointwise bounded and equicontinuous on $K$, then
 	\begin{enumerate}[(a)]
 		\item $\{f_n\}$ is uniformly bounded on $K$,
 		\item $\{f_n\}$ contains a uniformly convergent subsequence.
 	\end{enumerate}
 }

 \section{The Stone-Weierstrass Theorem}
 \subsection{Theorems}
 \thm{
 	If $f$ is a continuous complex function on $[a,b]$, there exists a sequence of polynomials $P_n$ such that $$\lim_{n \to \infty} P_n(x) = f(x)$$ uniformly on $[a,b]$. If $f$ is real, the $P_n$ may be taken real.
 }
 \cor{
 	For every interval $[-a,a]$ there is a sequence of real polynomials $P_n$ such that $P_n(0)=0$ and such that $$\lim_{n \to \infty} P_n(x) = |x|$$ uniformly on $[-a,a]$.
 }







\input{Mathematics/Analysis/AdvancedCalculus/Chapter8}
\chapter{Functions of Several Variables}
\section{The Contraction Principle}
\subsection{Definitions}
\deff{
	Let $X$ be a metric space, with metric $d$. If $\phi$ maps $X$ into $X$ and if there is a number $c < 1$ such that $$d(\phi(x),\phi(y)) \le c ~ d(x,y)$$ for all $x, y \in X$, then $\phi$ is said to be a {\underline{{\cem{contraction}} of $X$ into $X$}}. \smallmarginpar{If $f$ is a contraction mapping then it is also a continuous mapping. The reverse is not true.}
}

\subsection{Theorems}
\theo{
	If $X$ is a complete metric space, and if $\phi$ is a contraction of $X$ into $X$, then there exists one and only one $x \in X$ such that $\phi(x) = x$.
}
\section{Exercises}
\subsection{Concept Questions}
\prob{
	A sequence $\{a_n\}$ converges if and only if it is bounded.
\\
	- FALSE. $\{sin(n)\}$ is bounded but not convergent. However, if a sequence converges, then it is bounded. See Theorem~\ref{theo:convergent seq}
}


\documentclass{article}
\usepackage{amssymb,amsmath,mathtools}


\usepackage{algorithm2e,algorithmic}

\usepackage{mathrsfs}
\usepackage{paralist}

\usepackage{esint} % for \fint

\allowdisplaybreaks

\usepackage{color}		% enable color characters
\usepackage{graphicx} 	% insert image files
\usepackage{enumerate} 	% enumerate items
\usepackage{caption}
\usepackage{subcaption}
\usepackage{multirow,multicol}
\usepackage[makeroom]{cancel}

\usepackage[colorlinks=true,linkcolor=blue,citecolor=blue]{hyperref}

\usepackage{makeidx}

\newcommand{\cem}[1]{\color{magenta}{\em{#1}}} % color emphasize
\newcommand{\dual}[2]{{#1}^{(#2)}} % color emphasize

\newcommand{\tr}{{\mathrm{tr}}}
\renewcommand{\vec}{{\mathrm{vec}}}

\newcommand{\dd}{{\,\text{d}\,}}
%%
%% horizontal and vertical centering in table p mode
%%
\usepackage{array}
\newcolumntype{P}[1]{>{\centering\arraybackslash}p{#1}} % horizontal centering
\newcolumntype{M}[1]{>{\centering\arraybackslash}m{#1}} % vertical centering

%%
%% define bold font for the alphabet
%%
\usepackage{pgffor}
\foreach \letter in {a,...,z}{ % bold font for a..z
\expandafter\xdef\csname \letter \endcsname{\noexpand\ensuremath{\noexpand\mathbf{\letter}}}
}
\foreach \letter in {A,...,Z}{ % bold font for A..Z
\expandafter\xdef\csname \letter \endcsname{\noexpand\ensuremath{\noexpand\mathbf{\letter}}}
}
\foreach \letter in {A,...,Z}{ % `field' font for AA..ZZ
\expandafter\xdef\csname \letter\letter \endcsname{\noexpand\ensuremath{\noexpand\mathcal{\letter}}}
}
\foreach \letter in {A,...,Z}{ % `field' font for AAA..ZZZ
\expandafter\xdef\csname \letter\letter\letter \endcsname{\noexpand\ensuremath{\noexpand\mathbb{\letter}}}
}
\newcommand{\balpha}{{\boldsymbol{\alpha}}}
\newcommand{\bbeta}{{\boldsymbol{\beta}}}
\newcommand{\bgamma}{{\boldsymbol{\gamma}}}
\newcommand{\bkappa}{{\boldsymbol{\kappa}}}
\newcommand{\bmu}{{\boldsymbol{\mu}}}
\newcommand{\btheta}{{\boldsymbol{\theta}}}
\newcommand{\bTheta}{{\boldsymbol{\Theta}}}
\newcommand{\bPi}{{\boldsymbol{\Pi}}}
\newcommand{\bSigma}{{\boldsymbol{\Sigma}}}
\newcommand{\bPhi}{{\boldsymbol{\Phi}}}
\newcommand{\bLambda}{{\boldsymbol{\Lambda}}}
\newcommand{\bdeta}{{\boldsymbol{\eta}}}
\newcommand{\bphi}{{\boldsymbol{\phi}}}



%%
%% add definitions and theorems
%%
\usepackage[thmmarks,amsmath]{ntheorem}
\theorembodyfont{\normalfont}
\newtheorem{deff}{Definition}[section]
\newtheorem{thm}{Theorem}[section]
\newtheorem{prop}{Proposition}[section]
\newtheorem{lem}{Lemma}[section]
\newtheorem{cor}{Corollary}[section]
\newtheorem{rmk}{Remark}[section]
\newtheorem{alg}{Algorithm}[section]
\newtheorem{ex}{Example}[section]
\newtheorem{ques}{Question}[section]
\newtheorem{ans}{Answer}[section]
\newtheorem{prob}{Problem}[section]
\newtheorem{sol}{Solution}[section]
\newtheorem*{prof}{Proof}[section] 

\title{Analysis}
\author{Xi Tan (tan19@purdue.edu)}
\date{\today}

\begin{document}
\maketitle

\tableofcontents
\newpage

\section*{Preface}
This book reviews calculus, advanced calculus, real analysis, and functional Analysis. The main references to be used are \cite{Stewart} for calculus, \cite{Rudin} for advanced calculus, \cite{Royden} for real analysis, and \cite{Kreyszig} for functional analysis. Other useful texts include: \cite{Folland} and \cite{Torchinsky} for real analysis.
\newpage

\section{Introduction}
The core material of real analysis is that of Lebesgue integral, which extends the application of Riemann integral to a larger family of functions. The prerequisite of Lebesgue integral is measure theory. We begin from important concepts of sets, point topology, and the real number system, then continue with measurable functions before discussing Lebesgue integral.

\section{Set Theory}


\section{Point Topology}
\section{Real Number System}
The real number system can be characterized by three axioms: 1) the filed axiom, 2) the order axiom, and 3) the completeness axiom.

Of particular interest is the completeness axiom. Depending on the construction of real numbers, it can take the form of axioms (the completeness axiom), or a theorem from the construction. These include:
\begin{enumerate}
	\item Lease upper bound property
	\item Dedekind completeness
	\item Cauchy completeness
	\item Nested intervals theorem
	\item Monotone convergence theorem
	\item Bolzano-Weierstrass theorem
\end{enumerate}

\section{Measure Theory}

\section{Measurable Sets and Measurable Functions}

\section{Lebesgue Integration}

\begin{thebibliography}{100} % 100 is a random guess of the total number of %references  
\bibitem{Stewart} James Stewart {\em{Calsulus - Early Transcendentals}}. Cengage Learning, 2012
\bibitem{Rudin} Walter Rudin {\em{Principles of Mathematical Analysis}}. McGraw-Hill Companies, Inc., 1976.
\bibitem{Royden} H. L. Royden {\em{Real Analysis}}. Pearson Eduction, Inc., 1988.
\bibitem{Kreyszig} Erwin Kreyszig {\em{Introductory Functional Analysis with Applications}}. Wiley, 1989.
\bibitem{Folland} Gerald B. Folland {\em{Real Analysis: Modern Techniques and Their Applications}}. Wiley, 1999.
\bibitem{Torchinsky} Alberto Torchinsky {\em{Real Variables}}. Westview Press, 1995.
\end{thebibliography}

\end{document}


\documentclass{book}
\usepackage{amssymb,amsmath}
\usepackage[]{algorithm2e}

\usepackage{mathrsfs}
\usepackage{hyperref}
\usepackage{mathrsfs}

\usepackage{color}		% enable color characters
\usepackage{graphicx} 	% insert image files
\usepackage{enumerate} 	% enumerate items
\usepackage{caption}
\usepackage{subcaption}

\usepackage[verbose,letterpaper,tmargin=1in,bmargin=1in,lmargin=1in,rmargin=1in]{geometry}
\setlength{\marginparwidth}{1.2in}
\newcommand{\smallmarginpar}[1]{\marginpar[#1]{\small{#1}}}


%%
%% enable margin note
%%
\usepackage{marginnote}
\renewcommand*{\marginfont}{\footnotesize}
\usepackage{manfnt}

%%
%% define bold font for the alphabet
%%
\usepackage{pgffor}
\foreach \letter in {a,...,z}{ % bold font for a..z
\expandafter\xdef\csname \letter \endcsname{\noexpand\ensuremath{\noexpand\mathbf{\letter}}}
}
\foreach \letter in {A,...,Z}{ % bold font for A..Z
\expandafter\xdef\csname \letter \endcsname{\noexpand\ensuremath{\noexpand\mathbf{\letter}}}
}
\foreach \letter in {A,...,Z}{ % `field' font for AA..ZZ
\expandafter\xdef\csname \letter\letter \endcsname{\noexpand\ensuremath{\noexpand\mathbb{\letter}}}
}
\newcommand{\bmu}{{\boldsymbol{\mu}}}
\newcommand{\btheta}{{\boldsymbol{\theta}}}
\newcommand{\bSigma}{{\boldsymbol{\Sigma}}}
\newcommand{\bdeta}{{\boldsymbol{\eta}}}
\newcommand{\nN}{{\mathcal{N}}}

%%
%% add definitions and theorems
%%
\usepackage[thmmarks,amsmath]{ntheorem}
\theorembodyfont{\normalfont}
\newtheorem{deff}{Definition}[section]
\newtheorem{thm}{Theorem}[section]
\newtheorem{prop}{Proposition}[section]
\newtheorem{cor}{Corollary}[section]
\newtheorem{rmk}{Remark}[section]
\newtheorem{ex}{Example}[section]
\newtheorem{ques}{Question}[section]
\newtheorem*{prof}{Proof}[section]


\title{Linear Algebra: Theory and Techniques}
\author{Xi Tan (tan19@purdue.edu)}
\date{\today}

\begin{document}
\maketitle
\tableofcontents
\newpage

\section*{Preface}
Some good books to consider:
\begin{enumerate}
	\item Linear Algebra Its Applications, by Strang
	\item Linear Algebra Right, by Axler
	\item Linear Algebra, by Lang
	\item Finite Dimensional Spaces Mathematics Studies, by Halmos
	\item Linear Algebra Problem Book, by Halmos
	\item Linear Algebra and Its Applications, by Lax
	\item http://joshua.smcvt.edu/linearalgebra/book.pdf
	\item http://www.math.brown.edu/~treil/papers/LADW/book.pdf
\end{enumerate}

\newpage

\part{Theory}
\chapter{Preliminaries}
\deff{
	A {\bf{field}} is a non-empty set $F$ {\em{closed}} under two operations, usually called {\em{addition}} and {\em{multiplication}}\footnote{Subtraction and division are defined implicitly in terms of the inverse operations of addition and multiplication.}, and denoted by $+$ and $\cdot$ respectively, such that the following {\em{nine}} axioms hold
	\begin{enumerate}
		\item[(1-2).] Associativity of addition and multiplication.
		\item[(3-4).] Commutativity of addition and multiplication.
		\item[(5-6).] Existence and uniqueness of additive and multiplicative identity elements.
		\item[(7-8).] Existence and uniqueness of additive inverses and multiplicative inverses.
		\item[(9).] Distributivity of multiplication over addition.
	\end{enumerate}
}
\deff{
	The characteristic of a ring $R$, $char(R)$, is the smallest positive integer $n$ such that
	$$\underbrace{1+\cdots+1}_{n \text{ summands}} = 0$$
}
\thm{
	Any finite ring has nonzero characteristic.
}

\chapter{Vector Calculus}
\section{Vector Algebra}
\subsection{Dot Product}
\subsection{Cross Product}
\subsection{Scalar Triple Product}
\subsection{Vector Triple Product}

\section{Line, Surface, and Volume Integrals}

\chapter{Vector Spaces}
\section{Vector Space}
\deff{
	A {\bf{vector space}} over a field $\FF$ is a {\em{nonempty}} set $V$ together with the operations of addition $V\times V \to V$ and scalar multiplication $\FF \times V \to V$ satisfying the following {\em{eight}} properties:
	\begin{enumerate}[(-)]
		\item Additive axioms. For every $\u,\v,\w \in V$, we have
		\begin{enumerate}[(1)]
			\item $\u+\v = \v+\u$
			\item $(\u+\v)+\w = \u+(\v+\w)$
			\item ${\bf{0}}+\u = \u+{\bf{0}}=\u$, where ${\bf{0}} \in V$ is unique for all $\u \in V$
			\item $(-\u)+\u = \u+(-\u) = {\bf{0}}$, where $-\u \in V$ is unique for every $\u \in V$
		\end{enumerate}
		\item Multiplicative axioms. For every $\u \in V$ and scalars $a, b \in \FF$, we have
		\begin{enumerate}[(1)]
			\item $1\x = \x$
			\item $(ab)\x = a(b\x)$
		\end{enumerate}
		\item Distributive axioms. For every $\u, \v \in V$ and scalars $a, b \in \FF$, we have
		\begin{enumerate}[(1)]
			\item a(\u+\v) = a\u + a\v
			\item (a+b)\u = a\u + b\u
		\end{enumerate}
	\end{enumerate}
}
\section{Subspaces}
\deff{
	A subspace of $\RR^n$ is any collection $S$ of vectors in $\RR^n$ such that
	\begin{enumerate}[(1)]
		\item The zero vector $\mathbf{0}$ is in $S$.
		\item If $\u$ and $\v$ are in $S$, then $\u+\v$ is in $S$. \footnote{$S$ is closed under addition.}
		\item If $\u$ is in $S$ and $c$ is a scalar, then $c\u$ is in $S$. \footnote{$S$ is closed under scalar multiplication.}
	\end{enumerate}
}

\deff{
	Let $S, T$ be two subspaces of $\RR^n$. We say $S$ is orthogonal to $T$ if {\em{every}} vector in $S$ is orthogonal to {\em{every}} vector in $T$. The subspace $\{\mathbf{0}\}$ is orthogonal to all subspaces. \footnote{A line can be orthogonal to another line, or it can be orthogonal to a plane, but a plane cannot be orthogonal to a plane.}
}

\deff{
	Let $A$ be an $m \times n$ matrix.
	\begin{enumerate}[(1)]
		\item The {\em{row space}} of $A$ is the subspace $row(A)$ of $\RR^n$ spanned by the rows of $A$.
		\item The {\em{column space}} (or {\em{range}}) of $A$ is the subspace $col(A)$ of $\RR^m$ spanned by the columns of $A$.
	\end{enumerate}
}
\subsection{Four Important Subspaces: the row, column, null, and left null space}
\deff{
	Let $A$ be an $m \times n$ matrix. The {\em{null space}} (or {\em{kernel}}) of $A$ is the subspace of $\RR^n$ consisting of solutions of the homogeneous linear system $A\x=\mathbf{0}$. It is denoted by {\em{null($A$)}}.
}
\deff{
	A {\em{basis}} for a subspace $S$ of $\RR^n$ is a set of vectors in $S$ that
	\begin{enumerate}[(1)]
		\item spans $S$ and 
		\item is linearly independent. \footnote{It does not mean that they are orthogonal.}
	\end{enumerate}
}
\deff{
	If $S$ is a subspace of $\RR^n$, then the number of vectors in a basis for $S$ is called the {\em{dimension}} of $S$, denoted {\em{dim $S$}}. \footnote{The zero vector $\mathbf{0}$ is always a subspace of $\RR^n$. Yet any set containing the zero vector is linearly dependent, so $\mathbf{0}$ cannot have a basis. We define {\em{dim $\mathbf{0}$}} to be 0.}
}
\deff{
	The {\em{rank}} of a matrix $A$ is the dimension of its row and column spaces and is denoted by {\em{rank($A$)}}. \footnote{The row and column spaces of a matrix $A$ have the same dimension.}
}
\deff{
	The {\em{nullity}} of a matrix $A$ is the dimension of its null space and is denoted by {\em{nullity($A$)}}.
}
\thm{
	The Rank Theorem. If $A$ is an $m \times n$ matrix, then $$rank(A) + nullity(A) = n$$.
}
\thm{
	If $A$ is invertible, then $A$ is a product of elementary matrices.
}
\thm{
	Let $A$ be an $m \times n$ matrix. Then $rank(A^TA) = rank(A)$.
}
\deff{
	Let $S$ be a subspace of $\RR^n$ and let $B=\{\v_1,\cdots,\v_k\}$ be a basis for $S$. Let $\v$ be a vector in $S$, and write $\v = c_1\v_1 + \cdots + c_k\v_k$. Then $c_1,\cdots,c_k$ are called the coordinates of $\v$ with respect to $B$, and the column vector $$[\v]_B = [c_1,\cdots,c_k]^T$$ is called the coordinate vector of $\v$ with respect to $B$. \footnote{This coordinate vector is unique.}
}
\deff{
	A transformation $T: \RR^n \to \RR^m$ is called a linear transformation if $$T(c_1\v_1 + c_2\v_2) = c_1T(\v_1) + c_2T(\v_2)$$ for all $\v_1, \v_2$ in $\RR^n$ and scalars $c_1, c_2$.
}
\section{Bases and Dimension}
\section{Coordinates}

\chapter{Vector and Matrix Calculus}
\section{Functions of Vectors}
\subsection{Inner Product}
\subsection{Outer Product}
\deff{$$\u \otimes \v = \u \v^T$$}
\rmk{The inner product is the trace of the outer product.}
\subsection{Cross Product}
\deff{$$\a \times \b = \|\a\| \|\b\| sin(\theta) \n$$ It is also called the vector product.}
\section{Functions of Matrices}
\subsection{Matrix Determinant}


\subsection{Matrix Exponential}



\section{Functions of Vectors and Matrices}
\subsection{Linear Forms: One Vector as Argument}
\subsection{Bilinear and Quadratic Forms: Two Vectors as Argument}
\section{Derivatives of Vectors and Matrices}
\subsection{Derivatives of a Vector or Matrix with Respect to a Scalar}
Let $\A$ be a matrix, as a matrix-valued function
\begin{align}
	\A(x): \RR \rightarrow \RR^{m \times n}
\end{align}

For vector- and matrix-valued functions there is a further manifestation of the linearity of the derivative: Suppose that $f$ is a fixed linear function defined on $\RR^n$ and that $\A$ is a differentiable vector- or matrix-valued function. Then
\begin{align}
	f(\A)' = f(\A')
\end{align}
A useful example is the trace of $\A$, which is the sum of the diagonal elements of $\A$ (differentiable real-valued functions)
\begin{align}
	tr(\A)' = tr (\A')
\end{align}

Another example is the inner product of two vectors, where we have \footnote{Actually, it should work for all dot product (not necessarily the inner product, which is in the context of Euclidean spaces.)}
\begin{align}
	(\a^T\b)' = \a'^T\b + \a^T\b'
\end{align}

An important derivative of a matrix $\A$ is the derivative of its inverse.
\thm{
	$$\left(\A^{-1}\right)' = -\A^{-1} \A' \A^{-1}$$
}
\prof{
	Since $$\frac{\A^{-1}(x+h) - \A^{-1}(x)}{h} = \frac{\A^{-1}(x+h)[\A(x+h)-\A(x)]\A^{-1}(x)}{h}$$

	Another easy proof is: $$\mathbf{0} = \I' = (\A^{-1}\A)' = (\A^{-1})'\A + \A^{-1}\A'$$ Post-multiply $\A^{-1}$ and obtain the desired proof.
}
\section{Integration of Vectors and Matrices}

\chapter{Some Intuitive Explanations}
\section{Eigenvalues and Singular Values}
\section{SVD, PCA, and Change of Basis}
\section{Special Square Matrices}
\section{Elementary Matrices}
There are three types of elementary matrices: {\color{red}{Row Switching, Row Multiplication, and Row Addition.}}

\begin{rmk}
	Left multiplication (pre-multiplication) by an elementary matrix represents elementary row operations, while right multiplication (post-multiplication) represents elementary column operations.
\end{rmk}

\begin{rmk}
	The inverse of elementary matrices has the same format as the orginal ones.
\end{rmk}

\section{Permutation  Matrices}
\begin{rmk}
	When a permutation matrix $P$ is multiplied with a matrix M from the left it will permute the rows of $M$, when $P$ is multiplied with $M$ from the right it will permute the columns of $M$.
\end{rmk}

\begin{rmk}
	The inverse of a permutation matrix is its transpose.
\end{rmk}

\section{Symmetric Matrices}

\section{Projection Matrices}
\begin{rmk}
	$P = A(A^TA)^{-1}A^T$, $P = \frac{aa^T}{\|a\|}$
\end{rmk}

\begin{rmk}
	$P^2=P$
\end{rmk}

\begin{rmk}
	Only two eigenvalues possible: 0 and 1. The corresponding eigenvectors form the kernel and range of $A$, respectively.
\end{rmk}

\begin{rmk}
	Projection is invertible.
\end{rmk}

\section{Normal Matrix}
\deff{
	A {\em{normal matrix}} is a square matrix which satisfies
	\begin{align}
		\A^T\A = \A\A^T
	\end{align}
}

\section{Orthogonal Matrices}
\deff{
	An {\em{orthogonal matrix}} ({\em{unitary}} for a complex matrix) is a normal matrix which further satisfies
	\begin{align}
		\A^T\A = \A\A^T = \I
	\end{align}
}
Or, alternatively,
%\rmk{An orthogonal matrix is a square matrix with orthonormal \smallmarginpar{orthogonal and unit vectors} columns.}

\begin{rmk}
	$Q^TQ = I$ even if $Q$ is rectangular (but then left-inverse).
\end{rmk}

\begin{rmk}
	Any permutation matrix $P$ is an orthogonal matrix.
\end{rmk}

\rmk{Orthogonal matrices can be categorized into either the reflection matrix $Ref(\theta)$ which has determinant 1, or the rotation matrix $Rot(\theta)$, which has determinant -1.}

\rmk{Geometrically, an orthogonal $Q$ is the product of a rotation and a reflection.}

\begin{rmk}
	Orthogonal matrix is invariant to 2-norm, that is, suppose $Q$ is an orthogonal matrix, and $x$ a vector, then
	\begin{align}
		\|Qx\| = \|x\|
	\end{align}
\end{rmk}

\begin{rmk}
	Projection matrices are usually not orthogonal, since they are not invariant to 2-norm.
\end{rmk}


\rmk{As a linear transformation, an orthogonal matrix preserves the dot product of vectors (therefore also norm and angle), and therefore acts as an isometry of Euclidean space, such as a rotation or reflection. In other words, it is a unitary transformation.}

\rmk{The product of two rotation matrices is a rotation matrix, and the product of two reflection matrices is also a rotation matrix. See figure~\ref{fig:rotation}.}
%\begin{figure}[h]
%\centering
%\includegraphics[scale=0.4]{rotation}
%\caption{The product of two reflection matrices is a rotation matrix.}
%\label{fig:rotation}
%\end{figure}

\section{Positive Definite Matrices}
\deff{
	Let $\A$ be an $n \times n$ square matrix. $\A$ is said to be positive definite if
	\begin{align}
		\x^T \A \x > 0, ~~ \forall \x \ne \mathbf{0}
	\end{align}
}

%\thm{\marginnote{\dbend}
%	If $\A$ is positive definite, then
%	\begin{enumerate}
%		\item The diagonal elements of a positive definite matrix are positive.
%		\item All eigenvalues of $\A$ is positive.
%		\item Its determinant is positive.
%		\item It is nonsingular.
%	\end{enumerate}
%}
\noindent
\prof{
	The diagonal elements are positive because $a_{kk} = \e_k^T \A \e_k > 0$. The eigenvalues of an s.p.d. matrix are all positive is easy to prove by observing that $$0 < \x^T\A\x = \x^T\lambda\x = \lambda \|\x\|_2^2$$ The positivity of determinant can be shown by looking at the LDU decomposition. Finally, it is nonsingular because the determinant is nonzero.
}	
\deff{
	Let $\A$ be an $n \times n$ square matrix. A principal submatrix of $\A$ is obtained by selecting some rows and columns with the {\em{same}} index subset of $\{1, \cdots, n\}$.
}
\deff{
	Let $\A$ be an $n \times n$ square matrix. A {\em{leading}} principal submatrix of $\A$ is a principal submatrix of $\A$ with the index subset $\{1,\cdots,m\}$, for some $m \le n$.
}

%\thm{\marginnote{\dbend}
%	If $\A$ is positive definite then every principle submatrix is s.p.d..
%}
\prof{
	Suppose $\A_p$ of size $p$ is a principle submatrix of $\A$. Since $\A$ is positive definite, for any nonzero vector $\x$ we have $\x^T \A \x > 0$. Remove the corresponding coordinates of $\x$, same as those removed when creating the principle submatrix, and call it $\x_p$. Then the resulting vector $\x_p^T \A_p \x_p = \x^T \A \x > 0$.
}


\section{Numerical Linear Algebra Algorithms}
\section{Matrix Inverse: Binomial inverse theorem, Schur Complement, Blockwise Inversion}
\begin{rmk}
	$\mathbf{A}^{-1} = \begin{bmatrix}
a & b \\ c & d \\
\end{bmatrix}^{-1} =
\frac{1}{\det{\mathbf{A}}} \begin{bmatrix}
\,\,\,d & \!\!-b \\ -c & \,a \\
\end{bmatrix} =
\frac{1}{ad - bc} \begin{bmatrix}
\,\,\,d & \!\!-b \\ -c & \,a \\
\end{bmatrix}$
\end{rmk}

Usually, $|\A\B| \ne |\B\A|$. For example
\ex{
    \begin{align*}
        \begin{vmatrix}\left[\begin{matrix}1 ~~ 1\end{matrix}\right] \left[\begin{matrix}0 \\ 1\end{matrix}\right]\end{vmatrix} = \begin{vmatrix}1\end{vmatrix} = 1
    \end{align*}

    \begin{align*}
        \begin{vmatrix}\left[\begin{matrix}0 \\ 1\end{matrix}\right] \left[\begin{matrix}1 ~~ 1\end{matrix}\right]\end{vmatrix} = \begin{vmatrix}\left[\begin{matrix}0 ~~ 0 \\ 1 ~~ 1\end{matrix}\right]\end{vmatrix} = 0
    \end{align*}
}

However, the {\bf{Sylvester's Determinant Theorem}} says, as long as $\A\B$ and $\B\A$ are both square matrices,
\begin{align}
    \begin{vmatrix}\I+\A\B\end{vmatrix} = \begin{vmatrix}\I+\B\A\end{vmatrix}
\end{align}

It is also not true in general that
\begin{align*}
    \begin{vmatrix}\left[\begin{matrix}\A ~~ \B \\ \C ~~  \D\end{matrix}\right]\end{vmatrix} = \begin{vmatrix}\A\D-\B\C\end{vmatrix}
\end{align*}
unless $\C$ and $\D$ are commutable, i.e., $\C\D = \D\C$. The general formula for block determinant is
\begin{align}
    \begin{vmatrix}\left[\begin{matrix}\A ~~ \B \\ \C ~~  \D\end{matrix}\right]\end{vmatrix} = \begin{vmatrix}\A\end{vmatrix}\begin{vmatrix}\D - \C\A^{-1}\B\end{vmatrix}
\end{align}
which is based on Schur complement.


\section{The $\A\x=\b$ Problem}
\section{Solving a Linear System of Equations}
\thm{
	If $A = \left[\begin{matrix}a & b\\ c & d\end{matrix}\right]$, then $A$ is invertible if $ad-bc\ne 0$, in which case $$A^{-1} = \frac{1}{ad-bc} \left[\begin{matrix}d & -b\\ -c & a\end{matrix}\right]$$
}
\begin{rmk}
	Three ways to solve a system of linear equations: by elimination, by determinants {\color{red}{(Cramer's Rule???)}}, or by matrix decomposition.
\end{rmk}

\begin{rmk}
	We prefer to use matrix decomposition to solve a linear system because
	\begin{enumerate}
		\item It takes $\mathcal{O}(n^3)$ to factorize, but once done it can be used to solve systems with different $\b$ (right hand side).
		\item It is numerically more stable than computing $\A^{-1}\b$.
		\item For a sparse matrix, the inverse may be dense and may hard to store in memory. Decomposition can overcome this problem.
	\end{enumerate}
\end{rmk}

\rmk{
	Cofactors and Minors. Laplace’s Theorem.
}

\begin{rmk}
	The computation of elimination is $\mathcal{O}(n^3)$, but can be (non-trivially) reduced to $\mathcal{O}(n^{\log_27})$.
\end{rmk}

\section{The Vector Spaces of a Matrix}
\begin{rmk}
	$Ax$ is a combination of the {\em{columns}} of $A$. $b^TA$ is a combination of the {\em{rows}} of $A$. Row picture can be seen as interchapter of (hyper-)planes. Column picture can be seen as combination of columns.
\end{rmk}

\begin{rmk}
	There are three different ways to look at matrix multiplication:
	\begin{enumerate}
		\item Each entry of $AB$ is the product of a row (of $A$) and a column (of $B$)
		\item Each {\em{column}} of $AB$ is the product of a matrix (of $A$) and a column (of $B$)
		\item Each {\em{row}} of $AB$ is the product of a row (of $A$) and a matrix (of $B$)
	\end{enumerate}
\end{rmk}

\begin{rmk}
	Column space is perpendicular to the left null space. Row space is perpendicular to the null space.
\end{rmk}




\section{The $\A\x=\lambda \x$ Problem}

\section{Matrix Decomposition}
\section{Decomposition related to solving $\A\x = \b$}
\subsection{LU Decomposition: Schur Complement}
Usually, $|\A\B| \ne |\B\A|$. For example
\ex{
    \begin{align*}
        \begin{vmatrix}\left[\begin{matrix}1 ~~ 1\end{matrix}\right] \left[\begin{matrix}0 \\ 1\end{matrix}\right]\end{vmatrix} = \begin{vmatrix}1\end{vmatrix} = 1
    \end{align*}

    \begin{align*}
        \begin{vmatrix}\left[\begin{matrix}0 \\ 1\end{matrix}\right] \left[\begin{matrix}1 ~~ 1\end{matrix}\right]\end{vmatrix} = \begin{vmatrix}\left[\begin{matrix}0 ~~ 0 \\ 1 ~~ 1\end{matrix}\right]\end{vmatrix} = 0
    \end{align*}
}

However, the {\bf{Sylvester's Determinant Theorem}} says, as long as $\A\B$ and $\B\A$ are both square matrices,
\begin{align}
    \begin{vmatrix}\I+\A\B\end{vmatrix} = \begin{vmatrix}\I+\B\A\end{vmatrix}
\end{align}

It is also not true in general that
\begin{align*}
    \begin{vmatrix}\left[\begin{matrix}\A ~~ \B \\ \C ~~  \D\end{matrix}\right]\end{vmatrix} = \begin{vmatrix}\A\D-\B\C\end{vmatrix}
\end{align*}
unless $\C$ and $\D$ are commutable, i.e., $\C\D = \D\C$. The general formula for block determinant is
\begin{align}
    \begin{vmatrix}\left[\begin{matrix}\A ~~ \B \\ \C ~~  \D\end{matrix}\right]\end{vmatrix} = \begin{vmatrix}\A\end{vmatrix}\begin{vmatrix}\D - \C\A^{-1}\B\end{vmatrix}
\end{align}
which is based on Schur complement.

Now suppose we have a homogeneous linear system
\begin{align}
	\left[\begin{matrix}\A & \B \\ \C & \D \end{matrix}\right] \left(\begin{matrix} \x \\ \y \end{matrix}\right) = \left(\begin{matrix} \mathbf{0} \\ \mathbf{0} \end{matrix}\right)
\end{align}
To solve for $\y$, if $\A$ is nonsingular, we may multiply the first row by $-\C\A^{-1}$ and add to the second, and obtain
\begin{align}\label{eq:shurA}
	\left[\begin{matrix}\I & \mathbf{0} \\ -\C\A^{-1} & \I \end{matrix}\right] \left[\begin{matrix}\A & \B \\ \C & \D \end{matrix}\right] = \left[\begin{matrix}\A & \B \\ \mathbf{0} & \D-\C\A^{-1}\B \end{matrix}\right]	
\end{align}

\deff{
Suppose $\M$ is a square matrix $$\M = \left[\begin{matrix}\A & \B \\ \C & \D \end{matrix}\right]$$ and $\A$ nonsingular.
We denote \footnote{It is easy to remember if you multiply the submatrices clockwise.}
\begin{align}
	\M/\A = \D-\C\A^{-1}\B
\end{align}
and call it {\em{the Schur complement of $\A$ in $\M$}}, or {\em{the Schur complement of $\M$ relative to $\A$}}.

%\rmk{
%A very useful identity can be revealed from equation \ref{eq:shurA} \marginnote{\dbend}
%\begin{align}
%	\M = \left[\begin{matrix}\A & \B \\ \C & \D \end{matrix}\right] = \left[\begin{matrix}\I & \mathbf{0} \\ \C\A^{-1} & \I \end{matrix}\right] \left[\begin{matrix}\A & \mathbf{0} \\ \mathbf{0} & \D-\C\A^{-1}\B \end{matrix}\right] \left[\begin{matrix}\I & \A^{-1}\B \\ \mathbf{0} & \I \end{matrix}\right]
%\end{align}
%}which gives us the following identities
\thm{
\begin{align}
	det(\M) &= det(\M/\A) \cdot det(\A)\\
	rank(\M) &= rank(\M/\A) + rank(\A)
\end{align}
}
\rmk{
For a non-homogeneous system of linear equations $$\left[\begin{matrix} \A & \B \\ \C & \D \end{matrix}\right] \left(\begin{matrix} \x \\ \y \end{matrix}\right) = \left(\begin{matrix} \u \\ \v \end{matrix}\right)$$ We may use Schur complements to write the solution as
\begin{align}
	\x = (\M/\D)^{-1}(\u-\B\D^{-1}\v)\\
	\y = (\M/\A)^{-1}(\v-\C\A^{-1}\u)
\end{align}
}
\thm{
	If $\M$ is a positive-definite symmetric matrix, then so is the Schur complement of $\D$ in $\M$.
}
\subsection{LDU Decomposition}
The LDU decomposition can be viewed as the matrix form of Gaussian elimination. It is used to find the inverse of a matrix, {\bf{or computing the determinant of a matrix}}.

\begin{rmk}
	The triangular factorization can be written $A=LDU$, where $L$ and $U$  have $1's$ on the diagonal and $D$ is the diagonal matrix of pivots.
\end{rmk}
\subsection{Rank Decomposition}
\subsection{Cholesky Decomposition}
\deff{
	The Cholesky decomposition of an s.p.d. matrix $\A$ is of the form
	\begin{align}
		\A = \L\L^*
	\end{align}
	where $\L$ is a lower triangular matrix, with {\em{real and positive diagonal elements}}.
}

\deff{
	The Cholesky decomposition of a s.p.d. matrix $\A$ is of the form
	\begin{align}
		\A = \L\L^T
	\end{align}
	where $\L$ is a lower triangular matrix, with {\em{real and positive diagonal elements}}.
}

Cholesky decomposition is unique. If $\A$ is symmetric semi-positive definite, it still has a decomposition of the form $\A = \L\L^*$, although may not be unique, if the diagonal entries of $\L$ are allowed to be zero. A closely related variant of the classical Cholesky decomposition is the {\em{$LDL^T$ decomposition}}:
\begin{align}
	\A = \L\D\L^T = (\L\D^{\frac{1}{2}}) ({\D^{\frac{1}{2}}}^T\L^T) = (\L\D^{\frac{1}{2}}) (\L\D^{\frac{1}{2}})^T
\end{align}
where the diagonal entries of $\L$ are all ones.


\subsection{QR Decomposition: Givens Rotation, Householder Transformation}
Any square matrix $\A$ may be decomposed as
\begin{align}
	\A = \Q\R
\end{align}
where $\Q$ is an orthogonal matrix, and $\R$ an upper triangular matrix. This is called the QR decomposition. It is essentially a change of basis process, and can be obtained by using the Gram-Schmidt process.


\section{Decomposition related to solving $\A\x = \lambda\x$}
\subsection{Eigendecomposition}
Suppose a square matrix $\M$ of order $n$ is diagonalizable, i.e., it has $n$ linearly independent eigenvectors, then since
\begin{align}
	\M\Q = \Q\bLambda
\end{align}
where the columns of $\Q$ are eigenvectors of $\M$ (hence invertible), and $\bLambda$ is a diagonal matrix with eigenvalues of $\M$ as entries. Then we have
\begin{align}
	\M = \Q\bLambda\Q^{-1}
\end{align}


\subsection{Jordan Decomposition}
\subsection{Schur Decomposition}
\subsection{Singular Value Decomposition (SVD)}
\subsection{QZ Decomposition}

\section{Other Decompositions}
\subsection{Polar Decomposition} 
\section{Minors and Cofactors}
\section{Definition}
\deff{General definition of a minor.

Let $\A$ be an $m \times n$ matrix and $k$ an integer with $0 < k \le \min{m,n}$. A $k \times k$ minor of $\A$ is the determinant of a $k \times k$ matrix obtained from $\A$ by deleting $m-k$ rows and $n-k$ columns. For such a matrix there are a total of ${m \choose k} \cdot {n \choose k}$ minors of size $k \times k$.

\deff{First minors and cofactors.

If $A$ is a square matrix, then the minor of the entry in the $i$-th row and $j$-th column (also called the $(i,j)$ minor, or a first minor, is the determinant of the submatrix formed by deleting the $i$-th row and $j$-th column. This number is often denoted $M_{ij}$. The $(i,j)$ cofactor is obtained by multiplying the minor by $(-1)^{i+j}$.
}

\ex{
To illustrate these definitions, consider the following 3 by 3 matrix,
\begin{align}
	\begin{bmatrix}
	1 & 4 & 7 \\
	3 & 0 & 5 \\
	-1 & 9 & 11 \\
	\end{bmatrix}
\end{align}

To compute the minor $M_{23}$ and the cofactor $C_{23}$, we find the determinant of the above matrix with row 2 and column 3 removed.

\begin{align*}
	M_{2,3} = \det \begin{bmatrix}
	1 & 4 & \Box \\
	\Box & \Box & \Box \\
	-1 & 9 & \Box \\
	\end{bmatrix}= \det \begin{bmatrix}
	1 & 4 \\
	-1 & 9 \\
	\end{bmatrix} = (9-(-4)) = 13
\end{align*}

So the cofactor of the (2,3) entry is $C_{23} = (-1)^{2+3}(M_{23}) = -13$.
}

\vspace{5mm}
\noindent
An important application of cofactors is the {\bf{Laplace's formula}} for the expansion of determinants.
\begin{align}
	\det(\A) = \sum_{i=1}^n a_{ij}C_{ij} = \sum_{j=1}^n a_{ij}C_{ij}
\end{align}
If $k \ne i$, we see that
\begin{align}
	\sum_{j=1}^n a_{kj}C_{ij} = 0	
\end{align}
Similarly, if $k \ne j$
\begin{align}
	\sum_{i=1}^n a_{ik}C_{ij} = 0 \label{cofactor}
\end{align}
This is essentially the determinant of a matrix with the $k$-th row the same as the $i$-th row, or the $k$-th column the same as the $j$-th column, which is zero.

\section{The Cramer's Rule and the Adjugate Matrix}
\begin{align}
	\begin{matrix}a_{11}x_1+a_{12}x_2+\cdots+a_{1n}x_n&=&b_1\\a_{21}x_1+a_{22}x_2+\cdots+a_{2n}x_n&=&b_2\\\vdots&\vdots&\vdots\\a_{n1}x_1+a_{n2}x_2+\cdots+a_{nn}x_n&=&b_n\end{matrix}
\end{align}
If we multiply the above by the row vector of cofactors of the $1^{st}$ column, $[C_{11},C_{21},\cdots,C_{n1}]$, we obtain
\begin{align}
	[\det(\A),0,\cdots,0]\begin{bmatrix}x_1\\ \vdots \\ x_n\end{bmatrix} = [C_{11},C_{21},\cdots,C_{n1}]\b
\end{align}
The left hand side used Equation \ref{cofactor}. The right hand side is nothing but the determinant of a matrix with the first column replaced by $\b$.

Similarly, we can multiply the linear system by the row vector of cofactors of the $2^{nd}, 3^{rd}, \cdots, n^{th}$, and we obtain
\begin{align}	
	\det(\A)\x &= \begin{bmatrix}C_{11}& \cdots & C_{n1}\\C_{12} & \cdots &C_{n2}\\ \vdots &  & \vdots\\C_{1n} & \cdots & C_{nn}\end{bmatrix}\b
\end{align}
which gives us
\begin{align}
	\det(\A) &= \begin{bmatrix}C_{11}& \cdots & C_{1n}\\C_{21} & \cdots &C_{2n}\\ \vdots &  & \vdots\\C_{n1} & \cdots & C_{nn}\end{bmatrix}^T\A
\end{align}
The matrix on the right
\begin{align}
	\mathrm{adj}(\A) = \C^T = \begin{bmatrix}C_{11}& \cdots & C_{1n}\\C_{21} & \cdots &C_{2n}\\ \vdots &  & \vdots\\C_{n1} & \cdots & C_{nn}\end{bmatrix}^T
\end{align}
is called the adjugate matrix of $\A$, which is the transpose of the cofactor matrix $\C$.


\end{document} 







\section{Integers and Equivalence Relations}

\thm{
	Well Ordering Principle. Every nonempty set of positive integers contains a smallest member.
}

\thm{
	Division Algorithm. Let $a$ and $b$ be integers with $b > 0$. Then there exist unique integers $q$ and $r$ with the property that $a = bq + r$, where $0 \le r < b$. (Note: $a$ and $q$ could be negative.)
}

\thm{
	GCD (Greatest Common Divisor) is a Linear Combination. For any nonzero integers $a$ and $b$, there exist integers $s$ and $t$ such that $gcd(a,b)=as+bt$. Moreover, $gcd(a,d)$ is the smallest positive integer of the form $as+bt$.
}

\cor{
	If $a$ and $b$ are relatively prime, then there exist integers $s$ and $t$ such that $as + bt = 1$.
}

\thm{
	If $a \mod n = a'$ and $b \mod n = b'$, then $(a+b) \mod n = (a' + b') \mod n$ and $(ab) \mod n = (a'b') \mod n$.
}










\part{Statistics and Machine Learning}
\input{Statistics/Statistics/Statistics}
\input{Statistics/Probability/Probability}
\documentclass{memoir}
\input{/Users/tan19/Dropbox/LaTeXMacros.tex}

\title{Probability}
\author{Xi Tan (tan19@purdue.edu)}
\date{\today}

\begin{document}
\maketitle
\tableofcontents

\newpage
\chapter*{Preface}
This booklet is divided into 7 Chapters. The first chapter introduces the definitions of basic concepts, such as event, sample space, and probability space. Followed in the next chapter, we will discuss the relationship between two or more events when they interplay with each other. The third chapter formally brings in random variables and vectors, as a basis to develop their quantitative measure and characteristic functions later in chapter four. Chapter five includes some well-known limit theorems, which is useful for asymptotic analysis. The last two chapters will discuss several selected topics in probability theory, and provide a summary of common distributions.

\newpage
\part{Elementary Theory of Probability}
\input{./Elementary_Theory_of_Probability/Combinatorial_Analysis.tex}
\input{./Elementary_Theory_of_Probability/Conditional_Probability.tex}
\input{./Elementary_Theory_of_Probability/Probaility_Space.tex}
\input{./Elementary_Theory_of_Probability/Random_Variables.tex}
\input{./Elementary_Theory_of_Probability/Useful_Distributions.tex}
\input{./Elementary_Theory_of_Probability/Quantitative_Measure_and_Characteristic_Functions.tex}
\input{./Elementary_Theory_of_Probability/Limit_Theorems.tex}
\input{./Elementary_Theory_of_Probability/Selected_Topics_of_Probability.tex}


\newpage
\part{Elementary Theory of Stochastic Processes}
\input{./Elementary_Theory_of_Stochastic_Processes/Introduction.tex}
\input{./Elementary_Theory_of_Stochastic_Processes/Markov_Chains.tex}
\input{./Elementary_Theory_of_Stochastic_Processes/Poisson_Processes.tex}
\input{./Elementary_Theory_of_Stochastic_Processes/Renewal_Theory.tex}   
\input{./Elementary_Theory_of_Stochastic_Processes/Selected_Topics_of_Stochastic_Processes.tex}

\newpage
\part{Measure-theoretical Probability}
\chapter{Introduction}
\section{Why Do We Need Rigorous Probability Theory?}
Let's recall the axioms from non-measure probability courses. We have a \emph{sample space} $\Omega$, the elements of which $w \in \Omega$ called \emph{outcomes}, and the subsets $E \subset \Omega$ called \emph{events}. \emph{Probability measure} is nothing but assigning probability $P(E)$ to each event $E \subset \Omega$, under the following constraints:
\begin{enumerate}
	\item $P(E) \in [0,1]$
	\item $P(\Omega) = 1$
	\item $P(\uplus_{i=1}^\infty E_i) = \sum_{i=1}^n P(E_i)$
\end{enumerate}
where in the last constraint, `$\uplus$' means the union of \emph{disjoint} sets.

One of the motivations\footnote{Add unification of discrete and continuous r.v's.} of developing measure-theoretical probability theory is triggered by the following question:

\vspace{2 mm}
\textbf{Q: \emph{How do we know the three listed axioms are consistent, in particular, the last constraint?}}
\vspace{0.5 mm}

Actually, the \emph{Banach-Tarski paradox} and the \emph{Vitali paradox} are such counter examples (TBA). We have two choices: either we could reject the \emph{axiom of choice}, which is one of the basic assumptions made in the two paradoxes; or we could claim there are some subsets $E \subset \Omega$ that is `non-measurable'. In this course, we take the second solution.

\section{Normal Numbers}
\deff{
	Every $X \in [0,1]$ has a binary decimal expansion $X = 0.d_1d_2d_3\cdots$ or $X = \sum_{k=1}^\infty d_k 2^{-k}$, $d_k \in \{0,1\}$. However, it is possible that $X$ may have two expansions, for example, $X=0.111\cdots = 0.1000\cdots$. We choose the expansion ending in all 1's to make it well-defined.
}

\deff{
	$X \in (0,1)$ is normal if $\lim\limits_{n \to \infty} \frac{1}{n} \sum_{k=1}^n d_k = \frac{1}{2}$.
}

\prop{
	If $X \sim U(0,1)$, then $P(X\mbox{ is normal}) = 1$.
}

\lem{
	Suppose that $X_1, X_2, \cdots$ are i.i.d $Ber(1/2)$. Then $X = \sum_{n=1}^\infty X_n 2^{-n}$ is uniform on $(0,1)$.
}
\prof{	
	We (only) need to show $P(X \in (a,b]) = b - a$.\\
	
	Case 1: $\exists k, s.t. (a,b] = ((k-1)2^{-n}, k2^n]$. Since $(k-1)2^{-n} = 0.d_1d_2 \cdots d_n000\cdots$,
	\begin{align}
		P(X \in ((k-1)2^{-n},k2^{-n}]) &= P(X_1=d_1,X_2=d_2,\cdots,X_n=d_n, X_i=1 for i > n) = 2^{-n} = b-a
	\end{align}

	Case 2: $(a,b] = (l2^{-n},k2^{-n}], l < k$.
	\begin{align}
		P(X \in (l2^{-n},k2^{-n}]) &= (k-l)2^{-n} = b- a
	\end{align}

	In general, if $a < b$, let $a_n, b_n \in 2^{-n}\Z$ be s.t. $a_n \le a < a_n + 2^{-n}, b_n - 2^{-n} \le b < b_n$, then
	\begin{align}
		P(X \in (a,b]) \le P(X \in (a_n,b_n]) = b_n - a_n
	\end{align}
	$b_n - a_n - 2^{-n+1} = P(X \in (a_n+2^{-n},b_n-2^{-n}) \le \epsilon$. Note, $b-a \le b_n - a_n \le b- a + 2^{-n+1}$.

	We need to show that $P(\lim\limits_{n \to \infty} \frac{1}{n} \sum_{k=1}^\infty = \frac{1}{2}) = 1$. This is the SLLN.
}

We know that $\{X \in (a,b]\}$ are events. Why is $\{X\mbox{ is normal}\}$ an event?
\begin{align}
	\{X\mbox{ is normal}\} = \left\{\lim\limits_{n \to \infty} \frac{1}{n} \sum_{k=1}^\infty = \frac{1}{2}\right\} = \bigcap_{l=1}^\infty \bigcup_{m=l}^\infty \bigcap_{n=m}^\infty \left\{\left\vert \frac{1}{n} \sum_{k=1}^\infty - \frac{1}{2} \right\vert < \frac{1}{l} \right\}
\end{align}
This is equivalent to $\forall \epsilon = \frac{1}{l} > 0, \exists m < \infty, s.t., |\frac{1}{n} \sum_{k=1}^\infty - \frac{1}{2}| < \epsilon, \forall n \ge m$.

\section{Formal Definition of Probability Space}
\deff{
	A collection $\FF$ of subsets of $\Omega$ is a $\sigma$-field (or algebra) if
	\begin{enumerate}
		\item $\FF$ is non-empty;
		\item If $A \subset \FF$, then $A^c \in \FF$ (closed under complement);
		\item If $\{A_i\}$ is a countable sequence of elements of $\FF$, then $\cup_i A_i \in \FF$ (closed under countable unions).
	\end{enumerate}	
}
Note, 1) $\Omega \in \FF, \emptyset \in \FF$ since $\Omega = A \cup A^c if A \in \FF$. $\emptyset = \Omega^c$. 2) $\FF$ is closed under countable intersections.

\deff{
	A \emph{measure space} is a pair $(\Omega, \FF)$ where $\FF$ is a $\sigma$-field of subsets of $\Omega$.
}

\deff{
	A non-negative measure $\mu$ on $(\Omega, \FF)$ is a function $\mu: \FF \to \bar \RR_+ = [0,\infty]$ s.t.
	\begin{enumerate}
		\item $\mu(\emptyset) = 0$;
		\item If $A_i$ is a sequence of disjoint sets in $\FF$, then $\mu(\uplus_{i=1}^\infty A_i) = \sum_{i=1}^\infty \mu(A_i)$.
	\end{enumerate}
}

\deff{
	A \emph{probability space} is a triple $(\Omega, \FF, P)$ where $(\Omega, \FF)$ is a measure space, and $P$ is a measure on $(\Omega, \FF)$ with $P(\Omega) = 1$.
}

\ex{
	$\FF = 2^{\Omega}$ (all subsets of $\Omega$). If $\Omega = \Z$, then usually, $\FF = 2^\Omega$ is the $\sigma$-field we use. If $\Omega=(0,1]$ or $\R$, then $\FF$ is usually too big.
}

\ex{
	Let $\AA \subset 2^\Omega$, then $\sigma(\AA)$ is the smallest $\sigma$-field containing $\AA$. If $\OO \subset \R$ is the collection of all open subsets of $\R$, then $\sigma(\OO) = \BB$ is called the \emph{Borel $\sigma$-field}.
}

\chapter{Lecture 5 (1/21/2015 Wednesday): }
(Today and Friday) in Appendix A.
\deff{
	A non-empty collection of subsets $\AA \subset 2^\Omega$ is called an algebra if
	\begin{enumerate}
		\item $A \in \AA$ then $A^c \in \AA$
		\item $A, B \in \AA$ then $A \cup B \in \AA$
		\item $A, B \in \AA$ then $A \cap B \in \AA$
	\end{enumerate}
}

\deff{
	$\mu: \AA \rightarrow [0,\infty]$ is a measure on the algebra $\AA$ if
	\begin{enumerate}
		\item $\mu(\emptyset) = 0$
		\item If $\uplus_{i=1}^\infty A_i \in \AA$ then $\mu(\uplus_{i=1}^\infty A_i) = \sum_{i=1}^\infty \mu(A_i)$
	\end{enumerate}
}

\deff{
	A measure $\mu$ is $\sigma$-finite (on an algebra or a $\sigma$-field) if $\exists$ a sequence $A_n \nearrow \Omega$ with $\mu(A_n) < \infty$ ($A_n \subset A_{n+1}$ and $\Omega = \cup_{i=1}^\infty A_n$)
}

\begin{thm}
	(Caratheodory Extension) Let $\mu$ be a $\sigma$-finite measure on an algebra $\AA$. Then $\mu$ has a unique extension to a measure on $(\Omega, \sigma(\AA))$.
\end{thm}

Note: Meausre on algebras also satisfy Themorem 1.1.1
\begin{enumerate}
	\item $A \subset B$ then $\mu(A) \le \mu(B)$
	\item $\mu(\cup_{i=1}^\infty A_i \le \sum_i \mu(A_i)$ if $\cup A_i \in \AA$
\end{enumerate}


\chapter{Probability Space and Measure}

\section{Algebra of Sets}

\subsubsection{Set Operations}
Given two sets $A, B \in \Omega$, there are four basic binary operations on sets:
\begin{enumerate}
	\item {\em{Union}}: $A \cup B = \{x: x \in A \mbox{ or } x \in B\}$
	\item {\em{Intersection}}: $A \cap B = \{x: x \in A \mbox{ and } x \in B\}$
	\item {\em{Set difference}}: $A \setminus B = \{x: x \in A \mbox{ and } x \notin B\}$
	\item {\em{Set complement}}: $A^c = \Omega \setminus A$
\end{enumerate}

Set complement is the ``strongest'' operation, because if a collection of sets $\AA$ is closed under complement, and if it is also closed under any one of the other three operations, $\AA$ is closed under the rest two. That is seen from,
\begin{align}
	\mbox{If closed under } union &
	\begin{cases}
		A \cap B = (A^c \cup B^c)^c\\
		A \setminus B = A \cap B^c = (A^c \cup B)^c
	\end{cases}	
	\\
	\mbox{If closed under } intersection &
	\begin{cases}
		A \cup B = (A^c \cap B^c)^c\\		
		A \setminus B = A \cap B^c
	\end{cases}
	\\
	\mbox{If closed under } difference &
	\begin{cases}
		A \cap B = A \setminus (A \setminus B)\\
		A \cup B = (A^c \cap B^c)^c = (A^c \setminus (A^c \setminus B^c))^c\\
	\end{cases}		
\end{align}

The difference operation is the second ``strongest'' operation, in that if a collection of sets $\AA$ is closed under difference, it is closed under intersection, that is seen from,
\begin{align}
	A \cap B = A \setminus (A \setminus B)
\end{align}

The third ``strongest'' operation is the union, which can not be implied from difference or intersection, or their combination.

The ``weakest'' operation is the intersection, which can be implied from the difference.

To summarize, the ``strongest'' pair would be the complement plus any one more, which implies everything else; the second ``strongest'' would be the difference plus the union, which implies the intersection but not the complement ($\Omega$ may not be in the collection). \smallmarginpar{\textdbend}

\subsubsection{Class of Set Collection}
\begin{deff}
	Given a set $\Omega$, a non-empty collection $\PP \subset 2^\Omega$ is called a {\em{$\pi$-system}} iff:
	
	\center{$\forall A, B \in \PP$, $A \cap B \in \PP$}
\end{deff}
Notice, this has the weakest requirement.

\begin{deff}
	Given a set $\Omega$, a non-empty collection $\QQ \subset 2^\Omega$ is called a {\em{semiring}} iff:
	
	\center{$\forall A, B \in \QQ \mbox{ and } A \supset B$}
	\center{$\exists C_k \subset \QQ, s.t., A \setminus B = \bigcup_{k=1}^n C_k$}
\end{deff}

\begin{deff}
	Given a set $\Omega$, a non-empty collection $\RR \subset 2^\Omega$ is called a {\em{ring}} iff:
	
	\center{$\forall A, B \in \RR$, $A \cup B \in \RR$ and $B \setminus A \in \RR$}
\end{deff}

There are two points need to mention: the empty set is in a ring, since $A \setminus A = \emptyset$; $\AA$ is also closed under intersection (the reverse need not be true).

\begin{deff}
	A {\em{ring}} $\AA$ is called an {\em{field}} iff $\Omega \in \AA$.
\end{deff}
So a field is closed under all {\em{finite}} combination of set operations.


\begin{deff}
	An {\em{field}} is called a {\em{$\sigma$-field}} if for any sequence ${A_n}$ of sets in $\AA$, $\cup_{n \ge 1} A_n \in \AA$.
\end{deff}

$\not \exists$

\chapter{Integration Theory}
\chapter{Random Variables}
\chapter{Law of Large Numbers}
\section{Types of Convergence}
\subsubsection{Convergence in Distribution}
\subsubsection{Convergence in Probability}
\subsubsection{Almost Surely Convergence}
\subsubsection{$L^p$ Convergence}

\section{Weak Law of Large Numbers (WLLN)}
\section{Strong Law of Large Numbers (SLLN)}

\chapter{Central Limit theorem}

\chapter{Some Tricks}
\section{Prove by Contraposition}

\section{Construct Finer Partition}
Given twofinite partitions $\{A_n\}$ and $\{B_m\}$, a finer partition can be constructed as
\begin{align*}
	(\cup_{i=1}^n A_i) \bigcap (\cup_{j=1}^m B_j) = \bigcup (\cap_{i=1}^n \cap_{j=1}^m A_i B_j)
\end{align*}

\section{Prove Equality}
To prove two numerical quantities are equal $X=Y$, often times we can do this by showing $X \le Y$ and $X \ge Y$. Similarly, to prove two sets are equal $E = F$, we can show $E \subset F$ and $E \subset F$.

\section{An Epsilon of Room}
If one has to show that $X \le Y$, try proving that $X \le Y + \epsilon, \forall \epsilon > 0$. This trick combines well with the ``Prove Equality'' trick.

In a similar spirit, if one needs to show that a quantity $X$ vanishes, try showing that $|X| \le \epsilon, \forall \epsilon > 0$.

If one wants to show that a sequence $x_n$ of real numbers converges to zero, try showing that $\limsup_{n \to \infty} |x_n| \le \epsilon, \forall \epsilon > 0$

{\textbf{One caveat:}} for finite $x$, and any $\epsilon > 0$, it is true that $x + \epsilon > x$ and $x - \epsilon < x$, but it is not true when $x$ is equal to $+\infty$ or $-\infty$. \smallmarginpar{\textdbend}


\section{Interpretations of Probability}
\subsection{Cox's theorem}
\subsection{Principle of Maximum Entropy}

\part{Measure-theoretical Stochastic Processes}

\part{Appendix}

\end{document} 







\documentclass{memoir}
\usepackage{amssymb,amsmath,mathtools}


\usepackage{algorithm2e,algorithmic}

\usepackage{mathrsfs}
\usepackage{paralist}

\usepackage{esint} % for \fint

\allowdisplaybreaks

\usepackage{color}		% enable color characters
\usepackage{graphicx} 	% insert image files
\usepackage{enumerate} 	% enumerate items
\usepackage{caption}
\usepackage{subcaption}
\usepackage{multirow,multicol}
\usepackage[makeroom]{cancel}

\usepackage[colorlinks=true,linkcolor=blue,citecolor=blue]{hyperref}

\usepackage{makeidx}

\newcommand{\cem}[1]{\color{magenta}{\em{#1}}} % color emphasize
\newcommand{\dual}[2]{{#1}^{(#2)}} % color emphasize

\newcommand{\tr}{{\mathrm{tr}}}
\renewcommand{\vec}{{\mathrm{vec}}}

\newcommand{\dd}{{\,\text{d}\,}}
%%
%% horizontal and vertical centering in table p mode
%%
\usepackage{array}
\newcolumntype{P}[1]{>{\centering\arraybackslash}p{#1}} % horizontal centering
\newcolumntype{M}[1]{>{\centering\arraybackslash}m{#1}} % vertical centering

%%
%% define bold font for the alphabet
%%
\usepackage{pgffor}
\foreach \letter in {a,...,z}{ % bold font for a..z
\expandafter\xdef\csname \letter \endcsname{\noexpand\ensuremath{\noexpand\mathbf{\letter}}}
}
\foreach \letter in {A,...,Z}{ % bold font for A..Z
\expandafter\xdef\csname \letter \endcsname{\noexpand\ensuremath{\noexpand\mathbf{\letter}}}
}
\foreach \letter in {A,...,Z}{ % `field' font for AA..ZZ
\expandafter\xdef\csname \letter\letter \endcsname{\noexpand\ensuremath{\noexpand\mathcal{\letter}}}
}
\foreach \letter in {A,...,Z}{ % `field' font for AAA..ZZZ
\expandafter\xdef\csname \letter\letter\letter \endcsname{\noexpand\ensuremath{\noexpand\mathbb{\letter}}}
}
\newcommand{\balpha}{{\boldsymbol{\alpha}}}
\newcommand{\bbeta}{{\boldsymbol{\beta}}}
\newcommand{\bgamma}{{\boldsymbol{\gamma}}}
\newcommand{\bkappa}{{\boldsymbol{\kappa}}}
\newcommand{\bmu}{{\boldsymbol{\mu}}}
\newcommand{\btheta}{{\boldsymbol{\theta}}}
\newcommand{\bTheta}{{\boldsymbol{\Theta}}}
\newcommand{\bPi}{{\boldsymbol{\Pi}}}
\newcommand{\bSigma}{{\boldsymbol{\Sigma}}}
\newcommand{\bPhi}{{\boldsymbol{\Phi}}}
\newcommand{\bLambda}{{\boldsymbol{\Lambda}}}
\newcommand{\bdeta}{{\boldsymbol{\eta}}}
\newcommand{\bphi}{{\boldsymbol{\phi}}}



%%
%% add definitions and theorems
%%
\usepackage[thmmarks,amsmath]{ntheorem}
\theorembodyfont{\normalfont}
\newtheorem{deff}{Definition}[section]
\newtheorem{thm}{Theorem}[section]
\newtheorem{prop}{Proposition}[section]
\newtheorem{lem}{Lemma}[section]
\newtheorem{cor}{Corollary}[section]
\newtheorem{rmk}{Remark}[section]
\newtheorem{alg}{Algorithm}[section]
\newtheorem{ex}{Example}[section]
\newtheorem{ques}{Question}[section]
\newtheorem{ans}{Answer}[section]
\newtheorem{prob}{Problem}[section]
\newtheorem{sol}{Solution}[section]
\newtheorem*{prof}{Proof}[section] 

\title{Probability}
\author{Xi Tan (tan19@purdue.edu)}
\date{\today}

\begin{document}
\maketitle
\tableofcontents

\newpage
\chapter*{Preface}
This booklet is divided into 7 Chapters. The first chapter introduces the definitions of basic concepts, such as event, sample space, and probability space. Followed in the next chapter, we will discuss the relationship between two or more events when they interplay with each other. The third chapter formally brings in random variables and vectors, as a basis to develop their quantitative measure and characteristic functions later in chapter four. Chapter five includes some well-known limit theorems, which is useful for asymptotic analysis. The last two chapters will discuss several selected topics in probability theory, and provide a summary of common distributions.

\newpage
\part{Elementary Theory of Probability}
\chapter{Combinatorial Analysis}
\section{Axioms} 
There are two important rules in combinatorics: the rule of sum, and the rule of product.

The rule of sum says, if we have $a$ ways to finish a task using one method and alternatively, $b$ ways to finish the same task using another method, then there are $ab$ ways of finish this task. More generally,
\begin{equation}
	|S_1 \cup S_2 \cup \ldots \cup S_n| = |S_1| + |S_2| + \ldots + |S_n|
\end{equation}

One extension of the rule of sum is the inclusion-exclusion principle, which does not require sets $A_i$ to be disjoint. This does include the rule of sum, in that if sets $A_i$ are disjoint, the terms from the second to the last are all zero.
\begin{equation}
\begin{split}
|\bigcup_{i=1}^n A_i| = & \sum_{i=1}^n|A_i| - \sum_{1 \le i < j \le n}|A_i\cap A_j| + \sum_{1 \le i < j < k \le n}|A_i\cap A_j\cap A_k|-\ \cdots\ \\
&  +  \left(-1\right)^{n-1} |A_1\cap\cdots\cap A_n|
\end{split}
\end{equation}

The rule of product says, if finishing one task requires two steps, and there are $a$ ways to choose in the first step and $b$ ways to choose in the second step, then there are $ab$ ways to finish this task. More generally,
\begin{equation}
|S_1 \times S_2 \times \cdots \times S_n| = |S_{1}| \cdot |S_{2}| \cdots |S_{n}|
\end{equation}

\section{Binomial Coefficient and Its Applications}
\subsection{Binomial Coefficient}
We list here some of the useful binomial identities, all numbers are nature number (not including 0).

\begin{equation}
	{n \choose k} = {n \choose n-k}
\end{equation}

\begin{equation}\label{eq:subset_count}
	\sum^n_{k=0} {n \choose k} = 2^n
\end{equation}

\begin{equation}
	{n \choose k} = \frac{n}{k}{n-1 \choose k-1}
\end{equation}

\noindent{From the famous Pascal's rule,}
\begin{equation}\label{eq:Pascal's_rule}
	{n \choose k} + {n \choose k+1} = {n+1 \choose k+1}
\end{equation}
There is another form which is equivalent to equation \ref{eq:Pascal's_rule},
\begin{equation}
	{n \choose k} = {n-1 \choose k-1} + {n-1 \choose k}
\end{equation}

\noindent Here is an example that uses the {\em{logarithmic differentiation}}, $f' = f \dot [\ln(f)]'$.
\begin{equation}
	\frac{d}{dt} {t \choose k} = {t \choose k} \sum^{k-1}_{i=0} \frac{1}{t-i}
\end{equation}

\noindent A list of series that involves binomial coefficients,
\begin{eqnarray}
	\sum^n_{k=0} {n \choose k} = 2^n \\
	\sum^n_{k=0} k {n \choose k} = n 2^{n-1} \\
	\sum^n_{k=0} k^2 {n \choose k} = (n+n^2) 2^{n-2}
\end{eqnarray}
These can all be obtained by examining the function value or derivatives of the function $(1+x)^\alpha$, where $\alpha$ could be any real number, and $|x| < 1$.

There are some identities that could be proved using combinatorial analysis, such as {\em{double counting}}. Here is an example,
\begin{equation}\label{eq:double_count}
	\sum^n_{k=1} {n \choose k} {k \choose q} = 2^{n-q} {n \choose q}
\end{equation}
The left side of equation \ref{eq:double_count} counts the number of ways of selecting $k$ elements first, and then choosing $q$ elements from the resulting subset. These $q$ elements could be identical for different $k$. The right hand side of the equation says this is equivalent to first choosing $q$ elements directly from the set, and merging them into one of the $2^{n-q}$ subset of the set containing all but those selected $q$ elements.

Another example is,
\begin{equation}
	\sum^{n_1}_{m_1=0} {n_1 \choose m_1} {n_2 \choose m-m_1} = {n \choose m}
\end{equation}
This simply means choosing $m=m_1+m_2$ objects from a set of $n=n_1+n_2$ objects is equivalent to choosing $m_1$ objects from $n_1$ objects, and $m_2$ objects from $n_2$ objects.

Sometimes, knowing the bounds and asymptotic formulae could be helpful.
\begin{equation}
	\left(\frac{n}{k}\right)^k \le {n \choose k} \le {\frac{n^k}{k!}} \le \left(\frac{n \cdot e}{k}\right)^k
\end{equation}

\begin{equation}
	{2n \choose n} \sim \frac{4^n}{\sqrt{\pi n}}, \mbox{ as } n \rightarrow \infty
\end{equation}

\subsection{Bernoulli Distribution}
\subsection{The i.i.d. Case: Binomial Distribution}
\subsection{The Batch Mode Case: Hypergeometric Distribution}



\section{Multinomial Coefficient and Its Applications}

\subsection{Multinomial Coefficient}
The notion of {\em{multinomial coefficient}} is a generalization of binomial coefficient, which is defined in the multinomial theorem:
\begin{equation*}
	(x_1+x_2+\ldots+x_r)^n = \sum_{(n_1,\ldots,n_r):n_1+\ldots,n_r=n} {n \choose n_1,n_2,\ldots,n_r}x_1^{n_1}x_2^{n_2} \ldots x_r^{n_r}
\end{equation*}
We call $n \choose n_1,n_2,\ldots,n_r$ the multinomial coefficient.

\begin{prob}
	A set of $n$ distinct items is to be divided into $r$ distinct groups of respective sizes $n_1, n_2, \ldots, n_r$, where $\sum^r_{i=1} n_i = n$. How many different divisions are possible?
\end{prob}
Note that there are $n \choose n_1$ possible choices for the first group; for each choice of the first group there are $n \choose n-n_1$ possible choices for the second group; and so on. Hence it follows that there are
\begin{equation*}
\begin{split}
	& {n \choose n_1} {n-n_1 \choose n_2} \ldots {n-n_1-n_2- \ldots -n_{r-1} \choose n_r} \\
	& = \frac{n!}{(n-n_1)!n_1!} \frac{(n-n_1)!}{(n-n_1-n_2)!n_2!} \ldots \frac{(n-n_1-n_2- \ldots -n_{r-1}!}{(0)!n_r!} \\
	& = \frac{n!}{n_1!n_2! \ldots n_r!}
\end{split}
\end{equation*}
possible divisions.

Alternatively, we can first permute these $n$ items, where there are $n!$ such orderings. The first $n_1$ elements are assigned to group $1$, the next $n_2$ elements are assigned to group $2$, and so on. However, for example, keeping all but $n_i$ group fixed, this method would generate $n_i!$ equivalent divisions (note the order within a group does not matter). Therefore, we need to cancel out the equivalent-group effect by dividing $n_1!n_2! \ldots n_r!$. Finally, the multinomial coefficient is,
\begin{equation}
	{n \choose n_1,n_2,\ldots,n_r} = \frac{n!}{n_1!n_2! \ldots n_r!}
\end{equation}

\section{Categorical Distribution}
\section{Multinomial Distribution}


\section{Multiset Coefficient and Its Applications}
\subsection{Multiset Coefficient}
The notion of multiset (or bag) is a generalization of the notion of set in which members are allowed to appear more than once.

The number of times an element belongs to the multiset is the {\em{multiplicity}} of that member. The total number of elements in a multiset, including repeated memberships, is the {\em{cardinality}} of the multiset. For example, in the multiset \{a, a, b, b, b, c\} the multiplicities of the members a, b, and c are respectively 2, 3, and 1, and the cardinality of the multiset is 6.

The number of multisets of cardinality $k$, with elements taken from a finite set of cardinality $n$, is called the {\em{multiset}} coefficient or {\em{multiset number}}, and is denoted as $\left(\!\!{n\choose k}\!\!\right)$. It is equivalent to asking, with replacement, the number of all possible combinations of making $k$ draws from a urn with $n$ distinguishable balls labeled $1 \dots n$.

\begin{prob}\label{prob:multiset}
	With replacement, how many possible combinations to make $k$ draws from a urn with $n$ distinguishable balls labeled $1 \dots n$?
\end{prob}
If we translate this prob directly to the combinatorial language, \smallmarginpar{It is NOT the sum of all the multinomial coefficients, which can be seen by computing $(1+ \dots +1)^k$, that is, $\sum {k \choose N_1, N_2, \ldots, N_n} = n^k$.} we may end up with counting the total number of the multinomial coefficients (actually, it is also correct). To solve this prob, let's first see a similar example.

\begin{prob}
	Suppose $k$ balls that are indistinguishable from each other are to be distributed into $n$ distinguishable (non-empty) urns, how many different outcomes are possible?
\end{prob}
We note that this prob is equivalent to selecting $n-1$ of the $k-1$ spaces between (fixed) adjacent objects as our dividing points, \smallmarginpar{Objects being adjacent ensures the non-emptiness.} e.g., $OOO|OOO|OO$. We count the "bars" as urns (in total $n$) and big "O" as balls (in total $k$). Therefore, there are $k-1 \choose n-1$ such outcomes.\\


Now, we are ready to go back to our original prob. prob \ref{prob:multiset} could be asked this way: if however, we allow empty urns, how many outcomes are possible?

One possible way of borrowing the non-empty case solution to solve the empty one is by starting with $k+n$ balls (instead of $k$), and place them into $n$ urns, \smallmarginpar{It is of probability 1 to remove one ball from each urn.} however at last remove one ball from each urn (so in total $n$). Because some of the urns may only contain one ball, this would give us the number of orderings $n+r-1 \choose r-1$. Another way to look at it is to allow all $k+n-1$ positions (not gaps) available to both symbols, and we count all balls between two bars the same type. Hence, the number of all possible combinations is ${n+k-1 \choose n-1} = {n+k-1 \choose k}$. Note, this scheme is allowing emptiness, because two ``bars'' can be adjacent to each other.

Another beautiful explanation is to construct an equivalent mapping. Note, the rule of drawing a series of $k$ numbers $a_1, a_2, \ldots, a_k$ from the set $\{1,2,\ldots,n\}$ with repetition is
\begin{equation*}
	1 \le a_1 \le a_2 \le \ldots \le a_k \le n
\end{equation*}
Now, a new series of $k$ numbers $b_1, b_2, \ldots, b_k$ can be constructed as follows
\begin{equation*}
\begin{aligned}
	1 \le a_1 & < & a_2+&1 < & \ldots & <  & a_k+&k-1 \le n+k-1 \\
	\downarrow & & \downarrow & & & & \downarrow \\
	b_1 & & b_2 & & \ldots & & b_k
\end{aligned}	
\end{equation*}
Note that $b_1 < b_2 < \ldots < b_k$. This is a model without replacement, and is a one-to-one mapping of the original prob. Under the model of drawing $k$ times from $n+k-1$ balls without replacement, the number of all possible combinations is $n+k-1 \choose k$, which is the same as what we obtained earlier.

A last thing worth noting is that, as we mentioned earlier, the number of all multinomial coefficients is also $k+n-1 \choose n-1$.

\section{Selected Topics}
\subsection{Double Factorial}
\subsection{Stirling Numbers}

\section{The Bertrand's Ballot prob}\label{chapter: Bertrand_Ballot_prob}
The Bertrand's ballot prob was first introduced by Joseph Bertrand in 1887, in the form of: "In an election where candidate A receives $p$ votes and candidate B receives $q$ votes with $p>q$, what is the probability that A will be strictly ahead of B throughout the count?"

J. Bertrand himself gave the solution $\frac{p-q}{p+q}$  by using mathematical induction in the original paper. First of all, let's consider the initial case. At the first count, the vote can be either for candidate A or B. If it {\em{could}} \smallmarginpar{The word "could" means it is possible for an event to happen, but does not indicate its necessity.} be for candidate A, the simplest scenario is $p=1, q=0$, which is of probability 1 for candidate A to win the vote. This agrees with the formula. If it {\em{could}} be for candidate B, the simplest scenario is $p=2,q=1$, now there are ${2+1 \choose 1} = 3$ counting orders, i.e., AAB, ABA, or BAA. Note "AAA" is the only favorable order out of three possible orders. This again agrees with the formula. Assume the theorem is true when $p=a-1 \mbox{ and } q=b$ (last vote would be for candidate A), and when $p=a \mbox{ and } q=b-1$ (last vote would be for candidate B), which is the two possible scenarios at the second to last count. Now, considering the case with $p=a \mbox{ and } q=b$, the last vote is either for candidate A with probability $a/(a+b)$, or for candidate B with probability $b/(a+b)$. So the probability of candidate A to always lead the count is:

\smallmarginpar{$\frac{(a-1)-b}{(a-1)+b} \mbox{ and } \frac{a-(b-1)}{a+(b-1)}$ are conditional probabilities, which are conditioned on the last count.}

\begin{equation*}
	\frac{a}{a+b} \frac{(a-1)-b}{(a-1)+b} + \frac{b}{a+b} \frac{a-(b-1)}{a+(b-1)} = \frac{a-b}{a+b}
\end{equation*}

This proves that the theorem is true for all $p>q\ge0$.

D\'esir\'e Andr\'e in the same year gave an elegant proof of this prob, and the method is now called "the principle of reflection", or "the Andr\'e's reflection method", or "the Bertrand's ballot theorem". There are three facts need to noted. One is that, the sequences start with A or B with probability $p/(p+q)$ and $q/(p+q)$, respectively. The second fact is that, sequences start with B will for sure to tie at some point because A will finally win, and they are all unfavorable, because A already "loses" at the first count. The third is that, sequences that start with A can be classified into two cases, one is the case that A leads the counting process from beginning to the end which is the favorable case, the other is the case when the sequence will tie at some point which is the unfavorable case, importantly, the number of sequences in this latter case is the same as that of the sequences starting with B, because there is a bijection mapping. \smallmarginpar{One possible mapping is to denote the sequence as LBR, where "L" and "R" is the left and right part of the first tie position (must be a "B"), and then converting each character in L to its alternative, e.g., {\color{red}{AAB}}BABAA would become {\color{red}{BBA}}BABAA.} The probability is $1 - 2 \times \frac{q}{p+q} = \frac{p-q}{p+q}$.

The number of the unfavorable cases is $2 \times {p+q-1 \choose q-1}$, of which half starts with "A" and another half starts with "B", as explained above. The number of favorable cases is ${p+q-1 \choose p-1} - {p+q-1 \choose q-1}$. Actually, $\frac{{p+q-1 \choose p-1} - {p+q-1 \choose q-1}}{{p+q \choose p}} = \frac{p-q}{p+q}$.

Consider now the prob to find the probability that the second candidate is never ahead (i.e. ties are allowed); the solution is $\frac{p+1-q}{p+1}$. This is simply seen by awaring the following equivalent description:
\begin{itemize}
	\item same as the basic version, ties are NOT allowed; but,
	\item there are $p+1$ votes for candidate A and $q$ votes for candidate B;
	\item the first vote is for candidate A;
\end{itemize}
The probability can then be computed as:
\begin{equation*}
\begin{split}
	P(\mbox{A winning \mbox{\em{with}} ties}) & = P(\mbox{A winning \mbox{\em{without}} ties} \mid \mbox{the first vote is A})\\
	& =  \frac{P(\mbox{A winning without ties} \mbox{\em{ and }} \mbox{the first vote is A})}{P(\mbox{the first vote is A})}\\
	& = \frac{(p+1-q)/(p+1+q)}{(p+1)/(p+1+q)} = \frac{p+1-q}{p+1}
\end{split}
 \end{equation*}
 
 Another way to look at this prob is to model it as the following: represent a voting sequence as a lattice path on the Cartesian plane and,
\begin{itemize}
	\item Start the path at (0, 0);
	\item Each time a vote for the first candidate is received move right 1 unit;
	\item Each time a vote for the second candidate is received move up 1 unit.
\end{itemize}
Each such path corresponds to a unique sequence of votes and will end at $(p, q)$. A sequence is ``good'' exactly when the corresponding path never goes above the diagonal line $y = x$; equivalently, a sequence is ``bad'' exactly when the corresponding path touches the line $y = x + 1$. For each ``bad'' path P, define a new path P' by reflecting the part of P up to the first point it touches the line across it. P' is a path from (-1, 1) to (p, q). The same operation applied again restores the original P. This produces a one-to-one correspondence between the ``bad'' paths and the paths from (-1, 1) to (p, q). The number of these paths is $p+q \choose q-1$. So the probability asked is $\frac{{p+q \choose q} - {p+q \choose q-1}}{{p+q \choose p}} = \frac{p+1-q}{p+1}$.

\begin{figure}[htb]
	\centering	
	\includegraphics[width=150pt]{2000px-AndreReflection.png}
	\caption{Bertrand's Ballot prob allowing Ties}
	\label{fig:Bertrand's_Ballot_prob}
\end{figure}

An interesting application of this is the famous Catalan number formula, which can be introduced under the random walk story. A random walk on the integers is to take $n$ steps of unit length, beginning at the origin and ending at the point $m$, that never become negative. Assuming $n$ and $m$ have the same parity and $n \ge m \ge 0$, this number is, according to the Bertrand's Ballot prob allowing ties,
\begin{equation*}
	{n \choose \frac{n+m}{2}} - {n \choose \frac{n+m}{2}+1} = \frac{m+1}{\frac{n+m}{2}+1} {n \choose \frac{n+m}{2}}
\end{equation*}
Here, $p+q=n$ and $p-q=m$, compared to our used settings. When $m=0$ and $n$ is even, this gives the Catalan number $\frac{1}{\frac{n}{2}+1} {n \choose \frac{n}{2}}$.

Let's tweak this prob a little bit more. Let's say candidate A starts at $a$ ``bonus'' votes, not 0. That is to say, the system goes from $(0,a)$ to $(p+q,p+a-q)$. If we do not allow ties, all paths hit the x-axis will be unfavorable, and the number of these paths equals the number of paths from $(0,a)$ to its ``mirror'' point $(p+q,-p-a+q)$. So, there are in total ${p+q \choose p+a}$ unfavorable paths, and the probability is $1 - \frac{{p+q \choose p+a}}{{p+q \choose p}}$. Note when $a=0$, there is already a tie, and the question really should be asked as: ``What if the first count is A, and then the process never has a tie''. So the probability should compute as:
\begin{equation*}
	1 - \frac{p}{p+q} \frac{{p-1+q \choose p-1}}{{p-1+q \choose p-1}} = \frac{p-q}{p+q}
\end{equation*}
It should have no prob with $a>0$.

If we allow ties, the ``mirror'' point should be reflected against $y=-1$, so it becomes $(p+q,-2-p-a+q)$. Now, suppose we go up $u$ steps and go down $d$ steps. Solving the following equations:
\begin{eqnarray*}
	u+d & = & p+q\\
	u+a-d & = & -2-p-a+q
\end{eqnarray*}
gives us $u=q-a-1$ and $d=p+a+1$. So there are ${p+q \choose p+a+1}$ paths unfavorable, and the probability is then $1-\frac{{p+q \choose p+a+1}}{{p+q \choose p}}$. When $a=0$, this becomes $\frac{p+1-q}{p+1}$, which agrees with what we obtained earlier.

In summary, if one starts at $(0,a)$ and ends at $(p+q,p+a-q)$, the number of unfavorable paths is $p+q \choose p+a$ if not allowing ties, $p+q \choose p+a+1$ if allowing ties.


\section{Catalan Number}
In chapter \ref{chapter: Bertrand_Ballot_prob}, we first met Catalan number from the generalized Bertrand's Ballot prob. We write here again the definition of Catalan number, with an intuitive interpretation.

The Catalan number is defined as,
\begin{equation}
	C_n = \frac{1}{n+1} {2n \choose n}
\end{equation}
The underlying story reads: Given two urns, one with $n$ red balls and the other with $n$ black balls, we want to draw one ball at a time (either red or black), such that at no time the number of pre-specified color is less than its alternative.

Since the Catalan number is associated with two equal-sized sets, it is oftentimes co-occurrent  with the words "pair", "full binary", and etc.
\chapter{Conditional Probability}
\begin{deff}
	If $P(F) > 0$, then
	\begin{equation}
		P(E|F) = \frac{P(E,F)}{P(F)}
	\end{equation}
	$P(E|F)$ is called the conditional probability of $E$ given $F$. Conditional probability agrees with Definition \ref{def:probability}, and should be treated in the same way.
\end{deff}

\begin{deff}
	The multiplication rule
	\begin{equation}
		P(E_1, E_2, \ldots, E_n) = P(E_1) P(E_2|E_1) \ldots P(E_n|E_1, \ldots, E_{n-1})
	\end{equation}
\end{deff}

\begin{deff}
	Bayes' Formula
	\begin{equation}
		P(A_i|B) = \frac{P(B|A_i)P(A_i)}{P(B)}
	\end{equation}
	where $P(A_i)$ is sometimes called the prior distribution, $P(B|A_i)$ the likelihood function, and $P(A_i|B)$ the posterior distribution. The partition function $P(B)$ can be computed using the law of total probability
	\begin{equation}
		P(B) = \sum^n_{j=1}P(B|A_j)P(A_j)
	\end{equation}¥
\end{deff}

\begin{deff}
	Two events $E$ and $F$ are said to be {\em{independent}} if	
	\begin{equation}
		P(E,F) = P(E)P(F)
	\end{equation}
	A set of events are independent if every finite subset of these events is independent.
\end{deff}

\section{Conditional Probability}
\section{Conditional Expectation}
\section{Conditional Independence}
\section{Probaility Space}
\section{Sample Space, Events, and Probability}
\begin{deff}
	A sample space $S$ is a set of all possible outcomes of an experiment. An outcome is also called a sample point.
\end{deff}
Note, when an experiment consists of several repetitions, each one of them is called a {\em{trial}}. As an example, if one decides to toss a coin 42 times, we can call each toss a trial of the experiment composed of 42 ones.

\begin{deff}
	An event $E$ is a subset of the sample space $S$. If the outcome of the experiment is contained in $E$, then we say that $E$ occurred.
\end{deff}

\begin{deff}\label{def:probability}
In short, a probability space is a measure space such that the measure of the whole space is equal to one. The expanded definition is following: a probability space is a triple $(\Omega,\; \mathcal{F},\; P)$ consisting of:
\begin{itemize}
\item the sample space $\Omega$ --- an arbitrary non-empty set,
\item the $\sigma-algebra$ $\FF \in 2^\Omega$ (also called $\sigma-filed$) --- a set of subsets of $\Omega$, called events, such that:
	\begin{itemize}
	\item $\FF$ contains the empty set: $\emptyset \in \FF$,
	\item $\FF$ is closed under complements: if $A \in \FF$, then also $(\Omega \setminus A) \in \FF$,
	\item $\FF$ is closed under countable unions: if $A_i \in \FF$ for $i=1,2,\ldots$, then also ($\bigcup A_i) \in \FF$,
	\end{itemize}
\item the probability measure $P: \FF \rightarrow [0,1]$ --- a function on $\FF$ such that:
\item $P$ is countably additive: if $\{A_i\} \in \FF$ is a countable collection of pairwise disjoint sets, then $P(\bigcup A_i) = \sum P(A_i)$, where $\bigcup$ denotes the disjoint union,
\item the measure of entire sample space is equal to one: $P(\Omega)=1$.
\end{itemize}
\end{deff}

\section{Probability Axioms}
\subsection{Law of Total Probability}
Suppose $B_n (n=1,2,3,\cdots$ is a finite or countably infinity partition of the sample space, then for an event $A$
\begin{align}
	Pr(A) = \sum_n Pr(A,B_n)
\end{align}
or,
\begin{align}\label{eq:Law of Total Probability}
	Pr(A) = \sum_n Pr(A|B_n) Pr(B_n)
\end{align}
It can also be stated for conditional probabilities. Suppose $X$ is an event in the same sample space, we have
\begin{align}
	Pr(A|X) = \sum_n Pr(A|B_n,X)Pr(B_n|X)
\end{align}
One application of Eq~\ref{eq:Law of Total Probability} is when calculating $Pr(A)$ is difficult, we can introduce an ``auxiliary variable'' $B$, in the hope that the conditional probability $Pr(A|B_n)$ is easier to compute.
\subsection{Law of Total Variance}
\subsection{Law of Total Covariance}
\subsection{Law of Total Expectation}
\subsection{Law of Total Cumulance}
\subsection{Probability Inequalities}

\section{Types of Probabilities: Frequentism and Bayesian} 
\chapter{Random Variables}
The Laplace distribution can be written as an infinite mixture of Gaussians with variance $w$ distribution according to an exponential distribution. An exponential distribution can be written as a $\chi^2$ distribution with two degrees of freedom.
\section{Continuous Random Variables}
\section{Discrete Random Variables}
\section{Joint Distributed Random Variables}
\section{Random Vectors/Matrices}
\section{Function of Random Variables}
\subsection{Transformation}
\subsection{Convolutions: Sum of Normally Distributed Random Variables}
\subsection{Product Distribution}
\subsection{Ratio Distribution}
\chapter{Useful Distributions}
\section{Discrete Distributions}
\subsection{Poisson Distribution}
\subsection{Bernoulli Distribution}
\subsection{Binomial Distribution}
\subsection{Negative Binomial Distribution}
\subsection{Categorical Distribution}
\subsection{Multinomial Distribution}
\subsection{Geometric Distribution}
\subsection{Hyper-Geometric Distribution}
\subsection{Poisson Distribution}
\section{Continuous Distributions}
\subsection{Uniform Distribution}
\subsection{Exponential Distribution}
\subsection{$\chi^2$ Distribution}

\subsection{Gaussian Distribution}
\subsubsection{Univariate Gaussian}
\subsubsection{Multivariate Gaussian}
\begin{align}
	f(\x|\bmu,\bSigma) = \frac{1}{\sqrt{2\pi|\bSigma|}} \exp\left\{-\frac{1}{2}(\x-\bmu)^T\bSigma^{-1}(\x-\bmu)\right\}
\end{align}
Using the technique of `completing the square', we have
\begin{align}
	f(\x_1|\x_2) &= \frac{f(\x_1,\x_2)}{f(\x_2)}\\
	&= \frac{1/\sqrt{2\pi|\bSigma|}\exp\left\{-\frac{1}{2}(\x-\bmu)^T\bSigma^{-1}(\x-\bmu)\right\}}{1/\sqrt{2\pi|\bSigma_{22}|}\exp\left\{-\frac{1}{2}(\x_2-\bmu)^T\bSigma_{22}^{-1}(\x_2-\bmu)\right\}}\\
	&= C \cdot \exp\left\{-\frac{1}{2}(\x-\bmu)^T\bSigma^{-1}(\x-\bmu)\right\}\\
	&= C \cdot \exp\left\{-\frac{1}{2}[(\x_1-\bmu_1)^T\bLambda_{11}(\x_1-\bmu_1)+(\x_1-\bmu_1)^T\bLambda_{12}(\x_2-\bmu_2)+(\x_2-\bmu_2)^T\bLambda_{21}(\x_1-\bmu_1)]\right\}\\
	&= C \cdot \exp\left\{-\frac{1}{2}\x_1^T\bLambda_{11}\x_1 + \x_1^T[\bLambda_{11}\bmu_1 - \bLambda_{12}(\x_2-\bmu_2)]\right\}	
\end{align}
Hence
\begin{align}
	\bSigma_{1|2} &= \bLambda_{11}^{-1}\\
	\bmu_{1|2} &= \bmu_1 - \bLambda_{11}^{-1}\bLambda_{12}(\x_2-\bmu_2)
\end{align}
or
\begin{align}
	\bSigma_{1|2} &= \bSigma_{11} - \bSigma_{12}\bSigma_{22}^{-1}\bSigma_{21}\\
	\bmu_{1|2} &= \bmu_1 + \bSigma_{12}\bSigma_{22}^{-1}(\x_2-\bmu_2)
\end{align}

\subsection{Dirichlet Distribution}
\subsubsection{Gamma Function and Beta Function}
The {\em{gamma function}} is defined for all complex numbers except the non-positive integers. For complex numbers with a positive real part, it is defined via an improper integral that converges:
\begin{equation}
	\Gamma(z) = \int_0^\infty e^{-t} t^{z-1} dt
\end{equation}
For all positive numbers $z$,
\begin{equation}
	\Gamma(z) = (z-1)\Gamma(z-1)
\end{equation}
and in particular, if $n$ is a positive integer:
\begin{equation}
	\Gamma(n) = (n-1)!
\end{equation}
Note, the gamma function shifts the normal definition of factorial by 1.

The {\em{beta function}} is defined by
\begin{equation}
	B(x,y) = \int_0^1 t^{x-1} (1-t)^{y-1} dt
\end{equation}
for $Re(x), Re(y) > 0$.

It can also be written as
\begin{equation}
	B(x,y) = \frac{\Gamma(x) \Gamma(y)}{\Gamma(x+y)}
\end{equation}

\subsubsection{Introduction}
The Dirichlet distribution of order $K \ge 2$ with parameters $\alpha_1, \dots, \alpha_K > 0$ has a probability density function with respect to Lebesgue measure on the Euclidean space $\RR^{K-1}$ given by
\begin{equation}
	f(x_1, \dots, x_{K-1}; \alpha_1, \dots, \alpha_K) = \frac{1}{B(\balpha)} \prod^K_{i=1} x_i^{\alpha_i-1}
\end{equation}
for all $x_1, \dots, x_K > 0$ and $x_1 + \dots + x_K = 1$. The density is zero outside this open\footnote{``Open'' here means none of the $x_i$'s can be 1, actually $x_i \in (0,1)$.} $(K-1)$-dimensional simplex.

The normalizing constant is the multinomial beta function, which can be expressed in terms of the gamma function:
\begin{equation}
	B(\balpha) = \frac{\prod_{i=1}^K \Gamma(\alpha_i)}{\Gamma(\sum_{i=1}^K \alpha_i)}, \mbox{ where } \balpha=(\alpha_1, \dots, \alpha_K)
\end{equation}


\subsection{$t$ Distribution}
\subsection{Inverse Gaussian Distribution}
\subsection{log-normal Distribution}
\subsection{Laplace Distribution}
\subsection{Beta Distribution}
\subsection{Gamma Distribution}
\subsection{Wishart Distribution}
\chapter{Quantitative Measure and Characteristic Functions}
\section{Describing Shape of a Distribution: Skewness, Kurtosis}
\section{Describing a Sample: Mean, Variance}
\section{Degrees of Freedom, Mean, Variance, and Moment, Central Moment, Cumulant, Law of the unconscious statistician}
\section{Percentile and Median}
\section{Coefficient of Variation}
\section{Covariance and Correlation}
\section{Moment Generating Function}
\section{Characteristic Function}

\section{Limit Theorems}
\section{Markov and Chebyshev Inequalities}
\begin{thm}
	{\underline{Markov's Inequality}}. If $X$ is a random variable that takes only nonnegative values, then for any value $a > 0$, $$P\{X \ge a\} \le \frac{E[X]}{a}$$
\end{thm}

\begin{thm}
	{\underline{Chebyshev's Inequality}}. If $X$ is a random variable with finite mean $\mu$ and variance $\sigma^2$, then, for any value $k > 0$, $$P\{|X-\mu| \ge k\} \le \frac{\sigma^2}{k^2}$$
\end{thm}

\section{Weak Law of Large Numbers}
\begin{thm}
	{\underline{The Weak Law of Large Numbers}}. Let $X_1, X_2, \cdots$ be a sequence of i.i.d. random variables, each having finite mean $E[X_i] = \mu$. Then, for any $\epsilon > 0$, $$P\left\{\left|\frac{X_1+\cdots+X_n}{n} - \mu\right| \ge \epsilon \right\} \to 0 \mbox{ as } n \to \infty$$
\end{thm}

\section{Central Limit theorem}
\begin{thm}
	{\underline{Central Limit theorem}}. Let $X_1, X_2, \cdots$ be a sequence of independent and identically distributed random variables, each having mean $\mu$ and variance $\sigma^2$. Then the distribution of $$\frac{\frac{X_1+\cdots+X_n}{n}-\mu}{\sqrt{\sigma^2/n}} = \frac{X_1+\cdots+X_n - n\mu}{\sigma \sqrt{n}}$$ tends to the standard normal as $n \to \infty$.
\end{thm} 
\section{Selected Topics of Probability}
\section{Indicator Variables}
\section{Ordered Statistics}
\section{Copula}
\section{coupling}
\section{The Reflection Principle}



\newpage
\part{Elementary Theory of Stochastic Processes}
%!TEX root = ../Probability.tex

\chapter{Introduction}
\deff{
	A \emph{stochastic process} is a collection of random variables $X(t)$ for $t \in T$, where the \emph{time} parameter $T$ is usually\footnote{It can be defined in a general space.} a subset of $\RR$. The set $\mathscr{S}$ containing all possible values of $X(t)$ is called the \emph{state space} of the process.
}

There are four common types of stochastic processes (with corresponding examples):
\begin{table}[h]
\centering
  \begin{tabular}{ | c | c | c | }
    \hline
     & \textbf{Finite or Countable State Space} & \textbf{Continuous State Space} \\ \hline
    \textbf{Discrete Time} & Markov Chains & Harris Chains \\ \hline
    \textbf{Continuous Time} & Poisson Processes and Hawkes Processes & Gaussian Processes and Wiener Processes \\ \hline    
  \end{tabular}
  \small{\caption{Four Common Types of Stochastic Processes.}}
\end{table}

\deff{
	A {\em{renewal process}} is an idealized stochastic model for events that occur randomly in time (generically called renewals or arrivals). The basic mathematical assumption is that the times between the successive arrivals are independent and identically distributed (i.i.d.). 
}

\deff{
	A {\em{delayed renewal process}} is just like an ordinary renewal process, except that the first arrival time is allowed to have a different distribution than the other inter-arrival times.
}


\deff{
	A {\em{counting process}} is a stochastic process $\{N(t), t \ge 0\}$ with values that are non-negative, integer, and increasing:
}

\deff{
	A {\em{point process}} is a collection of mathematical points randomly located on some underlying mathematical space such as the real line, the Cartesian plane, or more abstract spaces.
}

\deff{
	A {\em{diffusion process}} is a solution to a stochastic differential equation. For example, Brownian motion.
}

\deff{
	A {\em{jump process}} is a type of stochastic process that has discrete movements, called jumps, with random arrival times, rather than continuous movement, typically modeled as a simple or compound Poisson process.
}
\chapter{Markov Chains}
\section{Introduction}
\deff{
The \emph{Markov property} is defined by the requirement that
\begin{align}
	P(X_{n+1}=x_{n+1} \vert X_0=x_0,\cdots,X_n=x_n) = P(X_{n+1}=x_{n+1} \vert X_n=x_n)
\end{align}
}
\deff{
	The conditional probabilities $P(X_{n+1}=y \vert X_n=x)$ are called the \emph{transition probabilities} of the chain.
}

\deff{
	We say a Markov chain has \emph{stationary transition probabilities} if the transition probabilities $P(X_{n+1}=y \vert X_n=x)$ is independent of the time $n$.
}

From now on, when we say that $X_n, n \ge 0$ forms a Markov chain, we mean that these random variables satisfy the Markov property and have stationary transition probabilities.

\deff{
	A state $a$ of a Markov chain is called an \emph{absorbing state} if $P(a,a) = 1$ or, equivalently, if $P(a,y) = 0$ for all $y \ne a$.
}

\section{Discrete Time Markov Chains}
\subsection{Gambler's Ruin}
\subsection{Discrete Time Branching Processes}

\section{Continuous Time Markov Chains}

\section{Poisson Processes}
\section{Poisson Processes on the Line}
\section{Variable Rate Poisson Processes}
\section{Poisson Processes in Higher Dimensions} 
\chapter{Renewal Theory}   
\section{Renewal theory for positive lattice valued random variables as connected with Markov chains: Blackwell's renewal theorem for positive lattice valued random variables}   
\section{Selected Topics of Stochastic Processes}
\section{Martingales which are functions of discrete time Markov chains}
\section{Brownian motion: Path properties, reflection principle, random walk approximation.}
\section{Random Fields}
\section{Discrete and Continuous Time Birth and Death processes}
\section{Discrete and Continuous Time Queuing processes}
\section{Finite State Space Pure Jump Processes}
\section{Infinite Server Queue}


\newpage
\part{Measure-theoretical Probability}
\chapter{Introduction}
\section{Why Do We Need Rigorous Probability Theory?}
Let's recall the axioms from non-measure probability courses. We have a \emph{sample space} $\Omega$, the elements of which $w \in \Omega$ called \emph{outcomes}, and the subsets $E \subset \Omega$ called \emph{events}. \emph{Probability measure} is nothing but assigning probability $P(E)$ to each event $E \subset \Omega$, under the following constraints:
\begin{enumerate}
	\item $P(E) \in [0,1]$
	\item $P(\Omega) = 1$
	\item $P(\uplus_{i=1}^\infty E_i) = \sum_{i=1}^n P(E_i)$
\end{enumerate}
where in the last constraint, `$\uplus$' means the union of \emph{disjoint} sets.

One of the motivations\footnote{Add unification of discrete and continuous r.v's.} of developing measure-theoretical probability theory is triggered by the following question:

\vspace{2 mm}
\textbf{Q: \emph{How do we know the three listed axioms are consistent, in particular, the last constraint?}}
\vspace{0.5 mm}

Actually, the \emph{Banach-Tarski paradox} and the \emph{Vitali paradox} are such counter examples (TBA). We have two choices: either we could reject the \emph{axiom of choice}, which is one of the basic assumptions made in the two paradoxes; or we could claim there are some subsets $E \subset \Omega$ that is `non-measurable'. In this course, we take the second solution.

\section{Normal Numbers}
\deff{
	Every $X \in [0,1]$ has a binary decimal expansion $X = 0.d_1d_2d_3\cdots$ or $X = \sum_{k=1}^\infty d_k 2^{-k}$, $d_k \in \{0,1\}$. However, it is possible that $X$ may have two expansions, for example, $X=0.111\cdots = 0.1000\cdots$. We choose the expansion ending in all 1's to make it well-defined.
}

\deff{
	$X \in (0,1)$ is normal if $\lim\limits_{n \to \infty} \frac{1}{n} \sum_{k=1}^n d_k = \frac{1}{2}$.
}

\prop{
	If $X \sim U(0,1)$, then $P(X\mbox{ is normal}) = 1$.
}

\lem{
	Suppose that $X_1, X_2, \cdots$ are i.i.d $Ber(1/2)$. Then $X = \sum_{n=1}^\infty X_n 2^{-n}$ is uniform on $(0,1)$.
}
\prof{	
	We (only) need to show $P(X \in (a,b]) = b - a$.\\
	
	Case 1: $\exists k, s.t. (a,b] = ((k-1)2^{-n}, k2^n]$. Since $(k-1)2^{-n} = 0.d_1d_2 \cdots d_n000\cdots$,
	\begin{align}
		P(X \in ((k-1)2^{-n},k2^{-n}]) &= P(X_1=d_1,X_2=d_2,\cdots,X_n=d_n, X_i=1 for i > n) = 2^{-n} = b-a
	\end{align}

	Case 2: $(a,b] = (l2^{-n},k2^{-n}], l < k$.
	\begin{align}
		P(X \in (l2^{-n},k2^{-n}]) &= (k-l)2^{-n} = b- a
	\end{align}

	In general, if $a < b$, let $a_n, b_n \in 2^{-n}\Z$ be s.t. $a_n \le a < a_n + 2^{-n}, b_n - 2^{-n} \le b < b_n$, then
	\begin{align}
		P(X \in (a,b]) \le P(X \in (a_n,b_n]) = b_n - a_n
	\end{align}
	$b_n - a_n - 2^{-n+1} = P(X \in (a_n+2^{-n},b_n-2^{-n}) \le \epsilon$. Note, $b-a \le b_n - a_n \le b- a + 2^{-n+1}$.

	We need to show that $P(\lim\limits_{n \to \infty} \frac{1}{n} \sum_{k=1}^\infty = \frac{1}{2}) = 1$. This is the SLLN.
}

We know that $\{X \in (a,b]\}$ are events. Why is $\{X\mbox{ is normal}\}$ an event?
\begin{align}
	\{X\mbox{ is normal}\} = \left\{\lim\limits_{n \to \infty} \frac{1}{n} \sum_{k=1}^\infty = \frac{1}{2}\right\} = \bigcap_{l=1}^\infty \bigcup_{m=l}^\infty \bigcap_{n=m}^\infty \left\{\left\vert \frac{1}{n} \sum_{k=1}^\infty - \frac{1}{2} \right\vert < \frac{1}{l} \right\}
\end{align}
This is equivalent to $\forall \epsilon = \frac{1}{l} > 0, \exists m < \infty, s.t., |\frac{1}{n} \sum_{k=1}^\infty - \frac{1}{2}| < \epsilon, \forall n \ge m$.

\section{Formal Definition of Probability Space}
\deff{
	A collection $\FF$ of subsets of $\Omega$ is a $\sigma$-field (or algebra) if
	\begin{enumerate}
		\item $\FF$ is non-empty;
		\item If $A \subset \FF$, then $A^c \in \FF$ (closed under complement);
		\item If $\{A_i\}$ is a countable sequence of elements of $\FF$, then $\cup_i A_i \in \FF$ (closed under countable unions).
	\end{enumerate}	
}
Note, 1) $\Omega \in \FF, \emptyset \in \FF$ since $\Omega = A \cup A^c if A \in \FF$. $\emptyset = \Omega^c$. 2) $\FF$ is closed under countable intersections.

\deff{
	A \emph{measure space} is a pair $(\Omega, \FF)$ where $\FF$ is a $\sigma$-field of subsets of $\Omega$.
}

\deff{
	A non-negative measure $\mu$ on $(\Omega, \FF)$ is a function $\mu: \FF \to \bar \RR_+ = [0,\infty]$ s.t.
	\begin{enumerate}
		\item $\mu(\emptyset) = 0$;
		\item If $A_i$ is a sequence of disjoint sets in $\FF$, then $\mu(\uplus_{i=1}^\infty A_i) = \sum_{i=1}^\infty \mu(A_i)$.
	\end{enumerate}
}

\deff{
	A \emph{probability space} is a triple $(\Omega, \FF, P)$ where $(\Omega, \FF)$ is a measure space, and $P$ is a measure on $(\Omega, \FF)$ with $P(\Omega) = 1$.
}

\ex{
	$\FF = 2^{\Omega}$ (all subsets of $\Omega$). If $\Omega = \Z$, then usually, $\FF = 2^\Omega$ is the $\sigma$-field we use. If $\Omega=(0,1]$ or $\R$, then $\FF$ is usually too big.
}

\ex{
	Let $\AA \subset 2^\Omega$, then $\sigma(\AA)$ is the smallest $\sigma$-field containing $\AA$. If $\OO \subset \R$ is the collection of all open subsets of $\R$, then $\sigma(\OO) = \BB$ is called the \emph{Borel $\sigma$-field}.
}

\chapter{Lecture 5 (1/21/2015 Wednesday): }
(Today and Friday) in Appendix A.
\deff{
	A non-empty collection of subsets $\AA \subset 2^\Omega$ is called an algebra if
	\begin{enumerate}
		\item $A \in \AA$ then $A^c \in \AA$
		\item $A, B \in \AA$ then $A \cup B \in \AA$
		\item $A, B \in \AA$ then $A \cap B \in \AA$
	\end{enumerate}
}

\deff{
	$\mu: \AA \rightarrow [0,\infty]$ is a measure on the algebra $\AA$ if
	\begin{enumerate}
		\item $\mu(\emptyset) = 0$
		\item If $\uplus_{i=1}^\infty A_i \in \AA$ then $\mu(\uplus_{i=1}^\infty A_i) = \sum_{i=1}^\infty \mu(A_i)$
	\end{enumerate}
}

\deff{
	A measure $\mu$ is $\sigma$-finite (on an algebra or a $\sigma$-field) if $\exists$ a sequence $A_n \nearrow \Omega$ with $\mu(A_n) < \infty$ ($A_n \subset A_{n+1}$ and $\Omega = \cup_{i=1}^\infty A_n$)
}

\begin{thm}
	(Caratheodory Extension) Let $\mu$ be a $\sigma$-finite measure on an algebra $\AA$. Then $\mu$ has a unique extension to a measure on $(\Omega, \sigma(\AA))$.
\end{thm}

Note: Meausre on algebras also satisfy Themorem 1.1.1
\begin{enumerate}
	\item $A \subset B$ then $\mu(A) \le \mu(B)$
	\item $\mu(\cup_{i=1}^\infty A_i \le \sum_i \mu(A_i)$ if $\cup A_i \in \AA$
\end{enumerate}


\chapter{Probability Space and Measure}

\section{Algebra of Sets}

\subsubsection{Set Operations}
Given two sets $A, B \in \Omega$, there are four basic binary operations on sets:
\begin{enumerate}
	\item {\em{Union}}: $A \cup B = \{x: x \in A \mbox{ or } x \in B\}$
	\item {\em{Intersection}}: $A \cap B = \{x: x \in A \mbox{ and } x \in B\}$
	\item {\em{Set difference}}: $A \setminus B = \{x: x \in A \mbox{ and } x \notin B\}$
	\item {\em{Set complement}}: $A^c = \Omega \setminus A$
\end{enumerate}

Set complement is the ``strongest'' operation, because if a collection of sets $\AA$ is closed under complement, and if it is also closed under any one of the other three operations, $\AA$ is closed under the rest two. That is seen from,
\begin{align}
	\mbox{If closed under } union &
	\begin{cases}
		A \cap B = (A^c \cup B^c)^c\\
		A \setminus B = A \cap B^c = (A^c \cup B)^c
	\end{cases}	
	\\
	\mbox{If closed under } intersection &
	\begin{cases}
		A \cup B = (A^c \cap B^c)^c\\		
		A \setminus B = A \cap B^c
	\end{cases}
	\\
	\mbox{If closed under } difference &
	\begin{cases}
		A \cap B = A \setminus (A \setminus B)\\
		A \cup B = (A^c \cap B^c)^c = (A^c \setminus (A^c \setminus B^c))^c\\
	\end{cases}		
\end{align}

The difference operation is the second ``strongest'' operation, in that if a collection of sets $\AA$ is closed under difference, it is closed under intersection, that is seen from,
\begin{align}
	A \cap B = A \setminus (A \setminus B)
\end{align}

The third ``strongest'' operation is the union, which can not be implied from difference or intersection, or their combination.

The ``weakest'' operation is the intersection, which can be implied from the difference.

To summarize, the ``strongest'' pair would be the complement plus any one more, which implies everything else; the second ``strongest'' would be the difference plus the union, which implies the intersection but not the complement ($\Omega$ may not be in the collection). \smallmarginpar{\textdbend}

\subsubsection{Class of Set Collection}
\begin{deff}
	Given a set $\Omega$, a non-empty collection $\PP \subset 2^\Omega$ is called a {\em{$\pi$-system}} iff:
	
	\center{$\forall A, B \in \PP$, $A \cap B \in \PP$}
\end{deff}
Notice, this has the weakest requirement.

\begin{deff}
	Given a set $\Omega$, a non-empty collection $\QQ \subset 2^\Omega$ is called a {\em{semiring}} iff:
	
	\center{$\forall A, B \in \QQ \mbox{ and } A \supset B$}
	\center{$\exists C_k \subset \QQ, s.t., A \setminus B = \bigcup_{k=1}^n C_k$}
\end{deff}

\begin{deff}
	Given a set $\Omega$, a non-empty collection $\RR \subset 2^\Omega$ is called a {\em{ring}} iff:
	
	\center{$\forall A, B \in \RR$, $A \cup B \in \RR$ and $B \setminus A \in \RR$}
\end{deff}

There are two points need to mention: the empty set is in a ring, since $A \setminus A = \emptyset$; $\AA$ is also closed under intersection (the reverse need not be true).

\begin{deff}
	A {\em{ring}} $\AA$ is called an {\em{field}} iff $\Omega \in \AA$.
\end{deff}
So a field is closed under all {\em{finite}} combination of set operations.


\begin{deff}
	An {\em{field}} is called a {\em{$\sigma$-field}} if for any sequence ${A_n}$ of sets in $\AA$, $\cup_{n \ge 1} A_n \in \AA$.
\end{deff}

$\not \exists$

\chapter{Integration Theory}
\chapter{Random Variables}
\chapter{Law of Large Numbers}
\section{Types of Convergence}
\subsubsection{Convergence in Distribution}
\subsubsection{Convergence in Probability}
\subsubsection{Almost Surely Convergence}
\subsubsection{$L^p$ Convergence}

\section{Weak Law of Large Numbers (WLLN)}
\section{Strong Law of Large Numbers (SLLN)}

\chapter{Central Limit theorem}

\chapter{Some Tricks}
\section{Prove by Contraposition}

\section{Construct Finer Partition}
Given twofinite partitions $\{A_n\}$ and $\{B_m\}$, a finer partition can be constructed as
\begin{align*}
	(\cup_{i=1}^n A_i) \bigcap (\cup_{j=1}^m B_j) = \bigcup (\cap_{i=1}^n \cap_{j=1}^m A_i B_j)
\end{align*}

\section{Prove Equality}
To prove two numerical quantities are equal $X=Y$, often times we can do this by showing $X \le Y$ and $X \ge Y$. Similarly, to prove two sets are equal $E = F$, we can show $E \subset F$ and $E \subset F$.

\section{An Epsilon of Room}
If one has to show that $X \le Y$, try proving that $X \le Y + \epsilon, \forall \epsilon > 0$. This trick combines well with the ``Prove Equality'' trick.

In a similar spirit, if one needs to show that a quantity $X$ vanishes, try showing that $|X| \le \epsilon, \forall \epsilon > 0$.

If one wants to show that a sequence $x_n$ of real numbers converges to zero, try showing that $\limsup_{n \to \infty} |x_n| \le \epsilon, \forall \epsilon > 0$

{\textbf{One caveat:}} for finite $x$, and any $\epsilon > 0$, it is true that $x + \epsilon > x$ and $x - \epsilon < x$, but it is not true when $x$ is equal to $+\infty$ or $-\infty$. \smallmarginpar{\textdbend}


\section{Interpretations of Probability}
\subsection{Cox's theorem}
\subsection{Principle of Maximum Entropy}

\part{Measure-theoretical Stochastic Processes}

\part{Appendix}

\end{document} 







\documentclass{book}
\usepackage{amssymb,amsmath,mathtools}


\usepackage{algorithm2e,algorithmic}

\usepackage{mathrsfs}
\usepackage{paralist}

\usepackage{esint} % for \fint

\allowdisplaybreaks

\usepackage{color}		% enable color characters
\usepackage{graphicx} 	% insert image files
\usepackage{enumerate} 	% enumerate items
\usepackage{caption}
\usepackage{subcaption}
\usepackage{multirow,multicol}
\usepackage[makeroom]{cancel}

\usepackage[colorlinks=true,linkcolor=blue,citecolor=blue]{hyperref}

\usepackage{makeidx}

\newcommand{\cem}[1]{\color{magenta}{\em{#1}}} % color emphasize
\newcommand{\dual}[2]{{#1}^{(#2)}} % color emphasize

\newcommand{\tr}{{\mathrm{tr}}}
\renewcommand{\vec}{{\mathrm{vec}}}

\newcommand{\dd}{{\,\text{d}\,}}
%%
%% horizontal and vertical centering in table p mode
%%
\usepackage{array}
\newcolumntype{P}[1]{>{\centering\arraybackslash}p{#1}} % horizontal centering
\newcolumntype{M}[1]{>{\centering\arraybackslash}m{#1}} % vertical centering

%%
%% define bold font for the alphabet
%%
\usepackage{pgffor}
\foreach \letter in {a,...,z}{ % bold font for a..z
\expandafter\xdef\csname \letter \endcsname{\noexpand\ensuremath{\noexpand\mathbf{\letter}}}
}
\foreach \letter in {A,...,Z}{ % bold font for A..Z
\expandafter\xdef\csname \letter \endcsname{\noexpand\ensuremath{\noexpand\mathbf{\letter}}}
}
\foreach \letter in {A,...,Z}{ % `field' font for AA..ZZ
\expandafter\xdef\csname \letter\letter \endcsname{\noexpand\ensuremath{\noexpand\mathcal{\letter}}}
}
\foreach \letter in {A,...,Z}{ % `field' font for AAA..ZZZ
\expandafter\xdef\csname \letter\letter\letter \endcsname{\noexpand\ensuremath{\noexpand\mathbb{\letter}}}
}
\newcommand{\balpha}{{\boldsymbol{\alpha}}}
\newcommand{\bbeta}{{\boldsymbol{\beta}}}
\newcommand{\bgamma}{{\boldsymbol{\gamma}}}
\newcommand{\bkappa}{{\boldsymbol{\kappa}}}
\newcommand{\bmu}{{\boldsymbol{\mu}}}
\newcommand{\btheta}{{\boldsymbol{\theta}}}
\newcommand{\bTheta}{{\boldsymbol{\Theta}}}
\newcommand{\bPi}{{\boldsymbol{\Pi}}}
\newcommand{\bSigma}{{\boldsymbol{\Sigma}}}
\newcommand{\bPhi}{{\boldsymbol{\Phi}}}
\newcommand{\bLambda}{{\boldsymbol{\Lambda}}}
\newcommand{\bdeta}{{\boldsymbol{\eta}}}
\newcommand{\bphi}{{\boldsymbol{\phi}}}



%%
%% add definitions and theorems
%%
\usepackage[thmmarks,amsmath]{ntheorem}
\theorembodyfont{\normalfont}
\newtheorem{deff}{Definition}[section]
\newtheorem{thm}{Theorem}[section]
\newtheorem{prop}{Proposition}[section]
\newtheorem{lem}{Lemma}[section]
\newtheorem{cor}{Corollary}[section]
\newtheorem{rmk}{Remark}[section]
\newtheorem{alg}{Algorithm}[section]
\newtheorem{ex}{Example}[section]
\newtheorem{ques}{Question}[section]
\newtheorem{ans}{Answer}[section]
\newtheorem{prob}{Problem}[section]
\newtheorem{sol}{Solution}[section]
\newtheorem*{prof}{Proof}[section] 

\begin{document}
\title{Machine Learning}
\author{Xi Tan (xtan3.1415926@gmail.com)}

\maketitle
\tableofcontents
\chapter*{Preface}
This book project, which consists of four subjects: Finance, Mathematics, Statistics, and Computer Science, is tailored specifically to prepare someone for a quant career. It originated from my general belief of the hierarchy of solving a problem --- problems are solved at strategic, tactical, and operational levels.

{\em{Microeconomics}} and {\em{Macroeconomics}} explain the driving forces of capital markets, from a legislator's perspective. {\em{Accounting}} and {\em{Corporate Finance}} take a closer and necessary look at these forces, from a different angle. {\em{Stochastic Calculus}} and {\em{Asset Pricing}} provide with a set of tools and ideas that enables us to {\bf{strategically}} model one of the central problems in Quantitative Finance.

Generally speaking, there are two paths to solve a quantitative finance problem at the {\bf{tactical}} level: the mathematical way and the statistical way. There are only two pieces of math we need to know: {\em{Analysis}}, in particular measure-theoretical probability and differential equations; and {\em{Linear Algebra}}, with functional analysis in mind. Statistics, on the other hand, should start with {\em{Statistical Experiment Design}}, from which we learn how to collect data for statistical models. Next, the study of {\em{Random Variables}} and {\em{Stochastic Processes}} introduce the building blocks of the statistical ``pillbox'', with {\em{Mathematical Statistics}} the ``scaffold''. Once the ``pillbox'' is ready, we are equipped to tackle our problems using {\em{Machine Learning}}, which is essentially a collection of statistical models and optimization algorithms.

{\em{Computer Architecture}} and {\em{Operating System}} are respectively about the ``hardware'' and ``software'' of a single computer. The interaction of multiple computers is understood in {\em{Computer Network}}. Once we are comfortable with these concepts, we will be able to use {\em{Data Structure and Algorithms}} to solve problems at the {\bf{operational}} level, and use {\em{C++}} and/or {\em{Java}} to implement our ideas.

I am aware that it can take a while, and even multiple advanced degrees, to finish this curriculum, but let's remember the motto from the Leipzig Gewandhaus Orchestra: ``{\bf{\em{Res severa est verum gaudium}}}''.

\vspace{8mm}
\hfill {\em{Xi Tan}}

\hfill {\em{West Lafayette, IN}}

\hfill {\em{October, 2013}}
\addcontentsline{toc}{chapter}{Preface}


\chapter{Introduction}
There are two main types of Machine Learning problems: the {\bf{predictive}} or {\bf{supervised learning}}, and the {\bf{descriptive}} or {\bf{unsupervised learning}} \footnote{It is also called {\bf{Knowledge Discovery}} in the Data Mining literature}.

For supervised learning, there are two subtypes: the {\bf{classification}} problem, where the response variable is categorical (either ordinal or nomial); and the {\bf{regression}} problem, where the response variable is numerical (either discrete or continuous).

For unsupervised learning, it is also called {\bf{density estimation}} in Statistics literature. Essentially we want to build models of the form $p(\x_i|\theta)$, and supervised learning can be seen as to build models of the form $p(y_i|\x_i,\theta)$, which is a problem of conditional density estimation. \footnote{We see from the formulation that, unsupervised learning usually involves multivariate probability models while supervised learning usually involves univariate probability models.}


\section{Some Distinctions}
\subsection{Machine Learning v.s. Statistical Learning}
{\bf{Machine Learning}} focuses more on high dimension low noise situations, and performance and efficiency are the main concerns. {\bf{Statistical Learning}} focuses more on low dimension high noise situations, interpretation and statistical inference are the main concerns.
\subsection{Parametric v.s. Non-parametric Models}
A parametric model has a fixed {\em{finite}} number of parameters, while the number of parameters in a non-parametric model grows with the amount of training data. A model with infinite number of parameters is usually non-parametric.

\chapter{A Brief History of Machine Learning}
The perceptron model was invented in 1957, and it generated over optimistic view for AI during 1960s. After Marvin Minsky pointed out the limitation of this model in expressing complex functions, researchers stopped pursuing this model for the next decade.

In 1970s, the machine learning field was dormant, when expert systems became the mainstream approach in AI.  The revival of machine learning came in mid-1980s, when the decision tree model was invented and distributed as software. It is also in mid 1980s multi-layer neural networks were invented, With enough hidden layers, a neural network can express any function, thus overcoming the limitation of perceptron. We see a revival of the neural network study.

Around 1995, SVM was proposed and have become quickly adopted.

After year 2000, Logistic regression was rediscovered and re-designed for large scale machine learning problems . In the ten years following 2003, logistic regression has attracted a lot of research work.

We discussed the development of 4 major machine learning methods. There are other method developed in parallel, but see declining use today in the machine field: Naive Bayes, Bayesian networks, and Maximum Entropy classifier (most used in natural language processing). 

In addition to the individual methods, we have seen the invention of ensemble learning, where several classifiers are used together, and its wide adoption today. 


\url{http://en.wikipedia.org/wiki/Timeline_of_artificial_intelligence}

1950s-1960s: the Perceptron Model
1970s-1980s: the Expert Systems
mid 1980s: Multi-layer Neural Networks
1995: SVM
2000s: Logistic Regression Rediscovered



\chapter{Regression v.s. Classification}
Consider a supervised learning problem in which we wish to approximate an unknown target function $f: \XX \to \YY$, or equivalently $P(Y|X)$. One way to learn $P(Y|X)$ is to use the training data to estimate $P(X|Y)$ and $P(Y)$, and then use Bayes rule to determine $P(Y|X)$.
\chapter{Parametric v.s. Nonparametric Models}
\chapter{Decision Trees}

\chapter{Clustering}
\chapter{Dimension Reduction}






\section{Graphical Models}
The motivation of graphical models is, in the previous settings, variables are assume to be (conditionally) independent of each other: such as observed variables $\x$ and/or $\y$; or latent variables $\z$ (excluding parameters $\btheta$). Graphical models consider the case when variables are correlated. 
\chapter{Neural Networks}
\chapter{Kernel Methods}
\section{Support Vector Machines (SVM)}
\section{Gaussian Processes (GP)}

\chapter{Statistical Learning Theory}
Statistical learning theory studies the properties, in particular error-bounds, of learning algorithms in a statistical framework.

The No Free Lunch theorem says, if no assumptions about how training data are related to the testing data, prediction is impossible; furthermore, if no assumptions about the data to be expected, generalization is impossible.

Simply means we need to make the assumption that there is a stationary distribution of data.

\deff{
	Suppose $f: \RRR \to \RRR_+$ and $g: \RRR \to \RRR_+$, we write
	\begin{align}
		f = O(g)		
	\end{align}	
	if there exists $x_0, \alpha \in \RRR_+$ such that for all $x > x_0$ we have $f(x) \le \alpha g(x)$. Replacing $\le$ with $\ge$, we write $f = \Omega(g)$.
}

\deff{
	Suppose $f: \RRR \to \RRR_+$ and $g: \RRR \to \RRR_+$, we write
	\begin{align}
		f = o(g)
	\end{align}	
	if for every $\alpha > 0$ there exists $x_0$ such that for all $x > x_0$ we have $f(x) \le \alpha g(x)$. Replacing $\le$ with $\ge$, we write $f = \omega(g)$.
}

\deff{
	If $f = O(g)$ and $f = \Omega(g)$, then we write $f = \Theta(g)$.
}

\deff{
	We write $f = \tilde O(g)$ if there exists $k \in \NNN$ such that $f(x) = O(g(x)\log^k(g(x)))$.
}

\chapter{Latent Dirichlet allocation (LDA)}
\chapter{Principle Component Analysis (PCA)}
\chapter{Linear Discriminant Analysis (LDA)}

\chapter{Expectation Maximization (EM)}
\section{The EM algorithm}
\section{The ECM and ECME algorithms}
\section{The PX-EM algorithm}

\chapter{Expectation Propagation (EP)}


\chapter{Markov Chain Monte Carlo Methods}
\section{Introduction}
This note is based on Peter Orbanz's BNP notes:
\vspace*{5mm}
\\
\url{http://people.stat.sc.edu/hansont/stat740/MCMC.pdf}

\section{Notation}
Bold upper case letters represent matrices, e.g., $\X, \Y, \Z, \bTheta$. Bold lower case letters represent vector-valued random variables and their realizations (we do not distinguish between the two), e.g., $\x, \y, \z, \btheta$. Curly upper case letters represent spaces (i.e., possible values) of random variables, e.g., $\XX, \YY, \ZZ, \Theta$.

\section{Introduction}
Markov chain Monte Carlo (MCMC) methods can be used to draw random samples from a target distribution $p$. It is particularly useful in Bayesian data analysis, due to the difficulties of evaluating the denominator in the Bayes' formula, a.k.a. the partition function.

\begin{enumerate}
\item A discrete-time, discrete-space Markov chain is $\dual{X}{0}, \dual{X}{1}, \dots$ where $\dual{X}{t}$ obeys the Markov property that
\begin{align}
P\left[\dual{X}{t} \bigg| \dual{x}{0},\dots,\dual{x}{t-1}\right] = P\left[\dual{X}{t} \bigg| \dual{x}{t-1} \right]
\end{align}
\item A Markov chain is {\em{irreducible}} if any state $j$ can be reached from any state $i$ in a finite number of steps for all $i$ and $j$.
\item A Markov chain is {\em{periodic}} if it can visit certain portions of the state space only at regularly spaced intervals.
\end{enumerate}

The MCMC sampling strategy is to construct an irreducible, aperiodic Markov chain for which the stationary distribution equals the target distribution $p$.

Suppose we want to draw samples from $p(x)$. The M-H algorithm proceeds as follows: Draw a candidate state, $x^*$, according to the proposal distribution $g(x^*|x)$, by computing the acceptance probability
\begin{align}
\alpha(x^*, x) = \min\left[1, a(x^*,x) = \frac{p(x^*)g(x|x^*)}{p(x)g(x^*|x)}\right].
\end{align}
where $a(x^*,x)$ is called the M-H ratio, and $\alpha(x^*,x)$ the probability of move. With {\em{probability of move}} $\alpha(x^*, x)$, set the new state, $x'$ to $x^*$. Otherwise, let $x'$ be the same as $x$. The intuition behind the probability of move is that, if the detailed balance condition is satisfied: $p(x)g(x^*|x) = p(x^*)g(x|x^*)$, then we are done, otherwise, the denominator $g(x^*|x)p(x)$ is proportional to the probability of moving from $x$ to $x^*$, if it is large then the numerator, which is proportional to the probability of moving from $x^*$ to $x$, then we should penalize it.

The sampled sequence may contain duplicated copies of data points, the frequency of which is used to correct the difference between the proposal distribution and the target one. A well chosen proposal distribution produces candidate values that efficiently cover the support of the target distribution.

\section{Independence Chains}
If we choose the proposal distribution to be
\begin{align}
g(x^*|x) = g(x^*)
\end{align}
then the M-H ratio is
\begin{align}
a(x^*, x) = \frac{p(x^*)g(x)}{p(x)g(x^*)}.
\end{align}

For example, in a Bayesian framework, if the target distribution is the posterior $p(\theta|\y)$, where $\y$ is the data. Then, if we choose the proposal distribution to be the prior $p(\theta)$
\begin{align}
g(\theta^*|\theta) = p(\theta^*)
\end{align}
then the M-H ratio is
\begin{align}
a(\theta^*, \theta | \y) &= \frac{p(\theta^*|\y)p(\theta)}{p(\theta|\y)p(\theta^*)} = \frac{p(\y|\theta^*)p(\theta^*)/p(\y)}{p(\y|\theta)p(\theta)/p(\y)}\frac{p(\theta)}{p(\theta^*)} = \frac{p(\y|\theta^*)}{p(\y|\theta)}
\end{align}
So if the proposal distribution is the prior, the M-H ratio is the likelihood ratio.

\section{Random walk chains}
Let $x^*$ be generated by setting
\begin{align}
x^* = x + \epsilon,\quad \epsilon \sim h(\epsilon)
\end{align}
or equivalently,
\begin{align}
g(x^*|x) = h(x^* - x)
\end{align}
For example, $h$ can be the uniform, or the standard normal, or the Student's $t$ distribution.


\section{Gibbs sampler}
Suppose it is easy to sample from the univariate conditional distributions:
\begin{align}
x_i | \x_{-i} \sim f(x_i | \x_{-i})
\end{align}
then the basic Gibbs sampler can be described as follows:
\begin{enumerate}
\item Select starting values $\dual{x}{0}$ and set $t=0$.
\item Generate, in turn for $i = 1, \dots, n$:
\begin{align}
\dual{x_i}{t+1} | \x_{-i}^{(t)} \sim f\left(x_1 | \x_{-i}^{(t)}\right).
\end{align}
\item Increment $t$ and go to step 2.
\end{enumerate}
A hybrid MCMC may contain different types of samplers. For example, The M-H within Gibbs algorithm is typically useful when the univariate conditional density for one or more elements is not available in closed form.

\section{Test for Convergence}
\begin{enumerate}
\item Burn-in.
\item Run multiple chains, and if the within- and between-chain behaviors are similar, suggests that the chains are stationary. Gelman-Rubin statistic.
\item Plot samples against time, or log-likelihood against time.
\item Autocorrelation function (ACF) plot: lag versus correlation. Slow decay suggests poor mixing.
\item Re-parameterize the model may help.
\item Burn-in should be about $5000$ iterations, chain lengths should be about $100$ times the burn-in.
\item Standard error should be less than $5\%$ of the standard deviation.
\end{enumerate}

\section{How it is used}
Marginalization: just ignore others. Mean and variance: use samples. Probability estimates: estimated by the frequencies. Standard error of estimates: batch runs to obtain estimates and compute mean and standard error (divided by the square root of batch size). Density: kernel density or simply histogram.

\section{Advanced MCMC methods}
Slice sampling and other auxiliary variable methods, reversible jump MCMC, perfect sampling, Hit-and-run (choose a direction and then a distance to run), multi-try (choose from a set of candidates), Langevin M-H (random walk with drift) and etc.

\subsection{Slice sampling}
Introduce an auxilary varialbe $u$, and if we can sample from $f(x,u) = f(x)f(u|x)$ then dropping $u$ and retain $x$ as desired. The slice sampling works as follows:
\begin{align}
\dual{u}{t+1} | \dual{x}{t} &\sim \text{Unif}\left(0, f\left(\dual{x}{t}\right)\right)\\
\dual{x}{t+1} | \dual{u}{t+1} &\sim \text{Unif}\left(x: f(x) \ge \dual{u}{t+1}\right)
\end{align}
It is particularly useful for multi-modal problems (but not for high dimensional ones).

\subsection{Reversible Jump MCMC}
RJMCMC is suitable for nonparametric models where model dimensions change. The key is to use auxiliary variables to match the dimensions.

\section{Introduction}
\subsection{Sampling Methods in General}
\subsubsection{Inverse Transform Sampling}
\subsubsection{Importance Sampling}
\subsection{Rejection Sampling}
Suppose we want to sample from $P(X)$ by utilizing a {\em{proposal distribution}} $Q(X)$, from which we can take samples easier. Let $C = \sup \left\{\frac{P(x)}{Q(x)}, \forall x\right\}$, so $\frac{P(x)}{CQ(x)} \le 1, \forall x$.

We propose a new value $x'$ from $Q(X)$ and accept this new value with probability
\begin{align}
	A(x'|x) = \frac{P(x')}{CQ(x')} \label{rejection sampling ratio}
\end{align}
This is called the {\em{rejection sampling}} method.

\rmk{
	If we plug in Equation \ref{rejection sampling ratio} into Equation \ref{detailed balance equation}, then
	\begin{align}
			LHS = P(x)Q(x'|x)\frac{P(x')}{CQ(x')} = P(x)Q(x')\frac{P(x')}{CQ(x')} = \frac{1}{C}P(x)P(x')\\
			RHS = P(x')Q(x|x')\frac{P(x)}{CQ(x)} = P(x')Q(x)\frac{P(x)}{CQ(x)} = \frac{1}{C}P(x')P(x)
	\end{align}
	which shows that the rejection sampling ratio in Equation \ref{rejection sampling ratio} is a special solution to the detailed balance equation.
}


\section{Markov Chains}
\deff{
  A \emph{Markov chain} is a collection of random variables $\{X^{(0)},X^{(1)},X^{(2)},\cdots\}$, satisfying the Markov property
  \begin{align}
    P(X^{(n+1)} \vert X^{(n)}, \cdots, X^{(0)}) = P(X^{(n+1)} \vert X^{(n)})
  \end{align}
}

\deff{
  A distribution over the states of a homogeneous\footnote{The transition matrix is invariant of time.} Markov chain is \emph{invariant} (or \emph{stationary}) with respect to transition probabilities $T$ if
  \begin{align}
    \mathbf{\pi} = \mathbf{\pi} T
  \end{align}
}
A Markov chain can have more than one invariant distribution. If $T$ is the identity matrix, for example, then any distribution is invariant.

We are interested in designing a Markov chain (its initial probabilities and transition matrix) for which the distribution we wish to sample from, given by $\mathbf{\pi}$, is invariant. Hence, if we run the chain for sufficient time, it will converge to $\mathbf{\pi}$, and then we can sample from it and compute the quantities desired.

\emph{Why do we need detailed balance?}
\deff{
  Often, we will use \emph{time reversible} homogeneous Markov chains that satisfy the more restrictive condition of \emph{detailed balance}:
  \begin{align}
    \pi(x)T(x,x') = \pi(x')T(x',x) \label{detailed balance definition}
  \end{align}
}

\rmk{
One might be tempted to conclude that the detailed balance condition always holds, since
\begin{align}
	\pi(x)T(x,x') = P(x)P(x'|x) = P(x',x)\\
	\pi(x')T(x',x) = P(x')P(x|x') = P(x,x')
\end{align}
and $P(x',x) = P(x,x')$.

The problem here is, $x$ and $x'$ denote different values, not different random variables. That is, in the argument of $\pi(\cdot)$ or in the first argument of $T(\cdot,\cdot)$, it is the value of the random variable $X_0$ (stochastic process at current time); and in the second argument of $T(\cdot,\cdot)$, it is the value of the random variable $X_1$ (stochastic process at the next time step). For example, we may have
\begin{align}
	\pi(X_0=1)T(X_0=1,X_1=2) = P(X_0=1)P(X_1=2|X_0=1) = P(X_1=2,X_0=1)
\end{align}
and
\begin{align}	
	\pi(X_0=2)T(X_0=2,X_1=1) = P(X_0=2)P(X_1=1|X_0=2) = P(X_1=1,X_0=2)
\end{align}
which are usually not the same. Here $x=1$ and $x'=2$.
}

Note, the detailed balance condition implies $\pi$ is an invariant distribution:
\begin{align}
	\sum_{i=1}^n \pi(x_i) T(x_i,x) = \sum_{i=1}^n \pi(x) T(x,x_i) = \pi(x) \sum_{i=1}^n T(x,x_i) = \pi(x)
\end{align}
It is possible for a distribution to be invariant without detailed balance holding. For example, the uniform distribution ($=1/3$) on the state space $\{0,1,2\}$ is invariant with respect to the homogeneous Markov chain with transition probabilities $T(0,1)=T(1,2)=T(2,0)=1$ and all others zero, but detailed balance does not hold.

\deff{
  A Markov chain is \emph{ergodic} (or \emph{irreducible}) if it is possible to go from every state to every state (not necessarily in one move).
}
\thm{
  Let $T$ be the transition matrix for a regular chain. Then as $n \to \infty$, the powers $T^n$ approach a limiting matrix $W$ with all rows the same vector $\w$. The vector $\w$ is a strictly positive probability vector (i.e., the components are all positive and they sum to one).
}

\section{Metropolis-Hastings Algorithm}
Suppose we can evaluate a distribution $P(X)$, but lack of information about how to directly draw samples from it. We wish to construct a Markov chain, utilizing the detailed balance condition (Equation \ref{detailed balance definition}), such that the induced invariant distribution is nothing but $P(X)$.

Assume we can take a new sample $x'$ from $Q(x'|x)$, a known {\em{proposal distribution}} conditioned on the current state $x$ of the Markov chain, and then we accept this new value $x'$ based on a to-be-determined probability $A(x'|x)$. We define a Markov chain based on this procedure,
\begin{align}
	T(x,x') = Q(x'|x)A(x'|x)
\end{align}

If this Markov chain further satisfies the detailed balance condition (Equation \ref{detailed balance definition}),
\begin{align}
	P(x)Q(x'|x)A(x'|x) = P(x')Q(x|x')A(x|x') \label{detailed balance equation}
\end{align}
then we can take
\begin{align}
	A(x'|x) = \min \left\{ \frac{P(x')Q(x|x')}{P(x)Q(x'|x)}, 1 \right\} \label{hastings ratio}
\end{align}
The ratio in the ``$\min$'' function is known as the ``Hastings ratio'' (or acceptance ratio).

Now, running the chain for some sufficient time, we would obtain samples from the desired distribution $P(X)$, which is designed to be the same as the induced invariant distribution.

\rmk{
	The $A(x'|x)$ defined above satisfies the detailed balance condition, since
	\begin{align}
		P(x)Q(x'|x)A(x'|x) = P(x)Q(x'|x) \min \left\{ \frac{P(x')Q(x|x')}{P(x)Q(x'|x)}, 1 \right\} = \min \left\{ P(x')Q(x|x'), P(x)Q(x'|x) \right\}\\
		P(x')Q(x|x')A(x|x') = P(x')Q(x|x') \min \left\{ \frac{P(x)Q(x'|x)}{P(x')Q(x|x')}, 1 \right\} = \min \left\{ P(x)Q(x'|x), P(x')Q(x|x') \right\}
	\end{align}
	and $\min \left\{ P(x')Q(x|x'), P(x)Q(x'|x) \right\} = \min \left\{ P(x)Q(x'|x), P(x')Q(x|x') \right\}$.
}

\subsection{Metropolis Algorithm}
In the Hastings ratio, if the proposal distribution is symmetric $Q(x|x') = Q(x'|x)$, such as a Gaussian distribution, then Equation \ref{hastings ratio} becomes
\begin{align}
	A(x'|x) = \min \left\{ \frac{P(x')Q(x|x')}{P(x)Q(x'|x)}, 1 \right\} = \min \left\{ \frac{P(x')}{P(x)}, 1 \right\}
\end{align}
this special case is called the ``Metropolis Algorithm''.

\subsection{Gibbs Sampling}
Suppose we wish to sample a random vector $\X = (X_1,\cdots,X_n)$, and its full conditional distribution $Q(X_i|\X_{-i})$ is known, where $\X_{-i} = (X_1,\cdots,X_{i-1},X_{i+1},X_n)$. Then if we sample $X_i$ component-wise from $Q(\x'|\x) = Q(x_i'|\x_{-i})$, and bearing in mind that $\x'_{-i} = \x_{-i}$ when sample $x_i$, the Hastings ratio becomes,
\begin{align}
	\frac{P(\x')Q(\x|\x')}{P(\x)Q(\x'|\x)} &= \frac{P(x_i'|\x'_{-i})P(\x'_{-i})Q(x_i|\x'_{-i})}{P(x_i|\x_{-i})P(\x_{-i})Q(x_i'|\x_{-i})}\\
	&= \frac{P(x_i'|\x_{-i})P(\x_{-i})Q(x_i|\x_{-i})}{P(x_i|\x_{-i})P(\x_{-i})Q(x_i'|\x_{-i})}\\	
	&= 1
\end{align}

\subsection{Collapsed Gibbs Sampling}

\subsection{Metropolis-Within-Gibbs}
If not all full conditional probabilities are known or can be easily sampled from, we can sample those random variables with known conditional probabilities using the Gibbs algorithm, and others using Metropolis-Hastings algorithm. This is called {\em{Metropolis-Within-Gibbs}}.

\section{Slice Sampling}
\subsection{Elliptical Slice Sampling}

\section{Split-Merge Sampling}

\section{Hamiltonian Monte Carlo}

\section{Data Fusion and Particle Filter (Sequential MCMC)}
\section{Reversible jump MCMC}
\section{Convergence Diagnostics}



\chapter{Bayesian Nonparametrics}
The concept of functions can be generalized to that of algorithms, which describe procedures, with loops and conditional tests, of how to generate output from input.

A draw from a finite dimensional Gaussian distribution is a real number, while a real-valued function can be considered as a sequence of (uncountably) infinite number of real numbers. A Gaussian Process (GP) is an infinite dimensional generalization of a Gaussian distribution. It defines a prior over real-valued functions, and a sample of it is a particular example of such functions.

A draw from a finite dimensional Dirichlet distribution is a (discrete) probability measure. A Dirichlet Process (DP) is an infinite dimensional generalization of a Dirichlet distribution. It defines a prior over probability measures, and a sample of it is a probability measure.  Distributions drawn from a Dirichlet process are discrete, but cannot be described using a finite number of parameters, thus the classification as a nonparametric model.


Note, that we do not have a measurement of the function, as in the GP case but a sample of the true probability measure; this is the main difference between GP and DP.

\section{Introduction}
This note is based on Peter Orbanz's BNP notes:
\vspace*{5mm}
\\
\url{http://stat.columbia.edu/~porbanz/npb-tutorial.html}

\section{Notation}
Bold upper case letters represent matrices, e.g., $\X, \Y, \Z, \bTheta$. Bold lower case letters represent vector-valued random variables and their realizations (we do not distinguish between the two), e.g., $\x, \y, \z, \btheta$. Curly upper case letters represent spaces (i.e., possible values) of random variables, e.g., $\XX, \YY, \ZZ, \Theta$.

\section{Terminology}
\subsection{Parametric and nonparametric models}
In a set of probability spaces $\{(\YY, \FF, \PP_\Theta)\}$, a {\em{statistical model}} $\MM$ on a sample space $\YY$ is {\underline{a set of probability measures}} $\PP_\Theta$ on $\YY$. If we write $PM(\YY)$ for the space of all probability measure on $\YY$, a model is a subset $\MM \subset PM(\YY)$. Every element of $\MM$ has a one-to-one mapping (hence the model is {\em{identifiable}}) with its parameter $\btheta$ with values in a parameter space $\Theta$, that is,
\begin{align}
\MM(\y) = \{P_\btheta(\y) | \btheta \in \Theta\},\quad \y \in \YY.
\end{align}
For example, a first order polynomial is a model, and a second order polynomial is another model. We can of course fit a model to the observed data, but {\em{model}} itself is an abstract concept, where the parameter values of a model need not be specified. We call a model {\em{parametric}} if $\Theta$ has finite dimension, and {\em{nonparametric}} if $\Theta$ has infinite dimension.

To formulate statistical problems, we assume that $n$ observations $\y_1,\dots,\y_n$ with values in $\YY$ are observed, which are drawn i.i.d. from a measure $P_\btheta$ in the model, i.e.,
\begin{align}
\y_1, \dots, \y_n \sim_{iid} P_\btheta \qquad \text{for some}~~ \btheta \in \Theta
\end{align}
The objective of statistical {\em{inference}} is then to draw conclusions about the value of $\btheta$ (and hence about the distribution $P_\btheta$ of the data) from the observations.


\subsection{Bayesian and Bayesian nonparametric models}
In Bayesian statistics, all parameters are considered as random variables. Hence under a Bayesian model, data are generated in two stages, i.e.,
\begin{align}
\btheta &\sim P(\btheta)\\
\y_1, \dots, \y_n ~|~ \btheta &\sim_{iid} P_\btheta(\y)
\end{align}
The objective is then to determine the {\em{posterior distribution}} -- the conditional distribution of $\btheta$ given the observed data,
\begin{align}
\pi(\btheta | \y_1, \dots, \y_n)
\end{align}
A {\em{Bayesian nonparametric}} model is a Bayesian model whose parameter space $\Theta$ has infinite dimension. To define a Bayesian nonparametric model, we have to define a prior $\pi$ on an infinite-dimensional space, which is a stochastic process with paths (i.e. realizations) in $\Theta$.

\section{Clustering and the Dirichlet process}
\subsection{Finite mixture models}
The basic assumption of a clustering problem is that each observation $\y_i$ belongs to a single cluster $k \in \{1,\cdots, K\}$, which has a cluster distribution
\begin{align}
P_k(\y_i | z_i = k)
\end{align}
where we have defined a latent variable $z_i$, indicating the cluster assignment of observation $\y_i$. Note that under the Bayesian framework, the latent variable $z_i$ itself has a distribution
\begin{align}
p_k^i \equiv P(z_i = k)
\end{align}

The marginal distribution of the observation $\y_i$ is then
\begin{align}
P(\y_i) = \sum_{k=1}^K P(z_i = k) P_k(\y_i | z_i = k)
\end{align}
A model of this form is called a {\em{finite mixture model}}.

\subsection{Bayesian mixture models}
Suppose we know there are $K$ clusters, we first sample the cluster parameters from some base measure:
\begin{align}
\btheta_1, \dots, \btheta_K \sim_{iid} G(\bbeta)
\end{align}
We then independently sample the latent cluster assignment vectors and the actual observations:
\begin{align}
(p_1^i, \dots, p_K^i) &\sim \text{Dirichlet}_K(\alpha)\\
z_i &\sim \text{Categorical}(p_1^i, \dots, p_K^i)\\
\y_i &\sim P_k(\y_i | \btheta_k, z_i = k)
\end{align}

\subsection{Dirichlet Process}
\deff{
	If $\alpha > 0$ and if $G$ is a probability measure on $\Omega_\phi$, the random discrete probability measure $\Theta$ generated by
	\begin{align}
	V_1, V_2, \dots \sim_{iid} \text{Beta}(1,\alpha)\\
	C_k = V_k\prod_{j=1}^{k-1}(1-V_k)\\
	\Phi_1, \Phi_2, \dots \sim_{iid} G
	\end{align}
	is called a {\em{Dirichlet process (DP)}} with base measure $G$ and concentration $\alpha$, and denote its law by $\text{DP}(\alpha, G)$.
}

\appendix
\chapter{Glossary}
\url{http://alumni.media.mit.edu/~tpminka/statlearn/glossary/}


- Model Evidence: Model evidence, or *marginal likelihood*, or *normalization constant*, is a likelihood function in which all parameter variables have been marginalized, usually denoted by $P(D|M)$, or $P(D)$ for short. For example, in polynomial regression, $M$ may denotes the degree of the regression function, and parameter variables are the coefficients for any given M.

- Filtering is the task of tracking the posterior distribution of the latent state varialbe in state space models.

%!TEX root = ../MachineLearning2.tex

\chapter{Q \& A}
\ques{
    What is curse of dimensionality?
}

\ans{
The curse of dimensionality is that when the dimensionality increases, the volume of the space increases so fast that the available data become sparse. This sparsity is problematic for any method that requires statistical significance. In order to obtain a statistically sound and reliable result, the amount of data needed to support the result often grows exponentially with the dimensionality. Also, organizing and searching data often relies on detecting areas where objects form groups with similar properties; in high dimensional data, however, all objects appear to be sparse and dissimilar in many ways, which prevents common data organization strategies from being efficient.
}

\ques{
    Why over-fitted polynomial models (without regularization) tend to have larger coefficients?
}
\ans{
	Because coefficients essentially are measures of ``derivatives'' of different orders, the larger the coefficients, the larger the derivatives, hence the more the function changes.
}

\ques{
    What is bias-variance trade-off, and its solution?
}

\ans{
\begin{align}
	E[(y - \hat f)^2] = [Bias(\hat f)]^2 + Var(\hat f) + \sigma^2
\end{align}
where $Bias(\hat f) = E[\hat f - f]$ and $Var(\hat f) = E[(\hat f - E(\hat f))^2] = E[\hat f^2] - (E[\hat f])^2$.

One way of resolving the trade-off is to use mixture models and ensemble learning. For example, boosting combines many ``weak'' (high bias) models in an ensemble that has lower bias than the individual models, while bagging combines ``strong'' learners in a way that reduces their variance.
}

\ques{
    What are parametric models, non-parametric models, and etc.?
}

\ans{
A {\bf{parametric model}} $\mathcal{M}$ is a collection of probability distributions $P_\btheta$, each of which is described by a {\em{finite dimensional}} (vector) parameter $\btheta$. A parametric model is called identifiable if the mapping $\btheta \to P_\btheta$ is invertible.

\begin{align}
    \mathcal{M} = \{P_\btheta | \btheta \in \mathbf{\Theta} \subset \mathbb{R}^k\}
\end{align}

A {\bf{non-parametric model}} may refer to two interpretations: 1) it may refer to models that do not rely on data belonging to any particular distribution\footnote{Distribution-free methods are such examples, but they are not equivalent concepts.}, but reply on comparative properties (statistics) of the data, or population, such as the ``order statistics'', or 2) it may refer to models that do not assume the structure of a model is fixed, i.e., the model grows in size to accommodate the complexity of the data. In these techniques, individual variables are typically assumed to belong to parametric distributions, and assumptions about the types of connections among variables are also made.
}

\ques{
    What are generative models, discriminative models?
}

\ans{
For a supervised learning problem in which we wish to approximate an unknown target function $f: \XX \to \YY$, or equivalently $P(Y|X)$, one approach is to model $P(Y|X)$ directly, which is called a discriminative model; another approach is to model $P(X,Y)$, or equivalently $P(Y)$ and $P(X|Y)$, and then use the Bayes' rule, to obtain $P(Y|X)$ for each $X = x$ query.
}

\ques{
    What are log-linear models?
}

\ans{{
Such models have many names, including maximum-entropy models, exponential models, and Gibbs models;
Markov random fields are structured log-linear models, conditional random fields are Markov Random Fields with a specific training criterion.
}

\ques{
	Bayesian v.s. Frequentism?
}

\ans{
There are two main schools of statistical inference: Frequentist and Bayesian\footnote{Another being Fiducial inference, or Fisherian inference.}. The controversies arise when it comes to how to interpret the randomness of data point generating process.

Bayesians consider parameters to be \emph{random}, and the observed data are conditioned on a realization of such random variables. Notice the natural hierarchical structure in this interpretation. The goal is to inference $p(\btheta \vert D_{\btheta_0})$, where $\btheta$ are the parameters and $D_{\btheta_0}$ the observed data set conditioned on realized values $\btheta_0$ of $\btheta$. Frequentists consider parameters to be \emph{unknown but fixed}, and the observed data set is just a sample from the population. As in \cite{Efron 1978}, Efron said ``... Bayesian averages involve only the data value $\bar x$ actually seen, rather than a collection of theoretically possible other $\bar x$ values.''

Also, as answered by Michael Hochster in \cite{Hochster} and \cite{wiki}: Suppose $h$ is the unknown constant, and $H$ is the statistic computed from a sample. For Frequentists, it is valid to write $P(L \le h \le U) = 95\%$ or $P(70 \le H \le 74) = 95\%$, but not $P(70 \le h \le 74) = 95\%$ (this is $0$ or $1$). So the correct way to say is either ``if the same experiment procedure is repeated 100 times, 95 times of the CIs will cover the unknown true value $h$'', or ``before the experiment, the probability is 95\% that the CI to be obtained will cover $h$''.

Wasserman said in \cite{Wasserman} that the two schools of inference differ in their \emph{goals}, not the \emph{methods}: the goal of Frequentist inference is to construct procedures with frequency guarantees, and the goal of Bayesian inference is to quantify and manipulate degrees of beliefs.

Further Readings: Stein's example, Likelihood principle \cite{bayesian-inference advantage}, \cite{bayesian-inference likelihood}, \cite{quora CI}, \cite{quora diff}.
}

\ques{
	What are Learning, Machine Learning, Data Mining, Statistics, and their differences?
}

\ans{
Learning is a process of improving performance with experience. There are two main types of learning: deductive learning and inductive learning. Deductive learning learns to apply generalization concepts (rules) to examples; inductive learning learns to generalized concepts (rules) from examples.

Machine Learning is an example of inductive learning of machines (computers).

A big difference between Machine Learning and Data Mining is Reinforcement Learning. While one of the main goals of statistics is hypothesis testing, one of the main goals of data mining is the construction of hypotheses.
}

\ques{
	What is the difference between learning and inference?
}

\ans{
Inference reasons about unknown probability distributions; (parameter) learning is finding point estimates of quantities in the model. In Statistics, no distinction between learning and inference only inference (or estimation); and in Bayesian Statistics, all
quantities are probability distributions, so there is only the problem of inference. Inference in the Machine Learning community also includes making predictions.

So your inference algorithm gives you posteriors in functional forms, and learning algorithm estimates parameter values from data, and you then inference about predictions using the fitted model.
}

\chapter{Useful Resources}
\section{Data Sets}
\section{Packages and Source Codes}
\section{Important Papers}



\begin{thebibliography}{100} % 100 is a random guess of the total number of %references  
\bibitem{Stewart} James Stewart {\em{Calsulus - Early Transcendentals}}. Cengage Learning, 2012
\bibitem{Rudin} Walter Rudin {\em{Principles of Mathematical Analysis}}. McGraw-Hill Companies, Inc., 1976.
\bibitem{Royden} H. L. Royden {\em{Real Analysis}}. Pearson Eduction, Inc., 1988.
\bibitem{Kreyszig} Erwin Kreyszig {\em{Introductory Functional Analysis with Applications}}. Wiley, 1989.
\bibitem{Folland} Gerald B. Folland {\em{Real Analysis: Modern Techniques and Their Applications}}. Wiley, 1999.
\bibitem{Torchinsky} Alberto Torchinsky {\em{Real Variables}}. Westview Press, 1995.

\bibitem{Wasserman} {\url{http://normaldeviate.wordpress.com/2012/11/17/what-is-bayesianfrequentist-inference/}}
\bibitem{Hochster} {\url{http://www.quora.com/What-is-the-difference-between-Bayesian-and-frequentist-statisticians}}
\bibitem{bayesian-inference advantage} {\url{http://www.bayesian-inference.com/advantagesbayesian}}
\bibitem{bayesian-inference likelihood} {\url{http://www.bayesian-inference.com/likelihood#likelihoodprinciple}}
\bibitem{Rossi} Rossi P, Allenby G, McCulloch R. {\emph{Bayesian Statistics and Marketing (pp. 4)}}. John Wiley \& Sons, 2005.
\bibitem{Efron 1978} Efron, Bradley. {\emph{Controversies in the Foundations of Statistics}}. The American Mathematical Monthly, Vol. 85, No. 4 (Apr., 1978), pp. 231-246.
\bibitem{Efron 2013} Efron, Bradley. {\emph{A 250-year Argument: Belief, Behavior, and the Bootstrap}}. Bull. Amer. Math. Soc. 50 (2013), 129-146.
\bibitem{quora CI} {\url{http://www.quora.com/Statistics-academic-discipline/What-is-a-confidence-interval-in-laymans-terms}}
\bibitem{quora diff} {\url{http://www.quora.com/What-is-the-difference-between-Bayesian-and-frequentist-statisticians}}
\bibitem{wiki} {\url{http://en.wikipedia.org/wiki/Confidence_interval#Meaning_and_interpretation}}

\bibitem{Minka} Thomas P. Minka. {\em{Old and New Matrix Algebra Useful for Statistics}}. December 28, 2000.
\bibitem{Wikepedia} \url{http://en.wikipedia.org/wiki/Matrix_calculus}. Accessed on \today
\bibitem{Searle} S. R. Searle and H. V. Henderson. {\em{A Primer on Differential Calculus for Vectors and Matrices}}. BU-1047-MB, 1993.
\bibitem{Nydick} Steven W. Nydick. {\em{A Different(ial) Way Matrix Derivatives Again}}. May 17, 2012.
\bibitem{Nydick} Steven W. Nydick. {\em{With(out) A Trace Matrix Derivatives the Easy Way}}. May 16, 2012.
\bibitem{Roweis} Sam Roweis. {\em{Matrix Identities}}. June 1999.
\bibitem{Tao} Terry Tao. {\em{Matrix identities as derivatives of determinant identities}}. January 13, 2013
\end{thebibliography}

\end{document}


\part{Numerical Methods and Optimization}
\documentclass{memoir}
\usepackage{amssymb,amsmath,mathtools}


\usepackage{algorithm2e,algorithmic}

\usepackage{mathrsfs}
\usepackage{paralist}

\usepackage{esint} % for \fint

\allowdisplaybreaks

\usepackage{color}		% enable color characters
\usepackage{graphicx} 	% insert image files
\usepackage{enumerate} 	% enumerate items
\usepackage{caption}
\usepackage{subcaption}
\usepackage{multirow,multicol}
\usepackage[makeroom]{cancel}

\usepackage[colorlinks=true,linkcolor=blue,citecolor=blue]{hyperref}

\usepackage{makeidx}

\newcommand{\cem}[1]{\color{magenta}{\em{#1}}} % color emphasize
\newcommand{\dual}[2]{{#1}^{(#2)}} % color emphasize

\newcommand{\tr}{{\mathrm{tr}}}
\renewcommand{\vec}{{\mathrm{vec}}}

\newcommand{\dd}{{\,\text{d}\,}}
%%
%% horizontal and vertical centering in table p mode
%%
\usepackage{array}
\newcolumntype{P}[1]{>{\centering\arraybackslash}p{#1}} % horizontal centering
\newcolumntype{M}[1]{>{\centering\arraybackslash}m{#1}} % vertical centering

%%
%% define bold font for the alphabet
%%
\usepackage{pgffor}
\foreach \letter in {a,...,z}{ % bold font for a..z
\expandafter\xdef\csname \letter \endcsname{\noexpand\ensuremath{\noexpand\mathbf{\letter}}}
}
\foreach \letter in {A,...,Z}{ % bold font for A..Z
\expandafter\xdef\csname \letter \endcsname{\noexpand\ensuremath{\noexpand\mathbf{\letter}}}
}
\foreach \letter in {A,...,Z}{ % `field' font for AA..ZZ
\expandafter\xdef\csname \letter\letter \endcsname{\noexpand\ensuremath{\noexpand\mathcal{\letter}}}
}
\foreach \letter in {A,...,Z}{ % `field' font for AAA..ZZZ
\expandafter\xdef\csname \letter\letter\letter \endcsname{\noexpand\ensuremath{\noexpand\mathbb{\letter}}}
}
\newcommand{\balpha}{{\boldsymbol{\alpha}}}
\newcommand{\bbeta}{{\boldsymbol{\beta}}}
\newcommand{\bgamma}{{\boldsymbol{\gamma}}}
\newcommand{\bkappa}{{\boldsymbol{\kappa}}}
\newcommand{\bmu}{{\boldsymbol{\mu}}}
\newcommand{\btheta}{{\boldsymbol{\theta}}}
\newcommand{\bTheta}{{\boldsymbol{\Theta}}}
\newcommand{\bPi}{{\boldsymbol{\Pi}}}
\newcommand{\bSigma}{{\boldsymbol{\Sigma}}}
\newcommand{\bPhi}{{\boldsymbol{\Phi}}}
\newcommand{\bLambda}{{\boldsymbol{\Lambda}}}
\newcommand{\bdeta}{{\boldsymbol{\eta}}}
\newcommand{\bphi}{{\boldsymbol{\phi}}}



%%
%% add definitions and theorems
%%
\usepackage[thmmarks,amsmath]{ntheorem}
\theorembodyfont{\normalfont}
\newtheorem{deff}{Definition}[section]
\newtheorem{thm}{Theorem}[section]
\newtheorem{prop}{Proposition}[section]
\newtheorem{lem}{Lemma}[section]
\newtheorem{cor}{Corollary}[section]
\newtheorem{rmk}{Remark}[section]
\newtheorem{alg}{Algorithm}[section]
\newtheorem{ex}{Example}[section]
\newtheorem{ques}{Question}[section]
\newtheorem{ans}{Answer}[section]
\newtheorem{prob}{Problem}[section]
\newtheorem{sol}{Solution}[section]
\newtheorem*{prof}{Proof}[section] 

\title{Optimization Notes}
\author{Xi Tan (tan19@purdue.edu)}
\date{\today}

\makeindex

\begin{document}
\maketitle
\tableofcontents

\chapter{Introduction}
{\em{Quadratic optimization problems}} (including, e.g., least-squares) form the base of the hierarchy; they can be solved exactly by solving a set of linear equations. {\em{Newton's method}} is the next level in the hierarchy. In Newton's method, solving an unconstrained or equality constrained problem is reduced to solving a sequence of quadratic problems. {\em{The interior-point methods}}, which form the top level of the hierarchy, solve an inequality constrained problem by solving a sequence of unconstrained, or equality constrained, problems. Besides Newton's method, there are quasi-Newton, conjugate-gradient, bundle, cutting-plane algorithms, and etc.

Optimization problems can be broadly divided into two types: linear optimization and nonlinear optimization,  the later of which consists of unconstrained and constrained optimization problems. Obtaining necessary and sufficient conditions is one of the central problems of nonlinear optimization. The main theory is the study of Lagrange multipliers, including the Karush-Kuhn-Tucker (KKT) theorem and its extensions. However, the theory of Lagrange multipliers is far from adequate, since it does not take into account the difficulties associated with solving the equations resulting from the necessary conditions.

\deff{
	{\em{Global convergence analysis}}. The verification that a given algorithm will in fact generate a sequence that converges to a solution point.
	{\em{Local convergence analysis}} or {\em{complexity analysis}}. The rate at which the generated sequence of points converges to the solution.
}

\chapter{Linear Programming}
\section{Basic Properties of Linear Programs}
\deff{
	Let $A$ be an $m \times n$ matrix and $B$ be any {\em{nonsingular}} $m \times m$ sub-matrix made up of columns of $A$. Then, if all $n-m$ components of $x$ not associated with columns of $B$ are set equal to zero, the solution to the resulting set of equations is said to be a {\em{basic solution}} with respect to the basis $B$. The components of $x$ associated with columns of $B$ are called {\em{basic variables}}.
}

\deff{
	If one or more of the basic variables in a basic solution has value zero, that solution is said to be a {\em{degenerate basic solution}}.
}

\deff{
	A feasible solution that is also basic is said to be a {\em{basic feasible solution}}; if this solution is also a degenerate basic solution, it is called a {\em{degenerate basic feasible solution}}.
}

\thm{
	Fundamental Theorem of Linear Programming. Given a linear program in standard form. i) if there is a feasible solution, there is a basic feasible solution; ii) if there is an optimal feasible solution, there is an optimal basic feasible solution.
}


Simplex algorithm has an exponential worst-case complexity, but polynomial average-case complexity.

\chapter{Unconstrained Optimization}
\section{Univariate Problems (Bisection, Newton, Secant Methods)}
Note that, $\min(f(x)))$ can be converted to the root-finding $f'(x) = 0$ problem.

\subsection{Bisection Method}
\index{bisection method}
Suppose $g'$ is continuous on $[a_0,b_0]$ and $g'(a_0)g'(b_0) \le 0$, then the Intermediate Value Theorem implies that there exists at least one $x^*$ for which $g'(x^*) = 0$ and hence $x^*$ is a local optimum of $g$. To find this local optimum, the Bisection Method systematically halves the interval at each iteration, by checking the product of $g'$.

The updating equations are
\begin{align}
[a_{t+1},b_{t+1}] =
	\begin{cases}
		[a_t,x^{(t)}], & \mbox{if } g'(a_t)g'(x^{t}) \le 0 \\
		[x^{(t)},b_t], & \mbox{if } g'(a_t)g'(x^{t}) > 0
	\end{cases}
\end{align}
and $x^{t+1} = \frac{a_{t+1}+b_{t+1}}{2}$.

\subsection{Newton's Method}
\index{Newton's method}
Suppose $g$ twice differentiable. At iteration $t$, Newton's method approximates $g'(x^*)$ by the linear Taylor series expansion:
\begin{align}
	g'(x^*) = g'(x^{(t)}) + (x^*-x^{(t)})(g''(x^{(t)}))
\end{align}
which gives us
\begin{align}
	x^* = x - \frac{g'(x^{(t)})}{g''(x^{(t)})}
\end{align}

\subsection{The Secant Method}
\index{secant method}
Recall that the Newton's method requires the function's derivative, which is always available. The secant method approximates the derivative with difference. It works as follows:
\begin{itemize}
  \item Start with two approximations $x_0$ and $x_1$.
  \item Compute the $(k+1)$th approximation with 
  \begin{align}
    x_{k+1} & \equiv x_k - f(x_k)/\frac{f(x_k) - f(x_{k-1})}{x_k - x_{k-1}}
  \end{align}
  \item The convergence rate of the secant method is $1.618$.

\end{itemize}

\section{Quasi-Newton Methods}

\chapter{Convex Optimization}

\printindex

\end{document}








\part{Software Engineering and Algorithms}
\part{Interview Questions}
\input{Interview/Interview}

\part{Notes}



\begin{thebibliography}{100} % 100 is a random guess of the total number of %references
\bibitem{Stewart} James Stewart {\em{Calsulus - Early Transcendentals}}. Cengage Learning, 2012
\bibitem{Rudin} Walter Rudin {\em{Principles of Mathematical Analysis}}. McGraw-Hill Companies, Inc., 1976.
\bibitem{Royden} H. L. Royden {\em{Real Analysis}}. Pearson Eduction, Inc., 1988.
\bibitem{Kreyszig} Erwin Kreyszig {\em{Introductory Functional Analysis with Applications}}. Wiley, 1989.
\bibitem{Folland} Gerald B. Folland {\em{Real Analysis: Modern Techniques and Their Applications}}. Wiley, 1999.
\bibitem{Torchinsky} Alberto Torchinsky {\em{Real Variables}}. Westview Press, 1995.

\bibitem{Wasserman} {\url{http://normaldeviate.wordpress.com/2012/11/17/what-is-bayesianfrequentist-inference/}}
\bibitem{Hochster} {\url{http://www.quora.com/What-is-the-difference-between-Bayesian-and-frequentist-statisticians}}
\bibitem{bayesian-inference advantage} {\url{http://www.bayesian-inference.com/advantagesbayesian}}
\bibitem{bayesian-inference likelihood} {\url{http://www.bayesian-inference.com/likelihood#likelihoodprinciple}}
\bibitem{Rossi} Rossi P, Allenby G, McCulloch R. {\emph{Bayesian Statistics and Marketing (pp. 4)}}. John Wiley \& Sons, 2005.
\bibitem{Efron 1978} Efron, Bradley. {\emph{Controversies in the Foundations of Statistics}}. The American Mathematical Monthly, Vol. 85, No. 4 (Apr., 1978), pp. 231-246.
\bibitem{Efron 2013} Efron, Bradley. {\emph{A 250-year Argument: Belief, Behavior, and the Bootstrap}}. Bull. Amer. Math. Soc. 50 (2013), 129-146.
\bibitem{quora CI} {\url{http://www.quora.com/Statistics-academic-discipline/What-is-a-confidence-interval-in-laymans-terms}}
\bibitem{quora diff} {\url{http://www.quora.com/What-is-the-difference-between-Bayesian-and-frequentist-statisticians}}
\bibitem{wiki} {\url{http://en.wikipedia.org/wiki/Confidence_interval#Meaning_and_interpretation}}

\bibitem{Minka} Thomas P. Minka. {\em{Old and New Matrix Algebra Useful for Statistics}}. December 28, 2000.
\bibitem{Wikepedia} \url{http://en.wikipedia.org/wiki/Matrix_calculus}. Accessed on \today
\bibitem{Searle} S. R. Searle and H. V. Henderson. {\em{A Primer on Differential Calculus for Vectors and Matrices}}. BU-1047-MB, 1993.
\bibitem{Nydick} Steven W. Nydick. {\em{A Different(ial) Way Matrix Derivatives Again}}. May 17, 2012.
\bibitem{Nydick} Steven W. Nydick. {\em{With(out) A Trace Matrix Derivatives the Easy Way}}. May 16, 2012.
\bibitem{Roweis} Sam Roweis. {\em{Matrix Identities}}. June 1999.
\bibitem{Tao} Terry Tao. {\em{Matrix identities as derivatives of determinant identities}}. January 13, 2013
\bibitem{Harrison05} P. G. Harrison. {\em{Sloppy Derivations of It\^o's Formula and the Fokker-Planck Equations}}. April 2005.
\bibitem{Rouah} Fabrice Douglas Rouah. {\em{Heuristic Derivation of the Fokker-Planck Equation}}. Retrieved 2/2/2019.
\bibitem{Stewart} James Stewart {\em{Calsulus - Early Transcendentals}}. Cengage Learning, 2012
\bibitem{Rudin} Walter Rudin {\em{Principles of Mathematical Analysis}}. McGraw-Hill Companies, Inc., 1976.
\bibitem{Royden} H. L. Royden {\em{Real Analysis}}. Pearson Eduction, Inc., 1988.
\bibitem{Kreyszig} Erwin Kreyszig {\em{Introductory Functional Analysis with Applications}}. Wiley, 1989.
\bibitem{Folland} Gerald B. Folland {\em{Real Analysis: Modern Techniques and Their Applications}}. Wiley, 1999.
\bibitem{Torchinsky} Alberto Torchinsky {\em{Real Variables}}. Westview Press, 1995.
\bibitem{Gallian} Joseph A. Gallian {\em{Contemporary Abstract Algebra (7th Edition)}}. Cengage Learning, 2010.
\bibitem{Stewart} James Stewart {\em{Calsulus - Early Transcendentals}}. Cengage Learning, 2012
\bibitem{Rudin} Walter Rudin {\em{Principles of Mathematical Analysis}}. McGraw-Hill Companies, Inc., 1976.
\bibitem{Royden} H. L. Royden {\em{Real Analysis}}. Pearson Eduction, Inc., 1988.
\bibitem{Kreyszig} Erwin Kreyszig {\em{Introductory Functional Analysis with Applications}}. Wiley, 1989.
\bibitem{Folland} Gerald B. Folland {\em{Real Analysis: Modern Techniques and Their Applications}}. Wiley, 1999.
\bibitem{Torchinsky} Alberto Torchinsky {\em{Real Variables}}. Westview Press, 1995.

\bibitem{Wasserman} {\url{http://normaldeviate.wordpress.com/2012/11/17/what-is-bayesianfrequentist-inference/}}
\bibitem{Hochster} {\url{http://www.quora.com/What-is-the-difference-between-Bayesian-and-frequentist-statisticians}}
\bibitem{bayesian-inference advantage} {\url{http://www.bayesian-inference.com/advantagesbayesian}}
\bibitem{bayesian-inference likelihood} {\url{http://www.bayesian-inference.com/likelihood#likelihoodprinciple}}
\bibitem{Rossi} Rossi P, Allenby G, McCulloch R. {\emph{Bayesian Statistics and Marketing (pp. 4)}}. John Wiley \& Sons, 2005.
\bibitem{Efron 1978} Efron, Bradley. {\emph{Controversies in the Foundations of Statistics}}. The American Mathematical Monthly, Vol. 85, No. 4 (Apr., 1978), pp. 231-246.
\bibitem{Efron 2013} Efron, Bradley. {\emph{A 250-year Argument: Belief, Behavior, and the Bootstrap}}. Bull. Amer. Math. Soc. 50 (2013), 129-146.
\bibitem{quora CI} {\url{http://www.quora.com/Statistics-academic-discipline/What-is-a-confidence-interval-in-laymans-terms}}
\bibitem{quora diff} {\url{http://www.quora.com/What-is-the-difference-between-Bayesian-and-frequentist-statisticians}}
\bibitem{wiki} {\url{http://en.wikipedia.org/wiki/Confidence_interval#Meaning_and_interpretation}}

\bibitem{Minka} Thomas P. Minka. {\em{Old and New Matrix Algebra Useful for Statistics}}. December 28, 2000.
\bibitem{Wikepedia} \url{http://en.wikipedia.org/wiki/Matrix_calculus}. Accessed on \today
\bibitem{Searle} S. R. Searle and H. V. Henderson. {\em{A Primer on Differential Calculus for Vectors and Matrices}}. BU-1047-MB, 1993.
\bibitem{Nydick} Steven W. Nydick. {\em{A Different(ial) Way Matrix Derivatives Again}}. May 17, 2012.
\bibitem{Nydick} Steven W. Nydick. {\em{With(out) A Trace Matrix Derivatives the Easy Way}}. May 16, 2012.
\bibitem{Roweis} Sam Roweis. {\em{Matrix Identities}}. June 1999.
\bibitem{Tao} Terry Tao. {\em{Matrix identities as derivatives of determinant identities}}. January 13, 2013
\bibitem{Stewart} James Stewart {\em{Calsulus - Early Transcendentals}}. Cengage Learning, 2012
\bibitem{Rudin} Walter Rudin {\em{Principles of Mathematical Analysis}}. McGraw-Hill Companies, Inc., 1976.
\bibitem{Royden} H. L. Royden {\em{Real Analysis}}. Pearson Eduction, Inc., 1988.
\bibitem{Kreyszig} Erwin Kreyszig {\em{Introductory Functional Analysis with Applications}}. Wiley, 1989.
\bibitem{Folland} Gerald B. Folland {\em{Real Analysis: Modern Techniques and Their Applications}}. Wiley, 1999.
\bibitem{Torchinsky} Alberto Torchinsky {\em{Real Variables}}. Westview Press, 1995.

\bibitem{Wasserman} {\url{http://normaldeviate.wordpress.com/2012/11/17/what-is-bayesianfrequentist-inference/}}
\bibitem{Hochster} {\url{http://www.quora.com/What-is-the-difference-between-Bayesian-and-frequentist-statisticians}}
\bibitem{bayesian-inference advantage} {\url{http://www.bayesian-inference.com/advantagesbayesian}}
\bibitem{bayesian-inference likelihood} {\url{http://www.bayesian-inference.com/likelihood#likelihoodprinciple}}
\bibitem{Rossi} Rossi P, Allenby G, McCulloch R. {\emph{Bayesian Statistics and Marketing (pp. 4)}}. John Wiley \& Sons, 2005.
\bibitem{Efron 1978} Efron, Bradley. {\emph{Controversies in the Foundations of Statistics}}. The American Mathematical Monthly, Vol. 85, No. 4 (Apr., 1978), pp. 231-246.
\bibitem{Efron 2013} Efron, Bradley. {\emph{A 250-year Argument: Belief, Behavior, and the Bootstrap}}. Bull. Amer. Math. Soc. 50 (2013), 129-146.
\bibitem{quora CI} {\url{http://www.quora.com/Statistics-academic-discipline/What-is-a-confidence-interval-in-laymans-terms}}
\bibitem{quora diff} {\url{http://www.quora.com/What-is-the-difference-between-Bayesian-and-frequentist-statisticians}}
\bibitem{wiki} {\url{http://en.wikipedia.org/wiki/Confidence_interval#Meaning_and_interpretation}}

\bibitem{Minka} Thomas P. Minka. {\em{Old and New Matrix Algebra Useful for Statistics}}. December 28, 2000.
\bibitem{Wikepedia} \url{http://en.wikipedia.org/wiki/Matrix_calculus}. Accessed on \today
\bibitem{Searle} S. R. Searle and H. V. Henderson. {\em{A Primer on Differential Calculus for Vectors and Matrices}}. BU-1047-MB, 1993.
\bibitem{Nydick} Steven W. Nydick. {\em{A Different(ial) Way Matrix Derivatives Again}}. May 17, 2012.
\bibitem{Nydick} Steven W. Nydick. {\em{With(out) A Trace Matrix Derivatives the Easy Way}}. May 16, 2012.
\bibitem{Roweis} Sam Roweis. {\em{Matrix Identities}}. June 1999.
\bibitem{Tao} Terry Tao. {\em{Matrix identities as derivatives of determinant identities}}. January 13, 2013
\bibitem{Harrison05} P. G. Harrison. {\em{Sloppy Derivations of It\^o's Formula and the Fokker-Planck Equations}}. April 2005.
\bibitem{Rouah} Fabrice Douglas Rouah. {\em{Heuristic Derivation of the Fokker-Planck Equation}}. Retrieved 2/2/2019.
\end{thebibliography}

\printindex

\end{document}