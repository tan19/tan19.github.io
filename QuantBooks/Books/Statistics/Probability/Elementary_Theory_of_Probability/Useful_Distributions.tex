\section{Useful Distributions}
\section{Discrete Distributions}
\subsection{Poisson Distribution}
\subsection{Bernoulli Distribution}
\subsection{Binomial Distribution}
\subsection{Negative Binomial Distribution}
\subsection{Categorical Distribution}
\subsection{Multinomial Distribution}
\subsection{Geometric Distribution}
\subsection{Hyper-Geometric Distribution}
\subsection{Poisson Distribution}
\section{Continuous Distributions}
\subsection{Uniform Distribution}
\subsection{Exponential Distribution}
\subsection{$\chi^2$ Distribution}

\subsection{Gaussian Distribution}
\subsubsection{Univariate Gaussian}
\subsubsection{Multivariate Gaussian}
\begin{align}
	f(\x|\bmu,\bSigma) = \frac{1}{\sqrt{2\pi|\bSigma|}} \exp\left\{-\frac{1}{2}(\x-\bmu)^T\bSigma^{-1}(\x-\bmu)\right\}
\end{align}
Using the technique of `completing the square', we have
\begin{align}
	f(\x_1|\x_2) &= \frac{f(\x_1,\x_2)}{f(\x_2)}\\
	&= \frac{1/\sqrt{2\pi|\bSigma|}\exp\left\{-\frac{1}{2}(\x-\bmu)^T\bSigma^{-1}(\x-\bmu)\right\}}{1/\sqrt{2\pi|\bSigma_{22}|}\exp\left\{-\frac{1}{2}(\x_2-\bmu)^T\bSigma_{22}^{-1}(\x_2-\bmu)\right\}}\\
	&= C \cdot \exp\left\{-\frac{1}{2}(\x-\bmu)^T\bSigma^{-1}(\x-\bmu)\right\}\\
	&= C \cdot \exp\left\{-\frac{1}{2}[(\x_1-\bmu_1)^T\bLambda_{11}(\x_1-\bmu_1)+(\x_1-\bmu_1)^T\bLambda_{12}(\x_2-\bmu_2)+(\x_2-\bmu_2)^T\bLambda_{21}(\x_1-\bmu_1)]\right\}\\
	&= C \cdot \exp\left\{-\frac{1}{2}\x_1^T\bLambda_{11}\x_1 + \x_1^T[\bLambda_{11}\bmu_1 - \bLambda_{12}(\x_2-\bmu_2)]\right\}	
\end{align}
Hence
\begin{align}
	\bSigma_{1|2} &= \bLambda_{11}^{-1}\\
	\bmu_{1|2} &= \bmu_1 - \bLambda_{11}^{-1}\bLambda_{12}(\x_2-\bmu_2)
\end{align}
or
\begin{align}
	\bSigma_{1|2} &= \bSigma_{11} - \bSigma_{12}\bSigma_{22}^{-1}\bSigma_{21}\\
	\bmu_{1|2} &= \bmu_1 + \bSigma_{12}\bSigma_{22}^{-1}(\x_2-\bmu_2)
\end{align}

\subsection{Dirichlet Distribution}
\subsubsection{Gamma Function and Beta Function}
The {\em{gamma function}} is defined for all complex numbers except the non-positive integers. For complex numbers with a positive real part, it is defined via an improper integral that converges:
\begin{equation}
	\Gamma(z) = \int_0^\infty e^{-t} t^{z-1} dt
\end{equation}
For all positive numbers $z$,
\begin{equation}
	\Gamma(z) = (z-1)\Gamma(z-1)
\end{equation}
and in particular, if $n$ is a positive integer:
\begin{equation}
	\Gamma(n) = (n-1)!
\end{equation}
Note, the gamma function shifts the normal definition of factorial by 1.

The {\em{beta function}} is defined by
\begin{equation}
	B(x,y) = \int_0^1 t^{x-1} (1-t)^{y-1} dt
\end{equation}
for $Re(x), Re(y) > 0$.

It can also be written as
\begin{equation}
	B(x,y) = \frac{\Gamma(x) \Gamma(y)}{\Gamma(x+y)}
\end{equation}

\subsubsection{Introduction}
The Dirichlet distribution of order $K \ge 2$ with parameters $\alpha_1, \dots, \alpha_K > 0$ has a probability density function with respect to Lebesgue measure on the Euclidean space $\RR^{K-1}$ given by
\begin{equation}
	f(x_1, \dots, x_{K-1}; \alpha_1, \dots, \alpha_K) = \frac{1}{B(\balpha)} \prod^K_{i=1} x_i^{\alpha_i-1}
\end{equation}
for all $x_1, \dots, x_K > 0$ and $x_1 + \dots + x_K = 1$. The density is zero outside this open\footnote{``Open'' here means none of the $x_i$'s can be 1, actually $x_i \in (0,1)$.} $(K-1)$-dimensional simplex.

The normalizing constant is the multinomial beta function, which can be expressed in terms of the gamma function:
\begin{equation}
	B(\balpha) = \frac{\prod_{i=1}^K \Gamma(\alpha_i)}{\Gamma(\sum_{i=1}^K \alpha_i)}, \mbox{ where } \balpha=(\alpha_1, \dots, \alpha_K)
\end{equation}


\subsection{$t$ Distribution}
\subsection{Inverse Gaussian Distribution}
\subsection{log-normal Distribution}
\subsection{Laplace Distribution}
\subsection{Beta Distribution}
\subsection{Gamma Distribution}
\subsection{Wishart Distribution} 