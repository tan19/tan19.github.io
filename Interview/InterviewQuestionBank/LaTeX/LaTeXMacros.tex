\usepackage{amssymb,amsmath,mathtools}

\usepackage{algorithm2e,algorithmic}
\usepackage{listings}
\usepackage{xcolor}
\lstset { %
    language=C++,
    backgroundcolor=\color{black!5}, % set backgroundcolor
    basicstyle=\footnotesize,% basic font setting
}

\usepackage{mathrsfs}
\usepackage{paralist}


\usepackage{svg}

%% Draw dice
\usepackage{tikz}
\usetikzlibrary{
  decorations.pathreplacing,
  calc,
  patterns,
  datavisualization.formats.functions,
  external}
\usepackage{tikz}
\usetikzlibrary{shapes}
\tikzset{
  dot hidden/.style={},
  line hidden/.style={},
  dot colour/.style={dot hidden/.append style={color=#1}},
  dot colour/.default=black,
  line colour/.style={line hidden/.append style={color=#1}},
  line colour/.default=black
}

\usepackage{esint} % for \fint

\allowdisplaybreaks

\usepackage{color}
\usepackage{graphicx} 	% insert image files
\usepackage{enumerate} 	% enumerate items
\usepackage{caption}
\usepackage{subcaption}
\usepackage{multirow,multicol}
\usepackage[makeroom]{cancel}

\usepackage{import} % to import subfolder LaTeX files
\usepackage{imakeidx}

\usepackage{shorttoc}
\usepackage{minitoc}


\usepackage[colorlinks=true,linkcolor=blue,citecolor=blue]{hyperref} % need to be place after imakeidx

\newcommand{\cem}[1]{\color{magenta}{\em{#1}}} % color emphasize
\newcommand{\dual}[2]{{#1}^{(#2)}} % color emphasize

\newcommand{\tr}{{\mathrm{tr}}}
\renewcommand{\vec}{{\mathrm{vec}}}

\newcommand{\dd}{{\,\mathrm{d}\,}}
%%
%% horizontal and vertical centering in table p mode
%%
\usepackage{array}
\newcolumntype{P}[1]{>{\centering\arraybackslash}p{#1}} % horizontal centering
\newcolumntype{M}[1]{>{\centering\arraybackslash}m{#1}} % vertical centering

%%
%% define bold font for the alphabet
%%
\usepackage{pgffor}
\foreach \letter in {a,...,z}{ % bold font for a..z
\expandafter\xdef\csname \letter \endcsname{\noexpand\ensuremath{\noexpand\mathbf{\letter}}}
}
\foreach \letter in {A,...,Z}{ % bold font for A..Z
\expandafter\xdef\csname \letter \endcsname{\noexpand\ensuremath{\noexpand\mathbf{\letter}}}
}
\foreach \letter in {A,...,Z}{ % `field' font for AA..ZZ
\expandafter\xdef\csname \letter\letter \endcsname{\noexpand\ensuremath{\noexpand\mathcal{\letter}}}
}
\foreach \letter in {A,...,Z}{ % `field' font for AAA..ZZZ
\expandafter\xdef\csname \letter\letter\letter \endcsname{\noexpand\ensuremath{\noexpand\mathbb{\letter}}}
}
\newcommand{\balpha}{{\boldsymbol{\alpha}}}
\newcommand{\bbeta}{{\boldsymbol{\beta}}}
\newcommand{\bgamma}{{\boldsymbol{\gamma}}}
\newcommand{\bkappa}{{\boldsymbol{\kappa}}}
\newcommand{\bmu}{{\boldsymbol{\mu}}}
\newcommand{\btheta}{{\boldsymbol{\theta}}}
\newcommand{\bTheta}{{\boldsymbol{\Theta}}}
\newcommand{\bPi}{{\boldsymbol{\Pi}}}
\newcommand{\bSigma}{{\boldsymbol{\Sigma}}}
\newcommand{\bPhi}{{\boldsymbol{\Phi}}}
\newcommand{\bLambda}{{\boldsymbol{\Lambda}}}
\newcommand{\bdeta}{{\boldsymbol{\eta}}}
\newcommand{\bphi}{{\boldsymbol{\phi}}}

\newcommand{\bOne}{{\boldsymbol{1}}}
\newcommand{\bZero}{{\boldsymbol{0}}}



%%
%% add definitions and theorems
%%
\usepackage[thmmarks,amsmath]{ntheorem}
\theorembodyfont{\normalfont}
\newtheorem{deff}{Definition}[section]
\newtheorem{thm}{Theorem}[section]
\newtheorem{prop}{Proposition}[section]
\newtheorem{lem}{Lemma}[section]
\newtheorem{cor}{Corollary}[section]
\newtheorem{rmk}{Remark}[section]
\newtheorem{alg}{Algorithm}[section]
\newtheorem{ex}{Example}[section]
\newtheorem{ques}{Question}[section]
\newtheorem{ans}{Answer}[section]
\newtheorem{prob}{Problem}[section]
\newtheorem{sol}{Solution}[section]
\newtheorem*{prof}{Proof}[section] 



\usepackage{pgffor,xparse,listofitems}


\NewDocumentCommand{\drawdie}{O{}m}{%
\begin{tikzpicture}[x=1em,y=1em,radius=0.1,#1]
  \draw[rounded corners=0.5,line hidden] (0,0) rectangle (1,1);
  \ifodd#2
    \fill[dot hidden] (0.5,0.5) circle;
  \fi
  \ifnum#2>1
    \fill[dot hidden] (0.2,0.2) circle;
    \fill[dot hidden] (0.8,0.8) circle;
   \ifnum#2>3
     \fill[dot hidden] (0.2,0.8) circle;
     \fill[dot hidden] (0.8,0.2) circle;
    \ifnum#2>5
      \fill[dot hidden] (0.8,0.5) circle;
      \fill[dot hidden] (0.2,0.5) circle;
     \ifnum#2>7
       \fill[dot hidden] (0.5,0.8) circle;
       \fill[dot hidden] (0.5,0.2) circle;
      \fi
    \fi
  \fi
\fi
\end{tikzpicture}%
}  


% Make some ntheorem environments to hold the questions and answers

\theoremstyle{plain} % plain or break
\theoremheaderfont{\bfseries}
\theorembodyfont{}
\theoremnumbering{arabic}
\theoremseparator{:}
\newtheorem{ntquestion}{Question}

\theoremstyle{plain}
\theoremheaderfont{\bfseries}
\theorembodyfont{}
\theoremnumbering{arabic}
\theoremseparator{}
%\theoremindent=0cm
\newtheorem*{ntanswer}{Answer}


% Make new environments to ease the labelling
% use the xparse package, since we need the arguments in the last bit of the environment as well
\newcounter{indexlink}
\newcommand{\Index}[1]{\index{Tag: #1{\string\hypertarget{\theindexlink}{}}}\hyperlink{\theindexlink}{#1}}
  
\NewDocumentEnvironment{question}
{
	m % title
	m % difficulty level
	m % tags	
} 
{
	\begin{ntquestion}
	\label{q:#1}

	%	
	% process question name and level
	%
	\index{Question: #1}
	\index{Question Level: #2}	
	
	\underline{\emph{#1}} (#2) (Tags:\setsepchar{; }\readlist*\looplist{#3}\foreachitem \n \in \looplist{ \Index{\n};})\\
	
}
{	
	\begin{flushright}
		(\hyperref[a:#1]{Answer on page \pageref*{a:#1}})
	\end{flushright}
		
	\end{ntquestion}
}

\NewDocumentEnvironment{answer}{m}%
{
	\begin{ntanswer}{\bf{\ref*{q:#1}:}}
	\label{a:#1}
	{\color{blue} \underline{\emph{#1}}}
	\hfill(\hyperref[q:#1]{Question on page \pageref*{q:#1}})\\
	\\
}%
{
	\end{ntanswer}
}

%%%%%%%%%%%%%%%%%%%%%%%%%%%%%%%%%%%%%%%%%
% Usage
%%%%%%%%%%%%%%%%%%%%%%%%%%%%%%%%%%%%%%%%%
%\begin{question}{questionlabelname}
% Write stuff here
%\end{question}

%\begin{answer}{questionlabelname}
% Write stuff here
%\end{answer}

% this will automatically create the labels called `q:questionlabelname` and `a:questionlabelname`.
% To change the formatting of the question and answer blocks, change their corresponding ntheorem styles.

