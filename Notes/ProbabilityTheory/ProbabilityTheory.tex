\documentclass{article}
\usepackage{amssymb,amsmath}
\usepackage[thmmarks,amsmath]{ntheorem}

\usepackage{mathrsfs}
\usepackage{hyperref}

\usepackage{manfnt}
\usepackage{mathrsfs}

% Math commands by Thomas Minka
\newcommand{\var}{{\rm var}}
\newcommand{\Tr}{^{\rm T}}
\newcommand{\vtrans}[2]{{#1}^{(#2)}}
\newcommand{\kron}{\otimes}
\newcommand{\schur}[2]{({#1} | {#2})}
\newcommand{\schurdet}[2]{\left| ({#1} | {#2}) \right|}
\newcommand{\had}{\circ}
\newcommand{\diag}{{\rm diag}}
\newcommand{\invdiag}{\diag^{-1}}
\newcommand{\rank}{{\rm rank}}
% careful: ``null'' is already a latex command
\newcommand{\nullsp}{{\rm null}}
\newcommand{\tr}{{\rm tr}}
\newcommand{\vech}{{\rm vech}}
\renewcommand{\det}[1]{\left| #1 \right|}
\newcommand{\pdet}[1]{\left| #1 \right|_{+}}
\newcommand{\pinv}[1]{#1^{+}}
\newcommand{\erf}{{\rm erf}}
\newcommand{\hypergeom}[2]{{}_{#1}F_{#2}}

% boldface characters
\renewcommand{\a}{{\bf a}}
\renewcommand{\b}{{\bf b}}
\renewcommand{\c}{{\bf c}}
\renewcommand{\d}{{\rm d}}  % for derivatives
\newcommand{\e}{{\bf e}}
\newcommand{\f}{{\bf f}}
\newcommand{\g}{{\bf g}}
\newcommand{\h}{{\bf h}}
%\newcommand{\k}{{\bf k}}
% in Latex2e this must be renewcommand
\renewcommand{\k}{{\bf k}}
\newcommand{\m}{{\bf m}}
\newcommand{\n}{{\bf n}}
\renewcommand{\o}{{\bf o}}
\newcommand{\p}{{\bf p}}
\newcommand{\q}{{\bf q}}
\renewcommand{\r}{{\bf r}}
\newcommand{\s}{{\bf s}}
\renewcommand{\t}{{\bf t}}
\renewcommand{\u}{{\bf u}}
\renewcommand{\v}{{\bf v}}
\newcommand{\w}{{\bf w}}
\newcommand{\x}{{\bf x}}
\newcommand{\y}{{\bf y}}
\newcommand{\z}{{\bf z}}
\newcommand{\A}{{\bf A}}
\newcommand{\B}{{\bf B}}
\newcommand{\C}{{\bf C}}
\newcommand{\D}{{\bf D}}
\newcommand{\E}{{\bf E}}
\newcommand{\F}{{\bf F}}
\newcommand{\G}{{\bf G}}
\renewcommand{\H}{{\bf H}}
\newcommand{\I}{{\bf I}}
\newcommand{\J}{{\bf J}}
\newcommand{\K}{{\bf K}}
\renewcommand{\L}{{\bf L}}
\newcommand{\M}{{\bf M}}
\newcommand{\N}{{\cal N}}  % for normal density
%\newcommand{\N}{{\bf N}}
\renewcommand{\O}{{\bf O}}
\renewcommand{\P}{{\bf P}}
\newcommand{\Q}{{\bf Q}}
\newcommand{\R}{{\bf R}}
\renewcommand{\S}{{\bf S}}
\newcommand{\T}{{\bf T}}
\newcommand{\U}{{\bf U}}
\newcommand{\V}{{\bf V}}
\newcommand{\W}{{\bf W}}
\newcommand{\X}{{\bf X}}
\newcommand{\Y}{{\bf Y}}
\newcommand{\Z}{{\bf Z}}

% this is for latex 2.09
% unfortunately, the result is slanted - use Latex2e instead
%\newcommand{\bfLambda}{\mbox{\boldmath$\Lambda$}}
% this is for Latex2e
\newcommand{\bfLambda}{\boldsymbol{\Lambda}}

% Yuan Qi's boldsymbol
\newcommand{\bsigma}{\boldsymbol{\sigma}}
\newcommand{\balpha}{\boldsymbol{\alpha}}
\newcommand{\bpsi}{\boldsymbol{\psi}}
\newcommand{\bphi}{\boldsymbol{\phi}}
\newcommand{\bbeta}{\boldsymbol{\beta}}
\newcommand{\Beta}{\boldsymbol{\eta}}
\newcommand{\btau}{\boldsymbol{\tau}}
\newcommand{\bvarphi}{\boldsymbol{\varphi}}
\newcommand{\bzeta}{\boldsymbol{\zeta}}

\newcommand{\blambda}{\boldsymbol{\lambda}}
\newcommand{\bLambda}{\mathbf{\Lambda}}

\newcommand{\btheta}{\boldsymbol{\theta}}
\newcommand{\bpi}{\boldsymbol{\pi}}
\newcommand{\bxi}{\boldsymbol{\xi}}
\newcommand{\bSigma}{\boldsymbol{\Sigma}}

\newcommand{\bgamma}{\mathbf{\gamma}}
\newcommand{\bGamma}{\mathbf{\Gamma}}

\newcommand{\bmu}{\boldsymbol{\mu}}
\newcommand{\1}{{\bf 1}}
\newcommand{\0}{{\bf 0}}

%\newcommand{\comment}[1]{}

\newcommand{\bs}{\backslash}
\newcommand{\ben}{\begin{enumerate}}
\newcommand{\een}{\end{enumerate}}

 \newcommand{\notS}{{\backslash S}}
 \newcommand{\nots}{{\backslash s}}
 \newcommand{\noti}{{\backslash i}}
 \newcommand{\notj}{{\backslash j}}
 \newcommand{\nott}{\backslash t}
 \newcommand{\notone}{{\backslash 1}}
 \newcommand{\nottp}{\backslash t+1}
% \newcommand{\notz}{\backslash z}

\newcommand{\notk}{{^{\backslash k}}}
%\newcommand{\noti}{{^{\backslash i}}}
\newcommand{\notij}{{^{\backslash i,j}}}
\newcommand{\notg}{{^{\backslash g}}}
\newcommand{\wnoti}{{_{\w}^{\backslash i}}}
\newcommand{\wnotg}{{_{\w}^{\backslash g}}}
\newcommand{\vnotij}{{_{\v}^{\backslash i,j}}}
\newcommand{\vnotg}{{_{\v}^{\backslash g}}}
\newcommand{\half}{\frac{1}{2}}
\newcommand{\msgb}{m_{t \leftarrow t+1}}
\newcommand{\msgf}{m_{t \rightarrow t+1}}
\newcommand{\msgfp}{m_{t-1 \rightarrow t}}

\newcommand{\proj}[1]{{\rm proj}\negmedspace\left[#1\right]}
\newcommand{\argmin}{\operatornamewithlimits{argmin}}
\newcommand{\argmax}{\operatornamewithlimits{argmax}}

\newcommand{\dif}{\mathrm{d}}
\newcommand{\abs}[1]{\lvert#1\rvert}
\newcommand{\norm}[1]{\lVert#1\rVert}


\setlength{\marginparwidth}{1.2in}
\newcommand{\smallmarginpar}[1]{\marginpar[#1]{\small{#1}}}

\newcommand{\mA}{\mathscr{A}}
\newcommand{\FF}{\mathcal{F}}
\newcommand{\XX}{\mathcal{X}}
\newcommand{\YY}{\mathcal{Y}}
\newcommand{\HH}{\mathcal{H}}

\newcommand{\PP}{\mathscr{P}}
\newcommand{\QQ}{\mathscr{Q}}
\newcommand{\RR}{\mathscr{R}}

\newcommand{\ZZ}{\mathbb{Z}}


\theorembodyfont{\normalfont}
\newtheorem{definition}{Definition}[section]
\newtheorem{remark}{Remark}[section]
\newtheorem{example}{Example}[section]
\newtheorem{proof}{Proof}[section]


\title{MA538: Probability Theory I}
\author{Xi Tan (tan19@purdue.edu)}
\date{\today}

\begin{document}
\maketitle
\tableofcontents
\newpage

\section*{Preface}
TBD

\newpage
\section{Probability Space and Measure}
\subsection{Algebra of Sets}

\subsubsection{Set Operations}
Given two sets $A, B \in \Omega$, there are four basic binary operations on sets:
\begin{enumerate}
	\item {\em{Union}}: $A \cup B = \{x: x \in A \mbox{ or } x \in B\}$
	\item {\em{Intersection}}: $A \cap B = \{x: x \in A \mbox{ and } x \in B\}$
	\item {\em{Set difference}}: $A \setminus B = \{x: x \in A \mbox{ and } x \notin B\}$
	\item {\em{Set complement}}: $A^c = \Omega \setminus A$
\end{enumerate}

Set complement is the ``strongest'' operation, because if a collection of sets $\mA$ is closed under complement, and if it is also closed under any one of the other three operations, $\mA$ is closed under the rest two. That is seen from,
\begin{align}
	\mbox{If closed under } union &
	\begin{cases}
		A \cap B = (A^c \cup B^c)^c\\
		A \setminus B = A \cap B^c = (A^c \cup B)^c
	\end{cases}	
	\\
	\mbox{If closed under } intersection &
	\begin{cases}
		A \cup B = (A^c \cap B^c)^c\\		
		A \setminus B = A \cap B^c
	\end{cases}
	\\
	\mbox{If closed under } difference &
	\begin{cases}
		A \cap B = A \setminus (A \setminus B)\\
		A \cup B = (A^c \cap B^c)^c = (A^c \setminus (A^c \setminus B^c))^c\\
	\end{cases}		
\end{align}

The difference operation is the second ``strongest'' operation, in that if a collection of sets $\mA$ is closed under difference, it is closed under intersection, that is seen from,
\begin{align}
	A \cap B = A \setminus (A \setminus B)
\end{align}

The third ``strongest'' operation is the union, which can not be implied from difference or intersection, or their combination.

The ``weakest'' operation is the intersection, which can be implied from the difference.

To summarize, the ``strongest'' pair would be the complement plus any one more, which implies everything else; the second ``strongest'' would be the difference plus the union, which implies the intersection but not the complement ($\Omega$ may not be in the collection). \smallmarginpar{\textdbend}

\subsubsection{Class of Set Collection}
\begin{definition}
	Given a set $\Omega$, a non-empty collection $\PP \subset 2^\Omega$ is called a {\em{$\pi$-system}} iff:
	
	\center{$\forall A, B \in \PP$, $A \cap B \in \PP$}
\end{definition}
Notice, this has the weakest requirement.

\begin{definition}
	Given a set $\Omega$, a non-empty collection $\QQ \subset 2^\Omega$ is called a {\em{semiring}} iff:
	
	\center{$\forall A, B \in \QQ \mbox{ and } A \supset B$}
	\center{$\exists C_k \subset \QQ, s.t., A \setminus B = \bigcup_{k=1}^n C_k$}
\end{definition}

\begin{definition}
	Given a set $\Omega$, a non-empty collection $\RR \subset 2^\Omega$ is called a {\em{ring}} iff:
	
	\center{$\forall A, B \in \RR$, $A \cup B \in \RR$ and $B \setminus A \in \RR$}
\end{definition}

There are two points need to mention: the empty set is in a ring, since $A \setminus A = \emptyset$; $\mA$ is also closed under intersection (the reverse need not be true).

\begin{definition}
	A {\em{ring}} $\mA$ is called an {\em{field}} iff $\Omega \in \mA$.
\end{definition}
So a field is closed under all {\em{finite}} combination of set operations.


\begin{definition}
	An {\em{field}} is called a {\em{$\sigma$-field}} if for any sequence ${A_n}$ of sets in $\mA$, $\cup_{n \ge 1} A_n \in \mA$.
\end{definition}

$\not \exists$

\section{Integration Theory}
\section{Random Variables}
\section{Law of Large Numbers}
\section{Central Limit Theorem}

\section{Some Tricks}
\subsection{Prove by Contraposition}

\subsection{Construct Finer Partition}
Given twofinite partitions $\{A_n\}$ and $\{B_m\}$, a finer partition can be constructed as
\begin{align*}
	(\cup_{i=1}^n A_i) \bigcap (\cup_{j=1}^m B_j) = \bigcup (\cap_{i=1}^n \cap_{j=1}^m A_i B_j)
\end{align*}

\subsection{Prove Equality}
To prove two numerical quantities are equal $X=Y$, often times we can do this by showing $X \le Y$ and $X \ge Y$. Similarly, to prove two sets are equal $E = F$, we can show $E \subset F$ and $E \subset F$.

\subsection{An Epsilon of Room}
If one has to show that $X \le Y$, try proving that $X \le Y + \epsilon, \forall \epsilon > 0$. This trick combines well with the ``Prove Equality'' trick.

In a similar spirit, if one needs to show that a quantity $X$ vanishes, try showing that $|X| \le \epsilon, \forall \epsilon > 0$.

If one wants to show that a sequence $x_n$ of real numbers converges to zero, try showing that $\limsup_{n \to \infty} |x_n| \le \epsilon, \forall \epsilon > 0$

{\bf{One caveat:}} for finite $x$, and any $\epsilon > 0$, it is true that $x + \epsilon > x$ and $x - \epsilon < x$, but it is not true when $x$ is equal to $+\infty$ or $-\infty$. \smallmarginpar{\textdbend}

\end{document} 






