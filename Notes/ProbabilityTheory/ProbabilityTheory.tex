\documentclass{article}
\usepackage{amssymb,amsmath}
\usepackage[thmmarks,amsmath]{ntheorem}

\usepackage{mathrsfs}
\usepackage{hyperref}

\usepackage{manfnt}
\usepackage{mathrsfs}

\input{emacscomm.tex}

\setlength{\marginparwidth}{1.2in}
\newcommand{\smallmarginpar}[1]{\marginpar[#1]{\small{#1}}}

\newcommand{\mA}{\mathscr{A}}
\newcommand{\FF}{\mathcal{F}}
\newcommand{\XX}{\mathcal{X}}
\newcommand{\YY}{\mathcal{Y}}
\newcommand{\HH}{\mathcal{H}}

\newcommand{\PP}{\mathscr{P}}
\newcommand{\QQ}{\mathscr{Q}}
\newcommand{\RR}{\mathscr{R}}

\newcommand{\ZZ}{\mathbb{Z}}


\theorembodyfont{\normalfont}
\newtheorem{definition}{Definition}[section]
\newtheorem{remark}{Remark}[section]
\newtheorem{example}{Example}[section]
\newtheorem{proof}{Proof}[section]


\title{MA538: Probability Theory I}
\author{Xi Tan (tan19@purdue.edu)}
\date{\today}

\begin{document}
\maketitle
\tableofcontents
\newpage

\section*{Preface}
TBD

\newpage
\section{Probability Space and Measure}
\subsection{Algebra of Sets}

\subsubsection{Set Operations}
Given two sets $A, B \in \Omega$, there are four basic binary operations on sets:
\begin{enumerate}
	\item {\em{Union}}: $A \cup B = \{x: x \in A \mbox{ or } x \in B\}$
	\item {\em{Intersection}}: $A \cap B = \{x: x \in A \mbox{ and } x \in B\}$
	\item {\em{Set difference}}: $A \setminus B = \{x: x \in A \mbox{ and } x \notin B\}$
	\item {\em{Set complement}}: $A^c = \Omega \setminus A$
\end{enumerate}

Set complement is the ``strongest'' operation, because if a collection of sets $\mA$ is closed under complement, and if it is also closed under any one of the other three operations, $\mA$ is closed under the rest two. That is seen from,
\begin{align}
	\mbox{If closed under } union &
	\begin{cases}
		A \cap B = (A^c \cup B^c)^c\\
		A \setminus B = A \cap B^c = (A^c \cup B)^c
	\end{cases}	
	\\
	\mbox{If closed under } intersection &
	\begin{cases}
		A \cup B = (A^c \cap B^c)^c\\		
		A \setminus B = A \cap B^c
	\end{cases}
	\\
	\mbox{If closed under } difference &
	\begin{cases}
		A \cap B = A \setminus (A \setminus B)\\
		A \cup B = (A^c \cap B^c)^c = (A^c \setminus (A^c \setminus B^c))^c\\
	\end{cases}		
\end{align}

The difference operation is the second ``strongest'' operation, in that if a collection of sets $\mA$ is closed under difference, it is closed under intersection, that is seen from,
\begin{align}
	A \cap B = A \setminus (A \setminus B)
\end{align}

The third ``strongest'' operation is the union, which can not be implied from difference or intersection, or their combination.

The ``weakest'' operation is the intersection, which can be implied from the difference.

To summarize, the ``strongest'' pair would be the complement plus any one more, which implies everything else; the second ``strongest'' would be the difference plus the union, which implies the intersection but not the complement ($\Omega$ may not be in the collection). \smallmarginpar{\textdbend}

\subsubsection{Class of Set Collection}
\begin{definition}
	Given a set $\Omega$, a non-empty collection $\PP \subset 2^\Omega$ is called a {\em{$\pi$-system}} iff:
	
	\center{$\forall A, B \in \PP$, $A \cap B \in \PP$}
\end{definition}
Notice, this has the weakest requirement.

\begin{definition}
	Given a set $\Omega$, a non-empty collection $\QQ \subset 2^\Omega$ is called a {\em{semiring}} iff:
	
	\center{$\forall A, B \in \QQ \mbox{ and } A \supset B$}
	\center{$\exists C_k \subset \QQ, s.t., A \setminus B = \bigcup_{k=1}^n C_k$}
\end{definition}

\begin{definition}
	Given a set $\Omega$, a non-empty collection $\RR \subset 2^\Omega$ is called a {\em{ring}} iff:
	
	\center{$\forall A, B \in \RR$, $A \cup B \in \RR$ and $B \setminus A \in \RR$}
\end{definition}

There are two points need to mention: the empty set is in a ring, since $A \setminus A = \emptyset$; $\mA$ is also closed under intersection (the reverse need not be true).

\begin{definition}
	A {\em{ring}} $\mA$ is called an {\em{field}} iff $\Omega \in \mA$.
\end{definition}
So a field is closed under all {\em{finite}} combination of set operations.


\begin{definition}
	An {\em{field}} is called a {\em{$\sigma$-field}} if for any sequence ${A_n}$ of sets in $\mA$, $\cup_{n \ge 1} A_n \in \mA$.
\end{definition}

$\not \exists$

\section{Integration Theory}
\section{Random Variables}
\section{Law of Large Numbers}
\section{Central Limit Theorem}

\section{Some Tricks}
\subsection{Prove by Contraposition}

\subsection{Construct Finer Partition}
Given twofinite partitions $\{A_n\}$ and $\{B_m\}$, a finer partition can be constructed as
\begin{align*}
	(\cup_{i=1}^n A_i) \bigcap (\cup_{j=1}^m B_j) = \bigcup (\cap_{i=1}^n \cap_{j=1}^m A_i B_j)
\end{align*}

\subsection{Prove Equality}
To prove two numerical quantities are equal $X=Y$, often times we can do this by showing $X \le Y$ and $X \ge Y$. Similarly, to prove two sets are equal $E = F$, we can show $E \subset F$ and $E \subset F$.

\subsection{An Epsilon of Room}
If one has to show that $X \le Y$, try proving that $X \le Y + \epsilon, \forall \epsilon > 0$. This trick combines well with the ``Prove Equality'' trick.

In a similar spirit, if one needs to show that a quantity $X$ vanishes, try showing that $|X| \le \epsilon, \forall \epsilon > 0$.

If one wants to show that a sequence $x_n$ of real numbers converges to zero, try showing that $\limsup_{n \to \infty} |x_n| \le \epsilon, \forall \epsilon > 0$

{\bf{One caveat:}} for finite $x$, and any $\epsilon > 0$, it is true that $x + \epsilon > x$ and $x - \epsilon < x$, but it is not true when $x$ is equal to $+\infty$ or $-\infty$. \smallmarginpar{\textdbend}

\end{document} 






