\documentclass{article}
\usepackage{amssymb,amsmath}
\usepackage[thmmarks,amsmath]{ntheorem}

\usepackage{mathrsfs}


% Math commands by Thomas Minka
\newcommand{\var}{{\rm var}}
\newcommand{\Tr}{^{\rm T}}
\newcommand{\vtrans}[2]{{#1}^{(#2)}}
\newcommand{\kron}{\otimes}
\newcommand{\schur}[2]{({#1} | {#2})}
\newcommand{\schurdet}[2]{\left| ({#1} | {#2}) \right|}
\newcommand{\had}{\circ}
\newcommand{\diag}{{\rm diag}}
\newcommand{\invdiag}{\diag^{-1}}
\newcommand{\rank}{{\rm rank}}
% careful: ``null'' is already a latex command
\newcommand{\nullsp}{{\rm null}}
\newcommand{\tr}{{\rm tr}}
\newcommand{\vech}{{\rm vech}}
\renewcommand{\det}[1]{\left| #1 \right|}
\newcommand{\pdet}[1]{\left| #1 \right|_{+}}
\newcommand{\pinv}[1]{#1^{+}}
\newcommand{\erf}{{\rm erf}}
\newcommand{\hypergeom}[2]{{}_{#1}F_{#2}}

% boldface characters
\renewcommand{\a}{{\bf a}}
\renewcommand{\b}{{\bf b}}
\renewcommand{\c}{{\bf c}}
\renewcommand{\d}{{\rm d}}  % for derivatives
\newcommand{\e}{{\bf e}}
\newcommand{\f}{{\bf f}}
\newcommand{\g}{{\bf g}}
\newcommand{\h}{{\bf h}}
%\newcommand{\k}{{\bf k}}
% in Latex2e this must be renewcommand
\renewcommand{\k}{{\bf k}}
\newcommand{\m}{{\bf m}}
\newcommand{\n}{{\bf n}}
\renewcommand{\o}{{\bf o}}
\newcommand{\p}{{\bf p}}
\newcommand{\q}{{\bf q}}
\renewcommand{\r}{{\bf r}}
\newcommand{\s}{{\bf s}}
\renewcommand{\t}{{\bf t}}
\renewcommand{\u}{{\bf u}}
\renewcommand{\v}{{\bf v}}
\newcommand{\w}{{\bf w}}
\newcommand{\x}{{\bf x}}
\newcommand{\y}{{\bf y}}
\newcommand{\z}{{\bf z}}
\newcommand{\A}{{\bf A}}
\newcommand{\B}{{\bf B}}
\newcommand{\C}{{\bf C}}
\newcommand{\D}{{\bf D}}
\newcommand{\E}{{\bf E}}
\newcommand{\F}{{\bf F}}
\newcommand{\G}{{\bf G}}
\renewcommand{\H}{{\bf H}}
\newcommand{\I}{{\bf I}}
\newcommand{\J}{{\bf J}}
\newcommand{\K}{{\bf K}}
\renewcommand{\L}{{\bf L}}
\newcommand{\M}{{\bf M}}
\newcommand{\N}{{\cal N}}  % for normal density
%\newcommand{\N}{{\bf N}}
\renewcommand{\O}{{\bf O}}
\renewcommand{\P}{{\bf P}}
\newcommand{\Q}{{\bf Q}}
\newcommand{\R}{{\bf R}}
\renewcommand{\S}{{\bf S}}
\newcommand{\T}{{\bf T}}
\newcommand{\U}{{\bf U}}
\newcommand{\V}{{\bf V}}
\newcommand{\W}{{\bf W}}
\newcommand{\X}{{\bf X}}
\newcommand{\Y}{{\bf Y}}
\newcommand{\Z}{{\bf Z}}

% this is for latex 2.09
% unfortunately, the result is slanted - use Latex2e instead
%\newcommand{\bfLambda}{\mbox{\boldmath$\Lambda$}}
% this is for Latex2e
\newcommand{\bfLambda}{\boldsymbol{\Lambda}}

% Yuan Qi's boldsymbol
\newcommand{\bsigma}{\boldsymbol{\sigma}}
\newcommand{\balpha}{\boldsymbol{\alpha}}
\newcommand{\bpsi}{\boldsymbol{\psi}}
\newcommand{\bphi}{\boldsymbol{\phi}}
\newcommand{\bbeta}{\boldsymbol{\beta}}
\newcommand{\Beta}{\boldsymbol{\eta}}
\newcommand{\btau}{\boldsymbol{\tau}}
\newcommand{\bvarphi}{\boldsymbol{\varphi}}
\newcommand{\bzeta}{\boldsymbol{\zeta}}

\newcommand{\blambda}{\boldsymbol{\lambda}}
\newcommand{\bLambda}{\mathbf{\Lambda}}

\newcommand{\btheta}{\boldsymbol{\theta}}
\newcommand{\bpi}{\boldsymbol{\pi}}
\newcommand{\bxi}{\boldsymbol{\xi}}
\newcommand{\bSigma}{\boldsymbol{\Sigma}}

\newcommand{\bgamma}{\mathbf{\gamma}}
\newcommand{\bGamma}{\mathbf{\Gamma}}

\newcommand{\bmu}{\boldsymbol{\mu}}
\newcommand{\1}{{\bf 1}}
\newcommand{\0}{{\bf 0}}

%\newcommand{\comment}[1]{}

\newcommand{\bs}{\backslash}
\newcommand{\ben}{\begin{enumerate}}
\newcommand{\een}{\end{enumerate}}

 \newcommand{\notS}{{\backslash S}}
 \newcommand{\nots}{{\backslash s}}
 \newcommand{\noti}{{\backslash i}}
 \newcommand{\notj}{{\backslash j}}
 \newcommand{\nott}{\backslash t}
 \newcommand{\notone}{{\backslash 1}}
 \newcommand{\nottp}{\backslash t+1}
% \newcommand{\notz}{\backslash z}

\newcommand{\notk}{{^{\backslash k}}}
%\newcommand{\noti}{{^{\backslash i}}}
\newcommand{\notij}{{^{\backslash i,j}}}
\newcommand{\notg}{{^{\backslash g}}}
\newcommand{\wnoti}{{_{\w}^{\backslash i}}}
\newcommand{\wnotg}{{_{\w}^{\backslash g}}}
\newcommand{\vnotij}{{_{\v}^{\backslash i,j}}}
\newcommand{\vnotg}{{_{\v}^{\backslash g}}}
\newcommand{\half}{\frac{1}{2}}
\newcommand{\msgb}{m_{t \leftarrow t+1}}
\newcommand{\msgf}{m_{t \rightarrow t+1}}
\newcommand{\msgfp}{m_{t-1 \rightarrow t}}

\newcommand{\proj}[1]{{\rm proj}\negmedspace\left[#1\right]}
\newcommand{\argmin}{\operatornamewithlimits{argmin}}
\newcommand{\argmax}{\operatornamewithlimits{argmax}}

\newcommand{\dif}{\mathrm{d}}
\newcommand{\abs}[1]{\lvert#1\rvert}
\newcommand{\norm}[1]{\lVert#1\rVert}

\newcommand{\GP}{Gaussian process\xspace}
\newcommand{\GPs}{Gaussian processes\xspace}
\newcommand{\astar}{\alpha^{\star}}
\newcommand{\alphas}{{\alpha^{\star}}}
\newcommand{\betas}{\beta^{\star}}

\newcommand{\FF}{\mathcal{F}}
\newcommand{\XX}{\mathcal{X}}
\newcommand{\YY}{\mathcal{Y}}
\newcommand{\HH}{\mathcal{H}}
\newcommand{\LL}{\mathcal{L}}
\newcommand{\NN}{\mathcal{N}}
\newcommand{\DD}{\mathcal{D}}
\newcommand{\RR}{\mathbb{R}}

\theorembodyfont{\normalfont}
\newtheorem{problem}{Problem}[section]
\newtheorem{definition}{Definition}[section]

\title{Linear Regression Models}
\author{Xi Tan (tan19@purdue.edu)}
\date{\today}

\begin{document}
\maketitle
\tableofcontents
\newpage

\section*{Preface}
TBD

\newpage
\section{Introduction}
Generalized linear models include as special cases,  linear regression and analysis-of-variance models, logit and probit models for quantal responses, log linear models and multinomial response models for counts and some commonly used models for survival data.

The second-order properties of the parameter estimates are insensitive to the assumed distributional form: the second-order properties depend mainly on the assumed variance-to-mean relationship and on uncorrelatedness or independence.

Data types:

\newpage
\section{Simple Linear Regression}

\subsection{Model}
\begin{equation}
	Y_i = \beta_0 + \beta_1 X_i + \epsilon_i
\end{equation}
where $\epsilon_i \sim \NN(0,\sigma^2)$

\subsection{Estimators}
\begin{eqnarray}
	b_1 & = & \frac{1}{\sum_{i=1}^n (X_i - \bar X)^2 } \sum_{i=1}^n (X_i-\bar X) Y_i \\
	b_0 & = & \bar Y - b_1 \bar X \\
	\hat \sigma^2 & = & \frac{\sum_{i=1}^n (Y_i - \hat Y_i)^2}{n-2}
\end{eqnarray}
Notice, $\sum(X_i-\bar X)^2 = \sum X_i^2 - n\bar X^2$, and $b_1 = \rho \cdot \frac{s_Y}{s_X}$, where $\rho$ is the correlation between $X$ and $Y$ and $s_Y, s_X$ are standard error of $Y$ and $X$, respectively.

\subsection{Properties of Residuals}
\begin{eqnarray}
	e_i & = & Y_i - \hat Y_i \\
	\sum e_i & = & 0 \\
	\sum X_i e_i & = & 0 \\
	\sum \hat Y_i e_i & = & 0
\end{eqnarray}

\subsection{Properties of $b_1$ and $b_0$}
\begin{eqnarray}
	b_1 &\sim& \NN \left(\beta_1,\frac{\sigma^2}{\sum(X_i-\bar X)^2}\right) \\
	b_0 &\sim& \NN \left(\beta_0,\frac{\sigma^2}{n} + \frac{\sigma^2 {\bar X}^2}{\sum(X_i-\bar X)^2} \right)
\end{eqnarray}
where $\sigma^2$ can be estimated by the MSE, i.e., $\hat \sigma^2 = \frac{\sum_{i=1}^n (Y_i - \hat Y_i)^2}{n-2}$

\noindent
\\
Now, since
\begin{eqnarray}
	b_1 & = & \frac{1}{\sum_{i=1}^n (X_i - \bar X)^2 } \sum_{i=1}^n (X_i-\bar X) Y_i \\
	& = & \sum_{i=1}^n k_i Y_i
\end{eqnarray}
where $k_i = \frac{X_i-\bar X}{\sum_{i=1}^n (X_i - \bar X)^2 }$, we have
\begin{eqnarray}
	\sum k_i &=& 0 \\
	\sum X_i k_i &=& 1 \\
	\sum k_i^2 &=& \frac{1}{\sum_{i=1}^n (X_i - \bar X)^2 }
\end{eqnarray}
The first two identity hold as a requirement for the unbiasness, since $$E(b_1) = E\left(\sum k_i Y_i\right)= E\left(\sum k_i (\beta_0+\beta_1 X_i)\right) = E\left(\beta_0 \sum k_i \beta_0 + \beta_1 \sum k_i X_i \right) = \beta_1$$ requires $\sum k_i = 0$  and $\sum X_i k_i = 1$.
The third identity ensures the attainment of the minimum variance.

\subsection{Confidence Interval of $b_1$ and $b_0$}
Since $SSE/\sigma^2 \sim \chi_{n-2}^2$, and $\frac{s^2\{b_1\}}{\sigma^2\{b_1\}} \sim \frac{\chi_{n-2}^2}{n-2}$
\begin{eqnarray}
	\frac{b_1-\beta_1}{s\{b_1\}} = \left. \frac{b_1-\beta_1}{\sigma\{b_1\}} \middle/ \frac{s\{b_1\}}{\sigma\{b_1\}} \right. \sim \frac{z}{\sqrt{\frac{\chi_{n-2}^2}{n-2}}} = t_{n-2}
\end{eqnarray}
so the confidence interval for $b_1$, with confidence level $\alpha$ is
\begin{equation}
	b_1 \pm t(1-\alpha/2;n-2)s\{b_1\}\\
\end{equation}
or
\begin{equation}
	b_1 \mp t(\alpha/2;n-2)s\{b_1\}\\
\end{equation}

Similarly, the confidence interval for $b_0$, with confidence level $\alpha$ is
\begin{equation}
	b_0 \pm t(1-\alpha/2;n-2)s\{b_0\}\\
\end{equation}
or
\begin{equation}
	b_0 \mp t(\alpha/2;n-2)s\{b_0\}\\
\end{equation}

\noindent
\\
The power of testing $\beta_1 = \beta^{H_0}$ is $Power = P\{|t^*|>t(1-\alpha/2;n-2)|\delta\}$, where $\delta = \frac{|\beta_1 - \beta^{H_0}|}{\sigma\{b_1\}}$. Similar for $\beta_0$.


%\begin{landscape}
\begin{table}[h]
\centering
\begin{tabular}{c || c c c }
\hline
& {Estimate} & {Expectation} & {Variance} \\
\hline
\\
$Y_i$& $\hat Y_i$ & $\beta_0+\beta_1 X_i$ & $\sigma^2$\\
\\
\hline
\\
$b_1$& $\frac{\sum (X_i-\bar X) Y_i}{\sum (X_i-\bar X)^2}$ & $\beta_1$ & $\sigma^2 \cdot \frac{1}{\sum (X_i-\bar X)^2}$\\
\\
\hline
\\
$b_0$& $\bar Y - b_1 \bar X$ & $\beta_0$ & $\sigma^2 \cdot \left[\frac{1}{n} + \frac{{\bar X}^2}{\sum (X_i-\bar X)^2}\right]$\\
\\
\hline
\\
$\hat Y_i$& $\bar Y + b_1 (X_i-\bar X)$ & $\beta_0+\beta_1 X_i$ & $\sigma^2 \cdot \left[\frac{1}{n} + \frac{(X_i-\bar X)^2}{\sum (X_i-\bar X)^2}\right]$\\
\\
\hline
\\
$e_i$& $Y_i-\hat Y_i$ & $0$ & $TBD$\\
\\
\hline
\end{tabular}
\caption{Simple Linear Regression}
\end{table}
%\end{landscape}

\end{document} 






