\documentclass{article}
\usepackage{amssymb,amsmath}
\usepackage[thmmarks,amsmath]{ntheorem}


\usepackage{mathrsfs}



\input{emacscomm.tex}
\newcommand{\GP}{Gaussian process\xspace}
\newcommand{\GPs}{Gaussian processes\xspace}
\newcommand{\astar}{\alpha^{\star}}
\newcommand{\alphas}{{\alpha^{\star}}}
\newcommand{\betas}{\beta^{\star}}

\newcommand{\FF}{\mathcal{F}}
\newcommand{\XX}{\mathcal{X}}
\newcommand{\YY}{\mathcal{Y}}
\newcommand{\HH}{\mathcal{H}}
\newcommand{\LL}{\mathcal{L}}
\newcommand{\DD}{\mathcal{D}}
\newcommand{\RR}{\mathbb{R}}

\theorembodyfont{\normalfont}
\newtheorem{problem}{Problem}[section]
\newtheorem{definition}{Definition}[section]

\title{Review of Elementary Stochastic Processes Thoeries}
\author{Xi Tan (tan19@purdue.edu)}
\date{\today}

\begin{document}
\maketitle
\tableofcontents
\newpage

\section*{Preface}
TBD

\newpage
\section{Markov Chains}
A {\em{stochastic process}} can be defined quite generally as any collection of random variables $X(t)$, $t \in T$, defined on a common probability space, where $T$ is a subset of $(-\infty, \infty)$ and is thought of as the time parameter set. If the random variables $X(t)$ all take on values from the fixed set $\mathscr{S}$, then $\mathscr{S}$ is called the {\em{state space}} of the process.


\section{Stationary Distributions of a Markov Chain}

\section{Markov Pure Jump Processes}
\section{Second Order Processes}

\end{document} 






