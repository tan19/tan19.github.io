\documentclass{article}
\usepackage{amssymb,amsmath}
\usepackage[thmmarks,amsmath]{ntheorem}

\usepackage{graphicx}
\usepackage{color}

\usepackage{hyperref}


\input{emacscomm.tex}
\newcommand{\GP}{Gaussian process\xspace}
\newcommand{\GPs}{Gaussian processes\xspace}
\newcommand{\astar}{\alpha^{\star}}
\newcommand{\alphas}{{\alpha^{\star}}}
\newcommand{\betas}{\beta^{\star}}

\newcommand{\rBeta}{{\rm B}}


\newcommand{\FF}{\mathcal{F}}
\newcommand{\XX}{\mathcal{X}}
\newcommand{\YY}{\mathcal{Y}}
\newcommand{\HH}{\mathcal{H}}
\newcommand{\LL}{\mathcal{L}}
\newcommand{\DD}{\mathcal{D}}
\newcommand{\RR}{\mathbb{R}}

\theorembodyfont{\normalfont}
\newtheorem{problem}{Problem}[section]
\newtheorem{definition}{Definition}[section]

\setlength{\marginparwidth}{1.2in}
\newcommand{\smallmarginpar}[1]{\marginpar[#1]{\small{#1}}}

\newcommand{\notion}[1]{{\em{#1}}}


\title{Dirichlet Distribution and Its Applications}
\author{Xi Tan (tan19@purdue.edu)}
\date{\today}

\begin{document}
\maketitle

\section {Gamma Function and Beta Function}
The \notion{gamma function} is defined for all complex numbers except the non-positive integers. For complex numbers with a positive real part, it is defined via an improper integral that converges:
\begin{equation}
	\Gamma(z) = \int_0^\infty e^{-t} t^{z-1} dt
\end{equation}
For all positive numbers $z$,
\begin{equation}
	\Gamma(z) = (z-1)\Gamma(z-1)
\end{equation}
and in particular, if $n$ is a positive integer:
\begin{equation}
	\Gamma(n) = (n-1)!
\end{equation}
Note, the gamma function shifts the normal definition of factorial by 1.

The \notion{beta function} is defined by
\begin{equation}
	\rBeta(x,y) = \int_0^1 t^{x-1} (1-t)^{y-1} dt
\end{equation}
for $Re(x), Re(y) > 0$.

It can also be written as
\begin{equation}
	\rBeta(x,y) = \frac{\Gamma(x) \Gamma(y)}{\Gamma(x+y)}
\end{equation}



\section {Introduction}
The Dirichlet distribution of order $K \ge 2$ with parameters $\alpha_1, \dots, \alpha_K > 0$ has a probability density function with respect to Lebesgue measure on the Euclidean space \smallmarginpar{I'm not quite sure about this "Lebesque measure on the Euclidean space" thing.} $\RR^{K-1}$ given by
\begin{equation}
	f(x_1, \dots, x_{K-1}; \alpha_1, \dots, \alpha_K) = \frac{1}{B(\balpha)} \prod^K_{i=1} x_i^{\alpha_i-1}
\end{equation}
for all $x_1, \dots, x_K > 0$ and $x_1 + \dots + x_K = 1$. The density is zero outside this open \smallmarginpar{``Open'' here means none of the $x_i$'s can be 1, actually $x_i \in (0,1)$.} $(K-1)$-dimensional simplex.

The normalizing constant is the multinomial beta function, which can be expressed in terms of the gamma function:
\begin{equation}
	B(\balpha) = \frac{\prod_{i=1}^K \Gamma(\alpha_i)}{\Gamma(\sum_{i=1}^K \alpha_i)}, \mbox{ where } \balpha=(\alpha_1, \dots, \alpha_K)
\end{equation}

\end{document} 






